%chapter 13
%original by wqmeeker  3 aug 1994
%edited by wqmeeker  26 may 1995 bringing in meeker/hamada and lu/meeker 
%edited by wqmeeker  27/30 may 1995 smoothing
%edited by wqmeeker  3 june 1995 smoothing luis' comments
%edited by wqmeeker  5 june 1995 minor changes
%edited by driker 22 june 1995
%edited by driker 3 august 1995
%edited by wqmeeker  6 august 1995
%edited by wqmeeker  21 dec 1995 changing to sqrt(pi) model
%edited by wqmeeker  25 may 1996 additional figures from lu&meeker
%edited by driker 22 october 1996
%edited by driker 4 april 1997
%edited by driker 22 july 1997

\setcounter{chapter}{12}

%\setcounter{page}{m}

\chapter{Degradation Data, Models, and Data Analysis}
\label{chapter:degradation.data}


%{\Large {\bf William Q. Meeker and Luis A. Escobar}}\\
%Iowa State University and Louisiana State University\\[5ex]
{\large {\bf \today}}\\[2ex]
Part of 
{\em Statistical Methods for Reliability Data}\\
Copyright 1997 W. Q. Meeker and L. A. Escobar. \\[1ex]
To be published by John Wiley \& Sons Inc. in 1998.



%----------------------------------------------------------------------
%----------------------------------------------------------------------
\section*{Objectives}
This chapter explains
\begin{itemize} 
\item 
Some useful degradation models.
\item
The connection
between degradation models and failure-time models.
\item
How degradation measures, when available, can be used to advantage
in estimating reliability.
\item 
Methods for data analysis and reliability inference with
degradation data.
\item 
The differences between degradation data analysis and traditional
failure time data analysis.
\item
A simple method for degradation analysis that can be useful in
certain situations.
\end{itemize}

%----------------------------------------------------------------------
%----------------------------------------------------------------------
\section*{Overview}
This chapter introduces the concepts of degradation analysis as they relate
to product reliability. Many failure mechanisms can be traced to an 
underlying degradation process. Degradation eventually leads to a weakness
that can cause failure. When it is possible to measure degradation,
such measures often provide more information than failure-time data for
purposes of assessing and improving product reliability. For some 
products direct observation of degradation level is impossible, but it
may be that product performance data will be a useful substitute. 
This chapter, Chapter~\ref{chapter:accelerated.test.models}, and 
Chapter~\ref{chapter:accelerated.degradation} provide a brief introduction
to this important subject (a complete treatment would
require a separate book).  This chapter should be read before
Chapter~\ref{chapter:accelerated.degradation}.  Readers may skip this
chapter if their primary interest is in failure-time data, although
Section~\ref{section:degrad.models} does
give an introduction to some physics-of-failure concepts that provide
useful motivation for failure-time models.
Section~\ref{section:est.degrad.param} extends ML methods from earlier
chapters to deal with the more complicated degradation models.
Sections~\ref{section:degrad.and.failure},
\ref{section:eval.of.ft}, \ref{section:est.ft}, and \ref{section:degrad.ci} 
relate degradation and failure time and show how to estimate $F(t)$
from degradation data.  Section~\ref{section:compare.ft.da} uses an
example to compare degradation analysis with traditional
failure-time analysis.  Section~\ref{section:app.deg.anal} presents
a simple approximate method for degradation analysis that might be
appropriate in some applications.

%----------------------------------------------------------------------
%----------------------------------------------------------------------
\section{Introduction}
Design of high-reliability systems generally requires that the
individual system components have extremely high reliability, even
after long periods of time.  With short product development times,
reliability tests must be conducted with severe time constraints.
Frequently no failures occur during such tests.  Thus, it is difficult
to assess reliability with traditional life tests that record only
failure time.  For some components degradation measures can be
taken over time.  A relationship between component failure and amount
of degradation makes it possible to use degradation models and data to
make inferences and predictions about failure time.

%----------------------------------------------------------------------
%----------------------------------------------------------------------
\section{Models for Degradation Data}
\label{section:degrad.models}
%-------------------------------------------------------------------
\subsection{Degradation data}
In some reliability studies, it is possible to measure the physical
degradation as a function of time (e.g., tire wear).  In other
applications actual physical degradation cannot be observed
directly, but measures of product performance degradation (e.g.,
power output) may be available. Both kinds of data are generically
referred to as ``degradation data'' and we will follow this
convention.  Modeling performance degradation may be useful, but
could be complicated because performance may be affected by more
than one underlying degradation process.  Depending on the
application, degradation data may be available continuously or at
specific points in time where measurements are taken.


In most reliability testing applications, degradation data, if
available, will have important practical advantages. In particular,
\begin{itemize}
\item
Degradation data can, especially in
applications with few or no failures, provide
considerably more reliability information
than traditional censored failure-time data.
\item
Accelerated tests are commonly used to obtain reliability test
information more quickly.  Direct observation of physical
degradation process (e.g., tire wear) or some closely related
surrogate may allow direct modeling of the failure-causing
mechanism, providing more credible and precise reliability estimates
and a firmer basis for often-needed extrapolation.
\end{itemize}

\begin{example}
\label{example:bk.fatigue.data.points}
{\bfseries Fatigue crack-size data.} Recall the Alloy-A fatigue
crack-size data in Example~\ref{example:bk.fatigue.data}.
Figure~\ref{figure:bk.fatigue.data.points.ps} is similar to
Figure~\ref{figure:bk.fatigue.data.ps}, but includes the actual data
points on crack size given in
Appendix Table~\ref{atable:bk.fatigue.data}. The initial crack size (i.e., at
time 0) for each path was .9 inch, the size of the notch cut into
each specimen.  Suppose that investigators wanted to estimate the
material's crack growth parameters and the time (measured in number
of cycles) at which 50\% of the cracks would reach 1.6 inches (a
size considered to be dangerous).
%-------------------------------------------------------------------
\begin{figure}
\splusbookfigure{\figurehome/bk.fatigue.data.points.ps}
\caption{Alloy-A fatigue crack data.}
\label{figure:bk.fatigue.data.points.ps}
\end{figure}
%-------------------------------------------------------------------
For purposes of our degradation analysis the fatigue experiment for
each specimen was terminated at the first inspection after a unit's
cracks reached 1.6 inch or censored after .12 million cycles,
whichever came first.
\end{example}

%-------------------------------------------------------------------
\subsection{Degradation leading to failure}
\label{section:deg.to.fail}
Most failures can be traced to an underlying degradation process.
Figure~\ref{figure:degradation.comparison.ps} shows examples of three
general shapes for degradation curves in arbitrary units
of degradation and time: linear, convex, and concave.  The horizontal
line at degradation level .6 represents the level or approximate
level at which failure would occur.  In some applications there may be
more than one degradation variable or more than one underlying
degradation process. The following examples, however, use only a
single degradation variable.
%-------------------------------------------------------------------
\begin{figure}
\splusbookfigure{\figurehome/degradation.comparison.ps}
\caption{Possible shapes for univariate degradation curves.}
\label{figure:degradation.comparison.ps}
\end{figure}
%-------------------------------------------------------------------

The following examples describe some specific models for degradation curves.
Engineers and physical scientists must find such models in their literature
or develop them from basic principles relating to the underlying degradation
process. Usually such models start with a deterministic description of the
degradation process --- often in the form of a differential equation or
system of differential equations. Then randomness can be introduced, as
appropriate, using probability distributions to describe variability in
initial conditions and model parameters like rate constants or materials
properties.

\begin{example}{\bfseries Linear degradation.}
Linear degradation arises in some simple wear processes
(e.g., automobile tire wear). For example if $\degpath(t)$ 
is the amount of automobile tire tread wear
at time $t$ and
wear rate is $d\, \degpath(t)/d\, t = C$
then $\degpath(t)=\degpath(0)+C \times t$.
\end{example}
The parameters $\degpath(0)$ and $C$ could be taken as constant for
individual units, but random from unit to unit.

\begin{example}{\bfseries Convex degradation.}
Models for which the degradation rate increases with the level of
degradation are, for example, used in modeling the growth of fatigue
cracks. Let $a(t)$ denote the size of a crack at time $t$.  A simple
version of the deterministic Paris-rule model (e.g., Dowling~1993)
\begin{equation}
\label{equation:paris.diffyq}
\frac{d\,a(t)}{d\,t} = C  \times \left[ \Delta K(a) \right ]^{m}
\end{equation}
provides a useful model for cracks within a certain size range.  Here
$C$ and $m$ are materials properties and $\Delta K(a)$ (known as the
``stress intensity range function'') is a function of crack size $a$,
the range of applied stress, part dimensions, and geometry. For
example, to model a two-dimensional edge-crack in a plate with a crack
that is small relative to the width of the plate (say less than 3\%),
$\Delta K(a)=\text{Stress} \sqrt{\pi a}$. The deterministic solution to the
resulting differential equation is
\begin{equation}
\label{equation:paris.diffyq.solution}
a(t) = \left\{ \begin{array}{lll}
\left[\{a(0)\}^{1-m/2} + 
(1-m/2)  \times C  \times (\text{Stress}\sqrt{\pi})^{m}  \times 
t\right]^{\displaystyle 2/(2-m)}, & m \ne 2\\[3ex]
a(0)  \times  \exp \left [ C  \times (\text{Stress}\sqrt{\pi})^{2} \times
 t \right ], & m = 2 .
\end{array} \right.
\end{equation}
This solution is illustrated for $m=2.05$ with the convex curve in
Figure~\ref{figure:degradation.comparison.ps}.
\end{example}

\begin{example}{\bfseries Concave degradation.}
\label{example:concave.deg}
Meeker and LuValle~(1995) describe models for growth of
failure-causing conducting filaments of chlorine-copper compounds in
printed-circuit boards. These filaments cause failure when they reach
from one copper-plated through-hole to another.
In their model, $A_{1}(t)$ is the amount of
chlorine available for reaction and $A_{2}(t)$ is proportional to the
amount of failure-causing chlorine-copper compounds at time $t$. Under
appropriate conditions of temperature, humidity, and electrical
charge, copper combines with chlorine $A_{1}$ to produce 
the chlorine-copper compound $A_{2}$
with rate constant $k_{1}$. Diagrammatically,\\
\begin{center}
\setlength{\unitlength}{.1mm}
\begin{picture}(360,120)
\thicklines
\put(5,60){\makebox(0,0){$\A_{1}$}}
\put(30,60){\vector(1,0){210}}
\put(135,85){\makebox(0,0){$k_{1}$}}
\put(270,60){\makebox(0,0){$\A_{2}$.}}
\end{picture}\\
\end{center}
The rate equations for this process are
\begin{equation}
\label{equation:luvalle.meeker.diffyq}
\frac{d \A_{1}}{dt} =  -k_1 \A_{1} \quad \text{and} \quad
\frac{d \A_{2}}{dt} =  k_1  \A_{1}. 
\end{equation}
The solution of this system of differential equations
gives
\begin{eqnarray}
\label{equation:luvalle.meeker.diffyq.sol}
\A_{1}(t)&=&\A_{1}(0) \times \exp(-k_{1}t) \\
\A_{2}(t)&=&\A_{2}(0)+\A_{1}(0)\times[1-\exp(-k_{1}t)]
\end{eqnarray}
where $\A_{1}(0)$ and $\A_{2}(0)$ are initial amounts. To simplify notation,
let $A_{2}(\infty)=\A_{1}(0)+\A_{2}(0)$. Then if $A_{2}(0)=0$,
the solution for $\A_{2}(t)$
(the quantity of primary interest) can be expressed as
\begin{equation}
\label{equation:luvalle.meeker.diffyq.sol2}
A_{2}(t) = A_{2}(\infty)\times [1 -  \exp(-k_{1}t)].
\end{equation}
This function is illustrated by the concave curve in
Figure~\ref{figure:degradation.comparison.ps}.  The asymptote at
$A_{2}(\infty)$ reflects the finite amount of chlorine available for the
reaction producing the harmful compounds.
\end{example}
Meeker and LuValle~(1995) also suggest other more elaborate models
for this failure process.  Carey and Koenig~(1991) use similar models
to describe degradation of electronic components.
Chapter~\ref{chapter:accelerated.test.models} describes the ideas behind
acceleration of failure-causing processes like these.

%-------------------------------------------------------------------
\subsection{Models for variation in degradation and failure times}
\label{section:models.for.variation}
If all manufactured units were identical, operated 
under exactly the same conditions, and in exactly the same
environment, and if every unit failed as it reached a particular
``critical'' level of degradation, then, according to the simple deterministic
models above, all units would fail at
exactly the same time.  Of course, there is some degree of variability
in all of these model factors as well as in factors that are not in the
model. These factors combine to cause variability in the degradation 
curves and in the failure times.

%-------------------------------
\mbox{  }\\
\noindent
{\bf Unit-to-unit variability.}
The following are examples of sources of unit-to-unit variability:
%-------------------------------------------------------------------
\begin{figure}
\splusbookfigure{\figurehome/paris.degradation.fixedm.ps}
\caption{Plot of Paris model for growth of fatigue cracks
with unit-to-unit variability in the initial crack size $a_{0}$ but
with constant materials parameters ($C$ and $m$) and constant stress.}
\label{figure:paris.degradation.fixedm.ps}
\end{figure}
%-------------------------------------------------------------------
%-------------------------------------------------------------------
\begin{figure}
\splusbookfigure{\figurehome/paris.degradation.stochm.ps}
\caption{Plot of Paris model for growth of fatigue cracks
with unit-to-unit variability in the initial crack size and in
materials parameters $C$ and $m$, but with constant stress.}
\label{figure:paris.degradation.stochm.ps}
\end{figure}
%-------------------------------------------------------------------
%-------------------------------------------------------------------
\begin{itemize}
\item
{\bfseries Initial conditions.} Individual units will vary with
respect to the amount of material available to wear, initial level of
degradation, amount of harmful degradation-causing material, and so
on.  Figure~\ref{figure:paris.degradation.fixedm.ps} shows the Paris
model for growth of fatigue cracks, with simulated variability in the
size of the initial crack, but with all other of the unit's Paris model
characteristics and other factors held
constant.
\item
{\bf Material properties}.
Figure~\ref{figure:paris.degradation.stochm.ps} shows the Paris model
for growth of fatigue cracks, allowing for unit-to-unit variability in
the material properties parameters $C$ and $m$ and the size of the
initial crack. In this case, as shown in the Paris model in
(\ref{equation:paris.diffyq}), the rate of growth depends on $C$ and
$m$ which differ from unit to unit. This yields crossing of the
crack-growth curves (typical of what is observed in actual fatigue
testing).
\item
{\bf Component geometry or dimensions.} Unit-to-unit variability in
component geometry or dimensions can, for example, cause additional
unit-to-unit variability in degradation rates [e.g., through the
$\Delta K(a)$ function in (\ref{equation:paris.diffyq})].
\item
{\bf Within-unit variability.} Often there will be spatial
variability in materials properties within a unit (e.g., defects).
\end{itemize}

%-------------------------------
\mbox{  }\\
\noindent
{\bf Variability due to operating and environmental conditions.}
Besides the materials properties described above, the rate of degradation
will depend on operating and environmental conditions.  For example,
$K(a)$ in the Paris model (\ref{equation:paris.diffyq}) depends on
the amount of applied stress and the Paris parameters can depend on 
temperature. In laboratory fatigue tests, the stress
is either fixed or changing in a systematic manner (e.g., to keep
$K(a)$ nearly constant as $a$ increases). In actual operation
of most components, stress would generally be a complicated function over 
time. Such variations possibly described by a stochastic process model. 
Figure~\ref{figure:paris.degradation.stochm.stosig.ps}
shows the Paris model with degradation rate varying due to variations
in stress that might have been caused, for example, by variation in
driving conditions encountered, over time, by an automobile. In some
applications, shocks that occur randomly in time can dominate other sources
of variability in a failure-causing process.

The models described here are simple relative to more exact theory of
failure-causing processes that, almost certainly exist (but may not be
known or understood). For some purposes, however, such simple first-order
descriptions are useful.

%-------------------------------------------------------------------
\begin{figure}
\splusbookfigure{\figurehome/paris.degradation.stochm.stosig.ps}
\caption{Plot of Paris model for growth of fatigue cracks
with unit-to-unit variability in the initial crack size and
materials parameters $C$ and $m$, and with a stochastic process
model for the changes in stress over the life of the unit.}
\label{figure:paris.degradation.stochm.stosig.ps}
\end{figure}
%-------------------------------------------------------------------

\subsection{Limitations of degradation data}
Physical degradation or performance degradation are natural property
to measure for many testing processes (e.g., monitoring crack size
of a specimen subjected to stress cycling or power output of an
electronic device). Often, however, degradation measurement of a
unit requires destructive inspection (e.g., destructive strength
tests) or disruptive measurement (e.g., disassembly and reassembly
of a motor) that has the potential to change the degradation process. In
such situations one can obtain only a single measurement on each
unit tested. It is possible to extract useful information from such
data if a large number of units can be tested (for an example, see
Nelson~1990a, Chapter 11).

The advantages of degradation data can also be compromised when the
degradation measurements are contaminated with large amounts of
measurement error or when the degradation measure is not closely
related to failure. For example, when the degradation measurement is
on performance degradation, rather than physical degradation,
failures may occur for physical reasons that are not or cannot be
observed directly.

\begin{example}{\bfseries Laser degradation and defective lasers.}
\label{example:laser.defect.deg}
Over the life of some laser devices, degradation causes a decrease
in light output. Some lasers, however, contain a feedback mechanism
that will maintain nearly constant light output by increasing operating
current as the laser degrades. When operating current gets too high,
the device is considered to have failed.
Figure~\ref{figure:laser.degradation.strate.defect.ps} shows the
increase in operating current over time for a sample of GaAs lasers
tested at $80\degreesc$ (this temperature, much higher than the use
temperature, was used to accelerate the failure mechanism so that
degradation information would be obtained more rapidly---see
Chapter~\ref{chapter:accelerated.degradation}).  Some of the units
degrade gracefully; others fail suddenly.  Sudden failures, like
those in Figure~\ref{figure:laser.degradation.strate.defect.ps}
usually indicate manufacturing or other quality problems in an
immature product. Such behavior is common, especially in outputs of
electronic devices.  Possible reasons for sudden failures include
\begin{itemize}
\item
An {\em unobserved} sudden change in the physical state of the unit
that would lead to a subsequent increase in the degradation rate
(e.g., growth of a conducting path that suddenly causes a short circuit).
\item
Manufacturing defects (often observed in early development of a new
product). 
\item
A different failure mode actuated only at high temperatures.
\item
Inadvertent shocks to units.
\end{itemize}
As a product's design, manufacturing, and testing processes mature, such
problems are usually eliminated.
%-------------------------------------------------------------------
\begin{figure}
\splusbookfigure{\figurehome/laser.degradation.strate.defect.ps}
\caption{Plot of percent increase in operating current for GaAs
lasers tested at $80\degreesc$.}
\label{figure:laser.degradation.strate.defect.ps}
\end{figure}
%-------------------------------------------------------------------
\end{example}

When there is not a strong correlation between failure times and
degradation, there may be little to be gained by
using degradation data instead of traditional censored failure time
data. The limited amount of information in such degradation
measurements can be the result of monitoring a {\em
performance} variable (e.g., output voltage) rather than the actual
physical degradation (e.g., amount of material displaced by
electromigration).  An important but difficult engineering challenge of
degradation analysis is to find variables that are closely related to 
failure time and develop methods for accurately measuring these variables.

%----------------------------------------------------------------------
\subsection{General degradation path model}
\label{section:general.degradation.model}

The actual degradation path of a particular unit over time is
denoted by $\degpath(t), t > 0$. In applications, values of
$\degpath(t)$
are sampled at discrete points in time $t_{1}, t_{2}, \dots $.
The observed sample degradation $y_{ij}$ of unit $i$ at time $t_j$ is
\begin{equation}
\label{equation:general.deg.path.model}
y_{ij} = \degpath_{ij} + \epsilon_{ij},\quad i
= 1, \dots, n, \quad j = 1, \dots, m_{i}
\end{equation}
where $\degpath_{ij}=\degpath(t_{ij},\beta_{1i},\dots,\beta_{ki})$ 
is the actual
path of the unit $i$ at time $t_{ij}$ (the times need not be the
same for all units) and $\epsilon_{ij}\sim \NOR
\left(0,\sigma_{\epsilon}\right)$ is a residual deviation
for unit $i$ at time $t_j$. The
total number of inspections on unit $i$ is denoted by $m_{i}$.
Time $t$ could be real-time, operating time, or some other appropriate
measure of use like
miles for automobile tires or cycles in fatigue tests.  For the $i$th
unit, $\beta_{1i}, \dots , \beta_{ki}$ is a vector of $k$ unknown
parameters.  Typically sample paths have $k=1$, 2, 3 or 4 parameters.
As described in Section~\ref{section:models.for.variation}, some of
the $\beta_{1},
\dots,\beta_{k}$ parameters will be random from unit-to-unit.  One or
more of the $\beta_{1},
\dots,\beta_{k}$ parameters could, however, be modeled as
common across all units.

The scales of $y$ and $t$ can be chosen, as suggested by physical
theory and the data, to simplify the form of
$\degpath(t,\beta_{1},\dots,\beta_{k})$.  For example, the
relationship between the logarithm of degradation and the logarithm of
time might be modeled by the additive relationship in
(\ref{equation:general.deg.path.model}).  Degradation model choice
requires not only specification of the form of the
$\degpath(t,\beta_{1},\dots,\beta_{k})$ function, but also
specification of which of the $\beta_{1}, \dots, \beta_{k}$ are random
(differing from unit to unit) and which are fixed (common to all units). 
Because of the flexibility in specifying
the form of $\degpath(t,\beta_{1},\dots,\beta_{k})$, and of the way in
which the $\beta_{1}, \dots, \beta_{k}$ come into this form, we can,
for simplicity, model the unit-to-unit variability in
$\beta_{1},\dots,\beta_{k}$ with a multivariate
normal distribution with mean vector $\muvec_{\betavec}$ and covariance
matrix $\vcvmat_{\betavec}$.

It is generally assumed that the random $\beta_{1}, \dots, \beta_{k}$ 
are independent of the $\epsilon_{ij}$ deviations. Another common 
assumption is that $\sigma_{\epsilon}$ is constant. The adequacy of this
assumption can be affected by transforming $\degpath(t)$. Because
the $y_{ij}$ are taken serially on a unit, however, there is potential for
autocorrelation among the $\epsilon_{ij}, j=1, \dots, m_{i}$
values, especially if there are many closely-spaced readings.
In many practical applications involving inference on the degradation
of units from a population or process, however, the correlation is
weak and, moreover, dominated by the unit-to-unit variability in the
$\beta_{1}, \dots, \beta_{k}$ values and thus can be ignored. In situations
where autocorrelation cannot be ignored, one can use a time series model
for the residual term along with appropriate estimation methods.

%----------------------------------------------------------------------
\subsection{Degradation model parameters}
Although the values of $\beta_{1},\dots,\beta_{k}$ for the
individual units may be of interest in some applications (e.g., to
predict the future degradation of a particular unit, based on a few
early readings), subsequent development in this chapter will
concentrate on the use of degradation data to make inferences about
the population or process or predictions about future units.
In this case, the underlying
model parameters are $\muvec_{\betavec}$ and $\vcvmat_{\betavec}$, as well
as the residual standard deviation $\sigma_{\epsilon}$. For
shorthand, we will use
$\thetavec_{\betavec}=(\muvec_{\betavec},\vcvmat_{\betavec})$ to denote the
overall population/process parameters.

%----------------------------------------------------------------------
%----------------------------------------------------------------------
\section{Estimation of Degradation Model Parameters}
\label{section:est.degrad.param}

The likelihood for the random-parameter degradation model in
Section~\ref{section:general.degradation.model} can be expressed as
\begin{equation}
\label{equation:degradation.likelihood}
L(\thetavec_{\betavec},\sigma_{\epsilon} | \DATA)=
\prod_{i=1}^{n} \int_{ - \infty}^{\infty} \dots \int_{ -
\infty}^{\infty}
\left[ \prod_{j=1}^{m_{i}} \frac{1}{\sigma_{\epsilon}}
\phi_{\nor}(\stdnor_{ij}) \right] f_{\betavec}(\beta_{1i}, \dots ,
\beta_{ki};\thetavec_{\betavec}) 
	d\beta_{1i}, \dots , d\beta_{ki}
\end{equation}
where $\stdnor_{ij}=[y_{ij}
-\degpath(t_{ij},\beta_{1i},\dots,\beta_{ki})]/\sigma_{\epsilon}$ and
$f_{\betavec}(\beta_{1i}, \dots ,
\beta_{ki};\thetavec_{\betavec}) $ is the
multivariate normal distribution density function.  Each evaluation of
(\ref{equation:degradation.likelihood}) will, in general, require
numerical approximation of $n$ integrals of dimension $k$ (where $n$ is
the number of sample paths and $k$ is the number of 
random parameters in each
path).  Maximizing (\ref{equation:degradation.likelihood}) with
respect to $(\muvec_{\betavec},\vcvmat_{\betavec},\sigma_{\epsilon})$
directly, even with today's computational capabilities, is extremely
difficult unless $\degpath(t)$ is a linear function.  Pinheiro and Bates
(1995a) describe and compare estimation schemes that provide
approximate ML estimates of
$\thetavec_{\betavec}=(\muvec_{\betavec},\vcvmat_{\betavec})$ and
$\sigma_{\epsilon}$, as well as the unit-specific components in
$\beta_{1i},\dots,\beta_{ki}, i=1, \dots, n$.  Pinheiro and Bates~(1995b)
implement a modification of the method of Lindstrom and Bates~(1990).
The examples in this chapter were computed with the Pinheiro and Bates
(1995b) program implemented in \splus as function {\tt nlme}. 

\begin{example}{\bfseries Estimates of fatigue data model parameters 
for Alloy-A.}
\label{example:bk.fatigue.data.basic.estimate}
Continuing with Example~\ref{example:bk.fatigue.data.points}, we fit
the model in (\ref{equation:general.deg.path.model}) with
$\degpath_{ij}=a(t)$ in
(\ref{equation:paris.diffyq.solution}), $a(0)=.9$,
Stress=1, $\beta_{1}=m$ and $\beta_{2}=C$, modeling
$(\beta_{1},\beta_{2})$ with a bivariate normal distribution.  The
program of Pinheiro and Bates~(1995b) gives the following
approximate ML estimates.
\begin{displaymath}  
\muvechat_{\betavec} =
\left( \begin{array}{c}  5.17 \\3.73 
       \end{array}
\right), \qquad
\vcvmathat_{\betavec} =
\left( \begin{array}{rr}
       .251  & -.194 \\
      -.194 &   .519
      \end{array}
\right).
\end{displaymath}
and $\sigmahat_{\epsilon}=.0034.$
Figure~\ref{figure:bk.fatigue.data.fit.ps} shows the fitted Paris relationship
for each of the sample paths (indicated by the points on the plot) 
for the Alloy-A fatigue-crack data.
Figure~\ref{figure:bk.fatigue.data.bvn.ps} is a scatter plot of the
estimates of the Paris relationship parameters for each of the 21 sample
paths,
indicating the reasonableness of the bivariate normal distribution
model for this random-coefficients model.
%-------------------------------------------------------------------
\begin{figure}
\splusbookfigure{\figurehome/bk.fatigue.data.fit.ps}
\caption{Alloy-A fatigue crack size observations and fitted Paris-rule model.}
\label{figure:bk.fatigue.data.fit.ps}
\end{figure}
%-------------------------------------------------------------------
%-------------------------------------------------------------------
\begin{figure}
\splusbookfigure{\figurehome/bk.fatigue.data.bvn.ps}
\caption{Plot of $\betahat_{1i}$ 
versus $\betahat_{2i}$ for the $i=1,\dots, 21$ sample paths from the
Alloy-A fatigue crack size data. The contour lines represent the
fitted bivariate normal distribution for $\beta_{1}$ and $\beta_{2}$.}
\label{figure:bk.fatigue.data.bvn.ps}
\end{figure}
%-------------------------------------------------------------------
\end{example}

%----------------------------------------------------------------------
\section{Models Relating Degradation and Failure}
\label{section:degrad.and.failure}

%----------------------------------------------------------------------
\subsection{Soft failures: specified degradation level}
For some products there is a gradual loss of performance (e.g.,
decreasing light output from a fluorescent light bulb). Then failure would
be defined (in a somewhat arbitrary, but purposeful, manner)
at a specified level of degradation (e.g., 60\% of initial
output).  We call this a ``soft failure'' definition.

A fixed value of $\critdeg$ will be used to denote the critical
level for the degradation path above (or below) which failure is
assumed to have occurred.  The failure time $T$ is defined as the
time when the actual path $\degpath(t)$ crosses the critical
degradation level $\critdeg$.  We use $\censortime$ to denote the
planned stopping time in the experiment (as illustrated in
Figure~\ref{figure:bk.fatigue.data.points.ps}).  Inferences are
desired on the failure-time distribution of a particular product or
material. For soft failures it may be possible to continue observation
beyond $\critdeg$.

%--------------------------------------------------------------------------
\subsection{Hard failures: joint distribution of 
degradation and failure level.} For some products, the definition of the
failure event is clear---the product stops working (e.g., when the
resistance of a resistor deviates too much from its nominal value,
causing the oscillator in an electronic circuit to stop oscillating or
when an incandescent light bulb burns out).  These are called ``hard
failures.''
With hard failures, failure times will not, in general, correspond
exactly with a particular level of degradation (like the horizontal
line shown in Figures~\ref{figure:degradation.comparison.ps} through
\ref{figure:laser.degradation.strate.defect.ps}). Instead, the
level of degradation at which failure (i.e., loss of functionality)
occurs will be random from unit to unit and even over time.  This
could be modeled by using a distribution to describe unit-to-unit
variability in $\critdeg$ or, more generally, the joint distribution of
$\vecofbetas$ and the stochastic behavior in $\critdeg$.

%----------------------------------------------------------------------
%----------------------------------------------------------------------
\section{Evaluation of $\boldsymbol{F(t)}$}
\label{section:eval.of.ft}
%----------------------------------------------------------------------

A specified model for $\degpath(t)$ and $\critdeg$
defines a failure-time distribution. In
general, this distribution can be written as a function of the
degradation model parameters.  Suppose that a unit fails at time $t$
if degradation level first reaches $\critdeg$ at time $t$. Then
\begin{equation}
\label{equation:mixed.effect.to.ttf}
\Pr(T \leq t) = F(t) =
	F(t; \thetavec_{\betavec})= 
	\Pr[\degpath(t,\beta_{1},\dots,\beta_{k}) \geq  \critdeg ].  
\end{equation}
For a fixed $\critdeg$, the distribution of $T$ depends on the
distribution of the $\beta_{1},\dots,\beta_{k}$ which, in turn,
depends on the basic path parameters in $\thetavec_{\betavec}$. In
some simple cases it is possible to write down a closed form
expression for $F(t)$.  In general, however, such a closed form
expression will not exist.  For most practical path models,
especially when $\degpath(t)$ is nonlinear and more than one of the
$\beta_{1},\dots,\beta_{k}$ is random, it is necessary to evaluate
$F(t)$ with numerical methods.

%----------------------------------------------------------------------
\subsection{Analytical solution for $\boldsymbol{F(t)}$}
\label{section:evaluation.of.f.of.t}

For some particularly simple path
models, $F(t)$ can be expressed as a function of the basic
path parameters in a closed form.
Consider the following example.

\begin{example}{\bfseries Linear degradation with lognormal rate.}
\label{example:closed_form_lognormal}
Suppose the actual degradation path of a particular unit is given by
\begin{displaymath} 
\degpath(t) = \beta_{1} + \beta_{2} t
\end{displaymath} 
where
$\beta_{1}$ is fixed and $\beta_{2}$ varies from unit to unit according to a
$\LOGNOR(\mu, \sigma)$ distribution; that is,
\begin{displaymath} 
\Pr(\beta_{2} \leq b)  = 
\Phi_{\nor} \left[ \frac{\log(b)-\mu}{\sigma}\right].
\end{displaymath} 
The parameter $\beta_{1}$ represents
the common initial amount of degradation of all the test units at
time 0 and $\beta_{2}$ represents
the degradation rate, random from unit-to-unit.
Then
\begin{eqnarray*}
F(t;\beta_{1}, \mu, \sigma) &=& \Pr\left[\degpath(t) > \critdeg \right] = 
	\Pr(\beta_{1} + \beta_{2} t> \critdeg) = \Pr \left (\beta_{2} >
\frac{\critdeg - \beta_{1}}{t} \right ) \\[1ex]
	 &=& 1-  
\Phi_{\nor}\left(\frac{\log(\critdeg - \beta_{1}) - \log(t) - \mu}{\sigma}\right)  = 
\Phi_{\nor}\left(\frac{\log(t) - \left[ \log(\critdeg - \beta_{1} ) - \mu \right]}
{\sigma}\right), \quad t>0. 
\end{eqnarray*}
This shows that $T$ has a lognormal distribution with parameters
that depend on the basic path parameters
$\thetavec_{\betavec}=(\beta_{1}, \mu, \sigma)$, and $\critdeg$. That
is, $\exp[ \log(\critdeg - \beta_{1} ) - \mu ]$ is the
lognormal median and $\sigma$ is the lognormal shape parameter.
\end{example}

%----------------------------------------------------------------------
\subsection{Numerical evaluation of $\boldsymbol{F(t)}$}

\begin{algorithm}{\bfseries Evaluation
of $\boldsymbol{F(t)}$ by direct integration.}
\label{algorithm:evaluation.of.f.of.t.by.int} 
If $(\beta_{1}, \beta_{2})$ have a bivariate normal distribution
with parameters $\mu_{\beta_{1}},\mu_{\beta_{2}},
\sigma_{\beta_{1}}^{2}, \sigma_{\beta_{2}}^{2},
\rho_{\beta_{1},\beta_{2}}$,
then
\begin{eqnarray*}
F(t) = P(\rv \le \realrv) &=&
\int_{-\infty}^{\infty}
\Phi_{\nor}
\left [
-\, 
\frac{g(\critdeg, \realrv, \beta_{1})-\mu_{\beta_{2} | 
\beta_{1}}}
     {\sigma_{\beta_{2} | \beta_{1}}}
\right ]
\frac{1}{\sigma_{\beta_{1}}}
\phi_{\nor}  
\left (
\frac{\beta_{1} -\mu_{\beta_{1}}}
     {\sigma_{\beta_{1}}}
\right ) d\beta_{1}
\end{eqnarray*}
where $g(\critdeg, \realrv, \beta_{1})$ 
is the value of $\beta_{2}$ that gives $\degpath(\realrv)=\critdeg$
for specified $\beta_{1}$ and where
\begin{eqnarray*}
\mu_{\beta_{2} | \beta_{1}}&=&\mu_{\beta_{2}}+
\rho \sigma_{\beta_{2}}
\left (\frac{\beta_{1}-\mu_{\beta_{1}}}{\sigma_{\beta_{1}}}
\right)
\\
\sigma^{2}_{\beta_{2} | \beta_{1}}&=&\sigma_{\beta_{2}}^{2} (1-\rho^{2}).
\end{eqnarray*}
In principle, this approach can be extended in a straightforward
manner when there are more than 2 continuous random variables. The
amount of computational time needed to evaluate the multidimensional
integral will, however, increase exponentially with the dimension of
the integral.
\end{algorithm}

%----------------------------------------------------------------------
\subsection{Monte Carlo evaluation of $\boldsymbol{F(t)}$}
Monte Carlo simulation, as illustrated in
Figures~\ref{figure:paris.degradation.fixedm.ps},
\ref{figure:paris.degradation.stochm.ps}, 
and \ref{figure:paris.degradation.stochm.stosig.ps} is a particularly
versatile method for evaluating $F(t)$. Evaluation is done in the
following algorithm by generating a large number of random sample
paths from the assumed path model, using the proportion 
of path crossing
$\critdeg$ by time $t$ as an evaluation of $F(t)$.

\begin{algorithm}{\bfseries Monte Carlo evaluation
of $\boldsymbol{F(t)}$ from degradation model parameters.}
\label{algorithm:mc.evaluation.of.f.of.t}
\begin{enumerate}
\item
Generate $N$ simulated realizations
$\betahook_{1},\dots,\betahook_{k} $ of $\beta_{1},\dots,\beta_{k}$
from a multivariate normal distribution with mean
$\thetavechat_{\betavec}$
and variance-covariance matrix $\vcvmathat_{\betavec}$,
where $N$ is a large number (e.g., $N$=100,000).
\item
Compute the $N$ simulated failure times corresponding to the $N$
realizations of $\betahook_{1},\dots,\betahook_{k}$. Conceptually this
can be done by substituting the realizations of
$\betahook_{1},\dots,\betahook_{k}$ into
$\degpath(t,\beta_{1},\dots,\beta_{k})$, finding the crossing time for
each (often the crossing time can be expressed conveniently as a
function of the $\beta_{1},\dots,\beta_{k}$; otherwise the crossing
time can be found by using a numerical root-finding algorithm).
\item
For any desired values of $t$, use
\begin{displaymath} 
F(t) \approx \frac{\text{Number of Simulated First Crossing Times }  
\leq t}{N} 
\end{displaymath}
as an evaluation of $F(t)$.
\end{enumerate}
\end{algorithm}
The potential error in this Monte Carlo approximation is easy to
evaluate by using the binomial distribution. The error can be made
arbitrarily small by choosing $N$ large enough. In particular,
the standard deviation of the Monte Carlo error in $F(t)$ at a given
point
is $\sqrt{F(t)(1-F(t))/N}$. For example, if $F(t)=.01$ 
and $N$=100,000, the standard deviation of the Monte Carlo error is .0003.

%----------------------------------------------------------------------
%----------------------------------------------------------------------
\section{Estimation of $\boldsymbol{F(t)}$} 
\label{section:est.ft}

One can estimate the
failure-time distribution $F(t)$ by substituting the estimates
$\thetavechat_{\betavec}$ into (\ref{equation:mixed.effect.to.ttf}) giving
$\Fhat(t) =
F(t; \thetavechat_{\betavec})$.
This is straightforward for the case when $F(t)$ can be expressed in a
closed form.  When there is no closed-form expression for $F(t)$, and
when numerical transformation methods are too complicated, one can  
use Algorithm~\ref{algorithm:evaluation.of.f.of.t.by.int} 
or \ref{algorithm:mc.evaluation.of.f.of.t} to evaluate 
(\ref{equation:mixed.effect.to.ttf}) at $\thetavechat_{\betavec}$.

\begin{example}{\bfseries Degradation data estimate of $\boldsymbol{F(t)}$.}
\label{example:bk.degradation.fhat}
Figure~\ref{figure:bk.fatigue.fhat.and.ci.ps} shows $\Fhat(t)$ for
the Alloy-A data, estimated with
Algorithm~\ref{algorithm:evaluation.of.f.of.t.by.int}, using the
estimates
$\thetavechat_{\betavec}=(\muvechat_{\betavec},\vcvmathat_{\betavec})$
obtained in Example~\ref{example:bk.fatigue.data.basic.estimate}.
\begin{table}
\caption{Crossing times and plotting positions for the Alloy-A data.}
\centering\small 
\begin{tabular}{rrrrlc}
\multicolumn{1}{c}{} & \multicolumn{4}{c}{Crossing
 Time} & \multicolumn{1}{c}{Plotting Positions} \\
\multicolumn{1}{c}{Path} & \multicolumn{4}{c}{(Million
Cycles)} & \multicolumn{1}{c}{$(i-.5)/21$} \\
\hline
 1 &&&& $.088$ & .024 \\
 2 &&&& $.100$ & .071 \\
 3 &&&& $.101$ & .119 \\
 4 &&&& $.103$ & .167 \\
 5 &&&& $.103$ & .214 \\
 6 &&&& $.106$ & .262 \\
 7 &&&& $.106$ & .310 \\
 8 &&&& $.109$ & .357 \\
 9 &&&& $.113$ & .405 \\
10 &&&& $.115$ & .452 \\
11 &&&& $.118$ & .500 \\
12 &&&& $.118$ & .548 \\
13 &&&& $.129^{*}$  &       .595 \\
14 &&&& $.133^{*}$  &       .643 \\
15 &&&& $.138^{*}$ &        .690    \\
16 &&&& $.144^{*}$  &       .738    \\
17 &&&& $.146^{*}$  &       .786    \\
18 &&&& $.151^{*}$  &       .833    \\
19 &&&& $.160^{*}$  &       .881    \\
20 &&&& $.167^{*}$ &        .929    \\
21 &&&& $.170^{*}$  &       .976    \\
\hline
\end{tabular}\\
\begin{minipage}[t]{4in}
Observations marked with a ``*'' would have been censored
for a test that ended at .12 million cycles.
\end{minipage}
\label{table:alloya.cross.times}
\end{table}
The plotted points from Table~\ref{table:alloya.cross.times} (computed
as described in Chapter~\ref{chapter:probability.plotting}) provide a
nonparametric estimate based on the complete crossing time data (i.e.,
including the eventual crossing times beyond .12 million cycles, where
we assume that the test has ended).  This nonparametric estimate
provides a useful comparison with the parametric degradation and
failure time models based on the data available up to
$\censortime=.12$.  The confidence intervals in this figure will be
explained in the next section.
\end{example}
%-------------------------------------------------------------------
\begin{figure}
\splusbookfigure{\figurehome/bk.fatigue.fhat.and.ci.ps}
\caption{Degradation model estimate of $F(t)$ with pointwise
two-sided approximate 90\% and 80\% bootstrap bias-corrected
percentile confidence intervals, based on the Alloy-A crack-size data
censored at $\censortime=.12$.  The dots track the nonparametric
estimate of $F(t)$.}
\label{figure:bk.fatigue.fhat.and.ci.ps}
\end{figure}
%-------------------------------------------------------------------

%----------------------------------------------------------------------
%----------------------------------------------------------------------
\section{Bootstrap Confidence Intervals }
\label{section:degrad.ci}
Because there is no simple method of computing standard errors for
$\Fhat(t)$, we use the bias-corrected percentile bootstrap method in
this chapter.  This method is described briefly in
Section~\ref{section:percentile.bootstrap} and more fully in
Efron~(1985) and Efron and Tibshirani~(1993). The method is
implemented with the following algorithm.

\begin{algorithm}{\bfseries Bootstrap confidence intervals from
degradation data.}\\
\label{algorithm:degradation.bootstrap}
\begin{enumerate}
\item
Use the observed data from the $n$ sample paths to compute 
the estimates $\thetavechat_{\betavec}$ and $\sigmahat^{2}_{\epsilon}$.
\item
Use Algorithm~\ref{algorithm:evaluation.of.f.of.t.by.int} 
or \ref{algorithm:mc.evaluation.of.f.of.t} with
$\thetavechat_{\betavec}$ as input to compute 
the estimate $\Fhat(t)$
at desired values of $t$.
\item
Generate a large number $B$ (e.g., $B=4$000) of bootstrap samples
that mimic the original sample and compute the corresponding
bootstrap estimates $\Fhatboot(t)$ according to the following steps.
\begin{enumerate}
\item
Generate, from  $\thetavechat_{\betavec}$,
$n$ simulated realizations of the random path parameters 
$\beta_{1i}^{*},\dots,\beta_{ki}^{*}, i=1,\ldots,n$.
\item
Using the same sampling scheme as in the original experiment,
compute $n$ simulated observed paths from
\begin{displaymath}
y_{ij}^{*}=\degpath(t_{ij};\beta_{1i}^{*},\dots,\beta_{ki}^{*} ) +
\epsilon_{ij}^{*}
\end{displaymath}
up to the planned stopping time $\censortime$, where the
$\epsilon^{*}_{ij}$ values are independent simulated residual values
generated from $\NOR(0,\sigmahat_{\epsilon})$.
\item
Use the $n$ simulated paths to estimate parameters of the path
model, giving the bootstrap estimates $\thetavechat^{*}_{\betavec}$.
\item
Use Algorithm~\ref{algorithm:evaluation.of.f.of.t.by.int} 
or \ref{algorithm:mc.evaluation.of.f.of.t} with
$\thetavechat^{*}_{\betavec}$ as input to compute 
the bootstrap estimates $\Fhat^{*}(t)$
at desired values of $t$.
\end{enumerate}
\item
For each desired value of $t$, the bootstrap confidence interval
for $F(t)$ is computed using the following steps
\begin{enumerate}
\item
Sort the $B$ bootstrap estimates
$\Fhat^{*}(t)_{1}, \dots, \Fhat^{*}(t)_{B}$
in increasing order giving $\Fhat^{*}(t)_{(b)},b=1, \dots, B$.
\item
Following Efron and Tibshirani~(1993), 
the lower and upper bounds of pointwise
approximate $100(1-\alpha)\%$ confidence intervals for the distribution
function $F(t)$ are
\begin{displaymath}
\left[ \wideundertildex{F(t)}, \quad
\wideovertildex{F(t)} \right] = 
%\tilde{F(t)} \right] = 
\left[\Fhatboot(t)_{(l)},  \quad
\Fhatboot(t)_{(u)} \right]
\end{displaymath}
where
\begin{displaymath}
l=B \times \Phi_{\nor}\left[2\Phi_{\nor}^{-1}(q)
+\Phi_{\nor}^{-1}(\alpha/2)\right], \;\; u=B \times
\Phi_{\nor}\left[2\Phi_{\nor}^{-1}(q)+\Phi_{\nor}^{-1}(1-\alpha/2)\right]
\end{displaymath}
and $q$ is the proportion of the $B$ values of  $\Fhat^{*}(t)$
that are less than $\Fhat(t)$ (using $q=.5$ is equivalent to the
percentile bootstrap method).
\end{enumerate}
\end{enumerate}
\end{algorithm}

%-------------------------------------------------------------------
\begin{figure}
\splusbookfigure{\figurehome/bk.bootsamples.ps}
\caption{Bootstrap estimates of $F(t)$ for Alloy-A.}
\label{figure:bk.bootsamples.ps}
\end{figure}
%-------------------------------------------------------------------

\begin{example}{\bf Degradation data 
bootstrap confidence intervals for the Alloy-A $\boldsymbol{F(t)}$.}
\label{example:bk.degradation.ci}
Continuing with Example~\ref{example:bk.degradation.fhat},
Figure~\ref{figure:bk.fatigue.fhat.and.ci.ps} shows pointwise
two-sided approximate 90\% and 80\% bootstrap bias-corrected
percentile confidence intervals for $F(t)$, based on the crack-size
data censored at $\censortime=.12$.
Figure~\ref{figure:bk.bootsamples.ps} shows a subset of the bootstrap
estimates that were used to compute the bias-corrected percentile
confidence intervals for $F(t)$.
\end{example}

%----------------------------------------------------------------------
%----------------------------------------------------------------------
\section{Comparison with Traditional Failure Time Analyses}
\label{section:compare.ft.da}
This section compares the degradation and failure-time data analyses
for the Alloy-A data. Based on the degradation data and the
failure-time data censored at $\censortime=.12$,
Figure~\ref{figure:bk.fatigue.lognor.cen.ps} shows a lognormal
probability plot of the nonparametric estimate of $F(t)$ (the dots)
and the lognormal distribution ML estimate of $F(t)$ based on the
failure-time data available at time $\censortime=.12$ million cycles
(the line).
%-------------------------------------------------------------------
\begin{figure}
\splusbookfigure{\figurehome/bk.fatigue.lognor.cen.ps}
\caption{Lognormal probability plot of the nonparametric estimate
(dots) and the lognormal distribution ML fit (line) based on the
failure-time Alloy-A data censored at $\censortime=.12$.}
\label{figure:bk.fatigue.lognor.cen.ps}
\end{figure}  
%-------------------------------------------------------------------
Figure~\ref{figure:bk.fatigue.lognor.all.ps} shows the nonparametric
estimate of $F(t)$, the lognormal distribution ML
estimate of $F(t)$, and the corresponding approximate 90\% pointwise
confidence intervals.
%-------------------------------------------------------------------
\begin{figure}
\splusbookfigure{\figurehome/bk.fatigue.lognor.all.ps}
\caption{Lognormal
distribution ML estimate and pointwise approximate 90\% confidence
intervals.  The dots indicate the nonparametric estimate based on the
Alloy-A {\em uncensored} failure-time data (i.e., based on failure
times beyond the artificial censoring time used in the parametric
degradation and failure-time analyses).}
\label{figure:bk.fatigue.lognor.all.ps}
\end{figure}
%-------------------------------------------------------------------
Figure~\ref{figure:bk.fatigue.dist.comp.ps} compares the degradation
data/model estimates with the ML estimates of the lognormal, normal,
and Weibull failure-time distributions, based on the censored
failure-time data.  Figures~\ref{figure:bk.fatigue.lognor.all.ps}
and \ref{figure:bk.fatigue.dist.comp.ps} also show the nonparametric
estimate of the failure-time distribution using the actual crossing
times that occurred after $\censortime=.12$.
%-------------------------------------------------------------------
\begin{figure}
\splusbookfigure{\figurehome/bk.fatigue.dist.comp.ps}
\caption{Comparison of degradation and failure-time
data analyses based on the Alloy-A data censored at $\censortime$.
The dots indicate the nonparametric estimate based on the Alloy-A
{\em uncensored} failure-time data (i.e., based on failure times
beyond the artificial censoring time used in the parametric
degradation and failure-time analyses).}
\label{figure:bk.fatigue.dist.comp.ps}
\end{figure}
%-------------------------------------------------------------------
Some observations from these figures are:
\begin{enumerate}
\item
Figures~\ref{figure:bk.fatigue.lognor.cen.ps} and
\ref{figure:bk.fatigue.lognor.all.ps} show that the lognormal
distribution provides a good fit to the failure-time data up to
$\censortime=.12$, but not beyond.
\item
Figure~\ref{figure:bk.fatigue.dist.comp.ps} shows that the other
commonly used parametric models, which fit almost as well before
$\censortime=.12$, do not do any better beyond $\censortime=.12$.
The degradation analysis, however, does provide a reasonable
extrapolation beyond $\censortime=.12$.  This is because the
degradation analysis method directly models the relationship between
degradation and time and takes account of the amount of degradation
in the censored observations when estimating $F(t)$.  See the
distribution of crack lengths for the Alloy-A units that had not
failed before $\censortime=.12$, shown in
Figure~\ref{figure:bk.fatigue.data.points.ps}.  The traditional
failure-time data analysis ignores this important information.

\item
Comparing Figures~\ref{figure:bk.fatigue.fhat.and.ci.ps} and
\ref{figure:bk.fatigue.lognor.all.ps} shows that the confidence
intervals based on the degradation and failure-time data have
similar widths from $.10 < t < .12$.  Outside of this range, however,
the confidence intervals are narrower for the degradation method.
\end{enumerate}

%----------------------------------------------------------------------
\section{Approximate Degradation Analysis}
\label{section:app.deg.anal}
This section describes an alternative (but only approximately
correct) method of analyzing degradation data.  Consider the general
degradation model given in
Section~\ref{section:general.degradation.model}.  There are two steps in
the approximate method.  The first step consists of separate
analysis for each unit to predict the time at which the unit will
reach the critical degradation level corresponding to failure. These
times are called ``pseudo failure times''. In the second step, the
$n$ pseudo failure times are analyzed as a complete sample of
failure times to estimate $F(t)$.  Formally, the method is as
follows.
\begin{itemize}
\item
For the unit $i$, use the path model $y_{ij} = \degpath_{ij} +
\epsilon_{ij}$ and the sample path data $(\realrv_{i1}, y_{i1})$,
\dots, $(\realrv_{im_{i}}, y_{im_{i}})$ to find the (conditional) ML
estimate of $\betavec_{i}=(\beta_{1i}, \dots , \beta_{ki})$, say
$\betavechat_{i}$. This can be done using nonlinear least squares.
\item
Solve the equation
$
\degpath(\realrv,\betavechat_{i})=\critdeg
$
for $\realrv$ and call the solution $\that_{i}$.
\item
Repeat the procedure for each sample path to obtain 
the pseudo failure times $\that_{1}, \dots, \that_{n}$.
\item
Do a single distribution analysis of the 
data $\that_{1}, \dots, \that_{n}$ to estimate $F(\realrv)$.
\end{itemize} 

%----------------------------------------------------------------------
\subsection{Simple linear path}
For simple linear degradation processes the degradation path for a unit can
be written as $\degpath(t) = \beta_{1} + \beta_{2} t$.
In some cases log
transformations on the sample degradation values or on the time scale or
both will result in a simple linear-path model. In this case the pseudo failure times are obtained from
\begin{displaymath}
\that_{i}=\frac{
\critdeg-\betahat_{1i}
	       }
 	       {
\betahat_{2i}
		}
\end{displaymath}
where 
\begin{displaymath}
\betahat_{1i}=\ybar_{i}-\betahat_{2i}\times \tbar_{i},
\quad \quad
\betahat_{2i}=
\frac{
\sum_{j=1}^{m_{i}} (t_{ij}-\tbar_{i}) \times y_{ij}
     }
     {
\sum_{j=1}^{m_{i}} (t_{ij}-\tbar_{i})^{2}
     }
\end{displaymath}
and $\tbar_{i}$ and $\ybar_{i}$ are the means of
$t_{i1}, \dots, t_{im_{i}}$ and 
$y_{i1}, \dots, y_{im_{i}}$, respectively.
%---------------------------------------------------------------------

\subsection{Simple linear path through the origin}
For some degradation, all paths start at the origin ($t_{i1}=0,
y_{i1}=0$). If, in addition, the degradation rate is constant, then
the degradation path has the form $\degpath(t) = \beta_{2}
t$. Then the pseudo failure times are
obtained from
\begin{displaymath}
\that_{i}=\frac{
\critdeg
	       }
 	       {
\betahat_{2i}
		}
\end{displaymath}
where 
\begin{displaymath}
\betahat_{2i}=
\frac{
\sum_{j=1}^{m_{i}} t_{ij} \times y_{ij}
     }
     {
\sum_{j=1}^{m_{i}} t_{ij}^{2}
     }.
\end{displaymath}
%----------------------------------------------------------------------


%----------------------------------------------------------------------
\subsection{Comments on the approximate degradation analysis}
For simple problems the approximate degradation analysis is
attractive because the computations are relatively simple.  The
approximate method is less appealing when the degradation paths are
nonlinear.

The approximate method  may give  adequate analysis if
\begin{itemize}
\item
The degradation paths are relatively simple.
\item
The fitted path model is approximately correct.
\item
There are enough  data for precise estimation
of the $\betavec_{i}$'s.
\item
The amount of measurement error is small.
\item
There is not too much extrapolation in predicting
the $\that_{i}$ ``failure times.''
\end{itemize} 
There are, however, 
potential problems with the  approximate degradation analysis
because
\begin{itemize}
\item
The method ignores the prediction error in $\that$ and does not
account for measurement error in the observed sample paths.
\item
The distributions fitted to the pseudo failure times will not, in general,
correspond to the distribution induced by the degradation model.
\item
For some applications, there may be sample paths that do not contain
enough information to estimate all of the path parameters (e.g.,
when the path model has an asymptote but the sample path has not
begun to level off). This might necessitate fitting different models
for different sample paths in order to predict the crossing time.
\end{itemize}
Overall, extrapolation into the tails of
the failure-time distribution may be more valid
with the actual crossing distribution implied by the degradation
model (as used in
Sections~\ref{section:degrad.and.failure} through
\ref{section:est.ft}) than with the empirically 
predicted failure times.

\begin{example}
{\bf Laser life analysis.}
\label{example:simple.deg.laser}
The data in
Figure~\ref{figure:GaAs.laser.ps} and Appendix
Table~\ref{atable:GaAs.laser.degradation.data} are from a laser life
test similar to the one described in
Example~\ref{example:laser.defect.deg}, except that there were no
early failures.  For this device and the corresponding application,
a $\critdeg=10\%$ increase in current was the specified failure
level.
%-------------------------------------------------------------------
\begin{figure}
\splusbookfigure{\figurehome/GaAs.laser.ps}
\caption{Plot of laser operating current as a function of time.}
\label{figure:GaAs.laser.ps}
\end{figure}
The failure times (for paths exceeding $\critdeg=10\%$ increase in
current before 4000 hours) and the pseudo failure times were
obtained by fitting straight lines through the data for each
path. These pseudo failure times are 3702, 4194, 5847, 6172, 5301,
3592, 6051, 6538, 5110, 3306, 5326, 4995, 4721, 5689, and 6102
hours.  Using methods from Chapter~\ref{chapter:parametric.ml.ls},
Figure~\ref{figure:GaAs.laser.lognor.mleprobplot.ps} is a Weibull
probability plot of the laser
pseudo failure times showing the ML estimate for $F(t)$ and
approximate 95\% pointwise confidence intervals.
%-------------------------------------------------------------------
\begin{figure}
\splusbookfigure{\figurehome/GaAs.laser.lognor.mleprobplot.ps}
\caption{Weibull probability plot of the laser pseudo 
failure times showing the ML estimate of $F(t)$
and approximate 95\% pointwise confidence intervals.}
\label{figure:GaAs.laser.lognor.mleprobplot.ps}
\end{figure}
\end{example}


%----------------------------------------------------------------------


%----------------------------------------------------------------------
%----------------------------------------------------------------------
\section*{Bibliographic Notes}
Gertsbakh and Kordonsky~(1969) discuss the degradation approach from
an engineering point of view.  They point out the value of analyzing
degradation measures in terms of sample paths to assess product
reliability.  They present the Bernstein distribution which
describes the failure-time distribution for a simple linear model
with random intercept and random slope.  Knezevic~(1993) presents
similar probability models based on what he calls a ``condition
parameter.''  Nelson~(1990a, Chapter 11) discusses a special
situation in which the degradation measurement is destructive (only
one measurement could be made on each item). Tomsky~(1982) uses a
multivariate normal regression model to evaluate component
degradation.  Linear degradation models were used in Suzuki, Maki,
and Yokogawa~(1993) to model the increase in a resistance
measurement over time and in Tseng, Hamada, and Chiao~(1995) to
model lumens output from fluorescent light bulbs over time.
Yanagisawa (1997) fits models to degradation data for silicon solar
cells where the accelerating factors were light intensity and
temperature.

Murray~(1993, 1994) and Murray and Maekawa~(1996) describe
degradation data for disk storage media (e.g. compact disc) error
rates in accelerated testing. These papers, Tseng et al.~(1995) and
Tobias and Trindade~(1995) use the approximate analysis method
described in Section~\ref{section:app.deg.anal} to analyze their
degradation data. Tseng and Yu~(1997) and Yu and Tseng~(1998)
propose methods for choosing the time to terminate a degradation
test.

Much of the material in this chapter is based on methods presented
in Lu and Meeker~(1993), Lu, Meeker and Escobar~(1996), and ideas in
Meeker and Hamada~(1995).  Nelson~(1995b) describes models and
analysis methods for problems with random nonzero degradation
initiation times.  His methods assume destructive inspection so that
each sample unit will provide a single (possibly censored)
degradation response. It would be useful to extend this work to
allow for multiple readings on individual test units.
Crowder~(1997) describes methods for developing component-based
preventive maintenance plans.  One of these methods used
degradation-type data. Davidian and Giltinan~(1995) provide an
excellent description and development of methods for fitting
statistical models to nonlinear degradation-type models (also known
as ``growth curves'') and estimating random parameters.

%----------------------------------------------------------------------
\section*{Exercises}

%---------------------------------------
\begin{exercise}
Show that (\ref{equation:paris.diffyq.solution}) 
is the solution to (\ref{equation:paris.diffyq}).
\end{exercise}

%---------------------------------------
\begin{exercise}
Use the degradation equation (\ref{equation:paris.diffyq.solution}) 
to obtain an expression for the time that $\degpath(t)$
crosses a specified $\critdeg$.
\end{exercise}

%---------------------------------------
\begin{exercise}
Show that (\ref{equation:luvalle.meeker.diffyq.sol}) 
is the solution to (\ref{equation:luvalle.meeker.diffyq}).
\end{exercise}

%---------------------------------------
\begin{exercise}
Use  the degradation equation (\ref{equation:luvalle.meeker.diffyq.sol2}) 
to obtain an expression for the time that $\A_{2}(t)$
crosses a specified $\A_{2f}$.
\end{exercise}

%---------------------------------------
\begin{exercise}
Determine the value of $N$ needed in
Algorithm~\ref{algorithm:mc.evaluation.of.f.of.t} to evaluate $F(t)$ at
the point where $F(t)=.01$ so that the probability that Monte Carlo
error in evaluation is less than .0001, with probability approximately
.95.  Use the normal distribution approximation to the binomial
probability.
\end{exercise}

%---------------------------------------
\begin{exercise}
Discuss the advantages and disadvantages of
using Algorithm~\ref{algorithm:evaluation.of.f.of.t.by.int} 
versus \ref{algorithm:mc.evaluation.of.f.of.t} when estimating $F(t)$
from degradation data.
\end{exercise}
%---------------------------------------

%---------------------------------------
\begin{exercise}
Consider the analysis of the laser degradation data
in Example~\ref{example:simple.deg.laser}.
\begin{enumerate}
\item
\label{exerpart:5per}
Repeat the analysis using the time at
which current has increased by 5\% to define failure.
\item
\label{exerpart:15per}
Repeat the analysis using the time at
which current has increased by 15\% to define failure.
\item
Compare the results in parts 
\ref{exerpart:5per} and \ref{exerpart:15per}. Comment on the
differences in assumptions needed to estimate these two different
distributions.
\end{enumerate}
\end{exercise}
%---------------------------------------

%---------------------------------------
\begin{exercise1}
Example~\ref{example:simple.deg.laser} illustrates the simple
analysis of the data in Appendix
Table~\ref{atable:GaAs.laser.degradation.data}. Do the analysis of
these data using the random-coefficient degradation model.
\begin{enumerate}
\item
Identify and estimate a parametric distribution for the slopes.
\item
Derive an expression for the failure-time distribution based on the
degradation model, where failure is defined as the time at
which current has increased by 10\%.
\item
Plot the estimate of the failure time distribution. Compare it with
the simple estimate obtained in
Example~\ref{example:simple.deg.laser}.
\end{enumerate}
\end{exercise1}
%---------------------------------------

%---------------------------------------
\begin{exercise}
\label{exercise:block.error.rate.degradation.data}
Appendix Table~\ref{atable:block.error.rate.degradation.data} gives
degradation data on block error rates (the ratio of number of bytes
with errors to the total number of bytes tested) of magneto-optic data
storage disks tested for 2000 hours at 80$\degreesc$ and 85\%
relative humidity. Use the simple analysis method described in
Section~\ref{section:app.deg.anal} to estimate the failure-time
distribution of these disks at 80$\degreesc$ and 85\% relative
humidity where failure is defined as the time that it takes to reach
an error rate of $50 \times 10^{-5}$ (a safe level at which error
correcting codes can be expected to correct errors).
\end{exercise}
%---------------------------------------

\begin{exercise}
\label{exercise:asym.deg}
As an electronic device ages, its power output decreases.  Because the
degradation results from a simple one-step chemical reaction with a
limited amount of harmful material available for reaction, the
decrease in power can be described by the degradation model
$\degpath(t) = \beta_{2}[1 -  \exp(-\beta_{1} t )]$
where $\degpath(t)$ is the power output at time $t$, $\beta_{2} < 0
$ is nearly the same for all units, and $\beta_{1}$, comes from a
lognormal distribution.  System performance degrades noticeably when
$\degpath(t)$ falls below $\critdeg$. Thus we define a failure as
the point in time when $\degpath(t) < \critdeg$.
\begin{enumerate}
\item
Describe some possible physical reasons for the asymptotic behavior of
$\degpath(t)$.
\item
What happens, in the long run, if $\beta_{2} > \critdeg $?
\item
Assuming that $\critdeg$ is a fixed constant and that
$\beta_{2} < \critdeg $, derive an
expression for $F(t)$, the failure time cdf.
\end{enumerate}
\end{exercise}

%---------------------------------------
\begin{exercise}
Suppose the actual degradation path of a particular unit is given by
$\degpath(t) = \beta_{1} t$ where $-\log(\beta_{1})$ varies from
unit to unit according to a $\SEV(\mu, \sigma)$ distribution. If
failure occurs when $\degpath(t) > \critdeg$, where $\critdeg$ is a
fixed constant, show that the failure time distribution is Weibull.
\end{exercise}

%---------------------------------------
\begin{exercise1}
Suppose the actual degradation path of a particular unit is given by
$\degpath(t) = \beta_{1} t$ 
where $\beta_{1}$ varies from unit to unit according to a
$\LOGNOR(\mu_{1}, \sigma_{1})$ distribution. Also suppose that failure
occurs when $\degpath(t) > \critdeg$, and $\critdeg$ has a
$\LOGNOR(\mu_{2},
\sigma_{2})$ distribution. Derive an
expression for $F(t)$, the failure time cdf.
\end{exercise1}

%---------------------------------------
\begin{exercise1}
Suppose the actual degradation path of a particular unit is given by
$\degpath(t) = \beta_{1} + \beta_{2} t$ where $(\beta_{1},
\beta_{2})$ vary from unit to unit according to a bivariate normal
distribution with parameters $\mu_{\beta_{1}}$, $\mu_{\beta_{2}}$,
$\sigma_{\beta_{1}}$, $\sigma_{\beta_{2}}$, and
$\rho_{\beta_{1}\beta_{2}}$.
\begin{enumerate} 
\item
Assuming that failure occurs when
$\degpath(t) > \critdeg$, derive an expression for the failure time
distribution $F(t)=\Pr(T  \leq t) = \Pr[\degpath(t) > \critdeg]$.
\item
Explain why $\lim_{t \rightarrow \infty} F(t) < 1$.
\end{enumerate}
\end{exercise1}

