%chapter 3
%original by wqmeeker  12 Jan 94
%edited by wqmeeker  19 Feb 94
%edited by wqmeeker  28 Feb 94
%edited by wqmeeker   4 Mar 94
%edited by wqmeeker   20 Mar 94
%edited by driker     25 mar 94
%edited by wqmeeker   26 Mar 94
%edited by driker     30 Mar 94
%edited by wqmeeker   31 Mar 94
%edited by driker 13 july 94
%edited by wqmeeker 16 july 94 moved shock absorber data
%edited by wqmeeker  2 aug 94
%edited by wqmeeker  7 aug 94
%edited by wqmeeker  9 aug 94
%edited by wqmeeker 20 sept 94 shock absorber km example
%edited by wqmeeker 21 sept 94 added data analysis data sets exercises
%edited by wqmeeker 22 sept 94 cleaning up/removed hazard est
%edited by wqmeeker 03 oct 94 cleaning ci just and begin bootstrap
%edited by driker 21 oct 94
%edited by wqmeeker  21/22 oct 94
%edited by wqmeeker  29 oct 94 data versus DATA
%edited by wqmeeker  20 nov 94 
%edited by wqmeeker  25 nov 94  
%edited by wqmeeker  09 dec 94 first round of boot figures 
%edited by wqmeeker  17/18 dec 94 editing and remaining boot stuff
%edited by driker 22 dec 94
%edited by wqmeeker  23/29 dec 94  bootstrap
%edited by wqmeeker  30 dec 94 various improvements
%edited by driker 4 jan 95
%edited by driker 5 jan 95
%edited by driker 2 feb 95
%edited by wqmeeker  18 feb 95 stronger warnings on bootstrap
%edited by wqmeeker 17 june 95 moving the bootstrap
%edited by wqmeeker 18 june 95 smoothing and reacting to reviewers comments
%edited by driker 26 june 95
%edited by wqmeeker 13 oct 95 smoothing and censoring in interval
%edited by wqmeeker 4 nov 95 luis comments
%edited by driker 6 nov 95
%edited by driker 1 july 96
%edited by driker 20 nov 96
%edited by wqmeeker 8 mar 97 india changes
\setcounter{chapter}{2}

\chapter{Nonparametric Estimation}
\label{chapter:nonparametric.estimation}


%{\Large {\bf William Q. Meeker and Luis A. Escobar}}\\
%Iowa State University and Louisiana State University\\[5ex]
{\large {\bf \today}}\\[2ex]
Part of 
{\em Statistical Methods for Reliability Data}\\
Copyright 1997 W. Q. Meeker and L. A. Escobar. \\[1ex]
To be published by John Wiley \& Sons Inc. in 1998.



%----------------------------------------------------------------------
%----------------------------------------------------------------------
\section*{Objectives}
This chapter explains:
\begin{itemize} 
\item 
Simple statistical methods, based on the binomial distribution, to
estimate a cdf $F(\realrv)$ from interval and singly right censored data,
without having to assume an underlying parametric distribution. This
is called ``nonparametric'' estimation.
\item 
Standard errors of the nonparametric estimator and approximate
confidence intervals for $F(\realrv)$.
\item 
Life table methods to extend nonparametric estimation to allow for
combinations of interval-censored and multiply right-censored data.
\item
The Kaplan-Meier nonparametric estimator for data with observations
reported as exact failure times.
\item 
A generalized nonparametric estimator of $F(\realrv)$ for arbitrary
censoring (including combinations and mixtures of exact failure times
with left, right, and interval censoring).
\end{itemize}

%----------------------------------------------------------------------
%----------------------------------------------------------------------
\section*{Overview}
The nonparametric (model-free) estimates described in this chapter are
used throughout this book as a tool for reliability data analysis.
Section~\ref{section:binomial.np} starts with a simple method that
applies to problems with complete data or single censoring.
Section~\ref{section:basic.iease.of.infer} explains the basic ideas
of statistical inference and introduces the ideas behind the use of
confidence intervals, another statistical tool used throughout this
book.  Confidence intervals for complete data or single censoring
are described in Section~\ref{section:ci.simple.cen}. The
methods are generalized to the commonly encountered multiple censoring
in Sections~\ref{section:def.model.free.est},
\ref{section:np.stderror.and.ci}, and
\ref{section:def.of.nonparametric.estimate}. Simultaneous
confidence bands (used for helping to choose a model in
Chapter~\ref{chapter:probability.plotting}) are presented in
Section~\ref{section:np.simultaneous.cb}.
Sections~\ref{section:life.table} and
\ref{section:arbitrary.censoring}
provide other general methods for handling more complicated kinds of censoring.



%----------------------------------------------------------------------
%----------------------------------------------------------------------
\section{Introduction}
As explained in Section~\ref{section:strategy.for.data.analysis}, data
analysis should begin with analytical and graphical tools that do not
require strong model assumptions.  Such methods allow the data to
be interpreted without distortion that might be caused by using inadequate
model assumptions.  This chapter describes methods for computing
nonparametric estimates and confidence intervals for $F(\realrv)$.  In
some cases, such estimates are all that will be needed for an
analysis.  In other situations a nonparametric analysis provides an
intermediate step toward a more highly structured model allowing more
precise or more extensive inferences, provided that the additional
assumptions of such a model are valid.


%----------------------------------------------------------------------
%----------------------------------------------------------------------
\section{Estimation From Singly-Censored Interval Data}
\label{section:binomial.np}
This section shows how to compute a nonparametric estimate of a cdf
from interval-censored data when either all units fail or
all of the right censoring is at one point at the end of
the study (known as single right censoring).
%-------------------------------------------------------------------
\begin{example}
\label{example:heat.exchanger.p1}
{\bf Plant 1 heat exchanger data.}
%-------------------------------------------------------------------
%-------------------------------------------------------------------
\begin{figure}
\xfigbookfigure{\figurehome/heatex.plant1fig.ps}
\caption{Plant~1 heat exchanger data.}
\label{figure:heatex.plant1fig.ps}
\end{figure}
%-------------------------------------------------------------------
Figure~\ref{figure:heatex.plant1fig.ps} shows the Plant~1 data from
Example~\ref{example:heat.exchanger.data}. As a first step in our
analysis of the heat exchanger data we want to estimate
$F(\realrv)$ for just Plant~1.
\end{example}

Most studies involving inspection start at time zero with an initial sample of
units. Information is available on the status of the units
at the end of each time interval.  Let $n$ be the initial 
number of units (sample size) and let
$\deadin_{i}$ denote the number of units that {\em died} or failed 
in the $i$th interval $(\realrv_{i-1}, \realrv_{i}]$.
The nonparametric
estimator $\Fhat(\realrv_{i})$, based on the simple binomial distribution, is 
\begin{equation}
\label{equation:binomial.ecdf}
\Fhat(\realrv_{i})=\frac{
\# \,\, \mbox{of failures up to time } \,\, \realrv_{i}
                           }
			   {n}
=\frac{
       \sum_{j=1}^{i} \deadin_{j}
      } {n}.
\end{equation}

\begin{example}
\label{example:npe.heat.exchanger.p1}
{\bf Nonparametric estimator of $F(\realrv)$.}
The data from Plant~1 ($n=100, \deadin_{1}=1,
 \deadin_{2}=2,  \deadin_{3}=2 $) give:
\begin{displaymath}
\Fhat(1)=1/100, \quad \Fhat(2)=3/100, \quad \Fhat(3)=5/100.
\end{displaymath}
This estimate is shown graphically with the $\bullet$ symbols in
Figure~\ref{figure:heatexchplant1.cdf.pw.ps}.
%-------------------------------------------------------------------
\begin{figure}
\splusbookfigure{\figurehome/heatexchplant1.cdf.pw.ps}
\caption{Plot of the nonparametric estimate 
for the Plant~1 heat exchanger data 
with  pointwise approximate $95\%$ confidence
intervals based on binomial theory.}
\label{figure:heatexchplant1.cdf.pw.ps}
\end{figure}
%-------------------------------------------------------------------
\end{example}


In general, this nonparametric estimator $\Fhat(\realrv_{i})$ is
defined at all values of $\realrv_{i}$ (upper endpoints of all
intervals).  Additionally, if interval $i$ is known to have no
failures, then $\Fhat(\realrv)=\Fhat(\realrv_{i-1})$ for
$\realrv_{i-1} \leq \realrv \leq
\realrv_{i}$. If interval $i$ is known to contain one or more
failures, $\Fhat(t)$ increases from $\Fhat(\realrv_{i-1})$ to
$\Fhat(\realrv_{i})$ in the interval $(\realrv_{i-1},\realrv_{i}]$.
In this case $\Fhat(\realrv)$ is undefined for $\realrv_{i-1} <
\realrv < \realrv_{i}$.  Intuitively, this is because we do not know
the exact location of the failure(s) within the interval
$(\realrv_{i-1},\realrv_{i}]$ and thus we have no information on how
$\Fhat(\realrv)$ is increasing in the interval.  By using the binomial
distribution, it is easy to show that $\Fhat(\realrv_{i})$ is the
maximum likelihood estimator of $F(t)$.

%----------------------------------------------------------------------
%----------------------------------------------------------------------
 \section{Basic Ideas of Statistical Inference}
\label{section:basic.iease.of.infer}
 \subsection{The Sampling Distribution of $\Fhat(\realrv_{i})$}
\label{section:sampling.dist.of.fhat}
Estimates like $\Fhat(\realrv_{i}), i=1, \ldots , m$, computed from
a set of sample data can be interpreted in at least
two different ways.
\begin{itemize}
\item
Estimates can be viewed as
descriptive of the {\em particular data set} used to compute the estimate.
This is known as ``descriptive statistics.''
\item
More commonly, there is interest {\em beyond} the particular
sample units and estimates are used to make inferences about the process or
larger existing population of units from which the sample units were
chosen at random. This is an example of ``inferential statistics.''
\end{itemize}
In inferential applications, an estimate [say $\Fhat(t_{i})$ at a
particular time $\realrv_{i}$] will deviate from $F(\realrv_{i})$, the
actual population or process cdf at $\realrv_{i}$.  The standard
(non-Bayesian\footnote{Bayesian methods of statistical inference,
described in Chapter~\ref{chapter:singledist.bayes}, are based on the
specification of a prior distribution to describe prior knowledge or
opinion about the model parameters. As explained in
Chapter~\ref{chapter:singledist.bayes}, this alternative approach to
inference leads to inference statements with a somewhat different
interpretation.}) approach to quantifying the possible size of the
difference between $\Fhat(\realrv_{i})$ and $F(\realrv_{i})$ is to
consider what would happen if the inferential {\em procedure}
(sampling and estimation) were repeated a large number of times, each
time getting different data and thus a different $\Fhat(\realrv_{i})$.
The distribution of $\Fhat(\realrv_{i})$ values is called a sampling
distribution, and this distribution provides insight into the probable
deviation between $\Fhat(\realrv_{i})$ and $F(\realrv_{i})$.

%-------------------------------------------------------------------
\subsection{Confidence intervals} 
\label{section:confidence.intervals}
A point estimate, by itself, can be misleading, as it may or may not be
close to the quantity being estimated.
Confidence intervals are one of the most useful
ways of quantifying uncertainty due to ``sampling error'' arising from
limited sample sizes. Confidence intervals, however, generally do not quantify
possible errors and biases arising from an inadequate model or other invalid
model assumptions.

Confidence intervals have a specified ``level of
confidence,'' typically 90\% or 95\%, expressing one's confidence ({\em
not probability}) that a specific interval
contains the quantity of interest. A specific interval either contains
the quantity of interest or not; the truth is unknown.  It
is important to recognize that the confidence level pertains to a 
probability statement
about the performance of the confidence interval {\em procedure} rather
than a statement about any particular interval.  See Chapter 2 of Hahn
and Meeker~(1991) for further discussion of the interpretation of 
statistical intervals.

``Coverage probability'' is the probability that a confidence interval
{\em procedure} will result in an interval containing the quantity of
interest. When the specified level of confidence [generically
$100(1-\alpha)\%$] is not equal to the coverage probability, the procedure
and resulting intervals are said to be {\em approximate}.  In some
simple problems coverage probability for a given procedure can be
computed analytically and, correspondingly, {\em exact} confidence
interval methods can be developed. ``Conservative'' procedures have a
coverage probability
that is {\em at least} as large as the specified confidence level.
In most practical problems
involving censored data, there are no ``exact'' confidence interval
procedures. There are, however, a number of useful, and often simple,
approximate methods.  Better approximations generally require more
computations. The adequacy of the approximations can be checked with
repeated simulation. In turn, often simulation can be used to obtain
better approximations, as described in Chapter~\ref{chapter:bootstrap}.
Section~\ref{section:np.stderror.and.ci} describes simple methods
for computing approximate confidence intervals.  
Chapter~\ref{chapter:bootstrap}
presents a more accurate simulation-based approach.

%-------------------------------------------------------------------
%-------------------------------------------------------------------
\section{Confidence Intervals from Complete or Singly-Censored Data}
\label{section:ci.simple.cen}
%-------------------------------------------------------------------
\subsection{Pointwise 
binomial-based confidence interval for $F(\realrv_{i})$}
\label{section:binomial.confidence.intervals}

A conservative $100(1-\alpha)\%$ confidence interval $\left [
\undertilde{F}(\realrv_{i}), \quad
\tilde{F}(\realrv_{i})
\right ]$ for $F(\realrv_{i})$
based on binomial sampling is
\begin{eqnarray}
\label{equation:binomial.confidence.interval}
\undertilde{F}(\realrv_{i}) &=&
\left \{
1+
\frac{(n-n\Fhat+1) \Fquant_{(1-\alpha/2; 2n-2n\Fhat+2,
2 n \Fhat)}
     }
     {
n \Fhat
     }
\right \}^{-1}
\\ \nonumber
\tilde{F}(\realrv_{i})&=&
\left \{
1+
\frac
{
n-n\Fhat
}
{
(n\Fhat+1) \Fquant_{(1-\alpha/2; 2n\Fhat+2,
         2n-2n\Fhat)}
}
\right \}^{-1}
\end{eqnarray}
where $\Fhat=\Fhat(\realrv_{i})$ and $\Fquant_{(p; \nu_{1},
\nu_{2})}$ is the $p$ quantile of the F distribution
with $(\nu_{1}, \nu_{2})$ degrees of freedom.
Elementary statistics textbooks provide
Tables of F distribution quantiles. The confidence
interval in (\ref{equation:binomial.confidence.interval}) is
conservative in the sense that the coverage probability is greater
than or equal to $1-\alpha$.  Theory for this confidence interval is
given, for example, in Brownlee~(1960, pages 119-120).  

A one-sided approximate $100(1-\alpha)\%$ confidence bound for $F(t)$
can be obtained by replacing $\Fquant_{(1-\alpha/2)}$ with
$\Fquant_{(1-\alpha)}$ in
(\ref{equation:binomial.confidence.interval}) and using the
appropriate endpoint of the two-sided confidence interval.  For
example a conservative 90\% confidence interval from
(\ref{equation:binomial.confidence.interval}) can also be viewed as
two conservative 95\% one-sided confidence bounds.

\begin{example}
{\bf Binomial confidence interval for $F(\realrv)$.}
To illustrate the computation of the binomial confidence intervals
from (\ref{equation:binomial.confidence.interval}), we
compute a 95\% confidence interval for F(2) at which point 3 of the
100 tubes had failed. Then with $n=100$ and $n\Fhat(2)=3$, 
$\Fquant_{(.975; 196, 6)}=4.8831$, and 
$\Fquant_{(.975; 8, 194)}=2.2578$, substituting into
%splus qf(.975, 8, 194) =2.257833
%splus qf(.975,196,6) =4.8831
(\ref{equation:binomial.confidence.interval}) gives
%splus (1+((100-3+1)*qf(.975,200-6+2,6)/(3)))^(-1)=0.006229972
%splus (1+(100-3)/((3+1)*qf(.975,6+2,200-6)))^(-1)=0.08517605
\begin{eqnarray*}
\undertilde{F}(2) &=&
\left \{
1+
\frac{(100-3+1) \Fquant_{(.975; 200-6+2,
6)}}{3}\right \}^{-1} = .0062
\\
\tilde{F}(2)&=&
\left \{1+\frac{100-3}{(3+1) \Fquant_{(.975; 6+2,200-6)}
}
\right \}^{-1}= .0852.
\end{eqnarray*}
Thus we are (at least) 95\% confident that the probability of failing
before the end of 2 years is between .0062 and .0852.
\end{example}

%-------------------------------------------------------------------
\subsection{Pointwise 
normal-approximation confidence interval~for~$F(\realrv_{i})$}
\label{section:nor.bino.confidence.intervals}

For a specified value
of $\realrv_{i}$, a simpler approximate $100 (1-\alpha)\%$ 
confidence interval
for $F(\realrv_{i}) $ is
\begin{equation}
\label{equation:nor.approc.binom.ci}
[\undertilde{F}(\realrv_{i}), \quad \tilde{F}(\realrv_{i})] =
\Fhat(\realrv_{i}) \pm \norquan_{(1-\alpha/2)} 
\sehat_{\Fhat}
\end{equation}
where $z_{(p)}$ is the $p$ quantile of the
standard normal distribution and

\begin{equation}
\label{equation:binomial.var.ecdf}
\sehat_{\Fhat}=\sqrt{ \Fhat(\realrv_{i}) \left [1- \Fhat(\realrv_{i}) \right ]/ n }
\end{equation}
is an estimate of the
standard error~of~$\Fhat(\realrv_{i})$.
This confidence interval is
based on the assumption that
\begin{displaymath}
Z_{\Fhat}  = 
\frac{
     \Fhat(\realrv_{i})-F(\realrv_{i})
     }
     {
   \sehat_{\Fhat}
     }
\end{displaymath}
can be adequately approximated by a $\NOR(0,1)$ (standard normal)
distribution.  For this approximation to be adequate,
$n\Fhat(\realrv_{i})$ should be at least 5 to 10 and no more than
$n-5$ or $n-10$.  Otherwise the approximation will be crude, and it is
even possible to get confidence limits that are outside
the interval 0 to 1.  The computations for
(\ref{equation:nor.approc.binom.ci}) are, however, simple and can be
done easily by hand.
\begin{example}
{\bf Normal-approximation confidence interval for the Plant~1 $F(\realrv)$.} To
illustrate these intervals we compute an approximate 95\% confidence
interval for $F(3)$ for the Plant~1 heat exchanger tubes. At $t=3$, 5
of the 100 tubes had failed. Then with $n=100$, $\Fhat(3)=.05$, and
$z_{.975}=1.960$, substituting into
(\ref{equation:nor.approc.binom.ci}) gives
\begin{displaymath}
[\undertilde{F}(3), \quad \tilde{F}(3)] =
.05  \pm 1.960 \times .02179 = [.0073, \quad .0927]
\end{displaymath}
where 
$
\sehat_{\Fhat}=\sqrt{ .05 \left (1- .05\right )/ 100 } = .02179
$
is an estimate of the
standard error of $\Fhat(\realrv_{3})$.

Table~\ref{table:plant1.heat.exch} also shows confidence intervals for 
$F(1)$ and $F(2)$ (both the conservative and the
normal-approximation).
Note that the normal-approximation intervals for $F(1)$ and $F(2)$
have nonsensical negative lower endpoints.
Section~\ref{section:np.stderror.and.ci} provides a method of
improving the normal-approximation interval for situations in which
the conservative binomial interval does not apply. 
\end{example}
%-------------------------------------------------------------------
\begin{table}
\caption{Nonparametric estimates and 
approximate confidence intervals for the heat exchanger tube $F(\realrv)$.}
\centering\small
\begin{tabular}{*{5}{r}c}
\\[-.5ex]
\hline
\\[-.8ex]
 & & & & & \multicolumn{1}{c}{Pointwise Approximate}
\\[-.1ex]
 & & & & & \multicolumn{1}{c}{Confidence Intervals} 
\\[-.1ex]
Year & $t_{i}$  &  $d_{i}$ & $\Fhat(\realrv_{i})$ &
$\sehat_{\Fhat}$ & $\undertilde{F}(\realrv_{i}) \quad \quad 
\tilde{F}(\realrv_{i})$
\\
\cline{1-6}
\\
$ (0-1]$ & 1 & 1 & .01& .00995 \\[1ex]
\multicolumn{5} {l} { 95\% Confidence Intervals for $F(1)$} \\
\multicolumn{5} {l} {\hspace{1em} Based on \,\,\,
        Binomial Theory}
& [ .0003, .0545 ] \\
\multicolumn{5} {l} {\hspace{1em} Based on \,\,\,
        $Z_{\Fhat} \approxdist \NOR(0,1)$ }
& [$-.0095$, .0295 ] \\[3ex]
$ (1-2]$ & 2 & 2 & .03& .01706 \\[1ex]
\multicolumn{5} {l} { 95\% Confidence Intervals for $F(2)$} \\
\multicolumn{5} {l} {\hspace{1em} Based on \,\,\,
        Binomial Theory}
& [ .0062,  .0852 ] \\
\multicolumn{5} {l} {\hspace{1em} Based on \,\,\,
        $Z_{\Fhat} \approxdist \NOR(0,1)$ }
& [$-.0034$, .0634 ] \\[3ex]
$ (2-3]$ & 3 & 2 & .05& .02179
\\[1ex]
\multicolumn{5} {l} { 95\% Confidence Intervals for $F(3)$} \\
\multicolumn{5} {l} {\hspace{1em} Based on \,\,\,
        Binomial Theory}
& [ .0164, .1128 ] \\
\multicolumn{5} {l} {\hspace{1em} Based on \,\,\,
        $Z_{\Fhat} \approxdist \NOR(0,1)$ }
& [ .0073, .0927 ] 
\\[1ex] \hline
\end{tabular}
\label{table:plant1.heat.exch}
\end{table}

\begin{example}
{\bf Nonparametric confidence intervals for integrated circuit
failure data.} The integrated circuit failure data in
Table~\ref{table:lfp.data} are singly censored at 1370 hours. Ties
in the data suggest that failures were found at points in times
where there was an inspection. For this example, however, as is
commonly done in practice when the intervals are small relative to
the spread in the data, this discreteness in the data will be
ignored.  Thus the simple binomial methods can be used to estimate
$F(t)$ at any specified value of $t$.
Figure~\ref{figure:lfp.cdf.pw.ps} shows the nonparametric estimate
along with normal-approximation 95\% confidence intervals for each
estimated point.  The normal-approximation intervals are
nonsymmetric because they are based on an approximation-improving
transformation explained in
Section~\ref{section:np.stderror.and.ci}.
\end{example}



%-------------------------------------------------------------------
\begin{figure}
\splusbookfigure{\figurehome/lfp.cdf.pw.ps}
\caption{Plot of the nonparametric estimate 
of the cdf of the integrated circuit failure times . Also shown are
a set of normal-approximation 95\% confidence intervals.}
\label{figure:lfp.cdf.pw.ps}
\end{figure}
%------

%----------------------------------------------------------------------
\section{Estimation From Multiply Censored Data}
\label{section:def.model.free.est}

This section shows how to compute a nonparametric estimate of a cdf
from data with multiple right censoring (failures occur after some
units have been censored).  Suppose that an initial sample of $n$ units
start operating at time zero. If a unit does not fail in interval $i$,
it is either censored at the end of interval $i$ or it continues into
interval $i+1$.  Information is available on the status of the units
at the end of each interval. The intervals may be large or small and
need not be of equal length, as long 
as the intervals for different units do not overlap
(Section~\ref{section:arbitrary.censoring}
extends the method to data with different, overlapping intervals,
which arise, for example, when units are not subject to the same inspection
schedule). Let $\deadin_{i}$ denote the number of units that {\em
died} or failed in the $i$th interval $(\realrv_{i-1}, \realrv_{i}]$.
Also, let $\rcensorat_{i}$ denote the number of units that survive
interval $i$ and are {\em right} censored at
$\realrv_{i}$.  The units that are alive at the beginning of interval
$i$ are called the ``risk set'' for interval $i$ (i.e., those at risk
to failure) and the size of this risk set at the beginning of interval
$i$ is
\begin{equation}
\label{equation:psubihat}
n_{i}= n-\sum_{j=0}^{i-1} \deadin_{j}-
\sum_{j=0}^{i-1} \rcensorat_{j},\,\,\,\,\
i=1,\ldots,m
\end{equation}
where $m$ is the number of intervals and
it is understood that $\deadin_{0}=0$ and $\rcensorat_{0}=0$.
An estimator of the conditional probability of failing in 
interval $i$, given that a unit enters this interval, is the sample
proportion failing
\begin{displaymath}
\phat_{i}=\frac{\deadin_{i}}{n_{i}},\,\,\,\,\
i=1,\ldots,m.
\end{displaymath}
Substituting these into (\ref{equation:discrete.survival.function})
provides an estimator of the survival function:
\begin{equation}
\label{equation:nonparametric.estimator}
\Shat(\realrv_{i})=\prod_{j=1}^{i}\left [ 1 - 
\phat_{j} \right ] ,\,\,\,\,\
i=1,\ldots,m.
\end{equation}
Then the corresponding nonparametric estimator of $F(\realrv_{i})$
is
\begin{equation}
\label{equation:nonparametric.estimator.for.F}
\Fhat(\realrv_{i})=1-\Shat(\realrv_{i}),\,\,\,\,\
i=1,\ldots,m.
\end{equation}
Here $\phat_{i}$ is the maximum likelihood (ML) estimator of the
conditional probability $p_{i}$ from (\ref{equation:pvec.def}).  This
implies that $\Fhat(\realrv_{i})$ is the ML estimator of
$F(\realrv_{i})$ (see Exercise~\ref{exercise:km.ml}).  The
nonparametric estimator $\Fhat(\realrv_{i})$ is defined at all
$\realrv_{i}$ values (upper endpoint of all intervals).  Additionally,
if interval $i$ is known to have zero failures, then
$\Fhat(\realrv_{i})=\Fhat(\realrv_{i-1})$ for $\realrv_{i-1}\leq
\realrv \leq
\realrv_{i}$. If interval $i$ is known to contain one or more
failures, $\Fhat(t)$ increases from $\Fhat(\realrv_{i-1})$ to
$\Fhat(\realrv_{i})$ in the interval $(\realrv_{i-1},\realrv_{i}]$ but
as before, $\Fhat(\realrv)$ is not defined over the interval.
Note that when there are no censored observations before
the last failure, (\ref{equation:nonparametric.estimator.for.F}) is
numerically equivalent to (\ref{equation:binomial.ecdf}).

%-------------------------------------------------------------------
\begin{example}
\label{example:npe.heat.exchanger}
{\bf Nonparametric estimate of $F(\realrv)$ for the pooled heat 
exchanger tube data.}
%-------------------------------------------------------------------
\begin{figure}
\xfigbookfigure{\figurehome/heatex.plcomp1fig.ps}
\caption{Pooling of the heat exchanger data in preparation for 
computing the nonparametric estimate of $F(\realrv)$. The number of
units at risk in each cell is shown in the small rectangles.}
\label{figure:heatex.plcomp1fig.ps}
\end{figure}
%-------------------------------------------------------------------
Returning to the heat exchanger data from Example
\ref{example:heat.exchanger.data},
Figure~\ref{figure:heatex.plcomp1fig.ps} displays the pooled
data from
Figure~\ref{figure:heatex.stime.datafig.ps} across the three different
plants. For each year of operation, the bottom of Figure
\ref{figure:heatex.plcomp1fig.ps} shows the number in the risk set (in
the upper left-hand corner), the number that cracked in each interval,
and the number censored at the end of each interval.
Table~\ref{table:pl.heat.exch} illustrates numerical computations.
Figures~\ref{figure:heatexch.cdf.pw.ps} and
\ref{figure:heatexch.cdf.sim.ps} show the estimate with different sets
of approximate confidence intervals and bands that are explained in
Sections~\ref{section:np.pointwise.ci} and
\ref{section:np.simultaneous.cb},
respectively.
%-------------------------------------------------------------------
\begin{table}
\caption{Calculations for the nonparametric estimate of $F(\realrv_{i})$
for the pooled heat exchanger tube data.}
\centering\small
\begin{tabular}{*{9}{r}}
\\[-.5ex]
\hline
&& \multicolumn{1} {c} {Failed} & 
	\multicolumn{1} {c} {Censored} & \multicolumn{1} {c} {Entered}&\\
\cline{3-5}
Year & $t_{i}$ & $d_{i}$ & $r_{i}$ & $n_{i}$ & $\phat_{i}$ & $1-\phat_{i}$
& $\Shat(\realrv_{i})$ & $\Fhat(\realrv_{i})$ \\[.8ex]
\hline
$ (0-1]$ & 1   & 4 & 99 &  300 &  4/300 & 296/300 & .9867 & .0133 \\ 
$ (1-2]$ & 2   & 5 & 95 &  197  & 5/197 & 192/197 & .9616 & .0384\\ 
$ (2-3]$ & 3   & 2 & 95 &  97  &  2/97 & 95/97 & .9418 & .0582
\\[1ex] \hline
\end{tabular}
\label{table:pl.heat.exch}
\end{table}
\end{example}
%-------------------------------------------------------------------



%----------------------------------------------------------------------
%----------------------------------------------------------------------
 \section{Pointwise Confidence Intervals from Multiply Censored Data}
\label{section:np.stderror.and.ci}
%-------------------------------------------------------------------
 \subsection{Approximate variance of $\Fhat(\realrv_{i})$}
\label{section:avar.nonpar.fhat}
Because $\Fhat(\realrv_{i})=1-\Shat(\realrv_{i})$,
$\var
\left [\Fhat(\realrv_{i}) \right ]=
\var \left [\Shat(\realrv_{i}) \right ].$
Using the delta method approach in Appendix
Section~\ref{asection:delta.method}, a first-order Taylor series
approximation for $\Shat(\realrv_{i})$ is
\begin{displaymath}
\Shat(\realrv_{i}) \approx
 S(\realrv_{i})+\sum_{j=1}^{i} 
 \left. \frac{\partial S}{\partial q_{j}} \right |_{q_{j}}
\left(\qhat_{j} - q_{j} \right )
\end{displaymath}
where $q_{j}=1-p_{j}$. 
Because the $\qhat_{j}$ values are approximately
uncorrelated binomial proportions (the $\qhat_{j}$ 
values are asymptotically, 
as $n \rightarrow \infty$, uncorrelated), it follows that
\begin{equation}
\label{equation:pl.asym.var}
 \var \left [\Fhat(\realrv_{i}) \right ]=\var \left [\Shat(\realrv_{i}) \right ] 
\approx [S(\realrv_{i})]^{2}
\sum_{j=1}^{i} \frac{p_{j}}{n_{j} (1-p_{j})}.
\end{equation}
The right-hand side of (\ref{equation:pl.asym.var}) is also an
asymptotic (large-sample approximate) variance denoted by
$\avar
\left [\Fhat(\realrv_{i}) \right ]$.  This can be shown by
using the large-sample approximation in
Appendix Sections~\ref{asection:asymptotic.theory.mle}
and~\ref{asection:asymptotic.theory.fmle}.

%-------------------------------------------------------------------
\subsection{Greenwood's formula}
Substituting $\phat_{j}$ for $p_{j}$ and $\Shat(\realrv_{i})$ for
$S(\realrv_{i})$
in (\ref{equation:pl.asym.var})
gives the following variance estimator:
\begin{equation}
\label{equation:greenwood}
 \varhat \left [\Fhat(\realrv_{i}) 
\right ]=\varhat \left [\Shat(\realrv_{i}) \right ] =
 [\Shat(\realrv_{i})]^{2}
\sum_{j=1}^{i} \frac{\phat_{j}}{n_{j}(1-\phat_{j})}.
\end{equation}
This is known as ``Greenwood's formula.''
An estimator of the standard error of
$\Fhat(\realrv_{i})$
is
\begin{equation}
\label{equation:sd.greenwood}
\sehat_{\Fhat} =
\sqrt{\varhat \left [\Fhat(\realrv_{i}) \right ]}.
\end{equation} 
Note that when there are no censored observations before
the last failure, (\ref{equation:sd.greenwood}) is
numerically equivalent to (\ref{equation:binomial.var.ecdf}).

%----------------------------------------------------------------------
\subsection{Pointwise normal-approximation
confidence interval for $F(\realrv_{i}) $}
\label{section:np.pointwise.ci}
Because $\Fhat(\realrv)$ is defined only at the upper endpoint of
intervals that contain failures, $\Fhat(\realrv)$ is generally
estimated only at such points (if there are no failures in an
interval, $\Fhat(\realrv)$ remains constant over that interval).  For
a specified upper endpoint $\realrv_{i}$ at which an estimate of
$F(\realrv)$ is desired, a normal-approximation $100 (1-\alpha)\%$
confidence interval for $F(\realrv_{i})$ is
\begin{equation}
\label{equation:np.normal.theory.ci}
[\Flower(\realrv_{i}), \quad \Fupper(\realrv_{i})] =
\Fhat(\realrv_{i}) \pm \norquan_{(1-\alpha/2)} 
\sehat_{\Fhat}
\end{equation}
where $z_{(p)}$ is the $p$ quantile of the 
standard normal distribution.
In general, a one-sided approximate $100(1-\alpha)\%$
confidence bound can be obtained by replacing
$\norquan_{(1-\alpha/2)}$ with $\norquan_{(1-\alpha)}$ and using the
appropriate endpoint of the two-sided confidence interval.

The approximate confidence intervals 
from (\ref{equation:nor.approc.binom.ci}) or
(\ref{equation:np.normal.theory.ci}) are based on the assumption that
the distribution of
\begin{equation}
\label{equation:def.zfhat.normal}
Z_{\Fhat}  = 
\frac{\Fhat(\realrv_{i})-F(\realrv_{i})}
{\sehat_{\Fhat}}
\end{equation}
can be approximated adequately by a $\NOR(0,1)$ distribution.
Then 
\begin{equation}
\label{equation:prob.zfhat.normal}
\Pr \left[ \norquan_{(\alpha/2)} < Z_{\Fhat} \leq
	\norquan_{(1-\alpha/2)} \right]	\approx 1-\alpha
\end{equation}
implies that
\begin{equation}
\label{equation:fhat.ci.normal}
\Pr \left[  \Fhat(\realrv_{i}) - \norquan_{(1-\alpha/2)} 
	\sehat_{\Fhat}  <
	F(\realrv_{i})   \leq
	\Fhat(\realrv_{i}) - \norquan_{(\alpha/2)} 
	\sehat_{\Fhat} \right]
	\approx 1-\alpha.
\end{equation}
This gives the approximate coverage probability for intervals computed
with the procedure in (\ref{equation:np.normal.theory.ci}). Note that
in (\ref{equation:fhat.ci.normal}), $F(\realrv_{i})$ is fixed while
$\Fhat(\realrv_{i})$ and $\sehat_{\Fhat}$ are
random.  The approximation in (\ref{equation:fhat.ci.normal}) is a
large-sample approximation and improves with increasing sample
size. Appendix Section~\ref{asection:convergence.in.distribution} 
provides more information on such large-sample approximations.

When the sample size is not large, however, the distribution of
$Z_{\Fhat}$ may be badly skewed and the normal distribution may not
provide an adequate approximation, particularly in
the tails of the distribution [where $\Fhat(\realrv)$ is close to 0
or 1].  For example, it is possible that 
(\ref{equation:np.normal.theory.ci}) gives 
$\undertilde{F}(\realrv) < 0$ or $\tilde{F}(\realrv) > 1$, a
result that is outside the possible range for $F(t)$. Generally
a better approximation might be obtained by using the logit
transformation ($\logit(p)=\log[p/(1-p)]$) and basing the confidence
intervals on the distribution of
\begin{equation}
\label{equation:fhat.logit.stand.z}
Z_{\logit(\Fhat)} = \frac{\logit[\Fhat(\realrv_{i})]-\logit
[F(\realrv_{i})]}
{\sehat_{\logit(\Fhat)}}.
\end{equation}
Because $\logit[\Fhat(\realrv_{i})]$, like a standard normal
random variable, is unrestricted (i.e.,  ranges between $-\infty$ and
$\infty$) (\ref{equation:fhat.logit.stand.z}) can be expected to be 
closer to $\NOR(0,1)$ than (\ref{equation:def.zfhat.normal}).  This leads
(the needed steps are left as an exercise)
to the two-sided approximate $100(1-\alpha)\%$ confidence interval
\begin{equation}
\label{equation:fhat.logit.ci}
[\undertilde{F}(\realrv_{i}), \quad \tilde{F}(\realrv_{i})] = 
\left[\frac{\Fhat}{\Fhat
+(1-\Fhat) \times w},  \quad \frac{\Fhat}{\Fhat+
(1-\Fhat)/w}
\right]
\end{equation}
where
$w=\exp\{\norquan_{(1-\alpha/2)}\sehat_{\Fhat}/
[\Fhat(1-\Fhat)]\}$. The endpoints of this interval
will always lie between 0 and 1.
A one-sided approximate $100(1-\alpha)\%$
confidence bound can be obtained by replacing
$\norquan_{(1-\alpha/2)}$ with $\norquan_{(1-\alpha)}$ and using the
appropriate endpoint of the two-sided confidence interval.

%-------------------------------------------------------------------
\begin{example}
\label{example:heat.ex.nonpar.nor}
{\bf Normal-approximation confidence intervals for the heat exchanger data.}
This example illustrates the computation of 
standard errors and nonparametric approximate confidence
intervals for the heat exchanger data, 
using both large-sample approximations in this section.
For the failure probability at $\realrv_{i}=\realrv_{1}=1$, we have
$\Fhat(1)=.0133$ and
\begin{displaymath}
\varhat \left [\Fhat(1) \right ]
=(.9867)^{2}
\left[ \frac{.0133}{300(.9867)} \right]=.0000438.
\end{displaymath}
Then
$\sehat_{\Fhat} =\sqrt{.0000438}=.00662$
and the approximate
95\% confidence interval for $F(1)$ from
(\ref{equation:np.normal.theory.ci})
is
\begin{displaymath}
[\undertilde{F}(1), \quad \tilde{F}(1)] = 
 .0133 \pm 1.960(.00662)=[.0003, \quad .0263].
\end{displaymath}
%splus:  .0133-(1.96)*(.00662)          = 0.0003248
%splus:  .0133+(1.96)*(.00662)          = 0.0262752
The corresponding interval from (\ref{equation:fhat.logit.ci}),
based on the logit transformation, is
\begin{displaymath}
[\undertilde{F}(1), \quad \tilde{F}(1)] = 
\left[\frac{.0133}{.0133
+(1-.0133) \times w}, \quad \frac{.0133}{.0133+
(1-.0133)/w}\right] = \left[.0050, \quad .0350\right]
\end{displaymath}
%splus:  (.0133)/(.0133+(1-.0133)*2.687816)       = 0.00498993
%splus:   (.0133)/(.0133+(1-.0133)/2.687816)      = 0.0349631
where $w=\exp\{1.960(.00662)/[.0133(1-.0133)]\}=2.687816$.
%splus:  exp(1.960*(.00662)/(.0133*(1-.0133)))    =2.687816
Differences between the two methods are large enough
to be of practical importance. The intervals based on the logit
transformation are expected to provide a better approximation
to the nominal $95\%$ confidence level.

For the failure probability at $\realrv_{i}=\realrv_{2}=2$, we have
\begin{displaymath}
\varhat \left [\Fhat(2) \right ]
=(.9616)^{2} \left[ \frac{.0133}{300(.9867)}+
\frac{.0254}{197(.9746)} \right] = .0001639
\end{displaymath}
so that $\sehat_{\Fhat} = \sqrt{.0001639}=.0128$.
The approximate 95\% confidence intervals for $F(2)$ are
\begin{displaymath}
[\undertilde{F}(2), \quad \tilde{F}(2)] 
= .0384 \pm 1.960(.0128)=[.0133, \quad .0635]
\end{displaymath}
%splus:  .0384-(1.96)*(.0128)          =  0.013312
%splus:  .0384+(1.96)*(.0128)          = 0.063488
and
\begin{displaymath}
[\undertilde{F}(2), \quad \tilde{F}(2)] = 
\left[\frac{.0384}{.0384
+(1-.0384) \times w}, \quad \frac{.0384}{.0384+
(1-.0384)/w}\right] = \left[.0198, \quad .0730\right]
\end{displaymath}
%splus:  (.0384)/(.0384+(1-.0384)*1.972739)       = 0.01984101
%splus:   (.0384)/(.0384+(1-.0384)/1.972739)      = 0.07302545
where $w=\exp\{1.960(.0128)/[.0384(1-.0384)]\}=1.972739$.
%splus:  exp((1.960)*(.0128)/(.0384*(1-.0384)))    =1.972739.

Figure~\ref{figure:heatexch.cdf.pw.ps} shows the nonparametric
estimates for $F(\realrv_{i})$ and pointwise approximate 95\%
confidence intervals.  The intervals are not symmetric around the
estimates because of the logit transformation. The intervals are wide
because of the heavy censoring and the small number of failures.
%-------------------------------------------------------------------
\begin{figure}
\splusbookfigure{\figurehome/heatexch.cdf.pw.ps}
\caption{Plot of the nonparametric estimate
for the heat exchanger data along with a set of pointwise
logistic-transform normal-approximation 95\% confidence intervals for
$F(t)$.}
\label{figure:heatexch.cdf.pw.ps}
\end{figure}
\end{example}
%-------------------------------------------------------------------


%-------------------------------------------------------------------
\begin{table}
\caption{Summary of calculations for nonparametric confidence intervals
for $F(\realrv)$
for the pooled heat exchanger tube data.}
\centering\small
\begin{tabular}{rrrrc}
\\[-.5ex]
\hline
\\[-1.5ex]
Year & $\realrv_{i}$ &  $\Fhat(\realrv_{i})$
&$\sehat_{\Fhat}$&
	\multicolumn{1}{c}{Pointwise Confidence Intervals }
\\[.5ex]
\hline
\\[-.5ex]
$ (0-1]$ & 1 & .0133 & .00662 \\[1ex]
\multicolumn{4} {l} { 95\% Confidence Intervals for $F(1)$} \\
\multicolumn{4} {l} {\hspace{1em} Based on \,\,\,
        $Z_{\logit(\Fhat)} \approxdist \NOR(0,1)$ }
& [.0050, .0350] \\[.1ex]
\multicolumn{4} {l} {\hspace{1em} Based on \,\,\,
        $Z_{\Fhat} \approxdist \NOR(0,1)$ }
& [.0004, .0133] \\[3ex]

$ (1-2]$&2&.0384&.0128\\[1ex]
\multicolumn{4} {l} { 95\% Confidence Intervals for $F(2)$} \\
\multicolumn{4} {l} {\hspace{1em} Based on \,\,\,
        $Z_{\logit(\Fhat)} \approxdist \NOR(0,1)$ }
& [.0198, .0730] \\[.1ex]
\multicolumn{4} {l} {\hspace{1em} Based on \,\,\,
        $Z_{\Fhat} \approxdist \NOR(0,1)$ }
& [.0133, .0635] \\[3ex]

$ (2-3]$&3&.0582&.0187\\[1ex]
\multicolumn{4} {l} { 95\% Confidence Intervals for $F(3)$} \\
\multicolumn{4} {l} {\hspace{1em} Based on \,\,\,
        $Z_{\logit(\Fhat)} \approxdist \NOR(0,1)$ }
& [.0307, .1076] \\
\multicolumn{4} {l} {\hspace{1em} Based on \,\,\,
        $Z_{\Fhat} \approxdist \NOR(0,1)$ }
& [.0216, .0949] \\[.1ex]
\\[.5ex] \hline
\end{tabular}
\label{table:conf.interval.heat.exch}
\end{table}

%-------------------------------------------------------------------
%----------------------------------------------------------------------
\section{Estimation from Multiply Censored Data with Exact Failures}
\label{section:def.of.nonparametric.estimate}
Failures are often reported at exact times.  In such cases, the
reported times are denoted by $t_{i}$.  This section shows how to
apply the methods of Sections~\ref{section:def.model.free.est} and
\ref{section:np.stderror.and.ci}
to estimate $F(t)$ for such exact failures.


%-------------------------------------------------------------------
\begin{example}
\label{example:shock.absorber.data}
{\bf Shock absorber failure data.} Appendix
Table~\ref{atable:shockabs.data} gives the failure times (in number of
kilometers of use) of vehicle shock absorbers, first reported in
O'Connor~(1985). The table shows two different failure modes, denoted
by M1 and M2. Engineers responsible for shock absorber manufacturing
and reliability would be interested in the distribution of time to
failure for the individual failure modes. Engineers responsible for
higher-level automobile system reliability and choosing among
alternative vendors would be more interested in the overall
failure-time distribution for the part.
\end{example}

Exact failure times arise from a continuous
inspection process (or, perhaps, from having used a very large number
of closely-spaced inspections).  In the limit, as the number of
inspections increases and the width of the inspection intervals
approaches zero, failures are concentrated in a relatively small
number of intervals. Most intervals will not contain any failures.
$\Fhat(t)$ is {\em constant} over all intervals that have no
failures. Thus with small intervals, $\Fhat$ will become a step
function with gaps over the intervals where there were failures and
with jumps at the upper endpoint of these intervals. In the limit, as
the width of the intervals approaches $0$, the size of the gaps
approaches 0 and the step function increases at the reported failure
times. This limiting case of the interval-based nonparametric
estimator is generally known as the product-limit or Kaplan-Meier
estimator.

%-------------------------------------------------------------------
\begin{example}
\label{example:shock.absorber.cdfest}
{\bf Nonparametric estimator and normal-approximation confidence
intervals for the shock absorber data.} 
%-------------------------------------------------------------------
\begin{table}[htb]
\caption{Nonparametric estimates 
for the shock absorber data up to 12,200 km.}
\centering\small
\begin{tabular}{*{8}{r}}
\\[-1.0ex]
\hline
& \multicolumn{1}{c}{Failed} & \multicolumn{1}{c}{Censored} & \multicolumn{1}{c}{Entered}&\\
\cline{2-4}
 $\realrv_{i} \quad \mbox{(km)}$ & $d_{j}$ & $r_{j}$ & $n_{j}$ & $\phat_{j}$ & $1-\phat_{j}$
& $\Shat(\realrv_{i})$ & $\Fhat(\realrv_{i})$ \\[.8ex]
\hline
 6,700   & 1 & 0  & 38 & 1/38  & 37/38  & .9737 & .0263 \\ 
 6,950   & 0 & 1  & 37 &       &       &       &  \\ 
 7,820   & 0 & 1  & 36 &       &       &       &  \\ 
 8,790   & 0 & 1  & 35 &       &       &       &  \\ 
 9,120   & 1 & 0  & 34 & 1/34  & 33/34 & .9451  & .0549 \\ 
 9,660   & 0 & 1  & 33 &       &       &       &  \\ 
 9,820   & 0 & 1  & 32 &       &       &       &  \\ 
11,310   & 0 & 1  & 31 &       &       &       &  \\ 
11,690   & 0 & 1  & 30 &       &       &       &  \\ 
11,850   & 0 & 1  & 29 &       &       &       &  \\ 
11,880   & 0 & 1  & 28 &       &       &       &  \\ 
12,140   & 0 & 1  & 27 &       &       &       &  \\ 
12,200   & 1 & 0  & 26 & 1/26  &  25/26& .9087  & .0913 \\ 
\hline
\end{tabular}
\label{table:pl.shockabs}
\end{table}
%-------------------------------------------------------------------
For the data from
Example~\ref{example:shock.absorber.data} and Appendix
Table~\ref{atable:shockabs.data} we do not
differentiate between the two different failure modes. Instead, we
estimate the time to failure when both mode M1 and M2 are acting.
Table~\ref{table:pl.shockabs} illustrates the computations for the
product-limit estimator up to 12,200 km.
Figure~\ref{figure:shockabsB.cdf.pw.ps} shows the nonparametric
estimator and a set of pointwise approximate 95\% confidence
intervals for $F(\realrv)$. Estimated standard errors were computed using
(\ref{equation:sd.greenwood}) and the confidence intervals were
computed using the logistic transformation method in
Section~\ref{section:np.pointwise.ci}. The wide confidence
intervals indicate a high degree of uncertainty.
\end{example}

%-------------------------------------------------------------------
\begin{figure}
\splusbookfigure{\figurehome/shockabsB.cdf.pw.ps}
\caption{Plot of the nonparametric estimate
for the shock absorber data along with a set of pointwise
logistic-transform normal-approximation 95\% confidence intervals for
$F(t)$.}
\label{figure:shockabsB.cdf.pw.ps}
\end{figure}
%-------------------------------------------------------------------

%-------------------------------------------------------------------
%-------------------------------------------------------------------
\section{Simultaneous Confidence Bands}
\label{section:np.simultaneous.cb}

%-------------------------------------------------------------------
\subsection{Motivation}
The {\em pointwise} confidence intervals defined in
Section~\ref{section:np.pointwise.ci} are useful for making a
statement about $F(\realrv_{i})$ at {\em one} particular specified
value of $\realrv_{i}$ (even though it is common practice to plot a
set of such intervals). In many applications, however, it is
necessary to quantify the sampling uncertainty, simultaneously, over
a range of values of $t$ (even though it is common practice to plot
a set of such intervals).  To do this, we can use simultaneous
confidence bands for $F(t)$. As explained in
Chapter~\ref{chapter:probability.plotting}, simultaneous confidence
bands are particularly useful for judging the magnitude of observed
departures from fitted parametric models.

The overall coverage probability
for the collection of pointwise intervals 
(e.g., Figure~\ref{figure:shockabsB.cdf.pw.ps}) is generally less than 
that for any individual interval.
Plotting an estimate of $F(\realrv)$ showing
simultaneous confidence bands more accurately reflects
the uncertainty over the range of times displayed on the plot.

%-------------------------------------------------------------------
\subsection{Large-sample simultaneous confidence 
bands for $F(\realrv)$}
Approximate $100 (1-\alpha)\%$ simultaneous confidence bands
for $F(\realrv)$ can be obtained from 
\begin{equation}
\label{equation:ep.bands.def}
 \left[\simFlower(\realrv), \quad \simFupper(\realrv) \right] = 
	\Fhat(\realrv) \pm e_{(a, b, 1-\alpha/2)} 
\sehat_{\Fhat}(\realrv) \,\,\,\,\mbox{for all}\,\,\, t \in[t_{L}(a),t_{U}(b)]
\end{equation}
where the range $[t_{L}(a),t_{U}(b)]$ is a complicated function of the
censoring pattern in the data, as described in
Section~\ref{section:simultaneous.a.b}.  With no
censoring the
range of $t$ is given by the values of $t$ for which $a
\leq \Fhat(\realrv) \leq b$.  The approximate factors $e_{(a, b,
1-\alpha/2)}$ given in Table~\ref{table:vijay.evalues} were computed
from a large-sample approximation given in Nair~(1984). Because the
factor $e_{(a, b, 1-\alpha/2)}$ is the same for all values of $t$,
this family of bands is known as the ``equal precision'' or ``EP''
simultaneous confidence bands. The factors $e_{(a, b, 1-\alpha/2)}$
are larger than the corresponding pointwise normal-approximation
$\norquan_{(1-\alpha/2)}$ values. Thus the width of the simultaneous
bands, at any given point $t$, is wider than the corresponding
pointwise confidence interval at that point. This is as expected (and
necessary) to account for the simultaneous nature of the bands
afforded by (\ref{equation:ep.bands.def}).

Simultaneous approximate confidence bands like those defined in
(\ref{equation:ep.bands.def}) are based on the approximate
distribution of 
\begin{equation}
\label{equation:nair}
Z_{\mbox{max} \Fhat} = 
\vstack{\max}{\realrv \in [t_{L}(a),t_{U}(b)]}
\left [
 \frac{\Fhat(t)-F(t)}
{ \sehat_{\Fhat(\realrv)}}
\right ] .
\end{equation}
As explained in Section~\ref{section:np.pointwise.ci}
for the pointwise confidence intervals, it is generally better to
compute the simultaneous confidence bands based on the logit
transformation of $\Fhat$. These can be computed from 
\begin{equation}
\label{equation:simul.logit.bands}
[\simFlower(\realrv), \quad \simFupper(\realrv)] = 
\left[\frac{\Fhat(t)}{\Fhat(t)
+[1-\Fhat(t)] \times w},  \quad \frac{\Fhat(t)}{\Fhat(t)+
[1-\Fhat(t)]/w}
\right]
\end{equation}
where
$w=\exp\{e_{(a, b, 1-\alpha/2)}\sehat_{\Fhat}/
[\Fhat(1-\Fhat)]\}$. The endpoints of these bands
will always lie between 0 and 1.
The bands computed from (\ref{equation:simul.logit.bands})
are based on the  approximate
distribution of the random function
\begin{equation}
\label{equation:fhat.sim.logit.stand.z}
Z_{\mbox{max}\, \logit(\Fhat)}= 
\vstack{\max}{\realrv \in [t_{L}(a),t_{U}(b)]}
	\left [
     \frac{\logit[\Fhat(\realrv)]-\logit
[F(\realrv)]}
{ \sehat_{\logit[\Fhat(\realrv)]}}
       \right ].
\end{equation}
The bands that we have computed for our examples use this approximation.

%-------------------------------------------------------------------
\begin{table}
\caption{Factors $e_{(a,b,1-\alpha/2)}$
for the EP simultaneous approximate confidence bands.}
\centering\small
\begin{tabular}{llrrrr}
\\[-.5ex]
\hline
\multicolumn{2}{c}{Limits} & \multicolumn{4}{c}{Confidence Level}\\
\hline
\multicolumn{1}{c}{$a$}&\multicolumn{1}{c}{$b$}
&\multicolumn{1}{c}{.80}
&\multicolumn{1}{c}{.90}
&\multicolumn{1}{c}{.95}
&\multicolumn{1}{c}{.99}\\
\hline
.005&.995&2.86&3.12&3.36&3.85\\
.01&.995&2.84&3.10&3.34&3.83\\
.05&.995&2.76&3.03&3.28&3.77\\
.1&.995&2.72&3.00&3.25&3.75\\
.005&.99&2.84&3.10&3.34&3.83\\
.01&.99&2.81&3.07&3.31&3.81\\
.05&.99&2.73&3.00&3.25&3.75\\
.1&.99&2.68&2.96&3.21&3.72\\
.005&.95&2.76&3.03&3.28&3.77\\
.01&.95&2.73&3.00&3.25&3.75\\
.05&.95&2.62&2.91&3.16&3.68\\
.1&.95&2.56&2.85&3.11&3.64\\
.005&.9&2.72&3.00&3.25&3.75\\
.01&.9&2.68&2.96&3.21&3.72\\
.05&.9&2.56&2.85&3.11&3.64\\
.1&.9&2.48&2.79&3.06&3.59\\
\hline
\end{tabular}
\label{table:vijay.evalues}
\end{table}
%-------------------------------------------------------------------
%-------------------------------------------------------------------
\subsection{Determining the time-range for simultaneous confidence 
bands for $F(\realrv)$} 
\label{section:simultaneous.a.b}
Specifying the quantities $a$ and $b$
determines the range $[t_{L}(a),t_{U}(b)]$ over which simultaneous
confidence bands for $F(\realrv)$ are defined. Let
\begin{displaymath}
\sigmahat(t) = n \sum_{j: t_{j} \leq t} \frac{d_{j}}{n_{j}(n_{j}-d_{j})},\quad
\mbox{and} \quad \Khat(t)=\frac{\sigmahat(t)}{1+\sigmahat(t)}.
\end{displaymath}
where the summation is over $j$ such that $t_{j}\leq t$.
Then the simultaneous confidence bands have a range covering all values
of $t$ such that $a \leq \Khat(t) \leq b$. The function $\Khat(t)$
behaves like a nonparametric estimate $\Fhat(t)$: it is nondecreasing,
$0 \leq \Khat(t) \leq 1$, and $\Khat(t)=\Fhat(t)$ when there is no
censoring.

%-------------------------------------------------------------------
\begin{example}
{\bf Simultaneous confidence bands for the shock absorber life cdf.}
Figure~\ref{figure:shockabsB.cdf.sim.ps} is similar to
Figure~\ref{figure:shockabsB.cdf.pw.ps}, but it displays approximate
95\% simultaneous confidence bands for $F(t)$ instead of a set of
pointwise confidence intervals.  Note that the upper limit of the
simultaneous confidence bands is constant from 6,700 km to 12,200 km,
even though there is a failure and a corresponding jump in $\Fhat(t)$
at 9,120 km.  This is due to an adjustment made to
(\ref{equation:simul.logit.bands}) so that the simultaneous confidence
bands for $F(t)$ do not decrease, thus agreeing with the
nondecreasing characteristic of cdfs. If the upper band is
decreasing on the left, it is made flat from $t_{L}(a)$ to the point
of the minimum. If the lower band is decreasing on the right, it made
flat from the point of maximum to $t_{U}(b)$.  These adjustments, if
needed, give tighter, more sensible bands and have no effect on the
actual coverage probability of the simultaneous bands.
%-------------------------------------------------------------------
\begin{figure}
\splusbookfigure{\figurehome/shockabsB.cdf.sim.ps}
\caption{Plot of the nonparametric estimate
for the shock absorber data along with simultaneous logistic-transform
normal-approximation 95\% confidence bands for $F(t)$ computed from
(\ref{equation:simul.logit.bands}).}
\label{figure:shockabsB.cdf.sim.ps}
\end{figure}
%-------------------------------------------------------------------
\end{example}



\begin{example}
{\bf Simultaneous confidence bands for the pooled heat exchanger tube data.}
Figure~\ref{figure:heatexch.cdf.sim.ps} for the
heat exchanger data, shows simultaneous confidence
bands. As expected, the simultaneous bands are much wider than the
set of pointwise confidence intervals in 
Figure~\ref{figure:heatexch.cdf.pw.ps}.
%-------------------------------------------------------------------
\begin{figure}
\splusbookfigure{\figurehome/heatexch.cdf.sim.ps}
\caption{Plot of the nonparametric estimate
for the heat exchanger data along with simultaneous logistic-transform
normal-approximation 95\% confidence intervals for $F(t)$ computed from
(\ref{equation:simul.logit.bands}).}
\label{figure:heatexch.cdf.sim.ps}
\end{figure}
%-------------------------------------------------------------------
\end{example}

%----------------------------------------------------------------------
%----------------------------------------------------------------------
%----------------------------------------------------------------------
\section{Uncertain Censoring Times}
\label{section:life.table}
The methods in earlier
sections of this chapter assume that all left- and right-censored
observations occur at upper endpoints of the
intervals. If all of the censoring times are {\em known}, this is not a
serious restriction because the data intervals can be defined to
accommodate all of the data.  If, however, censoring times are known
only to be {\em within} specified intervals, the risk set is
decreasing over the intervals in a manner that cannot be specified
precisely. If the censoring times are random and the
form of the distribution is known, a likelihood estimation
method could be based on the likelihood in
(\ref{equation:like.censoring.in.interval}). Without such knowledge,
another approach is needed.

Two extreme methods of handling the censored observations in the
intervals are
\begin{itemize}
\item
Assume that all censored observations are removed at $\realrv_{i}$,
the upper endpoint of the interval. This gives
$\phat_{i}=\deadin_{i}/n_{i}$ as used in
Section~\ref{section:def.model.free.est}.  This estimate of
$p_{i}$ is biased upward.
\item
Assume that all censored observations are removed at $\realrv_{i-1}$,
the lower endpoint of the interval. This gives
$\phat_{i}=\deadin_{i}/(n_{i}-\rcensorat_{i})$.
This estimate of $p_{i}$ is biased downward.
\end{itemize}
A commonly used compromise is
$\phat_{i}=\deadin_{i}/(n_{i}-\rcensorat_{i}/2)$, the harmonic mean of
the two more extreme estimates.  These compromise estimates
$\phat_{i}$ can be substituted into
(\ref{equation:nonparametric.estimator}) and
(\ref{equation:greenwood}) leading through
(\ref{equation:nonparametric.estimator.for.F}) and
(\ref{equation:sd.greenwood}) to the ``actuarial'' or
``life table'' nonparametric
estimate $\Fhat(t)$ and the corresponding standard error
$\sehat_{\Fhat}$.

\begin{example}
{\bf Nonparametric estimate for the prototype battery data with
uncertain censoring times}
\label{example:proto.battery.npmle}
%-------------------------------------------------------------------
\begin{table}
\caption{Calculations for the nonparametric life table estimate of $F(\realrv_{i})$
for the prototype battery data.}
\centering\small
\begin{tabular}{rrcccrrrrr}
\\[-.5ex]
\hline
\\[-1.5ex]
Interval&  & Failed in& Censored in&Entered& Adjusted& \\
in Hours &  &$(t_{i-1},t_{i}]$ &$(t_{i-1},t_{i}]$&$(t_{i-1},t_{i}]$ &
at Risk& \\
\cline{1-1}\cline{3-6}
$(t_{i-1},t_{i}]$ &  & $d_{i}$ & $r_{i}$ & $n_{i}$ & $n_{i}-\frac{r_{i}}{2}$ & $\phat_{i}$ & $1-\phat_{i}$
& $\Shat(\realrv_{i})$ & $\Fhat(\realrv_{i})$ \\[.8ex]
\hline
  $(0-50]$     & & 1 & 5 & 68 & 65.5 & 1/65.5 & 64.5/65.5 & .985 & .015 \\
  $(50-100]$   & & 0 & 6 & 62 & 59 & 0/59 & 59/59 & .985 & .015 \\ 
  $(100- 150]$ & & 1 & 1 & 56 & 55.5 & 1/55.5 & 54.5/55.5 & .967 & .033 \\ 
  $(150- 200]$ & & 4 & 6 & 54 & 51 & 4/51 & 47/51 & .891 & .109 \\
  $(200- 250]$ & & 1 & 2 & 44 & 43 & 1/43 & 42/43 & .870 & .130 \\
  $(250- 300]$ & & 1 & 1 & 41 & 40.5 & 1/40.5 & 39.5/40.5 & .849 & .151 \\
  $(300- 350]$ & & 1 & 2 & 39 & 38 & 1/38 & 37/38 & .827 & .173 \\
  $(350- 400]$ & & 4 & 2 & 36 & 35 & 4/35 & 31/35 & .732 & .268 \\ 
   $(450- 500]$& & 4 & 3 & 30 & 28.5 & 4/28.5 & 24.5/28.5 & .629 & .371 \\
  $(500- 550]$ & & 2 & 1 & 23 & 22.5 & 2/22.5 & 20.5/22.5 & .573 & .427 \\
 $ (550- 600]$ & & 2 & 0 & 20 & 20 & 2/20 & 18/20 & .516 & .484\\
\vdots &\vdots &\vdots &\vdots &\vdots &\vdots 
&\vdots &\vdots &\vdots &\vdots \\[1ex] \hline 
\end{tabular}
\label{table:battery.pl}
\end{table}
The prototype battery failure data in
Example~\ref{example:morgan.battery.data} and Appendix
Table~\ref{atable:morgan.battery.data} has both failure times and
censored observations within the given intervals.
Here we consider the life distribution
of the batteries, without distinguishing among the
different failure modes.
Table~\ref{table:battery.pl} illustrates the computations for the
nonparametric life table estimate of $F(t)$ up to 600 hours.
\end{example}
%-------------------------------------------------------------------

%\section{Other Nonparametric Estimators of $F(\realrv)$}
%\subsection{The hazard estimator of $F(\realrv)$}
%\label{section:hazard.estimator}
%\subsection{The reduced sample estimator of $F(\realrv)$}
%----------------------------------------------------------------------
%----------------------------------------------------------------------

%-------------------------------------------------------------------
\section{Arbitrary Censoring}
\label{section:arbitrary.censoring}
The nonparametric estimate in
(\ref{equation:nonparametric.estimator}) works only for some kinds of
censoring patterns (e.g., multiple right censoring and interval
censoring with intervals that do not overlap).  When censoring is
more complicated, an alternative is needed. 
This need arises because, with complicated censoring, we do not know
the $n_{i}$ values in (\ref{equation:psubihat}).

The Peto-Turnbull estimator provides the needed generalization of
the nonparametric ML estimator that can be used for: \begin{itemize}
\item
Arbitrary censoring (e.g., combinations of left and right censoring
and interval censoring with overlapping intervals).
\item
Truncated data (as described in Section~\ref{section:truncated.data}).
\end{itemize} The basic idea is to write the likelihood as in
(\ref{equation:probability.of.data}) and to maximize this likelihood
to estimate the vector $\pvec$ or $\pivec$ from which one can compute
an estimate of $F(\realrv)$.  We illustrate the basic idea in the
following example. With the simple right-censoring patterns used
previously, the Peto-Turnbull estimator is equivalent to the 
nonparametric estimator defined in 
Sections~\ref{section:def.model.free.est}
and \ref{section:def.of.nonparametric.estimate}.

%-------------------------------------------------------------------
\begin{example}
\label{example:turbine.cdfest}
{\bf Nonparametric estimate of the turbine wheel distribution of
time to crack initiation based on inspection data.} As explained in
Example~\ref{example:turbine.wheel.data}, the turbine wheel inspection
data on time-to-crack-initiation can be viewed as a collection of
overlapping right- and left-censored observations.
Figure~\ref{figure:turbine.rawcdf.ps} plots the raw observed
proportion failing as a function of hours of exposure.
%-------------------------------------------------------------------
\begin{figure}
\splusbookfigure{\figurehome/turbine.rawcdf.ps}
\caption{Plot of proportions failing 
versus hours of exposure for the turbine wheel inspection data.}
\label{figure:turbine.rawcdf.ps}
\end{figure}
%-------------------------------------------------------------------
Due to random variability, this crude estimate of $F(\realrv)$ is, in
several places, decreasing. The true cdf is, of course, a
nondecreasing function of time.  Although it is {\em not} possible to use
the product limit estimator to compute the nonparametric maximum
likelihood estimate of $F(\realrv)$, the general maximum likelihood
approach introduced in Section~\ref{section:general.likelihood} still
works.  Figure~\ref{figure:turbine.npmldatafig.ps} illustrates the
basic parameters $\pi_{i}$ used in computing the nonparametric estimate of
$F(\realrv_{i})$ for this example.
%-------------------------------------------------------------------
\begin{figure}
\xfigbookfigure{\figurehome/turbine.npmldatafig.ps}
\caption{Basic parameters used in computing the nonparametric ML
estimate of $F(\realrv_{i})$ for the turbine wheel inspection data.}
\label{figure:turbine.npmldatafig.ps}
\end{figure}
%-------------------------------------------------------------------
Using terms like those in (\ref{equation:left.censored.likelihood})
and (\ref{equation:right.censored}) and the data summarized in
Table~\ref{table:turbine.data} leads to
\begin{eqnarray*}
\like(\pivec)=
\like(\pivec;\DATA)&=& \likeconstant \times  
	[ \pi_{1}]^{0} \times  
	[ \pi_{2} + \cdots +\pi_{12}]^{39} \times  \\
&&	[ \pi_{1}+\pi_{2}]^{4} \times  
	[ \pi_{3} + \cdots +\pi_{12}]^{49} \times  \\
&&	[ \pi_{1}+ \pi_{2} +\pi_{3}]^{2} \times  
	[ \pi_{4} + \cdots +\pi_{12}]^{31} \times  \\
&& \vdots\\
&&	[ \pi_{1}+\cdots +\pi_{11}]^{21} \times  
	[\pi_{12}]^{15}
\end{eqnarray*}
where $\pi_{12}=1-\sum_{i=1}^{11}\pi_{i}$.  The elements of $\pivec = (\pi_{1},
\ldots ,\pi_{11})$ that maximize $\like(\pivec)$ gives $\pivechat$, the
ML estimator of $\pivec$.  Substituting the elements of $\pivechat$
into (\ref{equation:f.is.sum.of.pi}) provides the nonparametric
estimator of $F(\realrv)$.  The estimate, which is nondecreasing, is
plotted in Figure~\ref{figure:turbine.cdf.pw.ps}. The pointwise
confidence intervals in Figure~\ref{figure:turbine.cdf.pw.ps} were
computed using (\ref{equation:simul.logit.bands}). The needed
$\sehat_{\Fhat}$ values were computed based on general methods given
in Appendix Section~\ref{asection:observed.information}.
%-------------------------------------------------------------------
\begin{figure}
\splusbookfigure{\figurehome/turbine.cdf.pw.ps}
\caption{Plot of the nonparametric ML estimate
for the turbine wheel inspection data along with a set of pointwise
logistic-transform normal-approximation 95\% confidence intervals for
$F(t)$.}
\label{figure:turbine.cdf.pw.ps}.
\end{figure}
%-------------------------------------------------------------------
\end{example}


%----------------------------------------------------------------------

\section*{Bibliographic Notes}

For the estimator given in (\ref{equation:nonparametric.estimator}),
Kaplan and Meier~(1958) allowed the width of the intervals in
(\ref{equation:small.intervals}) to approach zero and the number of
intervals to approach $\infty$.  This is the origin of the alternative
name ``product-limit estimator.''  This estimator is also widely known
as the ``Kaplan-Meier estimator.''  Kaplan and Meier~(1958),
Elandt-Johnson and Johnson~(1980, page~172), and Lawless~(1982,
page~74) provide more detailed justification for
(\ref{equation:nonparametric.estimator}) as an ML estimator.

Nelson~(1969, 1972, 1982) defined and illustrated the use of a
nonparametric estimator for the cumulative hazard function.
Corresponding asymptotic theory is given by Aalen~(1976).  This
provides an alternative estimator for $F(\realrv)$, sometimes referred
to as the ``Nelson-Aalen estimator.'' The Nelson-Aalen nonparametric
estimator is asymptotically, as the sample size increases, equivalent
to the product-limit nonparametric estimator.  Some properties of this
estimator are explored in Exercise~\ref{exercise:nelson.aalen.estimator}.

Fleming and Harrington~(1992), and Anderson, Borgan, Gill, and
Keiding~(1993) give detailed treatment of the general theory for both
the product-limit estimator and the Nelson-Aalen estimator and outline
the related literature with biomedical applications.

Nair~(1981) gives asymptotic theory for the simultaneous confidence
bands described in Section~\ref{section:np.simultaneous.cb}. Nair
(1984) used simulation to compare these bands with alternatives
bands suggested in the literature.  Weston and Meeker~(1990)
suggest and use simulation to evaluate the modification to Nair's
bands, based on the logit transformation.  Their results showed that
the logit transformation provides a better approximation to the
nominal coverage probability.

Peto~(1973) defined the nonparametric maximum likelihood estimator for
arbitrary censoring, including complicated overlapping
interval-censored data.  Turnbull~(1976) further generalized the
estimator to cover ``truncated data'' (to be discussed in
Section~\ref{section:truncated.data}) and suggested an EM
(expectation-maximization) algorithm to compute the estimate.
Gentleman and Geyer~(1994) describe asymptotic theory and
computational issues for this estimator.

Thomas and Grunkemeier~(1975) compare several different
nonparametric confidence intervals for $F(\realrv)$. They
conclude that confidence intervals based on inverting a likelihood
ratio test for $F(\realrv)$ are more accurate than intervals based on
the normal approximation. Owen~(1990) describes ``empirical
likelihood,'' providing a theoretical basis for nonparametric
likelihood ratio confidence regions.  Li~(1995a), using the Thomas and
Grunkemeier~(1975) problem as a starting point, provides theory and an
algorithm for computing nonparametric confidence intervals based on
inverting a likelihood ratio test for $F(\realrv)$. The work is
extended to truncated data in Li~(1995b).

%----------------------------------------------------------------------
%----------------------------------------------------------------------
\section*{Exercises}


%------------------------------------------------------------------------
\begin{exercise}
\label{exercise:bb.ci75}
Use the ball bearing life test data in
Table~\ref{table:lz.bbearing.data} to do the following:
\begin{enumerate}
\item
Compute a nonparametric estimate of the population fraction failing by
75 million cycles.
\item
\label{exercise.part:bb.ci75p1}
Use the conservative interval in
(\ref{equation:binomial.confidence.interval}) to compute an
approximate 90\% confidence interval for the population
fraction failing by 75 million cycles.
\item
\label{exercise.part:bb.ci75p2}
Use the normal-approximation method in
(\ref{equation:nor.approc.binom.ci}) to compute an approximate 90\%
confidence interval for the population
fraction failing by 75 million cycles.
\item
Comparing the intervals from parts~\ref{exercise.part:bb.ci75p1} and
\ref{exercise.part:bb.ci75p2}, what do you conclude about the adequacy of
the normal-approximation method for these data?
\end{enumerate}
\end{exercise}

%------------------------------------------------------------------------
\begin{exercise}
Repeat Exercise~\ref{exercise:bb.ci75}, using the population
fraction failing by 25 million cycles. 
Why does the normal-approximation method not work so
well in this case?
\end{exercise}

%------------------------------------------------------------------------
\begin{exercise}
Show how (\ref{equation:np.normal.theory.ci}) follows from
(\ref{equation:fhat.ci.normal}) and how (\ref{equation:fhat.ci.normal})
follows from (\ref{equation:def.zfhat.normal}).
\end{exercise}


%------------------------------------------------------------------------
\begin{exercise}
\label{exercise:parida.1991}
Parida~(1991) gives data from a load-controlled high-cycle
fatigue test conducted on 130 chain links. The 130 links were randomly
selected from a population of different heats used to manufacture the
links. Each link was tested until failure or until it had run for 80
thousand cycles, whichever came first. There were 10 failures---one
each reported at 33, 46, 50, 59, 62, 71, 74, and 75 thousand cycles
and 2 reported at 78 thousand cycles. The other 120 links had not
failed by 80 thousand cycles.
\begin{enumerate}
\item
Use (\ref{equation:binomial.ecdf}) to compute the nonparametric estimate of
$F(\realrv)$ and corresponding standard errors. 
\item
Compute a set of pointwise approximate 90\% confidence intervals for
$F(\realrv)$.  Explain the proper interpretation of these intervals.
\item
For the first 3 failures, compare the numerical estimates
from (\ref{equation:nonparametric.estimator.for.F}) with the numerical
estimates from (\ref{equation:binomial.ecdf}).
\item
The original paper reported the number of cycles to failure, as given above.
Suggest reasons why the numbers of cycles to failures were not given with more
precision and the effect that this has on the results of the analysis.
\item
The original paper reported that the tested units had been selected
from a random sample of heats. What might have happened in the
experiment if all of the sample links had been selected from just one
or two heats?
\item
The original paper did not report the order in which the tests were
run. Typically fatigue tests require the use of 
one or a few expensive test
stands and tests are done in sequence. The order in which the
failures occurred was not described in the original paper. Is it
possible that there was some useful information in knowing the order
in which the 130 units had been tested? Discuss.
\end{enumerate}
\end{exercise}

%------------------------------------------------------------------------
\begin{exercise}
The supplier of an electromechanical control for a
household appliance ran an accelerated life
test on sample controls. In the test, 25 controls were put on test and run
until failure or until 30 thousand cycles had been been accumulated.
Failures occurred at 5, 21, and 28 thousand cycles. The other 22
controls did not fail by the end of the test.
\begin{enumerate}
\item
Compute and plot a nonparametric estimate for $F(t)$.
\item
\label{exer.part:elmech.ci.binom}
Compute an approximate 95\% confidence interval for the probability
that an electromechanical device from the same production process,
tested in the same way, would fail before 30 thousand cycles. Use the
conservative binomial distribution approach.
\item
\label{exer.part:elmech.ci.norm}
Compute an approximate 95\% confidence interval for the probability
that an electromechanical device from the same production process,
tested in the same way, would fail before 30 thousand cycles. Use the
normal-approximation method based on $Z_{\Fhat(30)} \approxdist
\NOR(0,1)$.
\item
Explain why, in this situation, the approach in
part~\ref{exer.part:elmech.ci.binom} would be preferred to the
approach in part~\ref{exer.part:elmech.ci.norm}.
\item
\label{exerpart:prod.mix}
The appliance manufacturer is really interested in the
probability of the number of days to failure for its product.
Use rate differs from household to household, but the average rate is
2.3 cycles per day. What can the manufacturer say about the proportion
of devices that would fail in 10 years of operation (the expected
technological life of the product)? 
\item
Refer to part~\ref{exerpart:prod.mix}.
Describe an appropriate model to use when use rate varies in the 
population of units. To simplify, start by assuming that there are
only two different use-rates. Discuss, using appropriate expressions.
\end{enumerate}
\end{exercise}

%------------------------------------------------------------------------
\begin{exercise}
\label{exercise:earth.orbit.cdf}
Over the past 18 months, ten separate
copies of an electronic system have been
deployed in earth-orbit where repair is impossible. Continuous remote
monitoring, however, provides information on the state of the system
and each of its main subsystems.  Each system contains 3 nominally
identical devices and it was learned, after deployment, that these
devices are, in the system's environment, failing unexpectedly. The
failures cause degradation to the overall system operation. For future
systems that are to be deployed, the problem will be fixed, but owners
of the systems have asked for information on the amount of degradation
that can be expected in future years of operation among these
currently deployed. To date, 5 of the 30 devices have failed. Due to
the staggered entry of the systems into service, the available data
are multiply censored. The following table summarizes the available
information with times given in hours.  Times of unfailed units are
marked with a ``$+$.''\\[5ex]
\begin{center}
{\centering\small
\begin{tabular}{rrrr}
\\[-.5ex]
System & Device 1 & Device 2 & Device 3\\
\hline
1  &   564\censormark    &   564\censormark        &  564\censormark  \\
2  &   1321\censormark   &   1104\nocensormark     &  1321\censormark \\
3  &   1933\censormark   &   1933\censormark       &  1933\censormark \\
5  &   1965\censormark   &   1965\censormark       &  1965\censormark \\
4  &   2578\censormark   &   2345\nocensormark     &  2578\censormark \\
6  &   3122\censormark   &   3122\censormark       &  3122\censormark \\
7  &   5918\censormark   &   5918\censormark       &  4467\nocensormark   \\
8  &   7912\censormark   &   7912\censormark       &  6623\nocensormark   \\
9  &   8156\censormark   &   8156\censormark       &  8156\censormark \\
10 &   7885\nocensormark &   12229\censormark      &  12229\censormark\\
\hline
\end{tabular}
}
\end{center}
\begin{enumerate}
\item
Compute a nonparametric estimate of  $F(t)$, the life
distribution of the devices, assuming that the devices are operating
and failing independently.
\item
Plot the nonparametric estimate of $F(t)$.
\item
Compute pointwise approximate $95\%$ confidence intervals for $F(t)$
and add these to your plot.
\item
Explain why it might be that the 30 devices are not operating and failing
independently and how this would affect conclusions drawn from the data.
\item
Describe possible reasons why this failure-causing problem was not
discovered earlier and what might be done to minimize the chance of
such problems occurring in the future. What questions would you ask of
a client who wants you to help interpret the information in the
failure data?
\end{enumerate}
\end{exercise}

%------------------------------------------------------------------------
\begin{exercise}
Consider the Plant 1 heat exchanger data in
Figure~\ref{figure:heatex.plcomp1fig.ps}.
\begin{enumerate}
\item
Write the likelihood for these data in terms of $\pi_{1}, \pi_{2},
\pi_{3},$ and $\pi_{4}.$
\item
Write the likelihood for these data in terms of $p_{1}, p_{2}, p_{3},$
and $p_{4}.$
\end{enumerate}
\end{exercise}

%------------------------------------------------------------------------
\begin{exercise1}
For a given $t_{i}$, $i=1,n$, show that the expression in
(\ref{equation:binomial.ecdf}) is the maximum likelihood estimator for
$\Fhat(t_{i}).$
\end{exercise1}

%------------------------------------------------------------------------
\begin{exercise1}
The expression in (\ref{equation:binomial.var.ecdf}) was obtained by
evaluating the square root of $\var[\Fhat (t_{i})]$ at
$\Fhat(t_{i}).$ Show this by deriving the expression for
$\var[\Fhat(t_{i})].$
\end{exercise1}

%------------------------------------------------------------------------
\begin{exercise}
\label{exercise:fhat.sim}
Some computer programs (e.g., statistical packages and spreadsheets)
can be used to generate pseudorandom samples from a uniform
distribution.  Let $U_{1},\ldots,U_{n}$ denote such a sample. Then
$T_{1}=-\log(1-U_{1}),\ldots,T_{n}=-\log(1-U_{n})$ is a pseudorandom
sample from an exponential distribution (to be described in more
detail in Chapter~\ref{chapter:ls.parametric.models}).  Simulate a
sequence of $50$ such samples each of size $n=200.$ For each sample:
\begin{enumerate}
\item
Compute and plot $\Fhat.$
\item
Make a histogram of the $50$ values of $\Fhat(1).$
\item
\label{exer.pt:fhat.sim1}
Make a histogram of the $50$ values of $Z_{\Fhat(1)}.$
\item
\label{exer.pt:fhat.sim2}
Make a histogram of the $50$ values of $Z_{\logit(\Fhat)}.$
\item
Compare the histograms in parts~\ref{exer.pt:fhat.sim1} and
\ref{exer.pt:fhat.sim2}. Which statistic seems to be better
approximated by a $\NOR(0,1)$ distribution?
\end{enumerate}
\end{exercise}

%------------------------------------------------------------------------
\begin{exercise}
Repeat Exercise~\ref{exercise:fhat.sim} using samples of size $n=20$
(and, perhaps other values of $n$).  Compare the plots
with those from
Exercise~\ref{exercise:fhat.sim} and describe how sample
size affects the distribution of statistics like $\Fhat,$ $Z_{\Fhat},$
and $Z_{\logit(\Fhat)}$.
\end{exercise}

%------------------------------------------------------------------------
\begin{exercise}
\label{exercise:weis.et.al.1986}
Weis, Caldararu, Snyder, and Croitoru~(1986) report on the results of
a life test on silicon photodiode detectors in which 28 detectors were
tested at
$85^\circ$C and 40 volts reverse bias. These conditions, which
were more stressful than normal use conditions, 
were used in order to get failures
quickly. Specified electrical tests were made at
0, 10, 25, 75, 100, 500, 750, 1000, 1500, 2000, 2500, 3000, 3600,
3700, and 3800 hours to determine if the detectors were still
performing properly. Failures were found after the inspections at
2500 (1~failure), 3000 (1~failure), 3500 (2~failures), 3600 
(1~failure), 3700 (1~failure), and 3800 (1~failure). The other 21
detectors had not failed after 3800 hours of operation.
Use these data to estimate the life distribution of 
such photodiode detectors running at the test conditions.
\begin{enumerate}
\item
From the description given above, the data would be useful
for making inferences about what particular populations or process?
Explain your reasoning.
\item
\label{exercise.part:photo.diode.npest}
Compute and plot a nonparametric estimate of the cdf for time
to failure at the test conditions.
\item
Compute standard errors for the nonparametric estimate in part
\ref{exercise.part:photo.diode.npest}.
\item
Compute a set of pointwise approximate 95\% confidence intervals for
$F(\realrv)$ and add these to your plot.
\item
Compute simultaneous approximate 95\% confidence bands for
$F(\realrv)$ over the complete range of observation.
\item
Provide a careful explanation of the differences in interpretation
and application of the pointwise confidence intervals and the
simultaneous confidence bands.
\end{enumerate}
\end{exercise}

%------------------------------------------------------------------------
\begin{exercise1}
Use the delta method (Appendix Section~\ref{asection:delta.method})
and the assumptions given in Section~\ref{section:avar.nonpar.fhat}
to derive (\ref{equation:pl.asym.var}).
\end{exercise1}

%------------------------------------------------------------------------
\begin{exercise1}
Show how (\ref{equation:fhat.logit.ci}) follows from
(\ref{equation:fhat.logit.stand.z}).  Begin by using the delta method
(e.g., Appendix Section~\ref{asection:delta.method}) to obtain an expression
for $\sehat_{\logit(\Fhat)}$ as a function of
$\sehat_{\Fhat}$.
\end{exercise1}

%------------------------------------------------------------------------
\begin{exercise}
Example~\ref{example:shock.absorber.cdfest} illustrates the
computations for the nonparametric estimation of the cdf for the shock
absorber data up to 12,220~km. Complete the computations for the rest
of the data (i.e., out to 28,100~km). Use $S(12200)=.9086984$ to continue
the cumulative product in (\ref{equation:nonparametric.estimator}).
\begin{enumerate}
\item
\label{exercise.part:npshock.plotcdf}
Plot the nonparametric estimate out to
28,100 km.
\item
Compute $\sehat_{\Fhat}$ out to
28,100 km.
\item
Compute a set of pointwise approximate 90\% confidence intervals for
$F(\realrv)$ out to 28,100~km and add these to the plot in part
\ref{exercise.part:npshock.plotcdf}.
\item
Explain why, with right-censored data, for the nonparametric
estimation method, there is only a limited range of time over which we
can estimate $F(\realrv)$.
\end{enumerate}
\end{exercise}


%------------------------------------------------------------------------
\begin{exercise1}
Show that with single censoring (i.e., all failures precede the first
censoring time) that
(\ref{equation:nonparametric.estimator.for.F}) simplifies to
(\ref{equation:binomial.ecdf}) and that (\ref{equation:greenwood})
simplifies to (\ref{equation:binomial.var.ecdf}).
\end{exercise1}


%------------------------------------------------------------------------
\begin{exercise}
Example~\ref{example:npe.heat.exchanger} illustrated the computations
for the nonparametric and approximate confidence intervals for $F(1)$ 
and $F(2)$ for the pooled
heat exchanger tube data. Complete similar computations for $F(3)$.
\end{exercise}

%------------------------------------------------------------------------
\begin{exercise}
Explain why the nonparametric estimate of $F(t)$ is a set of points
for the heat exchanger data in
Example~\ref{example:npe.heat.exchanger} but a step function for the
shock absorber data in Example~\ref{example:shock.absorber.cdfest}.
\end{exercise}

%------------------------------------------------------------------------
\begin{exercise}
Use the data in Table~\ref{table:v7.transmitter.tube} to do the
following for the V7 Transmitter tube:
\begin{enumerate}
\item
Compute an estimate of the conditional probability of failing
for each cell.
\item
Compute a
nonparametric estimate for $F(\realrv)$ for each cell. 
\item
Plot the estimate of $F(\realrv)$ 
along with a set of pointwise approximate 95\% confidence
intervals.
\item
For each interval,
compute and plot an estimate of the probability of failing in that
interval given a unit enters the interval.
\item
Explain how these estimates could be used to plan for preventive
maintenance
for a group of radio transmitters, each with one such tube.
\end{enumerate}
\end{exercise}

%------------------------------------------------------------------------
\begin{exercise}
\label{exercise:fan.km}
Use the diesel engine fan data in Appendix
Table~\ref{atable:fan.data} to compute the product-limit nonparametric
estimate of $F(\realrv)$ using (\ref{equation:nonparametric.estimator}).
\end{exercise}


%------------------------------------------------------------------------
\begin{exercise}
Example~\ref{example:proto.battery.npmle} illustrated the
computations for the nonparametric estimate of the 
cdf for prototype batteries with both failure times and
censored observations within the given intervals.
Complete the computations for the rest
of the data in Appendix Table~\ref{atable:morgan.battery.data} (i.e., out to
1700 hours).
Use $\Shat(600)=.51608$ to continue the cumulative product 
in (\ref{equation:nonparametric.estimator}).
\begin{enumerate}
\item
\label{exercise.part:proto.batt.plotcdf}
Plot the nonparametric estimate out to 1700 hours.
\item
Compute $\sehat_{\Fhat}$ out to 1700 hours.
\item
Compute a set of approximate pointwise 90\% confidence intervals for
$F(\realrv)$ out to 1700 hours and add these to the plot in part
\ref{exercise.part:proto.batt.plotcdf}.
\end{enumerate}
\end{exercise}


%------------------------------------------------------------------------
\begin{exercise1}
\label{exercise:km.ml}
Consider the model in Section~\ref{section:multinomial.failure.model}
and the data collection method described in
Section~\ref{section:def.model.free.est}.  
\begin{enumerate}
\item Show that the likelihood of the data, as a function of the
parameters, is
\begin{eqnarray*}
\like(\underline{\pi})
&=& \pi_{1}^{\deadin_{1}} \times \pi_{2}^{\deadin_{2}} \times
\cdots  \times
\pi_{m}^{\deadin_{m}} \times  \\
& & 
\left [S(\realrv_{1}) \right ]^{r_{1}} \times
\left [S(\realrv_{2}) \right ]^{r_{2}} \times
\cdots
\times
\left [S(\realrv_{m}) \right ]^{r_{m}}.
\end{eqnarray*}
\item 
Show that in
terms of the parameters $\pvec=(p_{1}, \ldots, p_{m})$,
\begin{eqnarray*}
\like(\pvec)
&=&\prod_{j=1}^{m} p_{j}^{\deadin_{j}} (1-p_{j})^{n_{j}-\deadin_{j}}
\end{eqnarray*}
where $n_{j}=n-\sum_{i=0}^{j-1} \deadin_{i} - \sum_{i=0}^{j-1} r_{i}$,
with the understanding that $d_{0}=0$ and $r_{0}=0$.
\item 
Show that the maximum likelihood estimators of the
parameters are
\begin{displaymath}
 \phat_{j}=\frac{d_{j}}{n_{j}}, \,\,\, j=1, \ldots, m.
\end{displaymath}
\item 
Show that the observed information matrix for the parameters $\pvec$
is diagonal and that the $i$th diagonal
element of the matrix is equal to 	
\begin{displaymath}  
- \, \frac{\partial^{2} \log  [ \like(\pvec) ]}
{\partial p_{i}^{2}} = \frac{n_{i}}{p_{i} (1-p_{i})}
\end{displaymath} 
evaluated at $\pvechat$.  This shows that, asymptotically (in large
samples), the components of $\pvechat$ are uncorrelated and
$\var(\phat_{i})=p_{i}(1-p_{i})/n_{i}$. Use these results and the
delta method to derive Greenwood's formula as given in
(\ref{equation:greenwood}).
\end{enumerate}
\end{exercise1}


%------------------------------------------------------------------------
\begin{exercise1}
\label{exercise:nelson.aalen.estimator}
Consider the relationship 
$S(\realrv_{i})=\exp[-H(\realrv_{i})]$, 
where $H(\realrv)$ is the cumulative hazard function. Note
that a nonparametric
ML estimator (based on the product-limit  estimator)
of $H(\realrv)$ without assuming a distributional form is 
\begin{displaymath}
    \Hhat(\realrv_{i}) = -\sum_{j=1}^{i}\log(1-\phat_{j}) 
	                          \approx \sum_{j=1}^{i}\phat_{j}
                                  =\sum_{j=1}^{i}\frac{d_{j}}{n_{j}}
=\Hhathat(\realrv_{i}).
\end{displaymath}
$\Hhathat(\realrv_{i})$ is known as the Nelson-Aalen
estimator of $H(t_{i})$. Thus $\Fhathat(\realrv_{i})=
1-\exp[-\Hhathat(\realrv_{i})]$ is another nonparametric
estimator for $F(t)$.
\begin{enumerate}
\item 
Give conditions to assure a good agreement between
$\Hhat(\realrv_{i})$ and $\Hhathat(\realrv_{i})$ and thus
between $\Fhat(\realrv)$ and $\Fhathat(\realrv)$.
\item 
Use the delta method to compute approximate expressions for  
$\var \left [\Hhat(\realrv_{i})\right ]$
and  $\var \left [\Hhathat(\realrv_{i})\right ]$.
Comment on the expression(s) you get. 	
\item
Compute Nelson-Aalen estimate of $F(\realrv)$ and compare with the
estimate computed in Exercise~\ref{exercise:fan.km}.  Describe
similarities and differences.  \item Show that
$\Hhathat(\realrv_{i})<\Hhat(\realrv_{i})$ and that
$\Fhathat(\realrv_{i}) <
\Fhat(\realrv_{i})$. 
\item
Describe suitable modifications of the estimator that can be used when
failure and censoring times are grouped into common intervals.
\end{enumerate} 
\end{exercise1}

