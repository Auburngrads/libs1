%chapter 18
%original by wqmeeker  12 Jan 94
%edited by wqmeeker 9 Apr 94
%edited by wqmeeker 26 Apr 94
%edited by driker 22 May 95
%edited by driker 11 Nov 95
%edited by wqmeeker 16 dec 95
%edited by wqmeeker 8-11 jan 96 new models and structure
%edited by driker 12 jan 96
%edited by wqmeeker 14 jan 96 smoothing
%edited by driker 1 july 96
%edited by driker 6 oct 96
%edited by wqmeeker 16 nov 96

\setcounter{chapter}{17}

\chapter{Accelerated Test Models}

\label{chapter:accelerated.test.models}


%{\Large {\bf William Q. Meeker and Luis A. Escobar}}\\
%Iowa State University and Louisiana State University\\[5ex]
{\large {\bf \today}}\\[2ex]
Part of 
{\em Statistical Methods for Reliability Data}\\
Copyright 1997 W. Q. Meeker and L. A. Escobar. \\[1ex]
To be published by John Wiley \& Sons Inc. in 1998.



%----------------------------------------------------------------------
%----------------------------------------------------------------------
\section*{Objectives}
This chapter explains:
\begin{itemize}
\item
Motivation and applications of accelerated testing.
\item 
Connections between degradation and physical failure.
\item 
Models for temperature acceleration.
\item
Models for voltage and pressure acceleration.
\item
How to compute time-acceleration factors.
\item
Other accelerated test models and their assumptions.
\end{itemize} 

%---------------------------------------------------------------------
%----------------------------------------------------------------------
\section*{Overview}
This chapter describes models used for accelerated 
tests and introduces concepts of physics of failure.
Some aspects of the models introduced here follow from the degradation
models introduced in Chapter~\ref{chapter:degradation.data}.  The
acceleration models described here are fitted to data in
Chapter~\ref{chapter:analyzing.alt.data} (accelerated life tests)
and Chapter~\ref{chapter:accelerated.degradation} (accelerated
degradation tests).  Section~\ref{section:intro.at.models} 
motivates and describes the general methods for
accelerating reliability tests.  Sections~\ref{section:use.rate.acc},
\ref{section:temp.acc}, and \ref{section:voltage.at.models}
describe, respectively, use-rate, temperature, and voltage
acceleration. Section~\ref{section:multi.factor.at.models}
describes some models with a combination of accelerating variables.

%----------------------------------------------------------------------
%----------------------------------------------------------------------
\section{Introduction}
\label{section:intro.at.models}
%----------------------------------------------------------------------
\subsection{Motivation}
Today's manufacturers face strong pressure to develop new, higher
technology products in record time, while improving productivity,
product field reliability, and overall quality.  This has motivated
the development of methods like concurrent engineering and
encouraged wider use of designed experiments for product and process
improvement.  The requirements for higher reliability have increased
the need for more {\em up-front} testing of materials, components,
and systems. This is in line with the modern quality philosophy for
producing high-reliability products: achieve high reliability by
improving the design and manufacturing processes; move away from
reliance on inspection (or screening) to achieve high reliability.

Estimating the failure-time distribution or long-term performance of
components of {\em high-reliability} products is particularly
difficult.  Most modern products are designed to operate without
failure for years, decades, or longer. Thus few units will fail or
degrade appreciably in a test of practical length at normal use
conditions. For example, the design and construction of a
communications satellite, may allow only 8 months to test components
that are expected to be in service for 10 or 15 years.  For such
applications, Accelerated Tests (ATs) are used widely in
manufacturing industries, particularly to obtain timely information
on the reliability of simple components and materials.  There are
difficult practical and statistical issues involved in accelerating
the life of a complicated product that can fail in different ways.
Generally, information from tests at high levels of one or more
accelerating variables (e.g., use rate, temperature, voltage, or
pressure) is extrapolated, through a physically reasonable
statistical model, to obtain estimates of life or long-term
performance at lower, normal levels of the accelerating variable(s).
In some cases, stress is increased or otherwise changed during the
course of a test (step-stress and progressive-stress ATs).  AT
results are used in the reliability-design process to assess or
demonstrate component and subsystem reliability, to certify
components, to detect failure modes so that they can be corrected,
compare different manufacturers, and so forth.  ATs have become
increasingly important because of rapidly changing technologies,
more complicated products with more components, higher customer
expectations for better reliability, and the need for rapid product
development.

%----------------------------------------------------------------------
\subsection{Different types of acceleration}
The term ``acceleration'' has many different meanings within the
field of reliability, but the term generally implies making ``time''
(on whatever scale is used to measure device or component life)
go more quickly, so that reliability information can be obtained more rapidly.
Different types of reliability tests and screens 
are used in different phases of
the product development/production processes.
Table~\ref{table:different.ats} outlines the kinds of reliability
tests done at different stages of product design and production.
Generally, there is a need to do all of these tests as quickly as
possible, and methods have been developed to accelerate all of
these different types of tests and screens.
\begin{table}
\caption{Reliability tests at different product stages.}
\centering\small
\begin{tabular} {cccc}
\\[-.5ex]
\multicolumn{2}{c}{Product Design} & & Product Production\\[.5ex]
\cline{1-2} \cline{4-4}
&&& Production and \\
Qualification Testing of & Prototype Testing of && Post-Production
Screens for \\ 
Materials and Components & Systems and Subsystems && Systems and Subsystems \\[.2ex] 
\hline 
\\[-.5ex]
Accelerated Degradation Tests& Robust Design Test &&  Component Certification \\[1ex] 
Accelerated Life Tests  & STRIFE Tests & &   Burn-in	\\ [1ex]
	& Test-and-Fix && Environmental  \\
		& for Reliability Growth &&		Stress Screening\\ 
\hline
\end{tabular} \\ 
\begin{minipage}[t]{4in}
\end{minipage}
\label{table:different.ats}
\end{table}
The main focus of this chapter and Chapters~\ref{chapter:analyzing.alt.data},
\ref{chapter:alt.test.planning}, and \ref{chapter:accelerated.degradation},
and Section~\ref{section:planning.accelerated.degradation.tests}
will be Accelerated Life Tests and Accelerated Degradation Tests
that are done during product design to assess reliability and
qualify the use of proposed materials and components.  A section at
the end of Chapter~\ref{chapter:analyzing.alt.data} provides
definitions and further discussion of the statistical aspects of the
other types of acceleration as well as references to sources of
further information .
%----------------------------------------------------------------------
\subsection{Types of responses}
It is useful to distinguish between ATs on the basis of what is
observed.

\begin{itemize}
\item
{\bf Accelerated Life Tests (ALTs):} One obtains information on the
failure time (actual failure time or an interval containing the
failure time) for units that fail and lower bounds for the failure
time (also known as a running times or runout time) for units that
do not fail.
\item
{\bf Accelerated Degradation Tests (ADTs):} As described in
Chapter~\ref{chapter:degradation.data}, one observes, at one or more
points in time, the amount of degradation for a unit (perhaps with
measurement error).
\end{itemize}
Many of the underlying physical model assumptions, concepts, and
practices are the same for ALTs and ADTs.  In some cases, analysts
use degradation-level data to define failure times, turning ADT data
into ALT data (generally simplifying analysis, but sacrificing
useful information).  There are close relationships between ALT and
ADT models.  Because of the different types of response, however,
the actual models fitted to the data and methods of analysis
differ. Analyses of ALT and ADT data are covered in
Chapters~\ref{chapter:analyzing.alt.data} and
\ref{chapter:accelerated.degradation}, respectively.
An important characteristic of both types of ATs is the need to
extrapolate outside the range of available data: tests are done at
accelerated conditions, but estimates are needed at use conditions.
Such extrapolation requires strong model assumptions.

%----------------------------------------------------------------------
\subsection{Methods of acceleration}
\label{section:methods.of.acceleration}
There are three different methods of
accelerating a reliability test:
\begin{itemize}
\item
Increase the use-rate of the product.  Consider the reliability of a
toaster, which is designed for a median lifetime of 20 years, assuming
a usage rate of twice each day.  If, instead, we  test the
toaster 365 times each day, we could reduce the median lifetime to
about 40 days.  Also, because it is not necessary to have all units
fail in a life test, useful reliability information  could be
obtained in a matter of days instead of months.
\item
Increase the aging-rate of the product. For example, increasing the
level of experimental variables like temperature or humidity can
accelerate the chemical processes of certain failure mechanism such as
chemical degradation (resulting in eventual weakening and failure) 
of an adhesive
mechanical bond or the growth of a conducting filament across an
insulator (eventually causing a short circuit).  
\item
Increase the level of stress (e.g., temperature cycling,
voltage, or pressure) under which
test units operate. A unit will fail when its {\em strength} drops
below applied stress.  Thus a unit at a high stress will generally
fail more rapidly than it would have failed at low stress.
\end{itemize}
Combinations of these methods of acceleration are also employed.
Variables like voltage and temperature cycling can both increase the
rate of an electro-chemical reaction (thus accelerating the aging
rate) and increase stress relative to strength. In such situations,
when the effect of an accelerating variable is complicated, there
may not be enough physical knowledge to provide an adequate physical
model for acceleration (and extrapolation).  Empirical models may or
may not be useful for extrapolation to use conditions.

%----------------------------------------------------------------------
\subsection{Acceleration models}

Interpretation of accelerated test data requires models that relate
accelerating variables like temperature, voltage, pressure, size,
etc.\  to time acceleration. For testing over some range of
accelerating variables, one can fit a model to the data to describe
the effect that the variables have on the failure-causing processes.
The general idea is to test at high levels of the accelerating
variable(s) to speed up failure processes and then to extrapolate to
lower levels of the accelerating variable(s). For some situations, a
physically reasonable statistical model may allow such
extrapolation.

{\bf Physical acceleration models.} For well-understood failure mechanisms,
one may have a model based on physical/chemical theory that describes
the failure-causing process over the range of the data and provides
extrapolation to use conditions. The relationship between accelerating
variables and the actual failure mechanism is usually extremely
complicated. Often, however, one has a simple model that adequately
describes the process.  For example, failure may result from a
complicated chemical process with many steps, but there may be one
rate-limiting (or dominant) step and a good understanding of this part
of the process may provide a model that is adequate for extrapolation.


{\bf Empirical acceleration models.} When there is little
understanding of the chemical or physical processes leading to
failure, it may be impossible to develop a model based on
physical/chemical theory.  An empirical model may be the only
alternative. An empirical model may provide an excellent fit to the
available data, but provide nonsense extrapolations (e.g., the
quadratic models used in Section~\ref{section:quad.regr}).  In some
situations there may be extensive empirical experience with
particular combinations of variables and failure mechanisms and this
experience may provide the needed justification for extrapolation to
use conditions.  In the rest of this chapter we will describe some
simple acceleration models that have been useful in specific
applications.

%----------------------------------------------------------------------
%----------------------------------------------------------------------
\section{Use-Rate Acceleration}
\label{section:use.rate.acc}
Increasing use-rate will, for some components, accelerate failure-causing
wear and degradation. Examples include:
\begin{itemize}
\item
Running automobile engines, appliances, and similar products
continuously or with higher than usual use rates.
\item
Higher than usual cycling rates for relays and switches.
\item
Increasing the cycling rate (frequency) in fatigue testing.
\end{itemize}

There is a basic assumption underlying simple use-rate acceleration models.
Useful life must be adequately modeled by cycles of operation and
cycling rate (or frequency) should not affect the cycles-to-failure
distribution. This is
reasonable if cycling simulates actual use and if the cycling
frequency is low enough that test units return to steady state after
each cycle (e.g., cool down).

\begin{example}
{\bf Increased cycling rate for low-cycle fatigue tests.}
Fatigue life is typically measured in cycles to failure.  To estimate
low-cycle fatigue life of metal specimens, testing is done using cycling
rates typically ranging between 10 Hz and 50 Hz (where 1 Hz is one
stress cycle per second), depending on material type and available
test equipment. At 50 Hz,
accumulation of $10^{6}$ cycles would require about 5 hours of testing.
Accumulation of  $10^{7}$ cycles would require about 2 days and
$10^{8}$ about 20 days. Higher frequencies are used in the study
of high-cycle fatigue.
\end{example}
Testing at higher frequencies could shorten test times but could also
affect the cycles-to-failure distribution due to specimen heating or
other effects. In some complicated
situations, wear rate or degradation rate depends on cycling frequency.
Also, a product may deteriorate in stand-by as well as during actual use.

For a certain type of fatigue test, notched test specimens are
subjected to cyclic loading, as shown in
Figure~\ref{figure:fatigue.test.setupfig.ps}. Because the notch is a point of
highest stress, a crack will initiate and grow out of it.
Cycling rates in such tests are generally increased to a point where
crack growth or cycles to failure (two common responses)
can still be measured without
distortion. There is a danger, however, that increased temperature 
due to cycling rate will affect crack growth.  This
is especially true if there are effects like creep-fatigue interaction
(see Dowling 1993, page 706, for further discussion).  In another
example, there was concern that changes in cycling rate would affect
the distribution of lubricant on a rolling bearing surface.
%-------------------------------------------------------------------
\begin{figure}
\xfigbookfiguresize{\figurehome/fatigue.test.setupfig.ps}{1in}
\caption{Fatigue test notched ``compact'' specimen to which cyclic stress will be applied.}
\label{figure:fatigue.test.setupfig.ps}
\end{figure}
%-------------------------------------------------------------------

%-------------------------------------------------------------------
\section{Temperature Acceleration}
\label{section:temp.acc}
It is sometimes said that high temperature is the enemy of
reliability.
Increasing temperature is one of the most commonly used methods to
accelerate a failure mechanism.  
\begin{example}
\label{example:resistance.degradation.data}
{\bf Resistance change of carbon-film resistors.} Appendix
Table~\ref{atable:resistance.degradation.data} and
Figure~\ref{figure:resistance.degradation.data} shows the percent
increase in resistance over time for a sample of carbon-film
resistors.  These data were previously analyzed by Shiomi and
Yanagisawa~(1979) and Suzuki, Maki, and Yokogawa~(1993).  Samples of
resistors were tested at each of three levels of temperature. At
standard operating temperature (e.g., $50 \degreesc$), carbon-film
resistors will slowly degrade. Changes in resistance can cause 
reduced product performance or even cause system failures.  The
test was run at high levels of temperature to accelerate
the chemical degradation process and obtain degradation data more
quickly.  Figure~\ref{figure:resistance.degradation.data} shows that
the resistors degrade more rapidly at high temperature.
%-------------------------------------------------------------------
\begin{figure}
\splusbookfigure{\figurehome/resistance.degradation.data.ps} 
\caption{Percent increase in resistance over time for a sample of carbon-film resistors.}
\label{figure:resistance.degradation.data}
\end{figure}
%-------------------------------------------------------------------
\end{example}


\subsection{Arrhenius relationship time-acceleration factor}
\label{section:arrhenius.af}
The Arrhenius relationship is a widely-used model describing the effect
that temperature has on the rate of a simple chemical
reaction.  This relationship can be written as
\begin{equation}
\label{equation:arrhenius.rate}
\Rate(\Temp)=\gamma_{0} \exp\left(
\frac{-\Ea}{\Boltzmann\times
\Tempk{}} \right)
=
\gamma_{0}\exp\left(
\frac{-\Ea \times 11605}{\Tempk{}} \right)
\end{equation}
where $\Rate$ is the reaction rate and $\Tempk{}= \Tempc{} + 273.15$
is temperature in the absolute Kelvin scale, $\Boltzmann = 8.6171
\times 10^{-5} = 1/11605$ is Boltzmann's constant in electron volts
per $\degreesc$, and $\Ea$ is the activation energy in electron
volts (eV).  The parameters $\Ea$ and $\gamma_{0}$ are product or
material characteristics.  The Arrhenius acceleration
factor is
\begin{equation}
\label{equation:arrhenius.af}
\AF(\Temp,\Temp_{U},\Ea) = \frac{\Rate(\Temp)}{\Rate(\Temp_{U})} =  \exp\left[
\Ea\left( \frac{11605}{\Tempk{U}} -  \frac{11605}{\Tempk{}}
 \right) \right].
\end{equation}
When $\Temp > \Temp_{U}$, $\AF(\Temp,\Temp_{U},\Ea) > 1$.  When
$\Temp_{U}$ and $\Ea$ are understood to be, respectively, product use
temperature and reaction-specific activation energy,
$\AF(\Temp)=\AF(\Temp,\Temp_{U},\Ea)$ will be used to denote a
time-acceleration factor.  Figure~\ref{figure:arrhenius.af.ps} gives the
acceleration factor
\begin{displaymath}
\AF(\Temp_{\High},\Temp_{\Low},\Ea) =  \exp\left(
\Ea \times \mbox{TDF}  \right)
\end{displaymath} 
as a function of $\Ea$ and the Temperature Differential Factor (TDF) values
\begin{displaymath}
\mbox{TDF}=
	\left(\frac{11605}
		   {\Tempk{\Low}}
	   -  \frac{11605}
		   {\Tempk{\High}}
	\right )
\end{displaymath}
given in Table~\ref{table:arrhenius.temp.diff}. 


\begin{table}
\caption{Temperature Differential Factors (TDFs) from the Arrhenius 
time-acceleration model.}
\centering\small
\begin{tabular} { crrrrrrrrr}
\\[-.5ex]
\multicolumn{1}{c}{Higher}\\
\multicolumn{1}{c}{Temperature}& \multicolumn{8}{c}{Lower Temperature in
$\degreesc$}\\
\cline{3-10}
$\degreesc$       & & 30 & 40 & 50 & 60 & 70 & 80 & 90 & 100 \\ 
\cline{3-10}\\
 80 &&   5.42 &  4.20 &  3.05  & 1.97 &  0.96 &  0.00 &       &   \\  
 85 &&   5.88 &  4.66 &  3.51 &  2.43 &  1.42 &  0.46 &       &    \\  
 90 &&   6.32 &  5.10 &  3.96 &  2.88 &  1.86 &  0.90 &  0.00 &   \\   
 95 &&   6.76 &  5.54 &  4.39 &  3.31 &  2.30 &  1.34 &  0.43 &    \\  
100 &&   7.18 &  5.96 &  4.81 &  3.73 &  2.72 &  1.76 &  0.86 &  0.00 \\[.5ex] 
105 &&   7.59 &  6.37 &  5.22 &  4.14 &  3.13 &  2.17 &  1.27 &  0.41 \\ 
110 &&   7.99 &  6.77 &  5.62 &  4.55 &  3.53 &  2.57 &  1.67 &  0.81 \\ 
115 &&   8.38 &  7.16 &  6.01 &  4.94 &  3.92 &  2.96 &  2.06 &  1.20 \\ 
120 &&   8.76 &  7.54 &  6.39 &  5.32 &  4.30 &  3.34 &  2.44 &  1.58 \\ 
125 &&   9.13 &  7.91&   6.76 &  5.69 &  4.67 &  3.71 &  2.81 &  1.95 \\[.5ex] 
130 &&   9.49 &  8.27 &  7.13 &  6.05 &  5.03 &  4.08 &  3.17 &  2.31 \\ 
135 &&   9.85 &  8.63 &  7.48 &  6.40 &  5.39 &  4.43 &  3.52 &  2.67 \\ 
140 &&  10.19 &  8.97 &  7.82 &  6.74 &  5.73 &  4.77 &  3.87 &  3.01 \\ 
145 &&  10.53 &  9.31 &  8.16 &  7.08 &  6.07 &  5.11 &  4.20 &  3.35 \\ 
150 &&  10.86 &  9.63 &  8.49 &  7.41 &  6.39 &  5.44 &  4.53 &  3.67 \\[.5ex] 
155 &&  11.18 &  9.95 &  8.81 &  7.73 &  6.71 &  5.76 &  4.85 &  3.99 \\ 
160 &&  11.49&  10.27 &  9.12 &  8.04 &  7.03 &  6.07 &  5.16 &  4.31 \\ 
165 &&  11.79 & 10.57 &  9.43 &  8.35 &  7.33 &  6.37 &  5.47 &  4.61 \\ 
170 &&  12.09 & 10.87&   9.72 &  8.65 &  7.63 &  6.67 &  5.77 &  4.91 \\ 
175 &&  12.39 & 11.16&  10.02 &  8.94 &  7.92 &  6.97 &  6.06 &  5.20 \\ [.5ex]
180 &&  12.67 & 11.45 & 10.30 &  9.22 &  8.21 &  7.25 &  6.35&   5.49 \\ 
185 &&  12.95&  11.73 & 10.58 &  9.50 &  8.49 &  7.53 &  6.63&   5.77 \\ 
190 &&  13.22 & 12.00&  10.85 &  9.78 &  8.76 &  7.80 &  6.90 &  6.04 \\ 
195 &&  13.49&  12.27 & 11.12 & 10.04 &  9.03 &  8.07 &  7.17 &  6.31 \\ 
200 &&  13.75 & 12.53 & 11.38 & 10.31 &  9.29 &  8.33 &  7.43&   6.57 \\ [.5ex]
205&&   14.01&  12.79&  11.64 & 10.56 &  9.55 &  8.59 &  7.69&   6.83 \\ 
210&&   14.26 & 13.04 & 11.89 & 10.81 &  9.80 &  8.84 &  7.94 &  7.08 \\ 
215&&   14.51 & 13.28 & 12.14 & 11.06&  10.04 &  9.09 &  8.18 &  7.33 \\ 
220 &&  14.75 & 13.53 & 12.38 & 11.30 & 10.29 &  9.33 &  8.42 &  7.57 \\ 
225 &&  14.98&  13.76 & 12.62 & 11.54 & 10.52 &  9.56 &  8.66 &  7.80 \\[.5ex] 
230 &&  15.22 & 13.99 & 12.85 & 11.77&  10.75 &  9.80 &  8.89 &  8.03 \\ 
235 &&  15.44&  14.22 & 13.07 & 12.00 & 10.98 & 10.02 &  9.12&   8.26 \\ 
240 &&  15.67 & 14.44 & 13.30 & 12.22&  11.20 & 10.25 &  9.34 &  8.48 \\ 
245 &&  15.88 & 14.66 & 13.51 & 12.44 & 11.42 & 10.46 &  9.56&   8.70 \\ 
250 &&  16.10&  14.88 & 13.73 & 12.65 & 11.64&  10.68 &  9.77 &  8.92 \\ 
 \hline 
\end{tabular} \\ 
\begin{minipage}[t]{4in}
{ $\mbox{TDF}=\left(\frac{11605}
		   {\Tempk{\Low}}
	   -  \frac{11605}
		   {\Tempk{\High}}
	\right )$ used as input to Figure~\ref{figure:arrhenius.af.ps}.}
\end{minipage}
\label{table:arrhenius.temp.diff}
\end{table}
%----------------------------------------------------------------------
\begin{figure}
\splusfigureportraitsize{\figurehome/arrhenius.af.ps}{5in}
\caption{Time-Acceleration Factor as a function of Temperature
Differential Factor from Table~\protect\ref{table:arrhenius.temp.diff}
and activation energy $\Ea$.}
\label{figure:arrhenius.af.ps}
\end{figure}
%----------------------------------------------------------------------
The Arrhenius relationship does not apply to all temperature
acceleration problems and will be adequate over only a limited
temperature range (depending on the particular application). Yet it is
satisfactorily and widely used in many applications. Nelson~(1990a, page 76)
comments that `` \dots in certain applications
(e.g., motor insulation), if the Arrhenius relationship \dots does
not fit the data, the data are suspect rather than the relationship.''

\begin{example}{\bf Arrhenius time-acceleration factor for a metallization 
failure mode.}
\label{example:arrhenius.af}
An accelerated life test will be used to study a metallization
failure mechanism for a transistor. Experience with this type of failure
mechanism suggests that the activation energy should be in the
neighborhood of $\Ea=1.2$. The usual operating junction temperature
for the transistor is 90$\degreesc$. To determine the acceleration
factor for testing at 160$\degreesc$, enter
Table~\ref{table:arrhenius.temp.diff} with these temperatures and
read TDF=5.16. Then enter Figure~\ref{figure:arrhenius.af.ps} with
this figure on the bottom and read up to the line with $\Ea=1.2$ eV,
giving an acceleration factor of approximately $4.9 \times 10^{2}$
(or computed more precisely using (\ref{equation:arrhenius.af}) as
491).
\end{example}
%splus af(160,90,1.2,"arrhenius3") 491.3586

\subsection{Eyring relationship time-acceleration factor}
\label{section:eyring.af}
The Arrhenius relationship (\ref{equation:arrhenius.rate}) was
obtained through empirical observation.  Eyring (e.g., Eyring,
Gladstones, and Laidler~1941 or Eyring~1980) gives physical theory
describing the effect that temperature
has on a reaction rate.  Written in terms of a reaction rate,
\begin{displaymath}
\Rate(\Temp)=\gamma_{0} \times A(\Temp) \times \exp\left(
\frac{-\Ea}{\Boltzmann\times
\Tempk{}} \right)
\end{displaymath}
where $A(\Temp)$ is a function of temperature depending on the
specifics of the reaction dynamics and $\gamma_{0}$ and $\Ea$ are
again constants (Weston and Schwarz~1972, for example, provides more
detail). Applications in the literature have typically used
$A(\Temp)= (\Tempk{})^{m}$ with a fixed value of $m$ ranging between
$m=0$ (Boccaletti et al.~1989, page 379), $m=.5$ (Klinger~1991a), to
$m=1$ (Nelson~1990a, page 100 and Mann Schafer and
Singpurwalla~1974, page 436).

The Eyring relationship temperature acceleration factor is
\begin{displaymath}
\label{equation:eyring.af}
\AF_{\rm Ey}(\Temp,\Temp_{U},\Ea) =  
\left(\frac{\Tempk{}}{\Tempk{U}}  \right)^{m} 
\times \AF_{\rm Ar}(\Temp,\Temp_{U},\Ea)
\end{displaymath}
where $\AF_{\rm Ar}(\Temp,\Temp_{U},\Ea)$ is the Arrhenius
acceleration factor from (\ref{equation:arrhenius.af}).  For use over
practical ranges of temperature acceleration, and when $m$ is close to 0,
the factor outside the exponential has relatively little effect on
the acceleration factor and the additional term is often dropped in
favor the simpler Arrhenius relationship.

\begin{example}
{\bf Eyring acceleration factor for a metallization failure mode.}
\label{example:eyring.af}
Returning to Example~\ref{example:arrhenius.af}, the Eyring
acceleration factor, using $m=1$, is 
\begin{displaymath}
\AF_{\rm
Ey}(160,90,1.2)= \left(\frac{160+273.15}{90+273.15}\right)\times
\AF_{\rm Ar}(160,90,1.2) =
1.1935 \times 491 = 586
\end{displaymath}
where $\AF_{\rm Ar}(160,90,1.2)=491$ from
Example~\ref{example:arrhenius.af}. We see that, for a {\em fixed}
value of $\Ea$, the Eyring relationship predicts, in this case, 
an acceleration that
is 19\% greater than the Arrhenius relationship. As explained below,
however, this figure exaggerates the practical difference
between these models.
\end{example}
%splus ((160+273.15)/(90+273.15))*af(160,90,1.2,"arrhenius3") 586.

When fitting models to limited data, the estimate of $\Ea$ depends
strongly on the assumed value for $m$ (e.g., 0 or 1). This
dependency will compensate for and reduce the effect of changing the
assumed value of $m$. Only with extremely large amounts of data
would it be possible to adequately separate the effects of $m$ and
$\Ea$ using data alone. If $m$ can be determined accurately on the
basis of physical considerations, the Eyring relationship could lead
to better low-stress extrapolations. With $m>0$ the Eyring gives a
larger acceleration factor.  One argument in favor of the Arrhenius
relationship (and perhaps a reason for its more common use) is that
extrapolation to use-levels of temperature will be more conservative
(i.e., predicting shorter life) than with the Eyring relationship
with $m>0$.

%----------------------------------------------------------------------
\subsection{Reaction-rate acceleration for
a nonlinear degradation path model}
\label{section:nonlin.deg.acc}
Some simple chemical degradation processes might be described by the
following path model (previously described in
Section~\ref{section:deg.to.fail})
\begin{equation}
\label{equation:gen.deg.model}
\degpath(t;\Temp) = \degpath_{\infty} \times
\left \{1 -  \exp \left [-\Rate_{U} \times \AF(\Temp) \times t \right ] \right \}
\end{equation}
where $\Rate_{U}$ is the reaction rate at use temperature
$\Temp_{U}$, $\Rate_{U} \times \AF(\Temp)$ is the rate reaction at a
general temperature $\Temp$, and for $\Temp > \Temp_{U}$,
$\AF(\Temp)>1$.  Figure~\ref{figure:nonlinear.saft.example.ps} shows
this function for fixed $\Rate_{U}$, $\Ea$, and $\degpath_{\infty}$,
but at different temperatures.
%-------------------------------------------------------------------
\begin{figure}
\splusbookfigure{\figurehome/nonlinear.saft.example.ps}
\caption{Nonlinear degradation paths at different temperatures
with a SAFT relationship.}
\label{figure:nonlinear.saft.example.ps}
\end{figure}
%-------------------------------------------------------------------
Note from (\ref{equation:gen.deg.model}) that
when $\degpath_{\infty} > 0$, $\degpath(\realrv)$ is increasing
and failure occurs when $\degpath(\realrv) >
\critdeg$. For the example in
Figure~\ref{figure:nonlinear.saft.example.ps},
however, $\degpath_{\infty} < 0$, $\degpath(\realrv)$ is decreasing,
and failure occurs when $\degpath(\realrv) <
\critdeg$.
In either case, equating $\degpath(\rv;\Temp)$ to $\critdeg$ and solving
for failure time gives
\begin{equation}
\label{equation.saft.time.scale1}
{\rv}(\Temp) =\frac{{\rv}(\Temp_{U})}{\AF(\Temp)}
\end{equation}
where 
${\rv}(\Temp_{U})= - \left (\frac{1}{\Rate_{U}}\right ) \log \left(1-
\frac{\critdeg}{\degpath_{\infty}} \right )$ 
is failure time at use conditions. Faster degradation shortens time
to any particular definition of failure (e.g., crossing $\critdeg$
or some other specified level) by a {\em scale factor} that depends
on temperature.  Thus changing temperature is similar to changing
the units of time. Consequently, the time to failure distributions
at $\Temp_{U}$ and $\Temp$ are related by
\begin{equation}
\label{equation.saft.time.scale2}
\Pr[\rv(\Temp_{U}) \leq t] =
\Pr[\rv(\Temp) \leq t/\AF(\Temp)].
\end{equation} 
Equations~(\ref{equation.saft.time.scale1}) 
and (\ref{equation.saft.time.scale2}) 
are forms of the Scale Accelerated
Failure Time (SAFT) model introduced in
Section~\ref{section:saft.model}.  


With a SAFT model, for example, if $\rv(\Temp_{U})$ (time at
use or some other baseline temperature) has a log-location-scale
distribution with parameters $\mu_{U}$ and $\sigma$, then
\begin{displaymath}
\Pr[\rv \leq t ;\Temp_{U} ] = 
	\Phi \left[\frac{\log(t)-\mu_{U}}{\sigma} \right].
\end{displaymath}
At any other temperature,
\begin{displaymath}
\Pr[\rv \leq t ; \Temp ] = \Phi \left[\frac{\log(t)-\mu}{\sigma} \right]
\end{displaymath}
where 
\begin{equation}
\label{equation:saft.mu.def}
\mu = \mu(x) = \mu_{U}-\log[\AF(\Temp)]=\beta_{0}+ \beta_{1} x,
\end{equation}
$x=11605/(\Tempk{})$, $x_{U}=11605/(\Tempk{U})$, $\beta_{1}=\Ea$,
and $\beta_{0}=\mu_{U}-\beta_{1}x_{U}$.  This is the same regression
model used in Section~\ref{section:exp.regr.model} (e.g., for the
lognormal, Weibull, and loglogistic distributions). LuValle,
Welsher, and Svoboda~(1988) and Klinger~(1992) describe more general
degradation model characteristics needed to assure that the SAFT
property holds. 
%-------------------------------------------------------------------
\begin{figure}
\splusbookfigure{\figurehome/arrhenius.lognor.alt.ps}
\caption{Example of the Arrhenius-lognormal life model.}
\label{figure:arrhenius.alt.ps}
\end{figure}
%-------------------------------------------------------------------
Figure~\ref{figure:arrhenius.alt.ps} shows a typical example of an
Arrhenius relationship between life and temperature. Using an Arrhenius
temperature axis and a log-life axis, the relationship plots as a family of
straight lines. Because of the SAFT relationship, the logarithms of
different lognormal distribution quantile lines all have the same slope.

%----------------------------------------------------------------------
\subsection{Acceleration for a linear degradation path model}
\label{section:linear.degradation}
If $\Rate_{U} \times \AF(\Temp) \times t$ in (\ref{equation:gen.deg.model})
is small so that $\degpath(t)$ is small relative to
$\degpath_{\infty}$, then
\begin{eqnarray}
\degpath(t;\Temp) &=& \degpath_{\infty} \times
\left \{1 -  \exp \left [-\Rate_{U} \times \AF(\Temp) \times t                \right ]
\right \} \nonumber \\
  &\approx& \degpath_{\infty} \times  \Rate_{U} \times \AF(\Temp) \times t =
\Rate^{\linearized}_{U} \times \AF(\Temp) \times t
\label{equation:linear.deg.path}
\end{eqnarray}
is approximately linear in $t$.
This is apparent when comparing the early-time behavior in
Figure~\ref{figure:nonlinear.saft.example.ps} with 
Figure~\ref{figure:linear.saft.example.ps}.
%-------------------------------------------------------------------
\begin{figure}
\splusbookfigure{\figurehome/linear.saft.example.ps}
\caption{Linear degradation model shown at different temperatures
with a SAFT relationship.}
\label{figure:linear.saft.example.ps}
\end{figure}
%-------------------------------------------------------------------
Also some degradation processes (e.g., automobile tire wear)
are approximately linear in time. In this case, if $\degpath(0;\Temp) =0$,
\begin{displaymath}
\degpath(t;\Temp) = \Rate^{\linearized}_{U} \times \AF(\Temp) \times t.
\end{displaymath}
Again, $\Rate^{\linearized}_{U}$ is the degradation rate at use
conditions and $\Rate^{\linearized}_{U} \times \AF(\Temp)$ is the
degradation rate at general temperature $\Temp$.  Failure occurs
when $\degpath(\realrv)$ crosses $\critdeg$.  Equating
$\degpath(\rv;\Temp)$ to $\critdeg$ and solving for the failure time
gives
\begin{eqnarray*}
{\rv}(\Temp) &=& \frac{{\rv}(\Temp_{U})}
                      {\AF(\Temp)}
\end{eqnarray*}
where ${\rv}(\Temp_{U})={\critdeg}/{\Rate^{\linearized}_{U}}$ is the
failure time at use conditions.  This is also a SAFT model. If
$\rv(\Temp)$ has a log-location-scale distribution, the parameters
of the distribution can, as in Section~\ref{section:nonlin.deg.acc},
be expressed as $\mu = \beta_{0}+ \beta_{1} x$ and a constant
$\sigma$.

%----------------------------------------------------------------------
\subsection{Acceleration of parallel chemical reactions}
\label{section:parallel.reaction}
Consider the more complicated chemical degradation path model
having two separate reactions contributing to failure
and described by
\begin{eqnarray*}
\degpath(t;\Temp) =& &\degpath_{1 \infty} \times
\left \{1 -  \exp \left [-\Rate_{1U} \times \AF_{1}(\Temp)
\times t \right ]
\right \} \\  & + & \degpath_{2 \infty} \times
\left \{1 -  \exp \left [-\Rate_{2U} \times \AF_{2}(\Temp) \times t \right ] \right \}.
\end{eqnarray*}
Here $\Rate_{1U}$ and $\Rate_{2U}$ are the use-condition rates of
the two parallel reactions contributing to failure. Suppose that the
Arrhenius relationship can be used to describe temperature dependence for
these rates,
providing acceleration functions $\AF_{1}(\Temp)$ and
$\AF_{2}(\Temp)$.
Then, unless $\AF_{1}(\Temp)=\AF_{2}(\Temp)$ for all $\Temp$, this
degradation model does {\em not} lead to a SAFT model.  Intuitively,
this is because temperature affects the two degradation processes
differently, inducing a nonlinearity into the acceleration function
relating times at two different temperatures.  To obtain useful
extrapolation models it is, in general, necessary to have adequate
models for the important individual degradation processes.

In some situations (e.g., when the individual
processes can be observed) it may be possible to use such a model by
estimating the effect that temperature (or other accelerating variable)
has on both $\Rate_{1U}$ and $\Rate_{2U}$.

%----------------------------------------------------------------------
%----------------------------------------------------------------------
\section{Voltage and Voltage-Stress Acceleration}
\label{section:voltage.at.models}
Increasing voltage or voltage stress (electric field)
is another commonly used method
to accelerate failure of electrical materials and components like
light bulbs, capacitors, transformers, heaters, and insulation.
Voltage is defined as the difference in electrical potential between
two points.  Physically it can be thought of as the amount of pressure
behind an electrical current. Voltage stress across a dielectric is
measured in units of volts/thickness (e.g., Volts/mm or kV/mm).

\begin{example}{\bf Accelerated life test of a mylar-polyurethane
insulation.}
\label{example:mylar.voltage.data}
Appendix
Table~\ref{atable:mylar.alt.data} and
Figure~\ref{figure:mylarpoly.altplot.ps} show data from an ALT on a
special type of mylar-polyurethane insulation used in high-performance
electro-magnets. The data, from Kalkanis and Rosso~(1989), give time
to dielectric breakdown of units tested at 100.3, 122.4,
157.1, 219.0, and 361.4 kV/mm.  The purpose of the
experiment was to evaluate the reliability of the insulating structure
and to estimate the life distribution at system design voltages.  The
figure shows that failures occur much sooner at high voltage
stress. Except for the 361.4 kV/mm data, the relationship between
log life and log voltage appears to be approximately linear.
%-------------------------------------------------------------------
\begin{figure}
\splusbookfigure{\figurehome/mylarpoly.altplot.ps}
\caption{Times to dielectric breakdown of mylar-polyurethane
insulation tested at 100.3, 122.4, 157.1, 219.0, and 361.4 kV/mm.}
\label{figure:mylarpoly.altplot.ps}
\end{figure}
%-------------------------------------------------------------------
\end{example}

%----------------------------------------------------------------------
\subsection{Voltage acceleration mechanisms}
Depending on the failure mode, raising voltage can:
\begin{itemize}
\item
Increase the voltage stress level relative to
dielectric strength of a specimen. The dielectric strength of certain
types of insulation will decline over time from chemical degradation.
\item
Increase the strength of the electric field, thereby accelerating some
failure-causing electro-chemical reactions or accelerating the growth
of failure-causing discontinuities in the dielectric material.
\end{itemize}
Sometimes one or the other of these effects will be
the primary cause of failure.
In other cases, both effects will be important.
%----------------------------------------------------------------------
\begin{figure}
\splusfigureportraitsize{\figurehome/inverse.power.af.ps}{5in}
\caption{Time-Acceleration Factor as a function of Stress Ratio
and exponent $-\beta_{1}$ for the inverse power relationship.}
\label{figure:inverse.power.af.ps}
\end{figure}
%----------------------------------------------------------------------

\subsection{The inverse power relationship}
\label{section:inverse.power.rule}
%----------------------------------------------------------------------
The most commonly used model for voltage acceleration is the ``inverse
power relationship'' (also known as the ``inverse power rule''
and the `inverse power law''). Let $\rv(\Volt)$ and $\rv(\Volt_{U})$
be the failure times that would result for a particular unit tested at
increased voltage and use voltage conditions, respectively. Then
the inverse power relationship is
\begin{equation}
\label{equation:inverse.power}
\rv(\Volt)=\frac{{\rv}(\Volt_{U})}{\AF(\Volt)}=
	\left (  \frac{  \Volt }{ \Volt_{U} }  
	\right)^{\beta_{1}}\rv(\Volt_{U})
\end{equation}
which is a SAFT model. The relationship in
(\ref{equation:inverse.power}) is known as the inverse power relationship
because, generally, $\beta_{1} < 0$. 

The inverse power relationship voltage acceleration factor can be
expressed as
\begin{equation}
\label{equation:voltage.acceleration}
\AF(\Volt)=\AF(\Volt,\Volt_{U}, \beta_{1})= 
	\frac{{\rv}(\Volt_{U})}{{\rv}(\Volt)}=
	\left (  \frac{  \Volt }{ \Volt_{U}}  \right) ^{-\beta_{1}}.
\end{equation}
When $\Volt > \Volt_{U}$, $\AF(\Volt,\Volt_{U},\beta_{1}) > 1$.  When
$\Volt_{U}$ and $\beta_{1}$ are understood to be, respectively,
product use (or other baseline) voltage and the material-specific
exponent, $\AF(\Volt)=\AF(\Volt,\Volt_{U},\beta_{1})$
denotes the acceleration factor.
Figure~\ref{figure:inverse.power.af.ps} gives $\AF$ as a function of
the stress ratio (e.g.,  $\Volt_{\High}$/$\Volt_{\Low}$) and
$\beta_{1}$.

If the model for $\rv(\Volt)$ is a
log-location-scale distribution, its parameters can be expressed as $\mu =
\beta_{0}+ \beta_{1} x$ with constant $\sigma$ 
where $x=\log(\Volt)$ and $\beta_{0}$ is the value of $\mu$ at
$\Volt=1$.  Then $\log(t_{p}) = \beta_{0}+ \beta_{1} x +
\Phi^{-1}(p) \sigma$.  The parameters $\beta_{0}$, $\beta_{1}$, and
$\sigma$ are product or material characteristics.
%-------------------------------------------------------------------
\begin{figure}
\splusbookfigure{\figurehome/inverse.power.weibull.altplot.ps}
\caption{Example of the inverse power relationship/Weibull life model.}
\label{figure:inverse.power.altplot.ps}
\end{figure}
%-------------------------------------------------------------------
Figure~\ref{figure:inverse.power.altplot.ps} shows a typical example
of an inverse-power relationship between quantiles of a Weibull life
distribution and voltage. Using log axes for time and voltage, the
relationship plots as a family of straight lines. Because of the
SAFT relationship, the quantile lines all have the same slope.

\begin{example}
{\bf Time-acceleration factor for glass capacitors.} From extensive
experience with glass capacitors, it is known that the power parameter
in the inverse power relationship model is in the neighborhood of $\beta_{1}=-2$.  For
capacitors rated at 100 volts, testing at 300 volts should provide an
acceleration factor of $(300/100)^{2}=9$. This can be seen directly
from Figure~\ref{figure:inverse.power.af.ps} entering with $300/100=3$
and reading $\AF \approx 9$ from the $\beta_{1}=-2$ line.
\end{example}

%----------------------------------------------------------------------
\subsection{Physical motivation for the inverse power relationship
for voltage-stress acceleration}
\label{section:ivp.physical.motivation}
The inverse power relationship is generally considered to be an
empirical model for the relationship between life and the
level of certain accelerating variables and especially those that are
pressure-like stresses.  This section presents a simple physical
motivation for the inverse power relationship for voltage-stress acceleration
under constant temperature situations.
Section~\ref{section:temp.volt.acc.model} describes a more general
model for voltage acceleration involving a combination of temperature
and voltage acceleration.

This discussion is for insulation. The ideas extend,
however, to other dielectric materials, products, and devices like
insulating fluids, transformers, and capacitors. In applications, an
insulation should not conduct an electrical current. An insulation has
a characteristic dielectric strength which can be expected to be
random from unit-to-unit. The dielectric strength of an insulation 
specimen operating in a specific
environment at a specific voltage may degrade with time.
Figure~\ref{figure:dielectric.breakdown.example.ps} shows a family of
simple curves to model unit-to-unit variability and 
degradation in dielectric strength over time.
%----------------------------------------------------------------------
\begin{figure}
\splusbookfigure{\figurehome/dielectric.breakdown.example.ps}
\caption{Dielectric strength degrading over time, relative to
voltage-stress
levels (horizontal lines).}
\label{figure:dielectric.breakdown.example.ps}
\end{figure}
%----------------------------------------------------------------------
The unit-to-unit variability could be caused, for example, by
materials or manufacturing variability. The horizontal lines
represent voltage-stress levels that might be present in actual
operation or in an accelerated test.  When a specimen's dielectric
strength falls below the applied voltage stress, there will be
flash-over, a short circuit, or other failure-causing damage to the
insulation.  Analytically, suppose that degrading dielectric strength
at age $t$ can be expressed as
\begin{displaymath}
\degpath(t) = \delta_{0} \times t^{1/\beta_{1}}.
\end{displaymath}
Here, as in Section~\ref{section:nonlin.deg.acc}, failure occurs
when $\degpath(\realrv)$ crosses $\critdeg$, the applied voltage
stress, denoted by $\Volt$.  In
Figure~\ref{figure:dielectric.breakdown.example.ps}, the
unit-to-unit variability is in the $\delta_{0}$ parameter. Equating
$\degpath(\rv)$ to $\Volt$ and solving for failure time $\rv$ gives
\begin{displaymath}
{\rv}(\Volt) =\left ( \frac{\Volt}{\delta_{0}} \right)^{\beta_{1}}.
\end{displaymath}
Then the acceleration factor for $\Volt$ versus $\Volt_{U}$ is
\begin{displaymath}
\AF(\Volt)=\AF(\Volt,\Volt_{U}, \beta_{1})= 
	\frac{{\rv}(\Volt_{U})}{{\rv}(\Volt)}=
	\left (  \frac{  \Volt }{ \Volt_{U}}  \right) ^{-\beta_{1}}
\end{displaymath}
which is an inverse power relationship, as in
(\ref{equation:voltage.acceleration}).

To extend this model, suppose that higher voltage also leads to an
increase in the degradation rate and that this increase is 
described with the degradation model
\begin{displaymath}
\degpath(t) = \delta_{0} \left [\Rate(\Volt) \times t \right ]^
{1/\gamma_{1}}
\end{displaymath}
where
\begin{displaymath}
\Rate(\Volt)= \gamma_{0} \exp\left[\gamma_{2} \log(\Volt) \right].
\end{displaymath}
Suppose failure occurs when $\degpath(\realrv)$ crosses $\critdeg$, the
applied voltage stress, denoted by $\Volt$. Then equating
$\degpath(\rv)$ to $\Volt$ and solving for failure time $\rv$ gives
the failure time
\begin{displaymath}
{\rv}(\Volt) =\frac{1}{\Rate(\Volt)} 
	\left ( \frac{\Volt}{\delta_{0}} \right)^{\gamma_{1}}.
\end{displaymath}
Then the ratio of failure times at $\Volt_{U}$ versus $\Volt$ 
is the acceleration factor
\begin{displaymath}
\AF(\Volt) = 
\frac{\rv(\Volt_{U})}
     {\rv(\Volt)}=\left ( \frac{ \Volt}{ \Volt_{U} } \right) ^
{\gamma_{2}-\gamma_{1}}.
\end{displaymath}
which is again an inverse power relationship 
with $\beta_{1}=\gamma_{1}-\gamma_{2}$.

\subsection{Other inverse power relationships}
The inverse power relationship is also commonly used for
other accelerating variables including pressure,
cycling rate, electric current, stress, and humidity.
Some examples are given in the next section.

%----------------------------------------------------------------------
\section{Acceleration Models with More than One Accelerating Variable}
\label{section:multi.factor.at.models}
Some accelerated tests use more than one accelerating variable.  Such
tests might be suggested when it is known that two or more potential
accelerating variables contribute to degradation and failure.
Using two or more variables may provide needed time-acceleration without
requiring levels of the individual accelerating variables to be too
high. Some accelerated tests include engineering variables that are not
accelerating variables. Examples include material type, design, operation,
and so on.

\subsection{Generalized Eyring relationship}
%----------------------------------------------------------------------
\label{section:g.eyring.model}

The generalized Eyring relationship extends the Eyring relationship in
Section~\ref{section:eyring.af}, allowing for one or more
non-thermal accelerating variables (such as humidity or voltage).  For
one additional non-thermal accelerating variable $X$, the model, in
terms of reaction rate, can be written as
\begin{equation}
\label{equation:g.eyring.model}
\Rate(\Temp,X)=\gamma_{0} \times \left(\Tempk{}\right)^{m} \times \exp\left(
\frac{-\gamma_{1}}{\Boltzmann  \times \Tempk{}}\right) \times \exp\left(\gamma_{2} X +
\frac{\gamma_{3} X}{\Boltzmann \times \Tempk{} }\right)
\end{equation}
where $X$ is a function of the non-thermal stress. The parameters
$\gamma_{1}=\Ea$ (activation energy) and $\gamma_{0}$, $\gamma_{2}$,
$\gamma_{3}$ are characteristics of the particular physical/chemical
process. Additional factors like the one on the right of
(\ref{equation:g.eyring.model}) can be added for other non-thermal
accelerating variables.

The following sections, following common practice, set
$\left(\Tempk{}\right)^{m}=1$, using what is essentially the Arrhenius
temperature-acceleration relationship. They describe some important
special-case applications of this more general model. 
If the underlying model relating the degradation process 
to failure is a SAFT model, then, as in
Section~\ref{section:arrhenius.af}, the generalized Eyring relationship
can be used to describe the relationship between times
at different sets of conditions $\Temp$ and $X$. In particular, the
acceleration factor relative to use conditions $\Temp_{U}$ and $X_{U}$
is 
\begin{displaymath}
\AF(\Temp,X) = \frac{\Rate(\Temp,X)}{\Rate(\Temp_{U},X_{U})}.
\end{displaymath}
The same approach used in Section~\ref{section:nonlin.deg.acc} shows the
effect of accelerating variables on time to failure. 
For example, suppose that $\rv(\Temp_{U})$ (time at
use or some other baseline temperature) has a log-location-scale
distribution with parameters $\mu_{U}$ and $\sigma$. Then
$\rv(\Temp)$ has the same log-location-scale distribution with
\begin{equation}
\label{equation:geyring.saft.mu.def}
\mu = \mu_{U} -  \log[\AF(\Temp,X)]= \beta_{0} + \beta_{1} x_{1}+
\beta_{2} x_{2}+ \beta_{3} x_{1} x_{2}
\end{equation}
where $\beta_{1}=\Ea$, $\beta_{2}= -\gamma_{2}$,  $\beta_{3}=
-\gamma_{3}$,
$x_{1}=11605/(\Tempk{})$, $x_{2}=X$, and
$\beta_{0}= \mu_{U} - \beta_{1}x_{1U}- \beta_{2}x_{2U}-
\beta_{3}x_{1U}x_{2U}$.

%----------------------------------------------------------------------
\subsection{Temperature-voltage acceleration}
%----------------------------------------------------------------------
\label{section:temp.volt.acc.model}
Example~\ref{example:zelen.data} describes an analysis of the
Zelen~(1959) data from a life test of glass capacitors at higher
than usual levels of temperature and voltage. That example used only
simple linear relationships between log-life and the accelerating
variables.  McPherson and Baglee~(1985) used accelerated life test data
to model the joint effect of thermal and electrical accelerating
variables for failure of thin-gate 100 \AA \,\,\, oxides.  Boyko and
Gerlach~(1989) investigate the effect of temperature and strength of
electrical field on the time to the generalized Eyring relationship.

To put the Eyring/Arrhenius temperature-voltage acceleration model in
the form of (\ref{equation:geyring.saft.mu.def}), let $x_{1}=11605/
\Tempk{}$, $x_{2}=\log(\Volt)$ and $x_{3}=x_{1} x_{2}$. The terms with
$x_{1}$ and $x_{2}$ corresponds, respectively, to the Arrhenius and
the power relationship acceleration models. The term with $x_{3}$, a
function of both temperature and voltage, is an interaction
suggesting that the temperature acceleration factor depends on the
level of voltage.  Similarly, a voltage-temperature interaction
suggests that the voltage-acceleration factor depends on the level
of temperature.  Klinger~(1991a) suggests an alternative
physically-motivated model for the Boyko and Gerlach data with
second order terms involving both temperature and voltage stress.

The dynamic voltage-stress/dielectric-strength model introduced in
Section~\ref{section:ivp.physical.motivation} can be generalized to
provide motivation for the generalized Eyring relationship in
(\ref{equation:g.eyring.model}) where $X=\log(\Volt)$. In particular
for the degradation path model
\begin{displaymath}
\degpath(t) = \delta_{0} \left [\Rate(\Temp,\Volt) \times t \right ]
^{1/\gamma_{1}}.
\end{displaymath}
Then, from (\ref{equation:g.eyring.model}),
\begin{displaymath}
\Rate(\Temp,\Volt)=\gamma_{0} \times \left(\Tempk{}\right)^{m} \times \exp\left(
\frac{-\Ea}{\Boltzmann  \times \Tempk{}}\right) \times \exp\left(\gamma_{2} 
\log(\Volt) +
 \frac{\gamma_{3} \log(\Volt)}{\Boltzmann \times \Tempk{} }\right).
\end{displaymath}
Failure occurs
when $\degpath(\realrv)$ crosses $\critdeg = \mbox{applied voltage
stress}$, denoted by $\Volt$.  Equating $\degpath(\rv)$ to $\Volt$ and
solving for failure time $\rv$ gives
\begin{displaymath}
{\rv}(\Temp,\Volt) =\frac{1}{\Rate(\Temp,\Volt)} \left ( \frac{\Volt}{\delta_{0}} \right)^{\gamma_{1}}.
\end{displaymath}
Then the ratio of failure times
at $(\Temp_{U},\Volt_{U})$ versus $(\Temp,\Volt)$ 
is the acceleration factor
\begin{eqnarray*}
\AF(\Temp,\Volt)&=&
\frac{\rv(\Temp_{U},\Volt_{U})}
     {\rv(\Temp,\Volt)}\\
&=& \exp[\Ea ( x_{1U} -  x_{1} )] \times   
\left (  \frac{ \Volt}{  \Volt_{U} } \right) ^{\gamma_{2}-\gamma_{1}}\times 
\left \{ \exp[x_{1} \log(\Volt)- 
x_{1U}\log(\Volt_{U}) \right]\}^{\gamma_{3}} .
\end{eqnarray*}
where $x_{1U}=11605/(\Tempk{U})$ and $x_{1}=11605/(\Tempk{})$.  For
the special case when $\gamma_{3}=0$ (no interaction) $\AF(\Temp,\Volt)$
is composed of separate factors for temperature and voltage
acceleration. Thus the voltage acceleration factor (holding
temperature constant) does not depend on the temperature level used in
the acceleration.

%----------------------------------------------------------------------
\subsection{Temperature-current density acceleration}
%----------------------------------------------------------------------
Accelerated tests for electromigration typically use temperature and
current density as the accelerating variables.  To put the
Eyring/Arrhenius temperature-current density acceleration model in the
form of (\ref{equation:geyring.saft.mu.def}), let
$x_{1}=11605/\Tempk{}$, $x_{2}=\log(\Current)$ and $x_{3}=x_{1}x_{2}$.
The terms with $x_{1}$ and $x_{2}$ correspond to Arrhenius and the
power relationship acceleration models. When the interaction term is omitted
(i.e., $\beta_{3}$ assumed to be 0), this is known as ``Black's
equation'' (described in Black~1969).

%----------------------------------------------------------------------
\subsection{Temperature-humidity acceleration}
\label{section:temp.hum.acc}
Humidity is another commonly used accelerating variable,
particularly for failure mechanisms involving corrosion and certain
kinds of chemical degradation.
\begin{example}{\bf Accelerated life test of a printed wiring board.} 
Example~\ref{example:caf.alt.data} introduced data, shown in
Figure~\ref{figure:luvalle.scatter.ps}, from an ALT of printed circuit
boards. It illustrates the use of humidity as an accelerating variable.
This is a subset of the larger experiment described by LuValle,
Welsher, and Mitchell~(1986), involving  acceleration with
temperature, humidity, and voltage.  The figure shows
clearly that failures occur earlier at higher levels of humidity.
%-------------------------------------------------------------------
\end{example}
A variety of different humidity models (mostly empirical but a few
with some physical basis) have been suggested for different kinds of
failure mechanisms. Much of this work has been motivated by concerns
about the effect of environmental humidity on plastic-packaged
electronic devices. Humidity is also an important factor in the
service-life distribution of paints and coatings. In most test
applications where humidity is used as an accelerating variable, it
is used in conjunction with temperature.  For example, Peck~(1986)
presents data and models relating life of semiconductor electronic
components to humidity and temperature.  See also Peck and
Zierdt~(1974) and Joyce et al.~(1985).  Gillen and Mead (1980)
describe a kinetic approach for modeling accelerated aging data.
LuValle, Welsher, and
Mitchell~(1986) describe the analysis of time to failure data on
printed circuit boards that have been tested at higher than usual
temperature, humidity, and voltage. They suggest ALT models based on
the physics of failure.  Chapter 2 of Nelson~(1990a) and Boccaletti
et al.~(1989) review and compare a number of different humidity
models.

The Eyring/Arrhenius temperature-humidity acceleration
relationship in the form of (\ref{equation:geyring.saft.mu.def}) uses
$x_{1}=11605/\Tempk{}$, $x_{2}=\log(\RH)$, and $x_{3}=x_{1}x_{2}$
where $\RH$ is relative humidity, expressed as a proportion.  An
alternative humidity relationship suggested by Klinger~(1991b), on
the basis of a simple kinetic model for corrosion, uses the term
$x_{2}=\log[\RH/(1-\RH)]$ (a logistic transformation) instead.




 
\section{Guidelines for the Use of Acceleration Models}

Because most applications of accelerated testing involve
extrapolation, users must exercise caution in planning tests
(accelerated test planning is described in
Chapter~\ref{chapter:alt.test.planning} and
Section~\ref{section:planning.accelerated.degradation.tests}) and
interpreting the results of data analyses (accelerated test data
analysis is described in Chapters~\ref{chapter:analyzing.alt.data}
and
\ref{chapter:accelerated.degradation}). Some guidelines for the use of
acceleration models include:
\begin{itemize}
\item
Accelerating variables should be chosen to correspond with
variables that cause actual failures.
\item
It is useful to
investigate previous attempts to accelerate failure mechanisms similar to
the ones of interest. There are many research reports and
papers that have been published in the physics of
failure literature.
\item
Accelerated tests should be designed, as much as possible, to minimize
the amount of extrapolation required (See
Chapter~\ref{chapter:alt.test.planning} and
Section~\ref{section:planning.accelerated.degradation.tests}).
High levels of accelerating variables can cause extraneous failure modes
that would never occur at use-levels of the accelerating variables.  If
extraneous failures are not recognized and properly handled, they can lead 
to seriously incorrect conclusions. Also, the relationship may not be
accurate enough over a wide range of acceleration.
\item
Generally, accelerated tests are used to obtain information about one
particular, relatively simple failure mechanism (or corresponding
degradation measure).  If there is more than one failure mode, it is
possible that the different failure mechanisms will be accelerated at
different rates. Then, unless this is accounted for in the modeling and
analysis, estimates could be seriously incorrect when extrapolating
to lower use-levels of the accelerating variables.
\item
In practice, it is difficult or impractical to verify acceleration
relationships over the entire range of interest.  Of course, accelerated
test data should be used to look for departures from the assumed
acceleration model.  It is important to recognize, however, that the
available data will generally provide very little power to detect
anything but the most serious model inadequacies. Typically there is
no useful diagnostic information about possible model inadequacies
at accelerating variable levels close to use conditions.
\item
Simple models with the right shape have generally proven to be more
useful than elaborate multi-parameter models.
\item
Sensitivity analysis should be used to assess the effect of perturbing
uncertain inputs (e.g., inputs related to model assumptions).
\item
Accelerated test programs should be planned and conducted by teams
including individuals knowledgeable about the product and its use
environment, the physical/chemical/mechanical aspects of the failure
mode, and the statistical aspects of the design and analysis of
reliability experiments.
\end{itemize}

%----------------------------------------------------------------------
%----------------------------------------------------------------------
\section*{Bibliographic Notes}
Nelson~(1990a) provides an extensive and comprehensive source for
background material, practical methodology, basic theory, and
examples for accelerated testing models. See Smith~(1996), Chapter 7
of Tobias and Trindade~(1995), Chapters 2 and 9 of Jensen~(1995),
and Klinger, Nakada, and Menendez~(1990) for additional discussion
of these topics.  Thomas~(1964) describes some practical aspects of
accelerated testing and describes what we have called SAFT as ``true
acceleration.'' Harter~(1977) provides a detailed review of the
literature on the effect that size has on reliability. Starke et
al.~(1996) describe the use of long-term elevated temperature
exposure and the prospects for the use of accelerated aging of
materials and structures.  Feinberg and Windom~(1995) describe the
reliability physics of thermodynamic aging and its relationship to
device reliability.  Fukuda~(1991) and Ueda (1996) describe
degradation models for Lasers and LEDs. Howes and Morgan~(1981),
Hakim~(1989), Pollino~(1989) and Christou~(1992, 1994a, 1994b)
describe degradation models for microelectronic devices. Gillen and
Clough~(1985) describe a kinetic model for predicting oxidative
degradation rates in combined radiation-thermal environments.

LuValle~(1990) and LuValle and Hines~(1992) show how to use
step-stress methods to extract information about the kinetics of
failure processes.  Drapella~(1992) provides a mathematical model
illustrating how a failure process with a kinetic model more
complicated than first order can lead to a break-down of the
commonly used Arrhenius relationship. Costa and Mercer~(1993)
describe degradation models for corrosion.  Cragnolino and
Sridhar~(1994) is a collection of papers describing the use of
accelerated corrosion tests for the prediction of service life. Bro
and Levy~(1990) describe kinetic models for degradation of
batteries.  Starke et al.~(1996) describe issues relating to aging
of materials and structures. Bayer~(1994) describes models for
prediction and prevention of wear.  Castillo and Galambos~(1987)
derive a regression model for fatigue failure, based on established
physical models.  The annual {\em Proceedings of the International
Reliability Physics Symposium}, sponsored by the IEEE Electron
Devices Society and the IEEE Reliability Society, contain numerous
articles describing physical models for acceleration and failure.

Often accelerated test models are derived or specified
through a system of differential equations.  When, as is often the
case, no closed form solution is available, it becomes necessary to
use numerical solutions. Nash and Quon~(1996) describe software for
fitting differential equation models to data.  

Evans~(1977) makes the important point that the need to make rapid
reliability assessments and the fact that accelerated tests may be
``the only game in town'' is not sufficient to {\em justify} the use
of the method.  Justification must be based on physical models or
empirical evidence. Evans~(1991) describes difficulties with
accelerated testing and suggests the use of sensitivity analysis. He
also comments that acceleration factors of 10 ``are not
unreasonable'' but that ``factors much larger than that tend to be
figments of the imagination and lots of correct but irrelevant
arithmetic.''
%----------------------------------------------------------------------
%----------------------------------------------------------------------
\section*{Exercises}
\begin{exercise}
For the toaster example in
Section~\ref{section:methods.of.acceleration}, toasters were cycled
365 times per day to get reliability information more quickly. Discuss
the practical limitations of increasing the cycling frequency to get
information even more quickly.
\end{exercise}

\begin{exercise}
\label{exercise:fet.acc}
Based on previous experience with similar products, the failure time
of a particular field effect transistor on a monolithic microwave GaAs
integrated circuit, operating at $100\degreesc$ (channel temperature),
is expected to have a lognormal distribution with a median time to
failure of 30 years. The primary failure mode is caused by a chemical
reaction that has an activation energy of $\Ea$=.6 eV. The value of the
lognormal scale parameter for this failure mode is expected to be
$\sigma=.7$.
\begin{enumerate}
\item
What are the lognormal parameters $\mu$ and $\sigma$ if time is
recorded in hours?
\item
For operation at $100\degreesc$ channel temperature, what is the time at which
$5\%$ of the units would fail? $10\%$? 90\%?
\item
Calculate the time-acceleration factor for testing at $250\degreesc$,
$200\degreesc$ and $150\degreesc$ channel temperature. Use
Table~\ref{table:arrhenius.temp.diff} and
Figure~\ref{figure:arrhenius.af.ps} and check with equation
(\ref{equation:arrhenius.af}).
\item
Obtain an expression for the temperature at which $100p\%$ of tested
units would be expected to fail in a 6000-hour test. Use this
expression to compute the temperatures at which $90\%$ and $10\%$
would be expected to fail.
\end{enumerate}
\end{exercise}


\begin{exercise}
Refer to Exercise~\ref{exercise:fet.acc}.  Obtain an expression for
the (average) FIT rate (in standard units of failures per hour in
parts per billion) for the first ten years of operation at
$100\degreesc$? How much would this improve if the operating channel
temperature is changed to only $90\degreesc$?
\end{exercise}


\begin{exercise}
\label{exercise:mech.adhesive}
A mechanical adhesive has been designed for 10-year life at 
$60 \degreesc$ ambient temperature. Over time, the bond will degrade
chemically and will eventually fail. The rate of the chemical
reaction can be increased by testing at higher levels of temperature.
Using an activation energy of $\Ea=1.2$ eV and the Arrhenius relationship,
calculate the time-acceleration factors for testing at
$120 \degreesc$, $90\degreesc$, and $80\degreesc$.
\end{exercise}


\begin{exercise}
Time to failure of incandescent light bulbs can be described
accurately with a lognormal distribution. A test engineer claims
that a $10\%$ increase in voltage decreases life by approximately $50\%$.
A particular brand of 100 watt bulb has a median life of 1200 hours at
110 volts.
\begin{enumerate}
\item
Give an inverse power relationship expression for the life of such light bulbs
as a function of voltage.
\item
Calculate the time-acceleration factors for operating the light bulb at
120 volts and 130 volts?
\end{enumerate}
\end{exercise}

\begin{exercise}
Refer to Exercise~\ref{exercise:ic.sing.alt}.  
\begin{enumerate}
\item
\label{ex.par:ic.accel}
Assume that the activation energy of the observed failure mode is
$\Ea=1.2$ eV and that the Arrhenius
relationship provides an adequate description of the effect of temperature on
the reaction rate. Compute estimates of
the lognormal distribution parameters
at $50\degreesc$, $80\degreesc$, and
$120\degreesc$.
\item
On lognormal probability paper, plot the estimate of
the life distributions at $50\degreesc$, $80\degreesc$, and
$120\degreesc$.
\item
Repeat part~\ref{ex.par:ic.accel}, using an
activation energy of $\Ea=.7$ eV and also plot these results.
\item
Explain the effect that an incorrect assumption about 
activation energy could have on estimates of life at low
temperature.
\end{enumerate}
\end{exercise}

\begin{exercise}
A certain kind of capacitor has an exponential life distribution
with a median life of 10 thousand hours at operating voltage of 400
volts. The relationship between life and voltage can be described by
the inverse power relationship with an exponent
$\beta=-10$. Determine the time-acceleration factors for accelerated
testing of these capacitors at 500 volts, 600 volts, and 800 volts.
\end{exercise}

\begin{exercise}
A particular type of integrated circuit is thought to have a dominant failure
mode with an activation energy of $\Ea=1.2$ eV.  This
circuit is designed to operate at $50\degreesc$.  If a
1000-hour life test is conducted at $120\degreesc$, under the
Arrhenius relationship, what is the equivalent amount of operating time for
this failure mode?
\end{exercise}

\begin{exercise}
Consider a failure mechanism modeled with an underlying degradation path model
in (\ref{equation:gen.deg.model}).  Suppose that the reaction
activation energy is $\Ea=1.8$ eV.  For $\Rate_{U}=.2$ and
$\degpath_{\infty}=.6$, compute the crossing times for all
combinations of $t=$1000, 2000 hours 
and $\critdeg=$.5, 1.  Use these results to
verify that the SAFT property holds in this case.
\end{exercise}



\begin{exercise}
Table~\ref{table:arrhenius.temp.diff} and
Figure~\ref{figure:arrhenius.af.ps} can be used together to
determine time-acceleration factors for different levels of use and
test temperatures.  Create a similar table and figure that can be
used to obtain time-acceleration factors for the
logit-transformation relative humidity model described in
Section~\ref{section:temp.hum.acc}.
\end{exercise}


\begin{exercise}
\label{exercise:arrhenius.scale}
Equation (\ref{equation:saft.mu.def}) gives the Arrhenius relationship
between the location parameter of the log-life distribution and
temperature. When needed, 
suppose that $\beta_{0}=-17$, $\beta_{1}=.86$, and $\sigma=1.2$.
Using this model:
\begin{enumerate}
\item
Give an expression for $\rvquan_{p}$, the $p$ quantile 
as a function of temperature in
$\degreesc$.
\item
\label{exer.part:linear.arrhen.relat}
Show that the relationship
between $\log(\rvquan_{p})$ versus $1/\kelvin$ is linear.
\item 
Starting with linear graph paper, make an ``Arrhenius plot'' for this
model, similar to Figure~\ref{figure:arrhenius.alt.ps}.  Start by
plotting the linear relationship in
part~\ref{exer.part:linear.arrhen.relat} on linear paper over the
range of interest. Use a range of temperature running from
50$\degreesc$ to 140$\degreesc$.  Note that in order to have the slope
of the plotted line decreasing in with temperature, it will be
necessary to have the $1/\kelvin$ axis running left to right from
largest to smallest values of $1/\kelvin$ (instead of the customary
increasing axis). Then, finally, add in new (nonlinear) axes for Time
and $\degreesc$.  Generally it is most convenient to do this on the
axis opposite to the corresponding linear axis. On these axes, use
tick and tick labels at major points on the scale (e.g., for
temperature at 50, 60, \dots , 140, corresponding to
$1/\kelvin$=.003094538, 0.003001651,\dots , 0.002420428).
\end{enumerate}
\end{exercise}



\begin{exercise}
Show why the relationships between the TDF and $\AF$
in Figure~\ref{figure:arrhenius.af.ps}
plot as straight lines.
\end{exercise}


\begin{exercise}
Show why the relationships between the voltage ratio and $\AF$
in Figure~\ref{figure:inverse.power.af.ps}
plot as straight lines.
\end{exercise}









