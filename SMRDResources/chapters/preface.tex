%preface
\chapter*{Preface}
\addcontentsline{toc}{chapter}{Preface}

Over the past 10 years there has been a heightened interest in
improving quality, productivity, and reliability of manufactured
products.  Global competition and higher customer expectations for
safe, reliable products are driving this interest. To meet this
need, many companies have trained their design engineers and
manufacturing engineers in the appropriate use of designed
experiments and statistical process monitoring/control. Now
reliability is being viewed as the product feature that has the
potential to provide an important competitive edge. A current
industry concern is in developing better processes to move rapidly
from product conceptualization to a cost-effective highly reliable
product.  A reputation for unreliability can doom a product, if not
the manufacturing company.

Data collection, data analysis, and data interpretation methods are
important tools for those who are responsible for product reliability
and product design decisions.  This book describes and illustrates the
use of proven traditional techniques for reliability data analysis and
test planning, enhanced and brought up to date with modern
computer-based graphical, analytical, and simulation-based methods.
The material in this book is based on our interactions with engineers
and statisticians in industry as well as on courses in applied
reliability data analysis that we have taught to MS-level statistics
and engineering students at both Iowa State University and Louisiana
State University.\\[3ex]

\noindent
{\bf Audience and assumed knowledge.} We have designed this book to be
useful to statisticians and engineers working in industry as well as
to students in university engineering and statistics programs. The
book will be useful for on-the-job training courses in reliability
data analysis. There is challenge in addressing such a wide-ranging
audience. Communications among engineers and statisticians, however,
is not only necessary, but essential in the industrial research and
development environment. We hope that this book will aid such
communication. To produce a book that will appeal to both engineers
and statisticians, we have placed primary focus on applications, data,
concepts, methods, and interpretation. We use simple computational
examples to illustrate ideas and concepts but, as in practical
applications, rely on computers to do most of the computations. We
have also included a collection of exercise problems at the end of
each chapter. These exercises will give readers a chance to test their
knowledge of basic material, explore conceptual ideas of reliability
testing, data analysis and interpretation, and to see possible
extensions of the material in the chapters.


It will be helpful for readers to have had a previous course in
intermediate statistical methods covering basic ideas of statistical
modeling and inference, graphical methods, estimation, confidence
intervals, and regression analysis.  Only the simplest concepts of
calculus are used in the main body of the text (e.g., probability for
a continuous random variable is computed as area under a density
curve; a first derivative is a slope or a rate of change; a second
derivative is a measure of curvature).  Appendix B and some
advanced exercises use calculus, linear algebra, basic optimization
ideas, and basic statistical theory. Concepts, however, are presented
in a relaxed and intuitive manner that we hope will also appeal to
interested non-statisticians. Throughout the book we have attempted to
avoid the heavy language of mathematical statistics.

A detailed understanding of underlying statistical theory is {\em not}
necessary to apply the methods in this book. Such details are,
however, often important to understanding how to extend methods to new
situations or developing new methods. Appendix B, at the end of the
book, outlines the general theory and provides references to more
detailed information.  Also, many derivations and interesting
extensions are covered in advanced guided exercises at the end of each
chapter.

Particularly challenging exercises (i.e., exercises requiring
knowledge of calculus or statistical theory) are marked with a
``\difficultexercise''. Exercises requiring computer programming
(beyond the use of standard statistical packages) are marked with a
``\computationalexercise.''\\[3ex]


\noindent
{\bf Special features of the book.}
Special features of this book include the following:

\begin{enumerate}
\item
We emphasize general methods that can be applied to the wide range of
problems found in industrial reliability data analysis---specifically,
nonparametric estimation of a failure-time distribution function,
probability plotting, and maximum likelihood estimation of important
reliability characteristics (failure probabilities, distribution
quantiles, and hazard functions) and associated statistical intervals.
In the basic chapters (\ref{chapter:nonparametric.estimation},
\ref{chapter:probability.plotting},
\ref{chapter:parametric.ml.one.par}, 
\ref{chapter:parametric.ml.ls}, 
\ref{chapter:regression.analysis}, and 
\ref{chapter:analyzing.alt.data})
we apply these methods to the most frequently encountered models in
reliability data analysis. In special chapters (which can be skipped
without loss of continuity or understanding), we apply the general
methods to important, but less frequently occurring situations
(e.g., problems involving truncation and prediction).

\item
Throughout the book we use computer graphics for displaying data,
for displaying the results of analyses, and for explaining technical
concepts.

\item 
We use simulation methods to complement large-sample asymptotic
theory (practical sample sizes are often small to moderate in size).
We explain and illustrate modern, more accurate (but computationally
demanding) methods of inference: likelihood and bootstrap methods
for constructing statistical intervals.

\item
For both nonparametric and parametric analyses, we illustrate the
use of general likelihood-based methods of handling {\em
arbitrarily} censored data (including left, right, and interval
censoring with overlapping intervals) and truncated data that
frequently arise in statistical reliability studies.

\item
We provide methods for planning reliability studies (length of test,
number of specimens, and levels of experimental factors).

\item
We cover methods for analyzing {\em degradation} data.
Such data are becoming increasingly important where there are
requirements for extremely high reliability.

\item
Almost all of our examples and exercises use real data, including many data
sets that have not previously appeared in any book. In order to
protect proprietary information, some data have been changed by a scale
factor and, in some cases, generic product names have been used (e.g.,
Device-A, Component-B, Alloy-A).

\item
Numerical examples in this book were done using the \splus
system for graphics and data analysis (a product of MathSoft, Inc.,
Seattle, WA).  A suite of special \splus functions was developed in
parallel with this book.  Although we have not included explicit
information about software-use in the chapters, the suite of special
\splus functions and a listing of the \splus commands used to do the
examples in the book are available from the authors via anonymous FTP.
\end{enumerate}

\noindent
{\bf Other software to use with the book.} Today there are many
commercial statistical software packages.  Unfortunately, only a few
of these packages have adequate capabilities for doing reliability
data analysis (e.g., the ability to do nonparametric and parametric
estimation with censored data). Nelson~(1990a, pages 237-240) outlines
the capabilities of a number of commercial and noncommercial packages
that were available at that time. As software vendors become more
aware of their customers' needs, capabilities in commercial packages
are improving. Here we describe briefly the capabilities of a few
packages that we and our colleagues have found to be useful.

MINITAB (1997), SAS PROC RELIABILITY (1997), SAS JMP (1995), \splus
(1996) and a specialized program, called WinSMITH (Abernethy 1996)
can do nonparametric and parametric product censored data analysis
to estimate a single distribution
(Chapters~\ref{chapter:nonparametric.estimation},
\ref{chapter:probability.plotting},
\ref{chapter:parametric.ml.one.par},
and \ref{chapter:parametric.ml.ls}).  SAS JMP can also analyze data
with more than one failure mode
(Chapter~\ref{chapter:system.reliability}).  MINITAB, SAS PROC
RELIABILITY, SAS JMP, and \splus can do parametric regression and
accelerated life test analyses
(Chapters~\ref{chapter:regression.analysis} and
\ref{chapter:analyzing.alt.data}), as well as semiparametric Cox
proportional hazards regression analysis.  SAS PROC RELIABILITY can,
in addition, do the nonparametric repairable systems analyses
(Chapter~\ref{chapter:repairable.system}).\\[3ex]

\noindent
{\bf Overview and paths through the book.} There are many possible
paths that readers and instructors might take through this book.
Chapters~\ref{chapter:reliability.data}-\ref{chapter:repairable.system}
cover single distribution models without any explanatory
variables.
Chapters~\ref{chapter:regression.analysis}-\ref{chapter:accelerated.degradation}
describe failure-time regression models.
Chapter~\ref{chapter:case.studies} presents case studies that
illustrate, in the context of real problems, the integration of
ideas presented throughout the book. This chapter also usefully
illustrates how some of the general methods presented in the earlier
chapters can be extended and adapted to deal with new problems.

Chapters~\ref{chapter:reliability.data}-\ref{chapter:nonparametric.estimation}
and \ref{chapter:probability.plotting}-\ref{chapter:parametric.ml.ls}
provide basic material that will be of interest to almost all readers
and should be read in sequence.
Chapter~\ref{chapter:ls.parametric.models} discusses parametric
failure-time models based on location-scale distributions and
Chapter~\ref{chapter:other.parametric.models} covers more advanced
distributional models.  It is possible to use only a light reading of
Chapter~\ref{chapter:ls.parametric.models} and to skip
Chapter~\ref{chapter:other.parametric.models} altogether before
proceeding on to the important methods in
Chapters~\ref{chapter:probability.plotting}-\ref{chapter:parametric.ml.ls}.
Chapter~\ref{chapter:bootstrap} explains and illustrates the use of
bootstrap (simulation-based) methods for obtaining confidence
intervals. Chapter~\ref{chapter:test-planning} focuses on test
planning: evaluating the effects of choosing sample size and length of
observation.
Chapters~\ref{chapter:ml.other.parametric}-\ref{chapter:repairable.system}
cover a variety of special more advanced topics for single
distribution models. Some of the material in
Chapter~\ref{chapter:other.parametric.models} is prerequisite for the
material in Chapter~\ref{chapter:ml.other.parametric}, but it is
possible to simply work in Chapter~\ref{chapter:ml.other.parametric},
referring back to Chapter~\ref{chapter:other.parametric.models} only
as needed.  Otherwise, each of Chapters~\ref{chapter:test-planning}
through
\ref{chapter:singledist.bayes} has only material up to
Chapter~\ref{chapter:parametric.ml.ls} as prerequisite.
Chapter~\ref{chapter:system.reliability} introduces some important
system reliability concepts and shows how the material in the first
part of the book can be used to make statistical statements about the
reliability of a system or a population of systems.
Chapter~\ref{chapter:repairable.system} explains and illustrates the
fundamental ideas behind analyzing system-repair and other recurrence data (as
opposed to data on components and other replaceable units).

There are several groups of chapters on special topics that can be
read in sequence.

\begin{itemize}
\item
{\bf Accelerated testing.} Chapter~\ref{chapter:regression.analysis}
introduces models and methods for regression analysis (assessing the
effects of explanatory variables) for failure-time data.
Chapter~\ref{chapter:accelerated.test.models} introduces physically
based reliability models used in accelerated testing.
Chapter~\ref{chapter:analyzing.alt.data} shows how to analyze data
from accelerated life tests.
Chapter~\ref{chapter:alt.test.planning} describes test planning for
regression and accelerated test applications with censored data.
\item
{\bf Degradation analysis.} Chapter~\ref{chapter:degradation.data}
provides methods for analyzing degradation reliability data. Use of
degradation data in accelerated tests is covered in
Chapter~\ref{chapter:accelerated.degradation}.
Chapter~\ref{chapter:case.studies} contains a case study that
describes a method for planning accelerated degradation tests.
\item
{\bf Bayesian methods.} Chapter~\ref{chapter:singledist.bayes}
introduces concepts and applications of Bayesian methods for
failure-time data. A case study in
Chapter~\ref{chapter:case.studies} extends these ideas to regression
with an application to accelerated life test data analysis.
\end{itemize}

Appendix~A provides a summary and index of notation used in the
book.  Appendix~B outlines the general maximum likelihood and other
statistical theory upon which the methods in the book are based.
Appendix~C gives tables for some of the larger data sets used in our
examples.\\[3ex]

\noindent
{\bf Use as a textbook.}  A two-semester course would be required
to cover thoroughly all of the material in the book.  For a
one-semester course, aimed at engineers and/or statisticians, an
instructor could cover
Chapters~\ref{chapter:reliability.data}-\ref{chapter:ls.parametric.models},
Chapters~\ref{chapter:probability.plotting}-\ref{chapter:parametric.ml.ls},
and
Chapters~\ref{chapter:regression.analysis}-\ref{chapter:analyzing.alt.data},
along with selected material from the appendices (according to the
background of the students), and a few other chapters according to
interests and tastes.

This book could be used as the basis for workshops or short courses
aimed engineers or statisticians working in industry.  For an audience
with a working knowledge of basic statistical tools,
Chapters~\ref{chapter:reliability.data}-\ref{chapter:nonparametric.estimation},
key sections in Chapter~\ref{chapter:ls.parametric.models}, and
Chapters~\ref{chapter:probability.plotting}-\ref{chapter:parametric.ml.ls}
could be covered in one day. If the purpose of the short course is
to introduce the basic ideas and illustrate with examples, then some
material from
\ref{chapter:regression.analysis}-\ref{chapter:alt.test.planning}
could also be covered. For a less experienced audience or for a
more relaxed presentation, allowing time for exercises and discussion,
two days would be needed to cover this material.  Extending the course
to three or four days would allow covering selected material in
Chapters~\ref{chapter:bootstrap}-\ref{chapter:case.studies}.







