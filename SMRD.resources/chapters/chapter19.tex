%chapter 19
%\batchmode
%original by wqmeeker  12 Jan 94
%edited by wqmeeker 9 Apr 94
%edited by wqmeeker 26 Apr 94
%edited by driker 22 May 95
%edited by driker 6 Sept 95
%edited by driker 2 July 96
%edited by driker 21 nov 96
%edited by driker 9 apr 97
\setcounter{chapter}{18}

\chapter{Accelerated Life Tests}

\label{chapter:analyzing.alt.data}


%{\Large {\bf William Q. Meeker and Luis A. Escobar}}\\
%Iowa State University and Louisiana State University\\[5ex]
{\large {\bf \today}}\\[2ex]
Part of 
{\em Statistical Methods for Reliability Data}\\
Copyright 1997 W. Q. Meeker and L. A. Escobar. \\[1ex]
To be published by John Wiley \& Sons Inc. in 1998.



%----------------------------------------------------------------------
%----------------------------------------------------------------------
\section*{Objectives}
This chapter explains
\begin{itemize} 
\item 
Nonparametric and graphical methods for presenting and analyzing
accelerated life test (ALT) data.
\item 
Likelihood methods for analyzing right-censored data from an ALT
with a single accelerating factor.
\item
Analysis of other kinds of ALT experiments including experiments
yielding interval data and experiments with two accelerating
variables.
\item
Some other common forms of accelerated testing.
\item
Some potential pitfalls of accelerated testing.
\end{itemize}

%----------------------------------------------------------------------
%----------------------------------------------------------------------
\section*{Overview}
This chapter describes and illustrates some basic data analysis
methods for accelerated life tests (ALT).  These tests are used to
characterize durability properties or the life distribution of
materials or simple components. The presentation employs the ALT
models described in Chapter~\ref{chapter:accelerated.test.models}.
Section~\ref{section:single.factor.alt} explains and illustrates
basic important ideas for a single-variable ALT with right-censored
data and exact failure times.
Section~\ref{section:further.examp.alt} presents several other
important examples with special models or data features.
Section~\ref{section:practical.suggest.alt} gives some suggestions
and cautions for drawing conclusions from AT data.
Section~\ref{section:other.kinds.alt} briefly describes other kinds
of ``accelerated tests.'' Section~\ref{section:alp.pitfalls} outlines
a number of potential pitfalls that can arise in the application of
accelerated testing.
%----------------------------------------------------------------------
%----------------------------------------------------------------------
\section{Introduction}
This chapter shows how to apply regression methods
from Chapter~\ref{chapter:regression.analysis} to the analysis
of accelerated life test (ALT) data.
%----------------------------------------------------------------------
%----------------------------------------------------------------------
\begin{example}{\bf Temperature-accelerated life test on Device-A.}
\label{example:devicea.data}
%----------------------------------------------------------------------
%----------------------------------------------------------------------
\begin{figure}
\splusbookfigure{\figurehome/devalt.weib.raltplot.ps}
\caption{Scatter-plot of temperature-accelerated 
life test data for Device-A. Censored observations are indicated by
$\Delta$. The number of censored/tested units were 30/30, 90/100,
11/20, 1/15 at 10, 40, 60, and $80\degreesc$, respectively.}
\label{figure:devalt.weib.raltplot.ps}
\end{figure}
%----------------------------------------------------------------------
Hooper and Amster~(1990) analyze the temperature-accelerated life test
data on a particular kind of device. Because they do not identify the
particular device, we will refer to it as Device-A. The data are given
in Appendix Table~\ref{atable:devalt.data} and
Figure~\ref{figure:devalt.weib.raltplot.ps}.  The purpose of the
experiment was to determine if Device-A would meet its failure rate
objective through 10,000 hours and 30,000 hours at its operating
ambient temperature of $10\degreesc$. In this context, failure rate is
usually taken to mean the proportion failing over the specified time
interval [see (\ref{equation:ahr.point}) in
Section~\ref{section:ttf.functions}]. In the following sections we
will show how to fit an accelerated life regression model to these
data to answer this and other questions.
\end{example}
%----------------------------------------------------------------------
%----------------------------------------------------------------------
\subsection{Accelerated life test models}
%----------------------------------------------------------------------
Most parametric ALT models have the following two components:
\begin{enumerate}
\item
A parametric distribution for the life of a population of units at a
particular level(s) of an experimental variable or variables.  It
might be possible to avoid this parametric assumption for some
applications, but when appropriate, parametric models (e.g., Weibull
and lognormal) provide important practical advantages for most
applications.

\item
A relationship between one (or more) of the distribution parameters
and the acceleration or other experimental variables. Such a
relationship models the effect that variables like temperature,
voltage, humidity, and specimen or unit size will have on the
failure-time distribution.  As described in
Chapter~\ref{chapter:accelerated.test.models}, this part of the
accelerated life model should be based on a physical model such as
one relating the accelerating variable to degradation, on a
well-established empirical relationship, or some combination.
\end{enumerate}

The examples in this section use the log-location-scale regression
models described in Section~\ref{section:exp.regr.model} and
illustrated with other examples in
Chapter~\ref{chapter:regression.analysis}. The relationships between
parameters and accelerating variables come from considerations like
those described in Chapter~\ref{chapter:accelerated.test.models}.

%----------------------------------------------------------------------
\subsection{Strategy for analyzing ALT data}
\label{section:alt.strategy}
This section outlines and illustrates a strategy that is useful for
analyzing ALT data consisting of a number of groups of specimens,
each having been run at a particular set of conditions. The basic
idea is to start by examining the data graphically. Use probability
plots to analyze each group separately and explore the adequacy of
candidate distributions. Then fit a model that describes the
relationship between life and the accelerating variable(s).
Briefly, the strategy is to

\begin{enumerate}
\item
Examine a scatter-plot of failure time versus the accelerating variable.
\item
\label{item.alt.gi}
Fit distributions individually to the data at separate levels of the
accelerating variable.  Plot the fitted ML lines on a multiple
probability plot along with the individual nonparametric estimates
at each level of the accelerating variable. Use the plotted points
and fitted lines to assess the reasonableness of the corresponding
life distribution and the constant-$\sigma$ assumption. Repeat with
probability plots for different assumed failure-time distributions.
\item
\label{item.alt.gm}
Fit an overall model with the proposed relationship between life and
the accelerating variable.
\item
Compare the combined model from Step~\ref{item.alt.gm} with the individual
analyses in  Step~\ref{item.alt.gi} to
check for evidence of lack of fit for the overall model.
\item
Perform residual analyses and other diagnostic checks of the model
assumptions.
\item
Assess the reasonableness of the ALT data to make the
desired inferences.
\end{enumerate}
The first examples in this chapter have just one accelerating
variable (the simplest and most common type of ALT).
Section~\ref{section:two.factor.alt} shows how to apply
the same general strategy to an ALT with two or more accelerating variables.
%----------------------------------------------------------------------
%----------------------------------------------------------------------
\section{Analysis of Single-Variable ALT Data}
\label{section:single.factor.alt}
This section describes methods for analyzing ALT data with a single
accelerating variable.  The subsections illustrate, in sequence, the
steps in the strategy described in Section~\ref{section:alt.strategy}.

\subsection{Scatter-plot of ALT data}
Start by examining a scatter-plot of failure-time data versus the
accelerating-variable data. A different symbol should be used
to indicate censored observations.
\begin{example}{\bf Scatter-plot of the Device-A data.}
Figure~\ref{figure:devalt.weib.raltplot.ps} is a scatter-plot of the
Device-A ALT data introduced in Example~\ref{example:devicea.data}. As
expected, units fail sooner at higher levels of temperature.  The
heavy censoring (note for example that there were no failures at
10$\degreesc$) makes it difficult to see the form of the
life/accelerating variable relationship from this plot.
\end{example}

\subsection{Multiple probability plot of
nonparametric cdf estimates at individual levels of the accelerating
variable}
%----------------------------------------------------------------------
\begin{figure}
\splusbookfigure{\figurehome/devalt.weib.groupi.ps}
\caption{Weibull multiple probability plot with individual Weibull ML fits
for each temperature for the Device-A data.}
\label{figure:devalt.weib.groupi.ps}
\end{figure}
%----------------------------------------------------------------------
%----------------------------------------------------------------------
\begin{figure}
\splusbookfigure{\figurehome/devalt.lnor.groupi.ps}
\caption{Lognormal multiple probability plot 
with lognormal ML fits
for each temperature for the Device-A data.}
\label{figure:devalt.lnor.groupi.ps}
\end{figure}
%----------------------------------------------------------------------
To make a multiple probability plot, first compute nonparametric
estimates of the failure-time distribution for each group of
specimens tested at the same level of the accelerating
variable. Then plot these on probability paper. Use the plot to
assess the distributional model for the different levels of the
accelerating variable (or variable-level combinations).

\begin{example}{\bf Multiple probability plot of the Device-A data.}
Figure~\ref{figure:devalt.weib.groupi.ps} is a Weibull multiple
probability plot of the Device-A data.
Figure~\ref{figure:devalt.lnor.groupi.ps} is a corresponding
lognormal probability plot.  Comparison of these plots indicates
that both the Weibull and the lognormal distributions provide a
reasonable fit to the failure data at the different levels of
temperature, but that the lognormal distribution provides a better
fit to the individual temperature groups.
\end{example}
%----------------------------------------------------------------------
\subsection{Adding ML estimates at each level of 
the accelerating variable(s) to a multiple probability plot} 

If a suitable parametric distribution can be found, then ML estimates of the
cdf at each level of the accelerating variable should be computed
and put on the probability plot along with the corresponding
nonparametric cdf estimates. This plot is useful for assessing the
commonly used assumptions that distribution shape does not depend on
the level of the accelerating variable and that the accelerating variable
only affects the distribution scale parameter. The slopes of the 
lines are related to the distribution shape parameter values. Thus
we can assess graphically the assumption that temperature has no effect on 
distribution  shape.
%----------------------------------------------------------------------
\begin{table}
\caption{Device-A ALT lognormal ML estimation 
results at individual temperatures.}
\centering\small
\begin{tabular}{lcrrrr}
\\[-.5ex] \hline
& & & & \multicolumn{2}{c}{95\% Approximate}\\
&&\multicolumn{1}{c}{ML} &Standard & \multicolumn{2}{c}{Confidence
Interval}\\ \cline{5-6}
& Parameter & Estimate&
\multicolumn{1}{c}{Error} & Lower & Upper \\
\hline 
40$\degreesc$
& $\mu$ &9.81 &.42 &8.9 &10.6 \\[.7ex] 
&$\sigma$ & $1.0$  &.27  & $.59$  & $1.72$ \\[.7ex]
\hline 
60$\degreesc$
& $\mu$ & 8.64 & .35 & $8.0$ & 9.3 \\[.7ex] 
&$\sigma$ &$1.19$  &.32  & $.70$  & $2.0$\\[.7ex] 
\hline 
80$\degreesc$
&$\mu$ &7.08 &.21 &$6.7$ &7.5 \\[.7ex] 
&$\sigma$ &$.80$  &.16 & $.55$  & $1.17$\\[.7ex]
\hline 
\\[-1.8ex]
\end{tabular}
\begin{minipage}[t]{4in}
The individual log likelihoods were
$\loglike_{40} = -115.46 $, $ \loglike_{60} =-89.72 $, 
and $  \loglike_{80} = -115.58 $. The confidence intervals 
are based on the normal-approximation method.
\end{minipage}
\label{table:deva.indiv.mles}
\end{table}


\begin{example}{\bf ML estimates of Device-A life
at 40, 60 and 80$\degreesc$.}
\label{example:devicea.mle}
The straight lines on Figures~\ref{figure:devalt.weib.groupi.ps} and 
\ref{figure:devalt.lnor.groupi.ps} depict, respectively, individual
Weibull and lognormal ML estimates of the cdfs at the different
levels of temperature.  Table~\ref{table:deva.indiv.mles} summarizes
the lognormal ML estimation results. In this plot there are small
differences among the slopes, but this could be due to sampling
error. This can be seen, informally, from the overlapping confidence
intervals for $\sigma$ (this issue is addressed more formally in
Example~\ref{example:devicea.model.lrt}).  Although both
distributions fit reasonably well, tradition and physical theory
(see Section~\ref{section:lognormal.distribution.def}) suggest the
lognormal distribution to describe the failure-time distribution for
such devices.  Subsequent Device-A examples will also use the
lognormal distribution.
\end{example}

%----------------------------------------------------------------------
\subsection{Multiple probability plot of
ML estimates with a fitted acceleration relationship}
\label{section:mpplot}
%----------------------------------------------------------------------
\begin{figure}
\splusbookfigure{\figurehome/devalt.lnor.groupm.ps}
\caption{Lognormal multiple probability plot depicting
the Arrhenius-lognormal regression model
ML fit to the Device-A ALT data.}
\label{figure:devalt.lnor.groupm.ps}
\end{figure}

%----------------------------------------------------------------------
In order to draw conclusions about life at low levels of
accelerating variables, one needs to use a life/accelerating
variable relationship to tie together results at the different
levels of the accelerating variable. The cdfs estimated from the
model fit can also be plotted on a probability plot along with the
data to assess how well the life/accelerating variable model fits
the data. Extrapolations to other levels of the accelerating
variable can also be plotted.

\begin{example}
{\bf ML estimates of Device-A data for the Arrhenius-lognormal model.}
\label{example:devicea.arr.log.model}
The Arrhenius-lognormal regression model, described in
Section~\ref{section:nonlin.deg.acc}, is
\begin{displaymath}
\Pr[\rv \leq t;\Temp] = \Phi_{\nor} \left[\frac{\log(t)-\mu}{\sigma} \right]
\end{displaymath}
where $\mu = \beta_{0}+ \beta_{1} x$, $x=11605/(\Tempk{})$, and
$\beta_{1}=\Ea$ is the activation energy.
Table~\ref{table:deva.arr.lognor.mles} contains ML estimates and
other information. The estimate of the variance-covariance matrix
for the ML estimates
$\thetavechat=(\betahat_{0},\betahat_{1},\sigmahat)$ is
\begin{eqnarray}
\label{equation:devicea.local.est.vcv}
\vcvmathat_{\thetavechat} &=&
\left[ 
\begin{array}{rrr}
 8.336  & $-0.239$ & $ -.195$\\
$-.239$ & $.0069$  & $.0059$\\
$-.195$ & $.0059$  & $.0176$\\
\end{array}
\right].
\end{eqnarray}
These quantities will be used in subsequent numerical examples.
Figure~\ref{figure:devalt.lnor.groupm.ps} is a lognormal probability
plot showing the
Arrhenius-lognormal model fit to the Device-A ALT data. The solid line at
the bottom of the graph is the ML estimate of the cdf at
10$\degreesc$, extrapolated from the ML fit.  

%----------------------------------------------------------------------
\begin{table}
\caption{ML estimates for the Device-A 
data and the Arrhenius-lognormal regression model.}
\centering\small
\begin{tabular}{crrrr}
\\[-.5ex] \hline
 & & & \multicolumn{2}{c}{95\% Approximate}\\
&\multicolumn{1}{c}{ML} &Standard  & \multicolumn{2}{c}{Confidence
Intervals}\\  \cline{4-5}
Parameter &  Estimate & \multicolumn{1}{c}{Error} & Lower & Upper \\
\hline 
 $\beta_{0} $ & $-13.5$ & $2.9$ & $-19.1$ & $-7.8$ \\[.7ex] 
$\beta_{1}$ &.63 &.08 &.47 &.79 \\[.7ex] 
$\sigma$ &.98 &.13 &.75 &1.28\\
\hline
\\[-1.8ex]
\end{tabular}
\begin{minipage}[t]{4in}
The log likelihood is $\loglike= -321.7$. The confidence intervals 
are based on the normal-approximation method.
\end{minipage}
\label{table:deva.arr.lognor.mles}
\end{table} 
%----------------------------------------------------------------------
The dotted curves are a set of pointwise 95\% normal-approximation
confidence intervals. They reflect the random
``sampling uncertainty'' arising from
the limited sample data. The necessary computations are illustrated in
Example~\ref{example:ci.fhat.comp}. It is important to note that these
intervals do {\em not} reflect model-specification and other
errors (and we know
that the model is only an approximation for the exact relationship).
%----------------------------------------------------------------------
\begin{figure}
\splusbookfigure{\figurehome/devalt.lnor.altplot.ps}
\caption{Scatter-plot showing hours to failure versus
$\degreesc$ (on an Arrhenius scale) and
the Arrhenius-lognormal regression model
fitted to the Device-A data.
Censored observations are indicated by $\Delta$.}
\label{figure:devalt.lnor.altplot.ps}
\end{figure}
%----------------------------------------------------------------------
Figure~\ref{figure:devalt.lnor.altplot.ps} shows directly the fitted
life/accelerating variable relationship and the estimated densities at
each level of temperature, and lines indicating ML estimates of
percent failing as a function of temperature. The density estimates 
are normal densities because time is plotted on a log scale.
\end{example}

It is useful to compare individual analyses with model analyses.  This
can be done both graphically and analytically. A likelihood ratio test
provides an analytical assessment about whether observed deviations
between the individual model fit and the overall life/accelerating
variable relationship can be explained by random variability or not.
\begin{example}
{\bf Analytical comparison of individual lognormal fits
and Arrhenius-lognormal model fit to the Device-A data.} 
\label{example:devicea.model.lrt}
Fitting individual lognormal distributions
(Table~\ref{table:deva.indiv.mles} and
Figure~\ref{figure:devalt.lnor.groupi.ps}) estimates
$\mu$ and $\sigma$ at each level of temperature without any
constraints. Fitting the Arrhenius-lognormal model
(Table~\ref{table:deva.arr.lognor.mles} and
Figure~\ref{figure:devalt.lnor.groupm.ps}) estimates
$\mu$ and $\sigma$ at each level of temperature with $\mu$ constrained
to be a linear function of $x=11605/(\Tempk{})$ and $\sigma$
constrained to be the same for all temperatures.  The total likelihood
for the unconstrained Arrhenius-lognormal
model will always be larger than the likelihood for
the constrained model.  If the total likelihood for the unconstrained
model is {\em much} larger than the total likelihood for the
constrained model, there is evidence of lack of fit for the
constrained Arrhenius-lognormal model. These two approaches for fitting
the data can be compared with an ``omnibus'' likelihood ratio test.
From Table~\ref{table:deva.indiv.mles}, for the
unconstrained model, $\loglike_{\rm unconst}=\loglike_{40} +
\loglike_{60} + \loglike_{80} = -320.76$ and from
Table~\ref{table:deva.arr.lognor.mles}, for the constrained model,
$\loglike_{\rm const}= -321.7$.  If the constrained model is
``correct'' then the test statistic
$Q=-2(\loglike_{\rm const}-\loglike_{\rm unconst})$
has a $\chisquare_{3}$ distribution (see Appendix
Section~\ref{section:profile.on.theta}).  In this case the 3 degrees of
freedom is the difference between the 6 parameters in the
unconstrained model and the 3 parameters in the constrained model.
Thus $Q=-2(-321.7 +320.76) =1.88 <
\chisquare_{(.75;3)} =4.1$ indicating that there is no evidence of inadequacy of the constrained model, relative to the unconstrained
model.
\end{example}

%----------------------------------------------------------------------
\subsection{Checking other model assumptions}
%----------------------------------------------------------------------
%----------------------------------------------------------------------
\begin{figure}
\splusbookfigure{\figurehome/devalt.residual.linear.lnor.ps}
\caption{Lognormal probability plot of the standardized residuals
from the Arrhenius-lognormal model
fit to the Device-A data.}
\label{figure:devalt.residual.linear.lnor.ps}
\end{figure}
%----------------------------------------------------------------------
\begin{figure}
\splusbookfigure{\figurehome/devalt.fitvsres.linear.lnor.ps}
\caption{Plot of standardized residuals
versus fitted values for the Arrhenius-lognormal regression model
fitted to the Device-A data. Censored residuals are indicated by
$\Delta$.}
\label{figure:devalt.fitvsres.linear.lnor.ps}
\end{figure}
%----------------------------------------------------------------------
Before drawing conclusions from a set of data, it is important to
check, as carefully as possible, model assumptions by using residual
analysis and other model diagnostics, as explained in
Section~\ref{section:resid.def}. A probability plot of residuals is
useful for assessing the overall adequacy of a fitted distribution.
A standard plot of residuals versus fitted values (e.g., $\muhat$)
can also be useful for identifying other departures from the fitted
model. Heavy censoring, however, makes such plots difficult to
interpret. Without censoring, the residuals should not show any
structure that might have been explained by a more elaborate model.
In interpreting such a plot with censoring, one has to make
allowances for the predictable patterns that censoring will induce.

\begin{example}
{\bf Residual analysis for the Arrhenius-lognormal model fit to the
Device-A data.} 
Figure~\ref{figure:devalt.residual.linear.lnor.ps} is a lognormal
probability plot of the (censored) standardized residuals from the
life/temperature model. In this case the deviation from linearity 
is not strong and can be attributed to randomness in the data. Thus,
the lognormal distribution appears to be adequate.

Figure~\ref{figure:devalt.fitvsres.linear.lnor.ps} is a plot of the
(censored) standardized residuals (as defined in
Section~\ref{section:censored.data.regression.diagnostics}) versus
the Arrhenius-lognormal model fitted values (as in other plots of failure
times, $\Delta$ represents right-censored residuals).  The heavy
censoring makes this plot more difficult to interpret.  All that we
know about the residuals marked with a $\Delta$ is that they are
larger than the plotted points. The appearance of a downward sloping
trend is mostly due to the right-censored observations. The smallest
observations also appear to slope downward, but there are only a few
observations involved in this trend. Although this might, at first,
suggest that there are outliers or some kind of other model
departure, there is no evidence of outliers in
Figure~\ref{figure:devalt.residual.linear.lnor.ps} and the observed
pattern could be a result of randomness from the
Arrhenius-lognormal model.
\end{example}

%----------------------------------------------------------------------
\subsection{Estimation at use conditions}
%----------------------------------------------------------------------
The methods presented in
Section~\ref{section:se.and.ci.for.regr.funct} can also be used to
compute estimates and
confidence intervals for quantities of interest at use
conditions.  The following example gives details on how to compute
an estimate of $F(t)$ but, as explained in
Section~\ref{section:se.and.ci.for.regr.funct}, the same ideas can be
applied to compute estimates of other quantities of interest such as
distribution quantiles or hazard function values.

\begin{example}
{\bf Confidence interval for the Device-A lognormal distribution 
$\boldsymbol{F(30000)}$ and $\boldsymbol{F(10000)}$ at 10$\degreesc$.}
\label{example:ci.fhat.comp}
%splus>      Location     Scale 
%splus> [1,]  12.2641 0.9778227
%splus>           Location      Scale 
%splus> Location 0.2867719 0.04781510
%splus>    Scale 0.0478151 0.01759516
As mentioned in Example~\ref{example:devicea.data},
the purpose of the ALT was to estimate the proportion failing at
30,000 and 10,000 hours. To illustrate the methods we use 
$\estimtime$=30,000 hours.
Using methods described in Sections~\ref{section:ci.for.loc.scale.cdf} and
\ref{section:se.and.ci.for.regr.funct}, and the numerical results in
Table~\ref{table:deva.arr.lognor.mles}, simple computations give, at
10$\degreesc$,
\begin{eqnarray*}
\muhat & = & \betahat_{0} + \betahat_{1}x\\ 
	& = & -13.469  + .6279 \times 11605/(10+273.15) = 12.2641\\[1ex]
\estimtimestdhat &= & [\log(\estimtime)-\muhat]/\sigmahat
 		= [\log(30000)-12.2641]/.9778=-2.000 \\[1ex]
\Fhat(30000)  &=  & \Phi_{\nor}(\estimtimestdhat) =
\Phi_{\nor}(-2.000) = .02281.
\end{eqnarray*}
%splus> -13.4693146 + .6279 * (11605/(10+273.15)) =  12.41985
%splus>   (log(30000)-12.2641)/.9778 = -2.000
%splus>  pnorm(-2.000) =  .02275013
%genmax 0.3000E+05  0.2281E-01  0.2253E-01  0.3206E-02  0.1448 
The estimated covariance matrix for $\muhat$ and $\sigmahat$ at
10$\degreesc$ can be computed as shown in equation
(\ref{equation:ls.regr.local.est.vcv}), using
(\ref{equation:devicea.local.est.vcv}). The result is
\begin{displaymath}
\vcvmathat_{\muhat,\sigmahat}=
\left[ 
\begin{array}{ll}
\varhat(\muhat)& \covhat(\muhat,\sigmahat)\\
\covhat(\muhat,\sigmahat)& \varhat(\sigmahat)
\end{array}
\right]=
\left[ 
\begin{array}{lr}
 .287 & .048 \\
 .048 & .0176
\end{array}
\right].
\end{displaymath}
The off-diagonal elements are zero for complete data but are
nonzero here because of the right censoring.
Then, following the general approach in
Section~\ref{section:ci.for.loc.scale.cdf}, equation
(\ref{equation:ls.se.fhat}) gives
%splus> ((dnorm(-2.000))/(.9778))* sqrt(.287-2*2.000*.048+(-2.000)^2*.0176)=.02245633
%splus>
\begin{eqnarray*}
	\sehat_{\Fhat}
	&=& \frac{  \phi(\estimtimestdhat )  }{\sigmahat} \left[ \varhat(\muhat) + 
	 2 \estimtimestdhat \covhat(\muhat,\sigmahat) +
	\estimtimestdhat^2 \varhat(\sigmahat) \right]^{\frac{1}{2}}\\
	&=& \frac{  \phi( -2.000 )  }{.9778  } \left[ .287 + 
	 2 \times  (-2.000) \times .048 +
	(-2.000)^2 \times .0176 \right]  ^ {\frac{1}{2}} = .0225.
\end{eqnarray*}
From this, the 95\% normal-approximation confidence interval for $F(30000)$, 
using (\ref{equation:normal.theory.ci.on.cdf}), is
\begin{eqnarray*}
	  [ \Flower(\estimtime), \quad	\Fupper(\estimtime) ] &= &
	\left[\frac{\Fhat}{\Fhat
	+(1-\Fhat) \times w}, \quad \frac{\Fhat}{\Fhat+
	(1-\Fhat)/w}\right]\\ &= &
	\left[\frac{.02281}{.02281
	+(1-.02281) \times w}, \quad \frac{.02281}{.02281+
	(1-.02281)/w}\right] =[.0032, \quad .14 ]
\end{eqnarray*}
%splus>  ((.02281)/(.02281+(1-.02281)* 7.231829))
%splus>  ((.02281)/(.02281+(1-.02281)/ 7.231829))
%splus>
%splus>
%splus>
where
\begin{eqnarray*}
w&=&\exp\{\norquan_{(1-\alpha/2)}
	\sehat_{\Fhat}/[\Fhat(1-\Fhat)]\}\\
  &=&\exp\{1.96 \times .0225/[.02281(1-.02281)]\} = 7.232.
\end{eqnarray*}
%splus>  exp((1.96 * .0225)/(.02281*(1-.02281))) =7.231829
This interval is based on the assumption that $Z_{\logit(\Fhat)}\approxdist
\NOR(0,1)$. The interval is wide
(also see Figure~\ref{figure:devalt.lnor.groupm.ps}),
but properly reflects the sampling uncertainty when the activation energy is
unknown. Additionally, it is important to note that the interval does
not reflect model uncertainty. On the other hand, if one 
assumed that the activation energy is known, this and other confidence
intervals would be much narrower.
A similar approximate 95\% confidence interval for $F(10000)$
(computations left as an exercise) is $[.00006, \quad .013]$.
\end{example}

Conclusions based on unverified assumptions are subject to
error. Although confidence intervals provide an assessment of sampling
uncertainty, they do not reflect possible model deviations.
When model assumptions are uncertain, repeating computations with
alternative assumptions provides informative sensitivity analyses.
The following example illustrates this by changing the fitted distribution
for the Device-A data.

\begin{example}
{\bf Confidence interval for the Weibull distribution $F(30000)$ and
$F(10000)$ for Device-A at 10$\degreesc$.} For the Weibull
distribution model (with other details omitted to save space), an
approximate 95\% confidence interval for $F(30000)$ is $[.0092,\quad
.126]$.  An approximate 95\% confidence interval for $F(10000)$ is
$[.0021,\quad.027]$.  The differences between the lognormal and
Weibull confidence intervals is not that large, relative to width of
these intervals.  Without some physical basis for a choice between
these two distributions, both sets of intervals would have to be
taken into consideration.  In this case, however, the lognormal was
favored on physical grounds, and the lognormal distribution also
provided a better fit.
%genmax>   0.1000E+05  0.7583E-02  0.4935E-02  0.2109E-02  0.2688E-01
%genmax>   0.3000E+05  0.3536E-01  0.2383E-01  0.9237E-02  0.1260    
\end{example}


%-----------------------------------------------------------------------------
%-----------------------------------------------------------------------------
\section{Further Examples}
\label{section:further.examp.alt}
%-----------------------------------------------------------------------------
\subsection{Voltage acceleration}
\label{section:volt.alt}
This section illustrates statistical methods for fitting a model to
and making inferences from voltage-accelerated life data.
Section~\ref{section:voltage.at.models} describes the inverse power
relationship for voltage acceleration used here.

\begin{example}{\bf Accelerated life test of a mylar-polyurethane insulating 
structure.} 
\label{example:mylar.voltage.ivp model}
Returning to the data introduced in
Example~\ref{example:mylar.voltage.data}, it was clear from the
scatter-plot in Figure~\ref{figure:mylarpoly.altplot.ps} that the
linear relationship for log life versus log voltage relationship
implied by the inverse power relationship
(Section~\ref{section:inverse.power.rule}) did not hold for the data
at 361.4 kV/mm.  This suggests that the failure mechanism might be
different at 361.4 kV/mm.  In this kind of situation, particularly
when primary interest is in extrapolating to lower ranges of
voltage, it is appropriate to drop the 361.4 kV/mm data and analyze
the remaining data.

%----------------------------------------------------------------------
\begin{figure}
\splusbookfigure{\figurehome/mylarpoly.lnor.groupi.ps}
\caption{Lognormal multiple 
probability plot and ML fit for each voltage in the 
mylar-polyurethane ALT.}
\label{figure:mylarpoly.lnor.groupi.ps}
\end{figure}
%----------------------------------------------------------------------
Figure~\ref{figure:mylarpoly.lnor.groupi.ps} is a lognormal
probability plot of the data at each of the 5 different
levels of voltage stress along with individual lognormal ML estimates.
Although the $\sigma$ estimates differ across voltage-stress levels, the 
differences are small and consistent with ordinary
random variability. A similar Weibull plot (not shown
here) also provided a reasonable fit to the data at the different
levels of voltage stress but, overall, the lognormal seemed to fit
better.
%----------------------------------------------------------------------
\begin{figure}
\splusbookfigure{\figurehome/mylarsub.lnor.groupm.ps}
\caption{Lognormal multiple probability plot 
and ML fit of the inverse power relationship-lognormal model to
the mylar-polyurethane data with the 361.4 kV/mm data omitted.}
\label{figure:mylarsub.lnor.groupm.ps}
\end{figure}
%----------------------------------------------------------------------
Figure~\ref{figure:mylarsub.lnor.groupm.ps} shows the
inverse power relationship-lognormal model fitted to the mylar-polyurethane
data. The model is
\begin{displaymath}
\Pr[\rv\leq t;\Volt] = \Phi_{\nor}\left[\frac{\log(t)-\mu}{\sigma} \right]
\end{displaymath}
where $\mu = \beta_{0}+ \beta_{1} x$, $x=\log(\text{\Volt})$, and
$\Volt$ is voltage stress in kV/mm. Numerical estimates are
summarized in Table~\ref{table:mylar.lognor.mles}.  The solid line
in the SE part of Figure~\ref{figure:mylarsub.lnor.groupm.ps} is the
ML estimate of the cdf at 50 kV/mm, extrapolated from the ML fit. As
in Example~\ref{example:devicea.arr.log.model}, the dotted lines are
pointwise 95\% normal-approximation confidence intervals, reflecting
sampling uncertainty.
%----------------------------------------------------------------------
\begin{figure}
\splusbookfigure{\figurehome/mylarsub.residual.linear.lnor.ps}
\caption{Lognormal
probability plot of the residuals from the inverse power
relationship-lognormal model fitted to the mylar-polyurethane data.}
\label{figure:mylarsub.residual.linear.lnor.ps}
\end{figure}
%----------------------------------------------------------------------
Figure~\ref{figure:mylarsub.residual.linear.lnor.ps} is a lognormal
probability plot of the residuals from the inverse power
relationship-lognormal model (Section~\ref{section:inverse.power.rule})
fitted to the mylar-polyurethane data. This plot does not
suggest any important departure from the assumed lognormal
distribution and is closer to linear than the corresponding plot for
the inverse power relationship-Weibull model (not shown here).
%----------------------------------------------------------------------
\begin{figure}
\splusbookfigure{\figurehome/mylarpoly.lnor.altplot.ps}
\caption{Plot of the inverse power relationship-lognormal model fitted to the
mylar-polyurethane data (also showing 361.4 kV/mm
data omitted from the ML estimation).}
\label{figure:mylarpoly.lnor.altplot.ps}
\end{figure}
%----------------------------------------------------------------------
Figure~\ref{figure:mylarpoly.lnor.altplot.ps} shows the fitted model
and the original data (showing the data at 361.4 kV/mm even though
these data were not used in the model fitting). This plot indicates
that there is a substantial probability of failure before 10,000
hours $[F(10000)=.076]$ for this insulating structure at 50 kV/mm. A
95\% confidence interval for $F(10000)$ is $[.0058, \quad
.54]$. Incorrectly including the 361.4 kV/mm data in the analysis
(details not shown here, but see
Exercise~\ref{exercise:mylar.lognormal.all}) changes the confidence
interval to $[.00012, \quad .064]$, resulting in a very optimistic
and misleading impression.
%lognormal sub data
%genmax>  0.1000E+05  0.7618E-01  0.9476E-01  0.5857E-02  0.5358    
%genmax>    0.3000E+05  0.3502      0.2376      0.6513E-01  0.8066 
%lognormal all data
%genmax>    0.1000E+05  0.2827E-02  0.4583E-02  0.1171E-03  0.6420E-01
%genmax>    0.3000E+05  0.3265E-01  0.3523E-01  0.3780E-02  0.2310
%genmax>
%genmax>
%----------------------------------------------------------------------
\begin{table}
\caption{Inverse power relationship-lognormal model ML estimates for the
mylar-polyurethane data.}
\centering\small
\begin{tabular}{crrrr}
\\[-.5ex] \hline
 & & & \multicolumn{2}{c}{95\% Approximate}\\
&\multicolumn{1}{c}{ML} &Standard  & \multicolumn{2}{c}{Confidence
Intervals}\\  \cline{4-5}
Parameter &  Estimate & \multicolumn{1}{c}{Error} & Lower & Upper \\
\hline 
 $\beta_{0} $ & 27.5 & 3.0 &21.6 &33.4 \\[.7ex] 
$\beta_{1}$ & $-4.29$ & .60 & $-5.46$ & $-3.11$ \\[.7ex] 
$\sigma$ & 1.05 & .12 & .83 & 1.32 \\
\hline
\\[-1.8ex]
\end{tabular}
\begin{minipage}[t]{4in}
The log likelihood is $\loglike= -271.4$. The confidence intervals 
are based on the normal-approximation method.
\end{minipage}
\label{table:mylar.lognor.mles}
\end{table}
%----------------------------------------------------------------------
%splus>ML estimation results
%splus>
%splus>Data: Mylar-Polyurethane Insulating Structure 
%splus>Data units: Minutes 
%splus>Distribution:  Lognormal 
%splus>Relationship:  log 
%splus>Log Likelihood at maximum: -271.4247 
%splus>
%splus>Parameter Estimation Results 
%splus>             MLE        se   z-ratio   95% lower  95% upper 
%splus>   b0  27.491758 2.9971602  9.172602  21.6173243  33.366192
%splus>   b1  -4.289109 0.5991313 -7.158879  -5.4634061  -3.114811
%splus>sigma   1.049793 0.1237193  8.485281   0.8332721   1.322576
%splus>
%splus>Variance-covariance matrix
%splus>                 b0            b1         sigma 
%splus>   b0  8.982969e+00 -1.792630e+00 -1.376619e-08
%splus>   b1 -1.792630e+00  3.589583e-01  2.385898e-09
%splus>sigma -1.376619e-08  2.385898e-09  1.530647e-02
\end{example}
%-----------------------------------------------------------------------------
\subsection{Analysis of interval ALT data}
\label{section:interval.data.alt}
Previous examples have shown that interval (or read-out) data occur
frequently in reliability studies.  Such data also arise in
accelerated tests. With a complicated evaluation process and limited
resources, it is often possible to do only a few inspections on each
unit.
\begin{example}
{\bf Analysis of ALT data on a new-technology IC device.}
\label{example:new.tech.ic}
Appendix Table~\ref{atable:icdevice2.data} gives data from an
accelerated life test on a new-technology integrated circuit (IC)
device. The device inspection involved an expensive electrical
diagnostic test.  Thus only a few inspections could be conducted on
each device. One common method of planning the times for such
inspections is to choose a first inspection time and then space the
inspections such that they are equally spaced on a log axis. In this
case, the first inspection was after one day with subsequent
inspections at two days, four days, and so on (except for one day
when the person doing the inspection had to leave early). Tests were
run at 150, 175, 200, 250, and 300$\degreesc$.  Failures had been
found only at the two higher temperatures.  After an initial
analysis based on early failures at 250$\degreesc$ and
300$\degreesc$, there was concern that no failures would be observed
at 175$\degreesc$ before the time at which decisions would have to
be made. Thus the 200$\degreesc$ test was started later than the
others to assure some failures and only limited running time on
these units had been accumulated by the time of the analysis.

The developers were interested in estimating the activation energy
of the failure mode and the long-life reliability of the
ICs. Initially engineers asked about ``MTTF'' at use conditions of
100$\degreesc$ junction temperature. After recognizing that the
estimate of the mean would be on the order of 6 million hours (more
than 700 years) decided that this would not be a useful reliability
metric. Subsequently they decided that the average hazard rate or
the proportion that would fail by 100 thousand hours (about 11
years) would be more useful for decision-making purposes.

Figure~\ref{figure:icdevice02.groupi.lognor.ps} is a lognormal
probability plot of the failures at 250$\degreesc$ and
300$\degreesc$ along with the ML estimates of the individual
lognormal cdfs.
%----------------------------------------------------------------------
\begin{figure}
\splusbookfigure{\figurehome/icdevice02.groupi.lognor.ps}
\caption{Lognormal
probability plot of the failures at 250$\degreesc$ and 300$\degreesc$
for the new-technology integrated circuit device ALT experiment.}
\label{figure:icdevice02.groupi.lognor.ps}
\end{figure}
%----------------------------------------------------------------------
Table~\ref{table:icdevice2.lognor.indiv.mles} summarizes the
individual lognormal ML estimates. 
\begin{table}
\caption{Individual lognormal ML estimation 
results for the new-technology IC device.}
\centering\small
\begin{tabular}{ccrrrr}
\\[-.5ex] \hline
 & & & & \multicolumn{2}{c}{95\% Approximate}\\ 
& &\multicolumn{1}{c}{ML}
&Standard & \multicolumn{2}{c}{Confidence Intervals}\\  \cline{5-6}
& Parameter & Estimate & \multicolumn{1}{c}{Error} & Lower & Upper \\
\hline 
250$\degreesc$ &$\mu $ & 8.54 & .33 &7.9 &9.2 \\[.7ex] 
 &$\sigma$ & .87 & .26 & .48 & 1.57 \\
\hline 
300$\degreesc$ &$\mu $ & 6.56 & .07 &6.4 &6.7 \\[.7ex] 
 	&$\sigma$ & .46 & .05 & .36 & .58 \\
\hline
\\[-1.8ex]
\end{tabular}
\begin{minipage}[t]{4in}
The log-likelihood values were $\loglike_{250}= -32.16$
and $\loglike_{300}= -53.85$. The confidence intervals 
are based on the normal-approximation method.
\end{minipage}
\label{table:icdevice2.lognor.indiv.mles}
\end{table}
%splus>Data: HTGB   ALT   Data subset 250 
%splus>Data units: Hours 
%splus>Distribution:  Lognormal 
%splus>Log Likelihood at maximum: -32.15036 
%splus>Parameter Estimation Results 
%splus>           MLE        se   z-ratio 95% lower 95% upper 
%splus>   mu 8.539997 0.3337729 25.586248 7.8858021  9.194192
%splus>sigma 0.870981 0.2621932  3.321906 0.4827976  1.571275
%splus>
%splus>Data: HTGB   ALT   Data subset 300 
%splus>Data units: Hours 
%splus>Distribution:  Lognormal 
%splus>Log Likelihood at maximum: -53.84794 
%splus>
%splus>Parameter Estimation Results 
%splus>            MLE        se   z-ratio 95% lower 95% upper 
%splus>   mu 6.5632715 0.0707790 92.729082 6.4245447 6.7019984
%splus>sigma 0.4571633 0.0548708  8.331632 0.3613304 0.5784133
The different slopes in the plot suggests that the lognormal shape
parameter $\sigma$ changes from 250 to 300$\degreesc$. Such a change
could be caused by the occurrence of a different failure mode at high
temperatures, casting doubt on the simple first-order Arrhenius model.
Failure modes with a higher activation energy, that
might never be seen at low levels of temperature, can appear at higher
temperatures (or other accelerating variables).  A 95\% confidence
interval for the ratio $\sigma_{250}/\sigma_{300}$ is $[1.01, \quad
3.53]$ (calculations requested in
Exercise~\ref{exercise:sigma.ratio}), suggesting that there could be a
real difference. These results also suggested that detailed physical
failure mode analysis should be done for at least some of the failed
units and that the accelerated test should be extended until some
failures are observed at lower levels of temperature. Even if the
Arrhenius model is questionable at 300$\degreesc$, it could be
adequate below 250$\degreesc$.

%----------------------------------------------------------------------
\begin{figure}
\splusbookfigure{\figurehome/icdevice02.groupm.lognor.ps}
\caption{Lognormal probability plot showing the ML fit of the 
Arrhenius-lognormal model for the new-technology IC device.}
\label{figure:icdevice02.groupm.lognor.ps}
\end{figure}
%----------------------------------------------------------------------
Table~\ref{table:icdevice2.lognor.arrhen.mles} gives
Arrhenius-lognormal model ML estimates for the new-technology
IC device.
\begin{table}
\caption{Arrhenius-lognormal model ML 
estimation results for the new-technology IC device.}
\centering\small
\begin{tabular}{crrrr}
\\[-.5ex] \hline
 & & & \multicolumn{2}{c}{95\% Approximate}\\ &\multicolumn{1}{c}{ML}
&Standard & \multicolumn{2}{c}{Confidence Intervals}\\ \cline{4-5} 
Parameter &
Estimate & \multicolumn{1}{c}{Error} & Lower & Upper \\
\hline 
 $\beta_{0}$ & $-10.2$ & $1.5$ & $-13.2$ & $-7.2$ \\[.7ex] 
$\beta_{1}$ & $.83$ & .07 & $.68$ & $.97$ \\[.7ex] 
$\sigma$ & $.52$ & $.06$ & $.42$ & $.64$ \\
\hline
\\[-1.8ex]
\end{tabular}
\begin{minipage}[t]{4in}
The log likelihood is $\loglike= -88.36$. The confidence intervals 
are based on the normal-approximation method.
\end{minipage}
\label{table:icdevice2.lognor.arrhen.mles}
\end{table}
% splus> Distribution:  Lognormal 
% splus> Relationship:  arrhenius2 
% splus> Log Likelihood at maximum: -88.3578 
% splus> 
% splus> Parameter Estimation Results 
% splus>               MLE         se    z-ratio   95% lower  95% upper 
% splus>    b0 -10.1721601 1.52700302  -6.661519 -13.1650861 -7.1792342
% splus>    b1   0.8265388 0.07319185  11.292770   0.6830828  0.9699948
% splus> sigma   0.5165083 0.05747376   8.986853   0.4152974  0.6423849
%----------------------------------------------------------------------
\begin{figure}
\splusbookfigure{\figurehome/icdevice02.altplot.lognor.ps}
\caption{Arrhenius plot showing the new-technology IC device data and 
the Arrhenius-lognormal model ML estimates.
Censored observations are indicated by $\Delta$.}
\label{figure:icdevice02.altplot.lognor.ps}
\end{figure}
%----------------------------------------------------------------------
Figure~\ref{figure:icdevice02.groupm.lognor.ps} is a lognormal
probability plot showing the Arrhenius-lognormal model fit to the
new-technology IC device ALT data. This figure shows lognormal cdf estimates
for all of the test levels of temperature as well at
the use-condition of 100$\degreesc$.

Following the approach used in
Example~\ref{example:devicea.model.lrt}, we use the ``omnibus'' test
to compare the constrained and unconstrained models for estimating the
lognormal distributions at the different levels of temperature.  From
Table~\ref{table:icdevice2.lognor.indiv.mles}, for the unconstrained
model, $\loglike_{\rm unconst}=\loglike_{250} +
% splus >   -2*(-88.36 +86.01 )
\loglike_{300}  = -86.01$ and from
Table~\ref{table:deva.arr.lognor.mles}, for the constrained model,
$\loglike_{\rm const}= -88.36$.  In this case the comparison has just
one degree of freedom (i.e., $dof=4-3=1$) and the test statistic is
$Q=-2(-88.36 +86.01 ) =4.7>
\chisquare_{(.95;1)} =3.84$. This indicates that there 
is some lack of fit in the constant-$\sigma$ Arrhenius-lognormal
model. Note that there were no failures at 150, 175, or
200$\degreesc$.  These results were implicitly in the likelihood.
Because the result of 0 failures at these temperatures is consistent
with the model and the other data, the computed likelihood at these
three temperatures would be very close to 1 and the computed values
of $\loglike$ are therefore very close to 0. Thus the results have
no direct effect on the ML fit. A result of 0 failures at any
temperature greater than 250$\degreesc$ combined with the rest of
these data would, however, have had a strong effect on the fit.

Figure~\ref{figure:icdevice02.altplot.lognor.ps} is an Arrhenius
plot of the Arrhenius-lognormal model fit to the IC new-technology
device data.  This plot shows the rather extreme extrapolation
needed to estimate the failure-time distribution at the use
conditions of 100$\degreesc$.  If the projections are close to the
truth, it appears unlikely that there will be any failures below
200$\degreesc$ during the remaining 3000 hours of testing and, as
mentioned before, this was the reason for starting some units at
200$\degreesc$.
\end{example}

In some applications, temperature-accelerated life tests are run
with only one level of temperature. Then a given value of activation
energy is used to compute an acceleration factor to estimate life at
use temperature. Resulting confidence intervals are generally
unreasonably precise because activation energy is generally not know
exactly.  For example, MIL-STD-883 provides reliability
demonstration tests based on a given value of $\Ea$.

\begin{example}
\label{example:new.tech.ic.beta.known}
{\bf Estimation with activation energy given.}
Figure~\ref{figure:icdevice02.groupm.lognor.fixedea.ps}, similar to
Figure~\ref{figure:icdevice02.groupi.lognor.ps} shows the effect of
assuming that $\Ea=.8$ eV and having to estimate only $\beta_{0}$
and $\sigma$ from the limited data. Using a given $\Ea$ results in a
set of approximate 95\% confidence intervals for $F(t)$ at
100$\degreesc$ that are unrealistically narrow. See
Section~\ref{section:bayes.acceleration} for alternative analyses of
these data.
\end{example}
%----------------------------------------------------------------------
\begin{figure}
\splusbookfigure{\figurehome/icdevice02.groupm.lognor.fixedea.ps}
\caption{Lognormal probability plot showing the Arrhenius-lognormal model ML 
estimates and 95\% confidence intervals for $F(t)$ at $100\degreesc$ for the 
new-technology IC device with given 
$\Ea=.8$.}
\label{figure:icdevice02.groupm.lognor.fixedea.ps}
\end{figure}
%----------------------------------------------------------------------

%-----------------------------------------------------------------------------
\subsection{Analysis of a two-variable ALT}
\label{section:two.factor.alt}
%----------------------------------------------------------------------
This section shows how to analyze AT data with two 
experimental variables. In this case, both variables
were thought to be accelerating. The methods
illustrated apply to experiments with any number of variables.

Appendix Table~\ref{atable:tantalum.alt.data} contains
temperature/voltage ALT data on tantalum electrolytic capacitors.
These data come from Singpurwalla, Castellino, and Goldschen~(1975).
Tests were conducted at temperature/voltage combinations that were
nonrectangular and with unequal allocations of units.
Figure~\ref{figure:tantalum.scatter.ps} is a scatter-plot of hours
to failure versus voltage with temperature indicated by different
symbols in the plot (with some jitter used in voltage to help in
viewing ties in the data).  The amount of censoring is indicated in
the top margin.  There were various censoring times that are given
in Table~\ref{atable:tantalum.alt.data}.
%----------------------------------------------------------------------
\begin{figure}
\splusbookfigure{\figurehome/tantalum.scatter.ps}
\caption{Scatter-plot of failures in the tantalum capacitor data
showing hours to failure versus voltage with temperature indicated by
different symbols.}
\label{figure:tantalum.scatter.ps}
\end{figure}
%----------------------------------------------------------------------

Figure~\ref{figure:tantalum.groupi.ps} is a multiple Weibull
probability plot for the individual combinations of voltage and
temperature for the tantalum capacitor data. The plot also shows
individual ML estimates of Weibull cdfs for those combinations
having more than one failure. The line for 85$\degreesc$ and 46.5
Volts is much steeper than the others, but this line results from
only two out of 50 capacitors failing. Thus, this deviation in the
slopes could be due to random variability.  The Weibull distribution
seems to provide a reasonable model for the failure-time
distribution at those conditions with enough failures to make a
judgment.  Figure~\ref{figure:tantalum.groupi.ps} shows that units
fail more rapidly at high voltage. Any possible temperature effect
is not as strong.

%----------------------------------------------------------------------
\begin{figure}
\splusbookfigure{\figurehome/tantalum.groupi.ps}
\caption{Weibull probability 
plot for the individual combinations of voltage and temperature for
the tantalum capacitor data, along with ML estimates of Weibull
cdfs.}
\label{figure:tantalum.groupi.ps}
\end{figure}
%----------------------------------------------------------------------
The Arrhenius-inverse power relationship-Weibull model
(Section~\ref{section:temp.volt.acc.model}) both without the
interaction term (Model 1) and with the interaction term (Model 2),
were fitted to these data. In particular, the fitted relationships were
\begin{eqnarray*}
\mu &=& \beta_{0} + \beta_{1} x_{1}+\beta_{2} x_{2}\\
\mu &=& \beta_{0} + \beta_{1} x_{1}+\beta_{2} x_{2}+ \beta_{3} x_{1} x_{2}
\end{eqnarray*}
where $x_{1}=11605/(\Tempk{})$, $x_{2}=\log(\Volt)$, and
$\beta_{1}=\Ea$.  The results for Model 1 are depicted in
Figure~\ref{figure:tantalum.groupm.ps} and the results for Models 1
and 2 are summarized numerically in Table~\ref{table:tantalum.mles}.
Comparing the log-likelihood values in Table~\ref{table:tantalum.mles}
indicates that the interaction term in Model 2 is not helpful in
explaining variability in these data (correspondingly, the
confidence interval for $\beta_{3}$ contains zero). There is
strong evidence for an important voltage effect on life.  There is
also some evidence for a temperature effect, but the evidence is not
strong. Physical theory, however, predicts the positive coefficient
$\beta_{1}$; the lack of strong evidence could be the result of a small
number of failures at most variable-level combinations and the
odd-shaped experimental region.

In Table~\ref{table:tantalum.mles}, it is interesting that the
coefficient estimates of the regression model are highly sensitive
to whether the interaction term is included in the model or
not. This is due, in part, to the highly unbalanced allocation in
the experiment.
Figure~\ref{figure:tantalum.residual.linear.weib.ps} is a Weibull
probability plot of the regression residuals for Model 1.  The
Weibull distribution appears to provide a reasonable description of
the variability in these data.
Figure~\ref{figure:tantalum.cellquan.ps} shows Model 1 estimates of
the .01 quantile of the life distribution of the tantalum capacitors
as a function of voltage, for the three temperatures in the data (5,
45, and 85$\degreesc$).  The plotted points give the individual ML
estimates of the .01 quantile at the temperature/voltage
combinations that had more than one failure. This figure shows the
relatively strong effect of voltage, relative to the effect of
temperature. The lines in this plot are parallel because there is no
interaction term in Model 1.
%----------------------------------------------------------------------
\begin{figure}
\splusbookfigure{\figurehome/tantalum.groupm.ps}
\caption{Weibull multiple probability plot showing the fitted
Arrhenius-inverse power relationship Weibull model (with no interaction) for
the tantalum capacitor data. The dotted lines
are 95\% confidence limits for $F(t)$ at use conditions.}
\label{figure:tantalum.groupm.ps}
\end{figure}
%----------------------------------------------------------------------

%----------------------------------------------------------------------
\begin{table}
\caption{Tantalum capacitor Weibull-inverse power relationship 
regression ML estimation results.}
\centering\small
\begin{tabular}{lcrrrr}
\\[-.5ex] \hline
& & & & \multicolumn{2}{c}{95\% Approximate}\\
&&\multicolumn{1}{c}{ML} &Standard & \multicolumn{2}{c}{Confidence
Interval}\\ \cline{5-6}
& Parameter & Estimate&
\multicolumn{1}{c}{Error} & Lower & Upper \\
\hline 
Model 1 
&$\beta_{0}$ &84.4 &13.6 &$57.8$ &111. \\[.7ex] 
&$\beta_{1}$ &$.33$  &.19 &$-.04$ &$.69$ \\[.7ex] 
&$\beta_{2}$ &$-20.1$  &4.4  & $-28.8$  & $-11.4$\\[.7ex] 
&$\sigma$ &2.33 &.36 &1.72 & 3.16 \\[1.2ex]
\hline 
Model 2 
&$\beta_{0}$ &-78.6 &109.0 &$-292.3$ &135.1 \\[.7ex] 
&$\beta_{1}$ &$5.13$ &3.3 & $-1.35$ &11.6 \\[.7ex]
&$\beta_{2}$ & $19.9$ &26.7 &$-32.5$ &72.35 \\[.7ex]  
&$\beta_{3}$ &$-1.17$ &$.80$ & $-2.8$ &$.40$ \\[.7ex] 
&$\sigma$ &2.33 &.36 &1.72 &3.16 \\[1.2ex]
\hline 
\\[-1.8ex]
\end{tabular}
\begin{minipage}[t]{4in}
The log likelihoods for Models 1 and 2 are, 
respectively, $\loglike_{1}=-539.63 $
and $\loglike_{2}=-538.40 $. The confidence intervals 
are based on the normal-approximation method.
\end{minipage}
\label{table:tantalum.mles}
\end{table}

%----------------------------------------------------------------------
\begin{figure}
\splusbookfigure{\figurehome/tantalum.residual.linear.weib.ps}
\caption{Weibull probability plot of the residuals from the Arrhenius-inverse 
power relationship Weibull model (with no interaction) for
the tantalum capacitor data.}
\label{figure:tantalum.residual.linear.weib.ps}
\end{figure}
%----------------------------------------------------------------------


%----------------------------------------------------------------------
\begin{figure}
\splusbookfigure{\figurehome/tantalum.cellquan.ps}
\caption{Log-log plot of the ML estimates of the
.01 quantile of tantalum capacitor life
based on data fitted to the 
Arrhenius-inverse power relationship Weibull model with no-interaction.}
\label{figure:tantalum.cellquan.ps}
\end{figure}
%----------------------------------------------------------------------

%----------------------------------------------------------------------
%----------------------------------------------------------------------
\section{Some Practical Suggestions for Drawing Conclusions
from ALT Data}
\label{section:practical.suggest.alt}
%----------------------------------------------------------------------

Due to their extrapolative nature,
drawing conclusions from ALT data can be difficult.
This section describes some cautions.

\subsection{Predicting product performance}

It is particularly difficult to use AT data to predict actual product
performance.  This is because most products tend to be complicated
combinations of materials and components, AT data are mostly
successful for estimating life distributions of simple components and
materials.  Extrapolation is needed to make predictions on the life
distribution of a product in the field environment (which, itself, may
be difficult to predict).  Extrapolation is difficult or impossible to
justify completely.

ALT experiments should be planned and executed with a great deal of
care.  Inferences and predictions should be made with a great deal of
caution.  Some particular suggestions for doing this are
\begin{itemize}
\item
Use previous experience with similar products and materials.
\item
Conduct initial studies (pilot experiments) to evaluate the effect
that the accelerating variable or variables will have on degradation and
the effect that degradation will have on life or performance.
Information from preliminary tests provide useful input for planning
ALTs (as described in Chapter~\ref{chapter:alt.test.planning}).
\item
Use failure mode analysis and physical/chemical theory to improve or
develop physical understanding to provide a better physical basis for
ALT models.
\item
Limit, as much as possible, the amount of extrapolation (in time and
in the accelerating variable). Methods for doing this are described in
Chapter~\ref{chapter:alt.test.planning}.

\end{itemize}

%-----------------------------------------------------------------------------
\subsection{Drawing conclusions from accelerated test experiments}

A typical AT is an extreme example of what Deming~(1975) calls an
``analytic study.''  As such, when making predictions from an AT, one
has to question the reasonableness of using the AT
manufacturing/testing process to represent the actual
manufacturing/use process and the adequacy of the life-acceleration
model.  Typically, confidence intervals based on analytic studies
provide, at best, only a lower bound on the total variability and
uncertainty; see Hahn and Meeker~(1991, Chapter 1 for more detail).
For example, such intervals do {\em not} account for model inadequacy.
Extrapolation will amplify model errors, often dominating
other sources of uncertainty.

%----------------------------------------------------------------------
%----------------------------------------------------------------------
\section{Other Kinds of Accelerated Tests}
\label{section:other.kinds.alt}
The first parts of this chapter have discussed traditional ATs that
are used to obtain timely information on a product life distribution
at use  conditions by testing units at one or more higher levels
of accelerating variables. Usually such tests are done on materials,
devices, other components, or relatively simple subsystems or
systems where the focus is one or a small number of known failure
modes.  The primary purpose of these tests has been to assess the
failure-time distribution of the failure mode(s) of interest. Such
information is then used to characterize product life at use
conditions and to make product design, warranty coverage, and other
decisions.

There are several other important types of accelerated reliability
testing, particularly in the electronics industry.  These additional
types have also been called ``Accelerated Tests.''  Their purpose is
generally other than direct estimation of reliability.
Nevertheless, these other tests do generate data that can, in some
cases, provide useful information for reliability estimation.

%----------------------------------------------------------------------
\subsection{Continuous operation product accelerated testing}
Assuring the reliability of individual components is generally not
sufficient to assure the reliability of the larger product or other
system within which the components are installed. In particular, it
is necessary to assure that components, subsystems, and various
interfaces work together.  Acceleration is more difficult when
testing a complete system and acceleration factors will generally be
small (i.e., 2 to 10).  During a system test, there are more
potential failure modes.  Relative to materials or component tests,
there generally has to be more stringent limitations on how much
accelerating variables can be increased.  One must be careful not to
cause damage to the system (e.g., melting of components,
over-heating causing a rapid change in material properties, or other
damage that would not accurately simulate degradation at use
conditions).  Usually acceleration is achieved by running the system
more or less continuously. For example a refrigerator might be
tested, with focus on the compressor, by running the cooling system
continuously.  Similar tests are used for systems like washing
machines and automobile engines (where the purpose of the test might
be either to accelerate the life of the engine itself or of the
engine's lubricant).


Usually accelerated systems tests are started on prototype units or
early production units, before the product is introduced into the
field. For some products, however, tests will continue for some
period of time after units have been introduced into the field (to
protect against failure modes that will not show up until later in
life).  If a design change is being contemplated new tests may
be needed.

Sometimes a manufacturer will design an accelerated system test to
be run on an audit basis to check the output of the production
process over time. For example, two units might be selected from
production at the beginning and end of each week.  These units would
be run at stressful conditions for some period of time (e.g., one
month) to see if any early failures occur (which might indicate that
there is a problem in the manufacturing process).

Manufacturers have developed a number of strategies for accelerated
testing of complete systems.  The characteristics of system
accelerated tests are often product and failure mode specific.  In
addition to testing units at higher than usual use rates, systems
might be placed in a more stressful environment (e.g., high
temperature and humidity, or higher or lower voltage).  A
manufacturer of electric garbage disposal system tests the unit by
having it ``chew'' on hard plastic cubes to provide a more stressful
kind of ``garbage.''  If the motor withstands an equivalent amount
of service with the more stressful garbage, then, in this case, the
reliability engineers believe that the product will not experience
motor or bearing wearout problems in actual use.

To test the starting system of an
automobile, the engine could be started and stopped. Such a
start-stop test would also tend to put special stresses on the
engine (or lubricant) itself, and might be used to study different
failure modes than the continuous test.  
The manufacturer of laser printers has the printer print a ream of
paper, rest two minutes and then continue printing. A sump pump
needs to be tested in humid environments. Continuous tests will
track some kinds of failure modes quite well. One does, however, run
the risk that continuous operation can actually inhibit some
failure-causing processes.

%----------------------------------------------------------------------
\subsection{Highly accelerated life tests}

When planning ALTs to make projections about life at use conditions,
Meeker and Hahn~(1985) suggest that tests should be planned to
minimize the amount of extrapolation in both the accelerating
variable and in time. In other applications, where life information
at use conditions is not needed, Highly Accelerated Life Tests
(HALTs) may be useful. Confer, Canner, and Trostle~(1991) discuss
the use of extremely high temperature (up to 150$\degreesc$) and
voltage (8 times rating) to achieve acceleration factors of up to
2555 times.  They suggest that such tests can be used as (a) a means
of sampling inspection for incoming component lots and (b) as
burn-in screening test (in this application there is danger that
high levels of the accelerating variable will damage units that are
to be put into service).  Other applications include (c) pilot tests
to get information needed for planning a more extensive ALT at lower
levels of the accelerating variable and (d) experiments to obtain
information on the relevance of failure modes discovered in STRIFE
testing (see next section).  When using very high levels of the
accelerating variable(s), one must watch for failure modes that
would never occur at use conditions and have concern for the
adequacy of the model (Chapter 7 of Nelson~1990a shows how modeling
might be used to deal with multiple failure modes in ALTs).


%----------------------------------------------------------------------
\subsection{Environmental stress testing}

The pressure to quickly develop new, high reliability products has
motivated the development of new product testing methods. The
purpose of these testing methods is to quickly identify and
eliminate potential reliability problems early in product
development.  One such testing method is known as STRIFE
(STRESS-LIFE) testing.  The basic idea of STRIFE testing is to
aggressively stress and test prototype or early production units to
force failures. It is common to test only one or two units, but more
test units can provide important additional information, for
example, on unit-to-unit variation.  Typical STRIFE tests use
combinations of temperature and vibration cycling. The amplitude of
the cycling is increased continuously until the end of the
test. When possible, use-rate may also be increased.  Such a test
could be run for days, provided appropriate fixes for detected
failure modes can be effected without long delays that might be
needed for a complicated redesign (as opposed to a simple part
substitution). Bailey and Gilbert~(1981) report an example in which
the complete STRIFE test and improvement program was successfully
completed in three weeks.  Nelson~(1990a, pages 37-39) describes
environmental stress tests as ``elephant tests'' and describes some
important issues.  The {\em Proceedings of the Institute of
Environmental Sciences} often contain papers on this subject as do
various journals on reliability of electronics systems.

Generally, failures in STRIFE testing are due to product or process
design flaws. When there is a failure in a STRIFE test it is necessary
to find and carefully study the failure's root cause.  First it is
necessary to assess whether if the failure could occur in actual use or not.
Knowledge and physical/chemical modeling of the particular failure
mode is useful for helping to make this assessment.  Nelson~(1990a,
page 38) describes an example where a costly effort was made to remove
a high-stress-induced failure mode that never would have occurred in
actual use.  The occurrence of such failures might indicate that the
test is using some combinations of accelerating variables that are too
high to be useful.  When it is determined that a failure could occur
in actual use, it is necessary to change the product design or
manufacturing process to eliminate that cause of failure.  In some
cases, the fix is obvious.  In other cases additional focused research
and experimentation at the component level or at component interfaces
may be required.

Because the results of STRIFE testing are used to make changes on
the product design and manufacturing process, it is difficult, or at
the very least, very risky to use the test data to predict what will
happen in normal use. Even so, ideas from statistical experimental
design and models relating stress to life could be useful in
choosing stresses, stress ramp speed, and other aspects of the test.

In other testing programs, the goal is to test prototype or
early-production units at somewhat accelerated conditions to obtain
information on field performance of a product. For example, a newly
designed automobile engine may be run continuously at high rpm to
simulate rapidly 50 thousand miles of service. The danger in
interpreting the results of such tests is that the assumed
acceleration factors can be seriously inaccurate. Different failure
modes have different acceleration factors. For example, such an
accelerated test may accurately predict a wear mechanism, but not even
discover a corrosion mechanism. In another accelerated test
application, humidity was used to accelerate a corrosion mechanism,
but had the unexpected effect of reducing wear rate. For these
reasons, it is important to have a good understanding of the
physics and/or chemistry of possible failure mechanisms.

Schinner~(1996) describes and gives examples of system-level and
system-level accelerated tests such as STRIFE.

%----------------------------------------------------------------------
\subsection{Burn-in}
The most common reliability problem for manufacturers and consumers
of electronic equipment has been early (infant mortality) failures.
Such failures are typically caused by manufacturing defects which
often appear in only a small proportion of the manufactured
units. In electronic manufacturing such defective components are
called ``freaks.'' The problem of early failure also arises in other
kinds of products. Such problems are sometimes referred to as a
``quality problem.''  Often such problems are cured as a product's
design and manufacturing process matures. Manufacturers would prefer
to ``build-in'' reliability by eliminating all manufacturing defects
from the start of production. With rapidly changing technology,
however, it has been difficult to achieve a goal of zero defects,
particularly early in the product development cycle.  To achieve
sufficiently high reliability, particularly in critical applications
(e.g., space and under-sea systems), it has been common practice to
use burn-in of components and systems to screen out the units that
would otherwise fail early in life. For components like integrated
circuits it is common to do burn-in at high humidity and
temperature. For system burn-in, it is generally necessary to avoid
the use of high levels of the accelerating variables to avoid
damaging sensitive components. In either case, burn-in may also
involve continuous operational exercising and monitoring of the
units. Such burn-in is useful for detecting intermittent failure
modes that have a low probability of being detected in ordinary
testing.

Burn-in can be viewed as a type of 100\% inspection or screening of
the product population to eliminate or reduce the number of
defective items going to customers.  Burn-in may be necessary if the
output of production does not meet reliability specifications.  It
is important that the burn-in stress or temperature not be so high
as to damage the good units.  Burn-in is expensive and thus the
length of the burn-in is typically limited.  The decision on how
long to run a burn-in can be based on the desired level of
reliability and the distribution of the observed failures during the
screening. For a stable production process, which has been
adequately characterized, this can be set in advance.  Otherwise a
sequential or dynamic stopping rule may be needed.  Kuo~(1984)
provides and applies a cost-optimization model for important burn-in
decisions.  He also reviews previous literature on this
subject. Most of the literature deals with burn-in at use
conditions.  Kuo, Chien, and Kim~(1998) describe methods for
implementing and optimizing the use of burn-in for electronic
components and systems.


Jensen and Petersen~(1982) provide an engineering approach to this
subject. Also, see Nelson~(1990a, page 43).  Statistical methods can
and have been useful for helping to choose stress levels, length of
burn-in, and in using burn-in data to assess the state of the
production process and the likely field reliability of a product
going into service.

%----------------------------------------------------------------------
\subsection{Environmental Stress Screening}

Environmental Stress Screening (ESS) was developed as an improvement
over traditional burn-in methods. ESS provides a more economical and
more effective means of removing defective units from a product
population when testing units at the system or subsystem (e.g.,
circuit board) level.  Because systems and subsystems cannot tolerate
high levels of stress for long periods of time, ESS uses mild, but
more complicated stressing. High levels of temperature and humidity at
the component level are replaced by more moderate temperature cycling,
physical vibration, and perhaps stressful operational regimes (e.g.,
running computer CPU chips at higher than usual clock speeds and lower
than usual voltages) to help identify the defective units.  These
tests can be viewed as generalizations of step-stress tests (but their
purposes are much different).  The tests are sometimes called ``shake
and bake'' tests.  Again, the goal is to screen out, as effectively
and as quickly as possible, the defective items without otherwise
doing harm to the product.

Numerous articles on ESS testing appear each year in the {\em
Proceedings of the Institute of Environmental Sciences}.
Tustin~(1990) provides a motivational description of the methodology
and several references.  Nelson~(1990a, page 39) gives additional
references, including military standards.  Kececioglu and Sun~(1995)
provide a comprehensive description of ESS methods, including
optimization and management of ESS programs. MIL-STD-2164
describes standard procedures for ESS of electronic equipment.

Statisticians have had little impact in the development of ESS
methods because they are mostly based on engineering
knowledge. There are, however, some areas where statistical methods
could have an impact on ESS. Some of these will require the
development of better models to relate the effect that complicated
stressing has on the life distribution of both the defective and the
nondefective units.  For example, advanced application of the
statistical principles of experimental design, modeling, and data
analysis can be used to help:
\begin{itemize}
\item
Choose stress conditions that are best for weeding out manufacturing
defects, but minimize the chance of doing damage to good units.
\item
Design screens to provide feedback that can be used to improve
product design or the manufacturing process by reducing the frequency of
manufacturing defects or eliminating them.
\item
Assess information that would allow prediction of field reliability.
The typically complicated stress patterns make it difficult or
impossible to use ESS data directly to make predictions about field
performance.  However, with enough screening data, correlated with
field data, one could find relationships that would allow such
predictions.
\end{itemize}
The development of physical/statistical models to describe the
effect of ESS stress patterns on product life would be useful.  This
task, will, however, not be easy.  LuValle and Hines~(1992) report
experimental evidence indicating that varying several accelerating
variables in typical ESS procedures leads to complicated effects on
life.  There can be interactions among the variables and there may
be ``memory'' of past stress patterns in that, from a particular
point in time, future degradation may depend not only on the current
amount of degradation, but also on how (i.e., under which stress
patterns) that degradation has accrued.  This is in contrast to
traditional step-stress failure models (e.g., Chapter 10 of Nelson
1990a).

Like burn-in, ESS is an inspection/screening scheme.  In line with the
modern quality precept of eliminating reliance on mass inspection,
most manufacturers would prefer not to use burn-in or ESS. They are
expensive and may not be totally effective.  By improving the
reliability through continuous improvement of the product design and
the manufacturing process, it is often possible to reduce or eliminate
reliance on screening tests except, perhaps, in the most critical
applications. Some companies apply
ESS only on an audit basis to monitor production quality on an
on-going basis. 


%--------------------------------------------------------------------------
%--------------------------------------------------------------------------
\section{Potential Pitfalls of Accelerated Life Testing}
%--------------------------------------------------------------------------
\label{section:alp.pitfalls}
As described earlier in this chapter, accelerated life testing can
be a useful tool for obtaining timely information about materials
and products.  There are, however, a number of important potential
pitfalls that could cause an ALT to lead to seriously incorrect
conclusions. Users of ALTs should be careful to avoid these pitfalls.

\subsection{Pitfall 1: Multiple (unrecognized) failure modes}

High levels of accelerating variables like temperature or voltage
can induce failure modes that would not be observed at normal
operating conditions.  In some cases new failure modes result from a
fundamental change in the way that the material or component
degrades or fails at high levels of the accelerating variable(s).
For example, instead of simply accelerating a failure-causing
chemical process, increased temperature may actually change certain
material properties (e.g., cause melting).  In less extreme cases,
high levels of an accelerating variable will change the relationship
between life and the accelerating variable (e.g., life at high
temperatures may not be linear in inverse absolute temperature, as
predicted by the Arrhenius relationship).

If other failure modes are caused at high levels of the accelerating
variables and this is recognized, it can be accounted for in the
data analysis by treating the failure for the new failure modes as a
censored observation (as long as the new failure mode does not
completely dominate the failure mode(s) of interest). Chapter 7 of
Nelson~(1990a) gives several examples. In this case, however, such
censoring can severely limit the information available on the
failure mode of interest.  If other failure modes are present but
not recognized in data analysis, seriously incorrect conclusions are
likely.

%--------------------------------------------------------------------------
\subsection{Pitfall 2: Failure to properly quantify uncertainty}
It is important to recognize that there is uncertainty in
statistical estimates. Basing decisions on point estimates alone
can, in many applications, be seriously misleading.  Standard
statistical confidence bounds quantify uncertainty arising from
limited data.  For example,
Figure~\ref{figure:icdevice02.groupm.lognor.ps} shows an enormous
amount of uncertainty in life at 100$\degreesc$, due to the small
number of failures, and the large amount of extrapolation in
temperature.  The corresponding analysis depicted in
Figure~\ref{figure:icdevice02.groupm.lognor.fixedea.ps} uses a given
value of activation energy for the life-temperature relationship.
Because the activation energy is not known exactly, the precision
exhibited in this plot is too small and potentially misleading. For
many applications, neither of these extremes would provide a proper
quantification of uncertainty.
Section~\ref{section:bayes.acceleration} describes an appropriate
compromise for situations where there is useful information about
activation energy.

It is also important to remember that statistical confidence bounds
do not account for model uncertainty (which can be tremendously
amplified by extrapolation in accelerated testing).  In general,
performing sensitivity analysis is an important step in any
quantitative analysis involving uncertainty and is particularly
useful for assessing the effects of model uncertainty.  For example,
one can rerun analyses under different assumed models to see the
effect that different model assumptions have on important
conclusions.

%--------------------------------------------------------------------------
\subsection{Pitfall 3: Multiple time scales and degradation affected
by more than one accelerating variable}

Section~\ref{section:def.of.time.scale} described issues relating to
time scales. These issues become even more important with
accelerated testing, and particularly when there is more than one
failure-causing mechanism that might be accelerated. Standard
acceleration methods generally will not accelerate all time scales
in the same manner. A serious pitfall of accelerated testing is to
assume a simple relationship between life and the accelerating
variables when the actual relationship if really very
complicated. Consider the following examples.

\begin{itemize}
\item
In an accelerated test to estimate the life-time characteristics of
a composite material, chemical degradation over time changes
material ductility.  Failures, however, are actually caused by
stress cycles during use, leading to initiation and growth of
cracks, and eventually to fracture. Thus there are two
failure-causing mechanisms. The acceleration model would be
complicated because the effect of cycling depends on the
material ductility and because increasing temperature would affect
the time scales of both mechanisms.
\item
An incandescent light bulb usually fails when its filament
breaks. During burn time the bulb's filament will go through an
evaporation process, eventually leading to failure.  There are,
however, other variables that can shorten a bulb's life. In
particular, on-off cycles can induce both thermal and mechanical
shocks that can, over time, lead to the growth of fatigue cracks in
the filament.  Thus the on-off frequency can have an effect on bulb
life.  Accelerating only the burn time (e.g., by testing at higher
voltage) may give misleading predictions of life in an environment
with many on-off cycles. Relatedly, light bulbs operated in an
environment with physical vibration (e.g., in automobiles, on large
ships, or in a motorized appliance) will often exhibit shorter life
times depending on the frequency and amplitude of the vibrations as
well as the bulb's design.

\item
The degradation of coatings like paint depends on a number of
different variables relating to time scales.  Most coatings degrade
chemically over time.  UV light accelerates the degradation process
of many kinds of coatings, as does high temperature and humidity.
The {\em number} of wet-dry and thermal cycles is also important to
coating life, but generally relates to a separation or peeling
failure-mechanism that is different from (but perhaps related to)
the chemical degradation mechanism. Each of these variables and each
failure mode has its own underlying time-scale.
\end{itemize}

Generally, there will be a distribution of product-use conditions
in the field.  For example, the number of fatigue cycles as a
function of the changing ductility of the composite material over
time or the ratio giving the number of on-off cycles per hour of
burn time for an incandescent light bulb. Similarly, some
automobiles are driven in the north and some in the south; some
spend substantial time in direct sunlight, others do not).  In these
situations, product-use environment plays an important, but
complicated, role in planning and making life predictions from
accelerated tests.

In a simple situations where the ratio of the time scales for
different mechanisms is known and reasonably constant in the product
population, an accelerated test could be conducted to simulate life
in that ratio.  When the ratio has a known distribution in the
product population, tests can be conducted over an appropriate range
of the ratio. In other applications, it will be necessary to use an
accelerated test in which accelerating variable (e.g., temperature,
humidity, and UV exposure simultaneously) are varied
simultaneously. To predict life at specified use conditions, one
needs an adequate physical model to describe the relationship among
these variables, the different degradation scales, and the
definition failure.

%--------------------------------------------------------------------------
\subsection{Pitfall 4: Masked failure mode}
\label{:section.masked.failure.mode}
Figure~\ref{figure:masked.failure.mode.ps} shows a graph of what
might illustrate the results of a typical accelerated life test
if there were just a single failure mode and if
increased temperature accelerated that failure mode in a simple
manner, described by the Arrhenius relationship.
%-------------------------------------------------------------------
\begin{figure}
\splusbookfigure{\figurehome/masked.failure.mode.ps} 
\caption{Possible results for a typical temperature-accelerated failure
mode on an IC device.}
\label{figure:masked.failure.mode.ps}
\end{figure}
%-------------------------------------------------------------------
It is possible that such an
accelerated test, while focusing on one known failure mode, 
may mask another!
This is illustrated in Figure~\ref{figure:unmasked.failure.mode.ps}.
%-------------------------------------------------------------------
\begin{figure}
\splusbookfigure{\figurehome/unmasked.failure.mode.ps} 
\caption{Failure Mode 2 with lower activation energy, masked at high
temperature and unmasked at low temperature.}
\label{figure:unmasked.failure.mode.ps}
\end{figure}
%-------------------------------------------------------------------
Moreover, as shown in Figure~\ref{figure:unmasked.failure.mode.ps},
it is often the masked failure mode that is the first one to show up
in the field. In such cases, the masked failure modes often
dominates among reported field failures.

%--------------------------------------------------------------------------
\subsection{Pitfall 5: Faulty comparison}

It is sometimes claimed that accelerated testing is not really
useful for predicting reliability, but is useful for comparing
alternatives (e.g., alternative designs, vendors, etc.). Consider
comparing similar products from two different vendors.  The thought
behind this claim is that laboratory accelerated tests generally
cannot be expected to adequately approximate actual use conditions,
but that if Vendor 1 is better than Vendor 2 in an accelerated test,
then the same would be true in field use, as illustrated in
Figure~\ref{figure:standard.comparison.ps}.
%-------------------------------------------------------------------
\begin{figure}
\splusbookfigure{\figurehome/standard.comparison.ps} 
\caption{Well-behaved comparison of two products.}
\label{figure:standard.comparison.ps}
\end{figure}
%-------------------------------------------------------------------
Comparisons based on ALTs are, however, subject to some of the same
difficulties as other ALTs. 
%-------------------------------------------------------------------
\begin{figure}
\splusbookfigure{\figurehome/faulty.comparison.ps} 
\caption{Comparison with evidence of different failure modes.}
\label{figure:faulty.comparison.ps}
\end{figure}
%-------------------------------------------------------------------
In particular, consider the results depicted in
Figure~\ref{figure:faulty.comparison.ps}.  In this case, Vendor 1
had longer life at both of the accelerated test conditions, but the
prediction at use conditions suggested that Vendor 2 would have
higher reliability.  An important decision on the basis of limited
results in this ALT would be, at best, difficult to justify.  It
would be most important to find out why the slopes are different and
to understand the life-limiting failure modes at use conditions. If
the failures at the use conditions are not the same as those at the
accelerated conditions, then the ALT results would be wrong. Also,
it might be possible that the early failures for Vendor 2 are
masking the failure mode that we see in Vendor 1's test results.
One cannot use an ALT to compare products that have different kinds
of failure modes.
%--------------------------------------------------------------------------
\subsection{Pitfall 6: Accelerating variables can cause deceleration!}

In some cases it is possible that increasing what is thought to be
an accelerating variable will actually cause deceleration! For
example, increased temperature in an ``accelerated'' circuit-pack
reliability audit predicted few field failures. The number of 
failures in the field was much higher than predicted
because the increased temperature resulted in lower humidity in the
``accelerated'' test and the primary failure mode in the field was
caused by corrosion that did not occur at high temperature and low
humidity.  It is for this reason that in most accelerated tests of
electronic equipment, both temperature and humidity need to be
controlled.

In another similar application, a higher than usual use-rate for a
mechanical device in an accelerated test inhibited a corrosion failure
mechanism. That corrosion mechanism
eventually caused a serious field problem that was
not predicted by the accelerated test.

In an accelerated test of a newly designed automobile air
conditioner, reliability, based on a series of constant-run accelerated
life tests, was predicted to be very high over a 5-year
period. However, after two years, a substantial number of the
in-service air conditioners failed due to a ``drying out'' material
degradation. These failure were caused by lack of use in winter and
had never been seen in the continuous accelerated testing.

\subsection{Pitfall 7: Untested design/production changes}

A new electro-mechanical device was to be used in a system designed
for 20 years of service in a protected environment.  
An accelerated test of the device was conducted and this test
``demonstrated'' 20 year life (no more than 10\% failing) under
normal operating conditions (typical use rate).
After the accelerated test, and as the product was going to
production, a material change was made by the device vendor.
The change led to a material-degradation failure mode
that caused (or would have caused) all in-service units to fail
within ten years. Eventually, all installed devices had to be replaced.

\subsection{Pitfall 8: Beware of drawing conclusions
on the basis of specially built prototype test units.}

Seriously incorrect conclusions can result from an accelerated life
test if test units will differ importantly from actual production
units.  For example, factory manufacturing conditions are different
from those in a laboratory. Cleanliness and care in building 
prototype versus production units may
differ substantially. Material and parts in prototype units might
differ from those that will be used in production. Highly trained
technicians may build prototype units that are importantly different
from units that would be made in the factory.

As much as possible, test units for an accelerated test should be
manufactured under actual production conditions, using raw materials
and parts, etc. that are the same as or as close as possible to
those that will be used in actual manufacturing of units. As much as
possible, the test units should reflect variabilities that will be
present in actual production.

In one situation, an accelerated test was conducted on 12 prototype
units.  The units contained epoxy that had to be cured in an oven
for a specified amount of time. The product passed its accelerated
test with a safe margin. In actual manufacturing operations,
however, the curing process was not well controlled. Uncured epoxy
can be highly reactive. For this product, a substantial proportion
of installed units eventually failed due to corrosion caused by
the improperly controlled curing.

%----------------------------------------------------------------------
%----------------------------------------------------------------------
\section*{Bibliographic Notes}
Nelson~(1990a) is an extensive and comprehensive source for further
material, practical methodology, basic theory, and examples for
accelerated testing.  Viertl~(1988) provides a briefer (and more
academic) overview of the available statistical methods for ALTs,
with more focus than Nelson~(1990a) on a large class of statistical
methods that, for a variety of practical reasons, seem not to have
been used widely in practice.  These methods include nonparametric
and semiparametric statistical methods. Viertl~(1988) also discusses
Bayesian methods for ALT planning and analysis.  Nelson~(1990a)
contains 431 references; Viertl~(1988) contains 208.  The
intersection is only 55 references.  Mann, Schafer, and
Singpurwalla~(1974, Chapter 9) overview ALT methods available at the
time.  Derringer~(1982) describes some important practical
considerations in the planning and analysis of ALTs. Meeker and
Escobar~(1993) survey important areas of research in Accelerated
Testing. LuValle~(1993) illustrated the use of graphical methods
that can be used to detect departures from the SAFT model.

Nelson~(1975a), Nelson~(1975b), and Nelson~(1990a, Chapter 7) describe
graphical and ML methods for analyzing ALT data with competing failure
modes.

%----------------------------------------------------------------------
%----------------------------------------------------------------------
\section*{Exercises}

%-----------------------------------------------------------------------------
\begin{exercise}
Explain the importance of having physical understanding of the physics
or chemistry of failure mechanisms when one is doing accelerated life
testing.
\end{exercise}

%-----------------------------------------------------------------------------
\begin{exercise}
\label{exercise:devicec.analysis}
The following table contains data from an accelerated life test on 
Device-C, an integrated circuit. Failures were caused by a chemical
reaction inside the circuit package. Reliability engineers tested 10
circuits at each temperature over a period of 3000 hours. The purpose of
the experiment was to estimate the activation energy of the
failure-causing reaction and to obtain an estimate of the integrated
circuit life distribution at $80\degreesc$ junction temperature.
\\[3ex]
\begin{tabular}{ccl}
Junction & \# of Units \\
Temperature &  Tested & Recorded Failure Times in Thousands of Hours \\
\hline
$80\degreesc$ &10& None by 3000 hours\\
$125\degreesc$&10& None by 3000 hours\\
$150\degreesc$&10  & 2.35, 2.56, 2.98\\
$175\degreesc$&10 & .80, 1.13, 1.21, 1.31, 
	1.35, 1.35, 1.37, 1.42, 1.77, 1.96\\
$200\degreesc$&10 & .22, .25, .28, .33, 
	.37, .38, .46, .46, .51, .61 \\
\hline
\\[2ex]
\end{tabular}
\begin{enumerate}
\item
\label{exer.part:device.c.lognormal}
For each temperature with failures, plot the ordered failure time
$t_{(i)}$ versus $(i-.5)/10$ on lognormal probability paper.
\item
Repeat part~\ref{exer.part:device.c.lognormal}, but use Weibull
probability paper.
\item
Make a judgment as to whether the lognormal or the Weibull
distribution is a more adequate distribution for these data.
\item
Using the lognormal probability plot, draw a set of parallel straight
lines through the plotted points, one line for each
temperature having failures.
\item
Use each line on the probability plot to obtain graphical estimates
of the .5 quantiles at the corresponding temperatures. 
\item
Use each line on the probability plot to obtain graphical estimates of
the .01 quantiles at the corresponding temperatures. Describe the
nature of the extrapolation in these estimates.
\item
\label{exer.part:device.c.arrhen.plot}
Plot the estimates of the .5 quantile versus temperature on Arrhenius
paper. Draw a straight line to estimate the relationship between life
and temperature. Do the same with the graphical estimates of the .01
quantile.
\item
Graphically estimate the slope of the lines drawn in
part~\ref{exer.part:device.c.arrhen.plot}, and use these to obtain a
graphical estimate of the failure mode's activation energy.
\item
Use the lines drawn in part~\ref{exer.part:device.c.arrhen.plot} to
obtain estimates of the .01 and .5 quantiles of the life distribution
of Device-C at $80\degreesc$. Describe the nature of the extrapolation
in these estimates.
\item
Predict the effect on the estimates of the Arrhenius relationship
if the data at $80\degreesc$ and $125\degreesc$ were to be omitted
from the analysis. Explain.
\end{enumerate}
\end{exercise}

%-----------------------------------------------------------------------------
\begin{exercise}
Provide a list of the different things that one can learn from
plotting individual nonparametric estimates and parametric ML
estimates on Weibull probability paper (such as
Figure~\ref{figure:mylarpoly.lnor.groupi.ps}).
\end{exercise}

%-----------------------------------------------------------------------------
\begin{exercise}
Suppose that failure time $T \sim \LOGNOR(\mu,\sigma)$ at a given
level of temperature and that the Arrhenius model can be used to get a
temperature/time acceleration factor as in
(\ref{equation:arrhenius.af}). Then show that the logarithm of quantiles
of the failure-time distribution will be a linear function of
$1/(\Tempc{} +273.15)$.
\end{exercise}

%-----------------------------------------------------------------------------
\begin{exercise}
Suppose that failure time $T \sim \WEIB(\mu,\sigma)$ at a given
level of voltage and that the inverse power relationship can be used
to get a voltage/time acceleration factor as in
(\ref{equation:voltage.acceleration}). Then show that the logarithm of quantiles
of the failure-time distribution will be a linear function of
$\log({\rm Voltage})$.
\end{exercise}



%-----------------------------------------------------------------------------
\begin{exercise}
An analyst has fitted the Arrhenius-lognormal model
\begin{displaymath}
\Pr[\rv\leq t;\Temp] = \Phi_{\nor} 
\left[\frac{\log(t)-\mu}{\sigma} \right]
\end{displaymath}
where $\mu =\beta_{0}+ \beta_{1} x$, $\sigma$ is constant and
$x=1/(\Tempk{})$.  Show how $\beta_{1}$ in this model is related to
activation energy $\Ea$ in the Arrhenius relationship.
\end{exercise}



%-----------------------------------------------------------------------------
\begin{exercise}
An analyst has fitted the Arrhenius-lognormal model
\begin{displaymath}
\Pr[\rv \leq t;\Temp] = \Phi_{\nor} 
\left[\frac{\log_{10}(t)-\mu}{\sigma} \right]
\end{displaymath}
where $\mu = \beta_{0}+ \beta_{1} x$, where $\sigma$ is constant and
$x=11605/(\Tempk{})$.  This differs from the model presented in
Example~\ref{example:devicea.arr.log.model} because base-10
logarithms have been used (something done quite commonly, and
usefully, in engineering and other non-mathematical disciplines).
Show how estimates of $\beta_{0}, \beta_{1},
\sigma,$ and $t_{p}$ in this model
relate to estimates in the traditional base-$e$ lognormal
distribution.
\end{exercise}



%-----------------------------------------------------------------------------
\begin{exercise}
Refer to Example~\ref{example:ci.fhat.comp}.  Write down expressions
that could be used to compute a normal-approximation confidence
interval for the $p$ quantile of the life distribution at a
specified level of temperature.
\end{exercise}

%-----------------------------------------------------------------------------
\begin{exercise}
\label{exercise:crisk.alt}
A particular type of IC has two different failure modes, both of
which can be accelerated by increasing temperature.  The random
failure-time of Mode 1 is $T_{{\rm M}_{1}}$ and the
random failure time due to Mode 2 is $T_{{\rm M}_{2}}$. In a unit,
suppose that the IC fails at $T=\min(T_{{\rm M}_{1}},T_{{\rm
M}_{2}})$. Failure mode ${\rm M}_{1}$ has an activation energy of
$\Ea=1.4$ electron volts and failure mode ${\rm
M}_{2}$ has an activation energy of $\Ea=.7$.  The physical nature
of the failure mechanisms suggests that $T_{{\rm M}_{1}}$ and
$T_{{\rm M}_{2}}$ are independent. The engineers involved believe
that both $T_{{\rm M}_{1}}$ and $T_{{\rm M}_{2}}$ have a lognormal
distribution.  At the proposed highest test temperature of
$120\degreesc$, with failure time measured in hours, assume that
$\mu_{1}=6.9$, $\sigma_{1}=.6$, $\mu_{2}=9.0$, and $\sigma_{2}=.8$.
\begin{enumerate}
\item
Assuming an Arrhenius ALT model, plot the median of the failure-time
distribution versus temperature for failure mode ${\rm M}_{1}$ and also
for failure mode ${\rm M}_{2}$. Plot using temperatures between the use
temperature of $40\degreesc$ and $120\degreesc$.
\item
Plot the .1 and .9 quantiles of the mode ${\rm M}_{1}$
failure-time distribution and of the mode ${\rm M}_{2}$ failure-time
distribution, as a function of temperature.
\item
Use the plots to help describe potential dangers of using a
temperature-accelerated life test on a component, ignoring failure mode
differences, when the  different
failure modes have vastly different activation energies.
\end{enumerate}
\end{exercise}

%-----------------------------------------------------------------------------
\begin{exercise1}
Refer to Exercise~\ref{exercise:crisk.alt}.  Compute and plot the
median, as well as the .1 and .9 quantiles of the failure-time 
distribution for the IC, as a function of temperature, when
both failure modes are active. That is, use the distribution of
$T=\min(T_{{\rm M}_{1}},T_{{\rm M}_{2}})$.  What do these results
suggest about appropriate methods for dealing with test acceleration
when there are 2 or more failure modes?  That is, suggest how a useful
ALT can be conducted and how the data could be analyzed when there are
2 or more failure modes.
\end{exercise1}


%-----------------------------------------------------------------------------
\begin{exercise1}
To describe the failure-time distribution of specimens of an
insulating material at use operating conditions $\Temp_{U}$, use a
Weibull distribution with cdf $F(t)=
1-\exp[-(t/\eta_{U})^\beta]$. Suppose that the Arrhenius model
applies. Then the model is SAFT and at some high temperature
$\Temp_{H}$, $\eta_{U} = \AF(\Temp_{H})
\eta_{H} $.
\begin{enumerate}
\item
Show that  $\log(t_{p})$, the logarithm of the Weibull quantile, is a linear
function of $1/(\Tempc{} +273.15)$.
\item
Derive an expression for the pdf at $\Temp_{H}$.
\item
Derive an expression for the hazard function at $\Temp_{H}$.
\item
Show that the ratio of the hazard function at $\Temp_{U}$
and at $\Temp_{H}$ does not depend on time. This implies that
the Weibull SAFT is also a proportional hazards model.
\end{enumerate}
\end{exercise1}


%-----------------------------------------------------------------------------
\begin{exercise}
For a particular kind of insulating material, life can be
described by the inverse power model 
\begin{displaymath}
\Pr[\rv\leq t;\Volt] = \Phi_{\sev}\left[\frac{\log(t)-\mu}{\sigma} \right]
\end{displaymath}
where $\sigma$ is constant, $\mu = \beta_{0}+ \beta_{1} x$, and
$x=\log(\Volt)$.
\begin{enumerate}
\item
Show that  $\log(t_{p})$, the logarithm of the Weibull quantile, is a linear
function of $\log(\Volt)$.
\item
Suppose that $100p_{1}$\% would fail at voltage $\Volt_{1}$
and $100p_{2}$\% would fail at voltage $\Volt_{2}$ before time
$\censortime$. Derive an expression for the proportion failing at
$\Volt_{3}$.
\item
Derive expressions
for the inverse power relationship parameters $\beta_{0}$ and
$\beta_{1}$
as functions of $p_{1}$, $p_{2}$, and $\sigma$.
\end{enumerate}
\end{exercise}

%-----------------------------------------------------------------------------
\begin{exercise}
Example~\ref{example:devicea.model.lrt} showed how to use an omnibus
likelihood ratio test to compare fitting individual lognormal
distributions at each level of temperature and fitting the
Arrhenius-lognormal ALT model.  This provides an overall test for
model adequacy.
\begin{enumerate}
\item
Explain, precisely, the hypothesis or hypotheses being tested in
Example~\ref{example:devicea.model.lrt}.
\item
Explain which constrained and unconstrained models you would
fit in order to test the assumption of a common $\sigma$ at all
levels of temperature. Do not assume anything about the
temperature-life relationship.
\item
Explain possible likelihood ratio tests on whether $\sigma$ is
constant, under the assumption that $\mu$ is linearly related to
$11605/\Tempk{}$.
\item
Explain how to do a likelihood ratio test to check the assumption
that $\mu$ is linearly related to $11605/\Tempk{}$ versus an
alternative relationship with curvature, assuming that $\sigma$ does
not depend on temperature.
\end{enumerate}
\end{exercise}


%-----------------------------------------------------------------------------
\begin{exercise}
Refer to Example~\ref{example:ci.fhat.comp} for the Device-A data.
Compute a confidence interval for the lognormal distribution
$F(10000)$ at 10$\degreesc$.
\end{exercise}



%-----------------------------------------------------------------------------
\begin{exercise}
\label{exercise:mylar.lognormal}
The ALT data on the mylar-polyurethane insulating structure in
Appendix Table~\ref{atable:mylar.alt.data} are complete data (no
censored observations). Thus it is possible to fit a lognormal
distribution to these data using the standard least-squares regression
analysis procedure in a standard statistical package.
Use these data, dropping the 361.4 kV/mm observations, to do the following.
\begin{enumerate}
\item
Use such a statistical package to compute the least squares estimates for
the inverse power relationship lognormal model.
\item
Plot the regression estimates of median of the failure-time
distribution on log-log axes. Also plot the sample median at each
level of voltage. What conclusions can you draw from this?
\item
Compare the least squares estimates with estimates obtained by using
ML estimation with the inverse power relationship lognormal
model. What differences do you notice in the parameter estimates?
\end{enumerate}
\end{exercise}

%-----------------------------------------------------------------------------
\begin{exercise}
Refer to Exercise~\ref{exercise:mylar.lognormal}.  Use a maximum
likelihood program to fit the inverse power relationship Weibull
model to these data. Compare Weibull and lognormal estimates of the
.1 quantile of failure at 50 kV/mm. What do you conclude about the
importance of the distribution used for this estimate?
\end{exercise}


%-----------------------------------------------------------------------------
\begin{exercise}
\label{exercise:mylar.lognormal.all}
Redo the analyses in Exercise~\ref{exercise:mylar.lognormal} but
include the 361.4 kV/mm observations. What do you conclude?
\end{exercise} 


%-----------------------------------------------------------------------------
\begin{exercise}
\label{exercise:sigma.ratio}
For Example~\ref{example:new.tech.ic}, using the results in
Table~\ref{table:icdevice2.lognor.indiv.mles}, compute a 95\%
normal-approximation confidence interval for the ratio
$\sigma_{250}/\sigma_{300}$. Base the interval on the large sample
distribution of $Z_{\log(\sigmahat_{250}/\sigmahat_{300})}$.
\end{exercise}


%-----------------------------------------------------------------------------
\begin{exercise}
In Example~\ref{example:new.tech.ic}, there is evidence that
$\sigma$ differs from one level of temperature to the other.
\begin{enumerate}
\item
Suppose that $\sigma$ is really changing (as suggested by the point
estimates) describe the effect that using a constant $\sigma$ model would
have on estimates of the life distribution at $100
\degreesc$. You do not need to do any numerical 
computations; answer using intuition or analytical arguments.
\item
With reference to the simple Arrhenius-lognormal failure-time model
described in Sections~\ref{section:arrhenius.af} and
\ref{section:nonlin.deg.acc} suggest possible physical reasons
(i.e., deviations from the simple model) that might cause $\sigma$
to change as a function of temperature.
\end{enumerate}
\end{exercise}


%-----------------------------------------------------------------------------
\begin{exercise1}
In Example~\ref{example:new.tech.ic}, the likelihood ratio test for
comparing the constrained and unconstrained models has only one
degree of freedom. In Example~\ref{example:devicea.model.lrt},
however, the comparison test has three degrees of freedom.
\begin{enumerate}
\item
Explain why there is a difference.
\item
If the 200 $\degreesc$ subexperiment had run to $10^{5}$ hours
without failure, would the number of degrees of freedom in the test
change?  Why or why not?
\end{enumerate}
\end{exercise1}

