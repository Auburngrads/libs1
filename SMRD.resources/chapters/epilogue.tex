%epilogue
%original by wqmeeker  12 Jan 95
%edited by wqmeeker  21/24 june 97

\addcontentsline{toc}{chapter}{Epilogue}
\chapter*{Epilogue}

The
material in this book is only part of a much larger picture. Our
examples, like the particular applications from which they
originated, focused on the reliability or durability of materials,
components, and relatively simple systems. Each of these
applications was, however, associated with larger reliability,
economic, and business questions like
\begin{itemize}
\item
Is Bearing-A durable enough to use in a new automobile transmission
or do the designers need to switch to the more durable (but somewhat
more expensive) Bearing-B? With the prospect of selling millions of
transmissions over several years, the decision has huge economic
consequences.
\item
How much redundancy is required for a critical device in the
repeater-transmitter subsystems for an under-sea telecommunications
system? Adding redundancy will add cost but also improve system
reliability and affect life-cycle cost. Promised life-cycle cost is
an extremely important part of contract negotiations. A critical
input to the analysis is the life distribution of the device. An
accelerated life test will be used to obtain needed information.
\end{itemize}
Numerous decisions like these are made in typical product design
processes. The methods presented in this book are essential for
providing some of the information needed for making such decisions
in the face of uncertainty. We close with some commentary on the
changing role of statistical methods in a Design for Reliability
program and as part of an overall Reliability Assurance process.

\subsubsection{The challenges of achieving high reliability}
In a conversation with one of us, an engineer commented that
``reliability is much more difficult today than it used to be.''  He
went on to explain that his company was facing real competition for
the first time. They were having to reduce engineering safety
factors and cut other corners to reduce cost. As a result, they were
beginning to see many more field failures.  New technology, new
materials, higher customer expectations, more competition, and the
need to get competitively priced products to the market quickly
complicate decision making. Indeed, reliability engineering {\em is}
more difficult today due to stringent cost constraints and enormous
time pressures caused by increased competition.

The engineer's comment might suggest that there is a
cost/reliability trade-off curve for product design and all one
needs to do is to find the right place on the design curve.  What is
really needed, however, is to move off of the old curve, find a
better curve, and then optimize it. The ``best curve'' in this case
depends on product design, manufacturing process design, the
collection and use of appropriate reliability data, and other
factors. There will always be cost/reliability tradeoff issues. To
be successful, products in competitive markets must provide high
reliability at low cost. The best curves (i.e., designs or products)
will be determined by good engineers who can properly use modern
statistical and other analytical engineering methods.

Achieving high reliability is complicated by the fact that the
impact of many reliability improvements will appear some time in the
future (often some years). Thus, it is extremely difficult to
quantify the effect on the current year's bottom line. For most
companies, this will require a change in metrics and mind-sets.

In spite of the difficult challenges involved, the leading companies
in a number of industries have continued to improve performance and
reliability while keeping cost low enough to maintain competitive
prices.  Examples include computer hardware, telecommunications
systems, and automobiles.

\subsubsection{High reliability at competitive cost} 
The necessary constant pressure on product and process designs to
reduce costs and/or improve performance has the potential to cause
reliability problems.  For example, reducing traditional safety
factors mandates more careful engineering and statistical
practice. Having large safety factors to protect against one
failure mode may well have prevented other un-thought-of failure
modes. A proposed design change for cost reduction may be analyzed
for the known risks (or failure modes), but the unknown risks may
not surface until product is in the field. There are also numerous
instances where all available energy was devoted to addressing one
failure mode; no resources or energy remained to address others
failure modes that turned out to be the most troublesome in the
field.

How can design engineers deal with failure modes that might
otherwise be in the un-thought-of category?  Three suggested ways
are:
\begin{itemize}
\item
Use careful, informed engineering (including knowledge of the
product's use environment). Identify and prevent most potential
failure modes before they have a chance to occur. See the discussion
of FMEA, FMECA, design reviews, and related reliability management
tools described in Section~\ref{section:fmea.fmeca}.
\item
Increase up-front experimentation and life testing to discover and
eliminate other potential failure modes before they get into a final
design.  It might be necessary to do such testing at the component
level, the subsystem level, or at the system level.
\item
Collect and carefully scrutinize early information from the
field. Compare with previous test results.
Detect and fix product weaknesses before they cause serious problems.
\end{itemize} 
As with the product design itself, the key is to find efficient
implementation of these activities and to balance costs against
potential risks.  The biggest payoff is in building high reliability
into the design before product introduction.  The cost of detecting
and eliminating failure modes increases as the product moves from
conceptualization through design, development, testing, and
production, and into the field. It is inevitable that there will be
some product reliability problems. Reliability assurance processes
should reduce risk by reducing the probability and severity of
field-failure problems.


\subsubsection{\bf Advancing engineering practice to achieve high reliability}
Today's engineers need to rely more heavily on modern tools like
probabilistic design and risk analysis and less on the traditional
easy-to-apply rules of thumb.  This modern approach to engineering
leads to a new set of concerns and costs.  Issues of model adequacy
and uncertainty in model inputs can be critical. More measurement,
experimentation, systems analysis, and sensitivity analysis will be
needed. Protection must still be provided where unacceptable
uncertainty exists. To address these issues, some or many engineers
need a command of basic experimental design and statistical
concepts.

Although reliability practice is primarily an engineering
discipline, statistics and other scientific disciplines (material
science, physical chemistry, and so on) play crucial supporting
roles.  Engineers need specialized training to analyze reliability
data (dealing with censored data, pitfalls of accelerated testing,
difficulties of interpreting warranty data, multiple failure modes,
physics of failure, and so on).

\subsubsection{\bf Useful tools and some specific suggestions}
There is no simple solution or magic for the challenge of achieving
high reliability at low cost. The process involves hard work. Some
suggestions, related to the technical material in this book, include
the following.
\begin{enumerate}

\item
Electronic design engineers use databases containing up-to-date
reliability-related information (component reliability, derating
functions, materials properties, etc.) linked with reliability modeling
software, embedded within CAD systems. More wide-spread development of
such systems would lead to better design practices.

\item
Computationally-based models of physical phenomena and increased use
of computer simulation have the potential to save time and money by
reducing reliance on expensive physical experimentation. Electronics
and fracture mechanics are the important success stories
here. Up-front investment in research and development is needed, but
there are important potential payoffs.
\item
Up-front testing of materials and components, as well as subsystems
and systems, is needed to reduce uncertainty about product
reliability. Such tests are an important part of any reliability
assurance program. What to test and how much testing to do (number
of units, test duration, and at what level of system integration)
requires careful consideration to balance risks with costs. It is
important to take into consideration the way in which the product is
used by customers.
\item
Efficiency in up-front testing (e.g., use of existing information,
properly designed experiments, etc.) is important.  Experimental
effort should be focused where engineering uncertainty implies
reliability uncertainty. For materials and components, the goal
should be to assess the failure-time distribution and/or to
determine allowable levels of stress. For subsystem and system tests
in the product development stage, the goal should be to apply
appropriate amounts of stresses in the right combinations to
discover (and then fix) potential failure modes.  Being able to
identify failure modes that would not be expected to occur in actual
operation is important and requires good engineering knowledge.
\item
Experiments in general and robust-design experiments, in particular,
have potential for leading to important improvements in product and
process design and resulting better reliability. The robust-design
methods of Taguchi (e.g., Phadke 1989) are important
here. Multifactor robust-design experiments (RDE) provide methods
for systematic and efficient reliability improvement.  These are
often conducted on prototype units and subsystems and focus on
failure modes involving interfaces and interactions among components
and subsystems.  Among many possible product-design factors that may
affect a system's reliability, RDEs empirically identify the
important ones and find levels of the product-design factors that
yield consistent high quality and reliability.  Graves and
Menten~(1996) provide an excellent description of experimental
strategies that can be used to help design products with higher
reliability.  Other important references relating to RDEs are
Condra~(1993) and Hamada~(1993, 1995a, 1995b).  Byrne and
Quinlan~(1993) present an interesting example to illustrate the
concepts.
\item
Product engineers, scientists, and statisticians can work more
effectively together to develop experimental strategies for
robust-design experiments for improving product performance and
reliability.  Those who follow the Taguchi approach seem to advocate
running a larger number of different simple experiments (and the
necessary confirmatory experiments) to obtain first-order
improvements.  Improvement is the goal. Traditional
statistical/engineering approaches might recommend a more extensive
sequential program of experimentation (perhaps requiring higher cost
and more time) to gain more fundamental scientific understanding.
The best approach depends on the potential for improvement,
available resources, and goals.
\item
Field data is a vital resource. There will always be risk of
failures with any product design.  Field tracking is expensive and
not always used. Warranty data usually have serious deficiencies and
often come too late. Nevertheless, it is necessary to develop
processes for the collection and appropriate use of field feedback
to quickly discover and correct potential problems before they
become wide spread, thereby reducing overall risk. Field-data
feedback should also be used to improve future designs.  Important
references describing the analysis of warranty and other field data
include Amster, Brush, and Saperstein~(1982), Suzuki~(1985),
Kalbfleisch and Lawless~(1988), Robinson and McDonald~(1991),
Lawless and Kalbfleisch~(1992), Lawless and Nadeau~(1995), Lawless,
Hu, and Cao~(1995), Blischke and Murthy~(1996), and Lawless~(1998).
\item
For field data, one general idea is to carefully monitor some number
of units conveniently located in the field. To be most effective,
the units should be operated in a reasonably use-intensive manner and
failures should be reported promptly.  All early failures should be
analyzed carefully to determine cause, whether the same failure mode
could be expected in the rest of the product population, and
relevant actions that might be taken to eliminate the failure mode.
Rather than just tracking failures, it is often useful, when
practicable, to go out periodically and inspect and take measurements
on field units (e.g., to assess degradation).
\item
In some circumstances there is need to use limited reliability audit
testing of ongoing production to catch the possible impact of
changes in raw materials, supposedly innocuous design changes, and
so on. This is especially important in today's manufacturing
environment where many producers are often just assemblers, who rely
heavily on  components provided by vendors.
\end{enumerate}

\subsubsection{\bf Statistics is much more than a collection of
formulas} Those who have had one or two courses in statistics are
often left with the impression that statistics is primarily a
collection of analytical techniques and formulas. Choose the
correct technique and formula and the problem is solved.

Viewed properly, statistics is the science of collecting and
extracting useful information from raw data and of dealing with
variability in quantitative information.  Statistical tools provide
the means for fitting and assessing the adequacy of models (physical
or empirical). Statistical models are used to describe the
relationships among variables as well as variability and
uncertainty.

\subsubsection{\bf Statistics is not magic}
The statistical methods described in this book are useful for
planning reliability studies and extracting useful information from
reliability data. Statistical methods also provide quantification of
sampling uncertainty and allow planning statistical studies so that
estimates and predictions can be obtained with a specified degree of
statistical uncertainty.

There is, however, no magic in statistics.  For example, we have
heard more than once a question that asks something like this:
``What kind of test can I run to demonstrate, with 95\% confidence,
that my system will have .99 reliability for its first year of
operation when I only have, at most, three systems to test and the
test has to be completed in two months?''  The answer is that such a
demonstration is impossible. Even if there were a given acceleration
method providing an acceleration factor of six (which would be
unlikely because for a complete system there are many failure modes
with different acceleration factors and typically one cannot
increase stress enough on a complete system to achieve an
acceleration factor as large as six), one would need to test
approximately 300 units for the two months with no failures to
have a successful demonstration.

When demonstration of desired reliability is impossible, there is an
inclination to take action the basis of the best estimate.  Then a
minimal requirement would be that there be {\em no} failures.  It is
for this reason that it is easier to prove lack of reliability when
you don't have it than it is to prove adequate reliability when you
do have it.

Reliability demonstration for systems, once popular in the military
and some other places, is difficult or impossible with today's
higher reliability standards and cost sensitivity. In new industrial
markets,``Reliability Assurance'' processes are needed instead. The
statistical methods in this book provide important tools for such
Reliability Assurance processes. For more discussion of this topic,
see Meeker and Hamada~(1995).
