%chapter 11
%\batchmode
%original by wqmeeker  12 Jan 94
%edited by wqmeeker 20 March 94
%edited by wqmeeker 17 April 94
%edited by wqmeeker  1 june 94
%edited by wqmeeker  30 nov 94 minor changes
%edited by wqmeeker  5 sept 95 minor changes
%edited by driker 11 nov 95
%edited by wqmeeker  1 dec 96 merging trunc, thresh and other dists
%edited by driker 22 apr 97
%edited by wqmeeker 28 apr 97
%edited by wqmeeker 14 may 97

\setcounter{chapter}{10}



\chapter{Parametric Maximum Likelihood:
Other Models}
\label{chapter:ml.other.parametric}


%{\Large {\bf William Q. Meeker and Luis A. Escobar}}\\
%Iowa State University and Louisiana State University\\[5ex]
{\large {\bf \today}}\\[2ex]
Part of 
{\em Statistical Methods for Reliability Data}\\
Copyright 1997 W. Q. Meeker and L. A. Escobar. \\[1ex]
To be published by John Wiley \& Sons Inc. in 1998.




%----------------------------------------------------------------------
%----------------------------------------------------------------------
\section*{Objectives}
This chapter explains
\begin{itemize} 
\item 
ML estimation for the gamma and
the extended generalized gamma (EGENG) distributions. 
\item 
ML estimation for the Birnbaum-Saunders (BISA) and 
the inverse Gaussian (IGAU) distributions.
\item 
ML estimation for the limited failure population (LFP) model.
\item 
How truncation arises in reliability data applications and ML
estimation for truncated data (or data from truncated distributions).
\item 
ML estimation for distributions with a threshold parameter like the
three-parameter lognormal and the three-parameter Weibull
distributions [using the generalized threshold-scale (GETS)
distribution].
\item 
Potential difficulties involved in using distributions with
threshold parameters and how to avoid them.
\end{itemize}

%----------------------------------------------------------------------
%----------------------------------------------------------------------
\section*{Overview}
This is an advanced chapter that can be skipped without loss of
continuity.  Section~\ref{section:other.dist.ml} describes the
extension of ML methods, introduced in
Chapter~\ref{chapter:parametric.ml.ls}, to distributions that are
not location-scale or log-location-scale.
Chapter~\ref{chapter:other.parametric.models} provides background
and technical details for the non-location-scale distributions used
in this chapter. Section~\ref{section:gamma.mle} illustrates ML
methods for the gamma distribution and
Section~\ref{section:ml.gen.gamma} shows how to fit the extended
generalized gamma distribution.  Section~\ref{section:bisa.igau.fit}
uses ML methods to fit and compare the Birnbaum-Saunders and inverse
Gaussian distributions.  Section~\ref{section:using.the.lfp.model}
shows how to fit and interpret the limited failure population (LFP)
distribution, a kind of mixture model.
Section~\ref{section:truncated.data} describes various applications
where truncated data arise and illustrates methods for analyzing
such data.  Section~\ref{section:threshold.dist.ml} illustrates
methods for fitting distributions with threshold parameters, showing
how to avoid potential difficulties.


%----------------------------------------------------------------------
%----------------------------------------------------------------------
\section{Introduction}
\label{section:other.dist.ml}

The methods presented in Chapter~\ref{chapter:parametric.ml.ls} were
based on fitting log-location-scale distributions. This chapter
describes the fitting of some important distributions that are not
log-location-scale.  For such distributions, things may be nearly as
simple, but they can also be considerably more complicated.

%----------------------------------------------------------------------
\subsection{Likelihood for other distributions and models}
General likelihood principles for fitting distributions and models
are as described in
Chapters~\ref{chapter:np.models.censoring.likelihood} and
\ref{chapter:parametric.ml.one.par}. 
For most distributions,
with data consisting of independent observations with exact failures
and right-censored observations, the standard density-approximation
form of the likelihood
\begin{equation}
\label{equation:general.parametric.den.likelihood}
\like(\thetavec) = \prod_{i=1}^{n} \like_{i}(\thetavec;\data_{i}) 
=
  \prod_{i=1}^{n}    
 \left [ 
 f(\realrv_{i};\thetavec)
\right ]^{\delta_{i}} 
\left [1- F(\realrv_{i};\thetavec) \right]^{1-\delta_{i}}
\end{equation}
works well. Here, as in previous chapters, 
data$_{i}=(t_{i}, \delta_{i})$,
\begin{eqnarray*}
\delta_{i}= \left \{
     \begin{array}{lll}
	1 & \quad \mbox{if $t_{i}$ is an exact failure} \\
        0 & \quad \mbox{if $t_{i}$ is a right censored  observation} \\
     \end{array}
            \right .
\end{eqnarray*}
and $F(\realrv;\thetavec)$ and
$f(\realrv;\thetavec)$ are the cdf and pdf, respectively, of the
specified distribution.

The likelihood can be adapted easily to accommodate left-censored
and interval-censored observations, as described in
Chapter~\ref{chapter:np.models.censoring.likelihood}.  As mentioned
in Section~\ref{section:loglikelihood}, operationally, it is usually
the log likelihood that is computed as the sum of log likelihoods for
individual independent observations. As illustrated in
Section~\ref{section:like.methods}, for some non-location-scale
distributions (e.g., GETS) the density approximation breaks down and
one should use instead the actual interval probability or ``correct
likelihood'' given in
(\ref{equation:general.parametric.likelihood}).

In some cases, the likelihood function can be poorly behaved and,
particularly in unfamiliar data/model situations, it is important to
investigate $\like(\thetavec)$ graphically.  Graphical exploration
of $\like(\thetavec)$ [or
$R(\thetavec)=\like(\thetavec)/\like(\thetavechat)$] is simple to do
when $\thetavec$ has length 1 or 2.  When the length of $\thetavec$
is 3 or more it is useful to view 1 (and 2) dimensional ``profiles''
of $\like(\thetavec)$. The needed computations can, however, become
quite demanding.

\subsection{Confidence intervals for other distributions and models}
%----------------------------------------------------------------------
%----------------------------------------------------------------------
Methods for computing confidence intervals and confidence regions
can be used in a manner that is similar to that used for
log-location-scale distributions in
Chapters~\ref{chapter:parametric.ml.ls}.  Normal-approximation
confidence intervals (using the delta method and appropriate
transformations) are simple and are adequate in large samples.  In
other situations these easy-to-compute intervals provide quick
analyses, but approximations can be a bit rough. Profile likelihoods
provide useful insight into the information available about a
particular parameter or function of a distribution's parameters.
Confidence limits based on the profile likelihood, as well as
bootstrap and simulation-based intervals described in
Chapter~\ref{chapter:bootstrap}, generally provide confidence
intervals with reasonably good approximations to nominal coverage
probabilities. The approximations will be adequate even with
moderately small samples (say samples large enough to yield at least
10 to 15 failures).  Such intervals will, however, require more
computer time (and may not be available in commercial software).

\section{Fitting the Gamma Distribution}
\label{section:gamma.mle}
The gamma distribution, introduced in
Section~\ref{section:gamma.distribution}, is
a commonly used failure-time distribution.  The gamma
distribution likelihood is obtained by substituting
(\ref{equation:gamma.pdf}) and (\ref{equation:gamma.cdf}) into
(\ref{equation:general.parametric.den.likelihood}).  For given
$\DATA$, the likelihood is a function of the scale parameter $\theta$
and the shape parameter $\gammashape$.


\begin{example}
\label{example:lzbearing.gamma}
{\bf Gamma distribution fit to the ball bearing fatigue data.}  This
example uses the ball bearing data from
Example~\ref{example:ball.bearing.data}.
Figure~\ref{figure:bearing.gamma.gmleprobplot.ps} shows a lognormal
probability plot of the bearing failure data, comparing ML estimates
of the gamma, lognormal, and Weibull distributions.  The gamma ML
estimates are $\thetahat=17.94$ and $\gammashapehat=4.025$.
%----------------------------------------------------------------------
\begin{figure}
\splusbookfigure{\figurehome/bearing.gamma.gmleprobplot.ps}
\caption{Lognormal probability plot of the ball bearing failure data, 
comparing gamma, lognormal, and Weibull ML estimates.  Approximate
95\% pointwise confidence intervals for the gamma $F(t)$ are also
shown.}
\label{figure:bearing.gamma.gmleprobplot.ps}
\end{figure}
%----------------------------------------------------------------------
%splus Orig-Parameters:
%splus  theta kappa 
%splus  17.94 4.025
Figure~\ref{figure:bearing.gamma.gmleprobplot.ps} also
shows approximate 95\% pointwise confidence intervals for $F(t)$,
based on the gamma ML estimates. The gamma distribution fits well.
Within the range of the data, however, there is very little difference
among these three different distributions.
\end{example}

%----------------------------------------------------------------------
%----------------------------------------------------------------------
\section{Fitting the Extended Generalized Gamma Distribution}
\label{section:ml.gen.gamma}
The extended generalized gamma (EGENG) distribution, introduced in
Section~\ref{section:gng.distribution}, includes the gamma,
generalized gamma, Weibull, exponential, and lognormal distributions
as special cases.  As such, it provides a flexible distribution
structure for modeling data and comparing among these widely-used
distributions. The EGENG distribution likelihood is obtained by
substituting (\ref{equation:egeng.pdf}) and
(\ref{equation:egeng.cdf}) into
(\ref{equation:general.parametric.den.likelihood}).  For given
$\DATA$, the likelihood is a function of the parameters $\mu$, $\sigma$, and
$\egengshape$.

\begin{example}
{\bf EGENG distribution fit to the ball bearing fatigue data.}
This example uses the ball bearing data from
Examples~\ref{example:ball.bearing.data} and
\ref{example:lzbearing.gamma}.
%----------------------------------------------------------------------
\begin{figure}
\splusbookfigure{\figurehome/bearing.egeng.gmleprobplot.ps}
\caption{Weibull probability plot of the bearing
failure data showing the exponential, Weibull, lognormal, and EGENG
ML estimates of $F(t)$.}
\label{figure:bearing.egeng.gmleprobplot.ps}
\end{figure}
%----------------------------------------------------------------------
Figure~\ref{figure:bearing.egeng.gmleprobplot.ps} is a Weibull
probability plot of the bearing failure data showing exponential,
Weibull, lognormal, and EGENG ML estimates of $F(t)$.  The EGENG ML
estimates are $\muhat=4.23$, $\sigmahat=.51$, and $\egengshapehat=.3076$.
Figure~\ref{figure:bearing.egeng.gmleprobplot.ps} illustrates that the EGENG
distribution provides a compromise between the lognormal and Weibull
distributions.
%splus Orig-Parameters:
%splus    mu sigma  delta 
%splus  4.23  0.51 0.3076
Figure~\ref{figure:bearing.egeng.delta.profile.ps}
%----------------------------------------------------------------------
\begin{figure}
\splusbookfigure{\figurehome/bearing.egeng.delta.profile.ps}
\caption{Profile likelihood plot for the EGENG shape parameter
$\egengshape$ for the bearing
failure data. The Weibull and lognormal distributions are shown as special
cases.}
\label{figure:bearing.egeng.delta.profile.ps}
\end{figure}
%----------------------------------------------------------------------
is a profile likelihood plot for the EGENG shape parameter
$\egengshape$ for the bearing failure data. The Weibull and lognormal
distributions are shown as special cases. This figure shows that the
lognormal relative likelihood is slightly higher than the Weibull.
The data, however, do not indicate a strong preference for one or the
other of these distributions.
\end{example}

The previous example illustrated the fitting of the EGENG distribution
to a moderate-size sample with {\em no} censoring.  The EGENG
distribution can also be fit to censored data, but the fitting can be
more delicate because of difficulty statistically separating the three
different parameters in (\ref{equation:egeng.cdf}).  Using a robust
optimization algorithm and/or a reparameterization to stable
parameters will, however, allow this distribution to be fit to
censored data, even with very heavy censoring (see the comments in the
bibliographic notes at the end of this chapter).

\begin{example}
{\bf EGENG distribution fit to the fan data.}
This example fits the EGENG distribution to the fan data from
Example~\ref{example:fan.data}.  
%----------------------------------------------------------------------
\begin{figure}
\splusbookfigure{\figurehome/fan.egeng.gmleprobplot.ps}
\caption{Lognormal probability plot of the fan
failure data showing EGENG ML estimates and corresponding 95\% pointwise 
confidence intervals for $F(t)$. Exponential, Weibull, and
lognormal ML estimates of $F(t)$ are also shown.}
\label{figure:fan.egeng.gmleprobplot.ps}
\end{figure}
%----------------------------------------------------------------------
During the period of observation there were 12 failures
out of 70 units.  Due to multiple censoring, however, the
nonparametric estimate extends to .29. 
Figure~\ref{figure:fan.egeng.gmleprobplot.ps} shows a lognormal
probability plot of the fan failure data showing EGENG ML estimates
and corresponding 95\% pointwise confidence intervals for $F(t)$. 
%splus Orig-Parameters:
%splus     mu sigma  delta 
%splus  9.332 2.375 -1.764 
The EGENG ML estimates are $\muhat=9.332$, $\sigmahat=2.375$, and
$\egengshapehat=-1.764$.  The pointwise confidence intervals for the
EGENG $F(t)$ become extremely wide outside the range of the data.
Exponential, Weibull, and lognormal ML estimates of $F(t)$ are also
shown.  These distributions,
and especially the lognormal distribution, also fit the data
reasonably well. Note that the EGENG departs quite strongly from the
other
distributions outside the range of the data.
Except for missing the first point,
the EGENG estimate also goes well with the nonparametric
estimate. To assess the strength
of the evidence in the data for choosing among these distributions,
consider the profile likelihood plot for EGENG $\egengshape$ for the
fan failure data in Figure~\ref{figure:fan.egeng.delta.profile.ps}.
%----------------------------------------------------------------------
\begin{figure}
\splusbookfigure{\figurehome/fan.egeng.delta.profile.ps}
\caption{Profile likelihood plot for the EGENG shape parameter
$\egengshape$ for the fan
failure data showing the Weibull and lognormal distributions as special
cases.}
\label{figure:fan.egeng.delta.profile.ps}
\end{figure}
%----------------------------------------------------------------------
The EGENG has a larger likelihood than the other distributions, but the
difference is statistically unimportant (the approximate 95\%
likelihood-based confidence interval endpoints for $\egengshape$
ranges from something less than $-8$ to something greater than 2, as
shown in Figure~\ref{figure:fan.egeng.delta.profile.ps}). Because of
the small number of failures, fitting a
three-parameter distribution to these data could be considered to be
``overfitting.''  Generally one should use the simplest model that
provides an adequate fit to the data. Overfitting such as this,
however, is useful to help demonstrate uncertainty 
in distribution choice when
the decision is to be based on data alone.
\end{example}

%----------------------------------------------------------------------
%----------------------------------------------------------------------
\section{Fitting the BISA and IGAU Distributions}
\label{section:bisa.igau.fit}
Fitting the Birnbaum-Saunders (BISA) or the inverse-Gaussian (IGAU)
distributions is very similar to fitting the gamma distribution (all
of these distributions have one scale and one shape parameter). The
likelihood function for the BISA and IGAU distributions can be
obtained by substituting the corresponding pdf and cdf from
Sections~\ref{section:igau.pdfcdf} or \ref{section:bisa.pdfcdf} into
(\ref{equation:general.parametric.den.likelihood}).  As described in
Section~\ref{section:bias.properties}, the BISA and IGAU
distributions were motivated by particular degradation models.

\begin{example}
\label{example:bkfat10.bisa.igau}
{\bf BISA and IGAU distributions fit to fatigue-fracture data.}
Yokobori~(1951) describes a fatigue-fracture test on .41\% carbon
steel cylindrical specimens, tested at $\pm 37.1 \mbox{kg/mm}^2$ stress
amplitude.  The data are given on page 224-225 of Bogdanoff and Kozin
(1985). Figure~\ref{figure:bkfat10.cf.ln.bisa.igau.gmleprobplot.ps} is
a lognormal probability plot of Yokobori's fatigue failure data.
%----------------------------------------------------------------------
\begin{figure}
\splusbookfigure{\figurehome/bkfat10.cf.ln.bisa.igau.gmleprobplot.ps}
\caption{Lognormal probability plot of 
Yokobori's fatigue failure data showing lognormal, BISA, and IGAU
distribution ML estimates.}
\label{figure:bkfat10.cf.ln.bisa.igau.gmleprobplot.ps}
\end{figure}
%----------------------------------------------------------------------
The plot also shows ML estimates of $F(t)$ for the lognormal, BISA, and IGAU
distributions. 
%splus bisa Orig-Parameters:
%splus     theta     beta 
%splus  109.2444 1.129053
%splus igau Orig-Parameters:
%splus     theta      beta 
%splus  179.9206 0.5949112
%splus 
%splus ML Estimates for lognormal
%splus            MLE         se  t.ratio 95% lower 95% upper 
%splus   mu 4.721612 0.12920139 36.54460 4.4683824  4.974843
%splus sigma 1.025504 0.09135918 11.22497 0.8612047  1.221149
The ML estimates for the BISA distribution are $\thetahat=109.24$ and
$\betahat=1.129$. The ML estimates for the IGAU distribution are
$\thetahat=179.9$ and $\betahat=.595$. The ML estimates for the
lognormal distribution are $\muhat=4.72$ and $\sigmahat=1.026$.
The closeness of the $F(t)$ estimates
from these three different distributions in
Figure~\ref{figure:bkfat10.cf.ln.bisa.igau.gmleprobplot.ps} is
striking.  This is not too surprising, given the correspondence
between the hazard shapes that can be seen in
Figures~\ref{figure:distplot.lnor.ps}, \ref{figure:distplot.igau.ps},
and \ref{figure:distplot.bisa.ps}.
\end{example}

Figure~\ref{figure:bisa.ln.cf.ps} is a plot of cdfs of the BISA
distribution with $\theta=1$ and different shape parameters on lognormal 
probability paper.
%----------------------------------------------------------------------
\begin{figure}
\splusbookfigure{\figurehome/bisa.ln.cf.ps}
\caption{Comparison of BISA cdfs on lognormal probability paper.}
\label{figure:bisa.ln.cf.ps}
\end{figure}
%----------------------------------------------------------------------
The approximate linearity of the cdfs, especially for small $\beta$
(small coefficient of variation) suggests that the lognormal and
BISA distributions will often give very similar results in the
center of the distribution.  It was noted in
Section~\ref{section:lognormal.distribution.def} that the lognormal
distribution is widely used to describe time to fracture from
fatigue crack growth in metals.  The similarity of the lognormal and
BISA distributions and the fatigue-fracture justification for the
BISA distribution (see Section~\ref{section:bisa.dist}) suggest why
the lognormal distribution has been found to be a useful model for
fatigue-fracture data.  Figure~\ref{figure:bisa.ln.cf.ps} indicates
that the lognormal distribution, when used to extrapolate into the
lower tail of a distribution, will give more conservative (i.e.,
smaller) estimates of distribution quantiles. Such extrapolation is
common, for example, when it is desired to estimate $t_{.001}$ on
the basis of tests on 300 specimens.

%----------------------------------------------------------------------
%----------------------------------------------------------------------
\section{Fitting a Limited Failure Population Model}
\label{section:using.the.lfp.model}

%----------------------------------------------------------------------
\subsection{Example and data}
\begin{example}
\label{example:lfp.ic.data}
{\bf The IC failure time data.}
We now return to the IC failure-time data 
from Example~\ref{example:lfp.data}, given in Table~\ref{table:lfp.data}.
Figure~\ref{figure:lfp.probability.plot.ps} is a Weibull probability
plot of the right-censored failure data. The last failure occurred at
593 hours and the test was stopped at 1370 hours.  We see that the
points on the probability plot are leveling off, apparently to
something less than 1\% failing.  The failures were caused by
manufacturing defects that could not be detected without a life test.
The reliability engineers responsible for this product wanted to
estimate $p$, the proportion of defects being manufactured by the
process, in its current state. Moreover, they were attempting to
make improvements to the process and wondered if informative life tests
could, in the future, be run without waiting so long.
\end{example}

%-------------------------------------------------------------------

%----------------------------------------------------------------------
\begin{figure}
\splusbookfigure{\figurehome/lfp.probability.plot.ps}
\caption{A Weibull probability plot of integrated circuit failure time
data with ML estimates of the Weibull/LFP model after 1370 hours and
after 100 hours of testing.  The asymptotes for the ML fits in the
plot correspond to the ML estimates for $p$, the proportion in the
process susceptible to failure.}
\label{figure:lfp.probability.plot.ps}
\end{figure}
%----------------------------------------------------------------------

%----------------------------------------------------------------------
\subsection{The limited failure population model}
\label{section:lfp.model}
The Limited Failure Population (LFP) model has a proportion $p$ of
units from a population or process that is defective and will fail
according to a distribution $F(\realrv;\mu,\sigma)$; the remaining
proportion $(1-p)$ will never fail.  This model has been found to be
useful for modeling the reliability of integrated circuit infant
mortality. Its use will be illustrated with the IC life test data
from Example~\ref{example:lfp.ic.data}. If $F(\realrv;\mu,\sigma)$
is Weibull, then the LFP failure-time model is
\begin{equation}
\label{equation:lfp.cdf}
\Pr(\rv \leq \realrv) = G(\realrv;\mu,\sigma,p) = pF(\realrv;\mu,\sigma)
= p \Phi_{\sev}\left[\frac{ \log(\realrv) - \mu}{\sigma}
\right].
\end{equation}
Note that as $\realrv \rightarrow \infty$, $G(\realrv) \rightarrow p$. The 
lognormal LFP model is obtained by using  $\Phi_{\nor}$
instead of $\Phi_{\sev}$ in (\ref{equation:lfp.cdf}).


%----------------------------------------------------------------------
\subsection{The likelihood function and its maximum}

The likelihood function for the Weibull LFP model is
\begin{equation}
\like(\mu,\sigma,p)=  \prod_{i=1}^{n}
\left\{ \frac{p}{t_{i}\sigma} \phi_{\sev}
\left[\frac{ \log(\realrv_{i}) -\mu}{\sigma}
\right]
\right\}^{\delta_{i}}
 \left\{ 1-p \Phi_{\sev}\left[\frac{ \log(\realrv_{i}) -\mu}{\sigma}
\right] \right\}^{1-\delta_{i}}
\label{equation:lfp.likelihood}
\end{equation}
where the notation is similar to that used in
Section~\ref{section:likelihood.for.locscale}.  In some situations it
will be difficult to estimate the parameters of this model. In
particular if the censoring time is before the nonparametric estimate
of $G(\realrv;\mu,\sigma,p)$ begins to level off, one cannot tell the
difference between a population with $p=1$ in which defective units
fail slowly and a population with small $p$ in which defective units
fail rapidly.

\begin{example}
{\bf Comparison of IC failure time data analyzed at 100 and at 1370 hours.}
Table \ref{table:lfp.results} summarizes and compares
the results of the analyses for the data that were available at 1370
hours and the data that would have been available after only 100
hours.  Figure~\ref{figure:lfp.probability.plot.ps} shows the ML
estimates and 95\% pointwise normal-approximation confidence intervals
for $G(\realrv)$.  As might be expected, there is close agreement
until approximately 100 hours, when the estimates of $G(\realrv)$
begin to differ importantly. The upper bounds of the pointwise
normal-approximation confidence intervals for $G(\realrv)$ are larger
for $\realrv > 100$.  As suggested below, however, similar confidence
intervals for the 100-hour data, computed with the likelihood-based
method, would be {\em much} wider.

The nonparametric estimate of $G(\realrv;\mu,\sigma,p)$ in
Figure~\ref{figure:lfp.probability.plot.ps} (i.e., the plotted
points) portends the LFP model estimation difficulties with the
100-hour data. Note that the curvature in the Weibull probability
plot becomes much more pronounced after 100 hours.
\end{example}
%-------------------------------------------------------------------
\begin{table}
\caption{Comparison of LFP model integrated circuit failure
data analyses.}
\centering\small
\begin{tabular}{*{4}{r}}
\hline
\multicolumn{2} {c} {}&
\multicolumn{2} {c} {Analysis with Test Run Until} \\
\multicolumn{2} {c} {}&
\multicolumn{1} {c} {1370 Hours}&
\multicolumn{1} {c} {100 Hours} \\
\cline{3-4}\\
\multicolumn{1} {c} {ML Estimate $\muhat$}
&&  3.34 &  4.05\\[1ex]
\multicolumn{1} {c} {Standard Error $ \sehat_{\muhat}$}
&&  .41 & 1.70 \\[1ex]
\multicolumn{1} {l} {95\% Confidence Intervals}
&&& \\
\multicolumn{1} {l} {for $\mu$ based on:}
&&& \\
\multicolumn{1} {l} {\hspace{1em} the Likelihood}
&& [2.50, 4.20] & [2.43, 24.99]  \\
\multicolumn{1} {l} {\hspace{1em}
        $Z_{\muhat} \approxdist \NOR(0,1)$ }
&& [2.55, 4.12] & [.72, 7.38]  \\[2ex]
\multicolumn{1} {c} {ML Estimate $\sigmahat$}
&& 2.02 & 2.12 \\[1ex]
\multicolumn{1} {c} {Standard Error $ \sehat_{\sigmahat}$}
&& .31 & .55 \\[1ex]
\multicolumn{1} {l} {95\% Confidence Intervals}
&&& \\
\multicolumn{1} {l} {for $\sigma$ based on:}
&&& \\
\multicolumn{1} {l} {\hspace{1em} the Likelihood}
&& [1.53, 2.82] & [1.40, 3.96]  \\
\multicolumn{1} {l} {\hspace{1em}
        $Z_{\log(\sigmahat)} \approxdist \NOR(0,1)$ }
&& [1.50, 2.71] & [1.28, 3.51]  \\
\multicolumn{1} {l} {\hspace{1em}
        $Z_{\sigmahat} \approxdist \NOR(0,1)$ }
&& [1.42, 2.62] & [1.05, 3.19]  \\[2ex]
\multicolumn{1} {c} {ML Estimate $\phat$}
&& .00674 & .00827 \\[1ex]
\multicolumn{1} {c} {Standard Error $ \sehat_{\phat}$}
&& .00127 & .00380 \\[1ex]
\multicolumn{1} {l} {95\% Confidence Intervals}
&&& \\
\multicolumn{1} {l} {for $p$ based on:}
&&& \\
\multicolumn{1} {l} {\hspace{1em} the Likelihood}
&& [.00455, .00955] & [.00463, 1.0000]  \\
\multicolumn{1} {l} {\hspace{1em}
        $Z_{\logit(\phat)} \approxdist \NOR(0,1)$ }
&& [.00466, .00975] & [.0033, .0203]  \\
\multicolumn{1} {l} {\hspace{1em}
        $Z_{\phat} \approxdist \NOR(0,1)$ }
&& [.00426, .00923] & [.00081, .0157]  \\
\hline
\end{tabular}
\label{table:lfp.results}
\end{table}
%-------------------------------------------------------------------


%----------------------------------------------------------------------
\subsection{Profile likelihood functions and likelihood-based
confidence intervals for $\mu$, $\sigma$, and $p$}
\label{section:lfp.ci}
\begin{example}
{\bf Likelihood-based confidence intervals for the IC data.}  Table
\ref{table:lfp.results} gives numerical values for the
likelihood-based confidence intervals for $\mu, \sigma$, and $p$
based on these and other profiles (not shown here). The table also
gives confidence intervals based on the normal
approximation. Figure~\ref{figure:lfp.psigma.conf.contour.ps} shows
approximate joint confidence regions for $p$ and $\sigma$ based on a
two-dimensional profile likelihood for $p$ and $\sigma$ for the
100-hour data. Figure~\ref{figure:lfp.p.profile.comparison.ps}
provides a comparison of the one-dimensional profiles for $p$ for
the 1370 and the 100-hour data.  The results of this comparison show
that for the 1370-hour data, the log likelihood is approximately
quadratic and the different methods of computing confidence
intervals give similar results.  For the 100-hour data, however, the
situation is quite different.  In particular, the leveling off of
the 100-hour profile likelihood for $p$ tells us that the data
available after 100 hours could reasonably have come from a
population with $p=1$. That is, the 100-hour data do not allow us to
clearly distinguish between a situation where there are many
defectives failing slowly and a situation with just a few defectives
failing rapidly.  The 1370-hour data, however, allow us to say with
a high degree of confidence, that $p$ is small.

For the 100-hour data, the likelihood and normal-approximation
confidence intervals for $p$ are vastly different.
This is because the log likelihood is not well approximated
by a quadratic function over the range of the
confidence interval. The approximate confidence intervals based 
on the likelihood can be expected to provide coverage probabilities
closer to the nominal values.
\end{example} 

%-------------------------------------------------------------------
\begin{figure}
\splusbookfigure{\figurehome/lfp.psigma.conf.contour.ps}
\caption{Approximate joint confidence
regions for  the LFP parameters $p$ and $\sigma$ based on
a two-dimensional profile likelihood after 100 hours of testing.}
\label{figure:lfp.psigma.conf.contour.ps}
\end{figure}
%-------------------------------------------------------------------

%-------------------------------------------------------------------
\begin{figure}
\splusbookfigure{\figurehome/lfp.p.profile.comparison.ps}
\caption{Comparison of profile 
likelihoods for $p$, the LFP proportion
defective after 1370 and 100 hours of testing.}
\label{figure:lfp.p.profile.comparison.ps}
\end{figure}
%-------------------------------------------------------------------
%-------------------------------------------------------------------

%----------------------------------------------------------------------
%----------------------------------------------------------------------
\section{Truncated Data and Truncated Distributions}
\label{section:truncated.data}

It is important to distinguish between truncated data and censored
data.  They are sometimes confused. Censoring occurs when there is a
bound on an observation (lower bound for observations censored on the
right, upper bound for observations censored on the left, and both
upper and lower bounds for observations that are interval censored).
Truncation, however, arises when even the {\em existence} of a
potential observation would be unknown if its value were to lie in a
certain range. Usually, truncation occurs to the left of a specified
point $\tau^{L}$ or to the right of a specified point $\tau^{U}$.


%-------------------------------------------------------------------
\subsection{Examples of left truncation}

\begin{example}
\label{example:ut.trun}
{\bf Ultrasonic inspection of material.} Ultrasonic inspection is
used to detect flaws in titanium alloys during several stages of
manufacturing of jet engine turbine disks.  Undetected flaws in such
parts could cause early initiation of a fatigue crack and thus
increase the risk of failure.  Ultrasonic signal amplitude is
generally positively correlated with flaw size. Thus the
distribution of signal amplitudes reflected from flaws provides
information on the distribution of flaw sizes. Titanium is a
``noisy'' material.  Titanium grain-boundaries reflect about as well
as small flaws.  Thus below a specified threshold $\tau^{L}$, it is
impossible to be sure whether a signal is from a flaw or a grain
boundary.  Suppose that interest centers on the distribution of
signal strengths (including signals below $\tau^{L}$ that would be
observed in the absence of material noise). Consider the following
two possibilities.

\begin{itemize}
\item
In a laboratory test of the inspection process, specimens with
seeded flaws of known size are inspected. In this case when a
reading is taken, it is known that a flaw is present. The signal's
amplitude is, however, measured only when the amplitude is above the
threshold $\tau^{L}$.  In some applications, $\tau^{L}$ will change
from time to time (depending on the local material noise level), but
generally the value $\tau^{L}$ can be recorded. Then the number of
signals that were below $\tau^{L}$ is known, and these are
left-censored observations.
\item
In an operating inspection process, a flaw is not detected when the
signal's amplitude lies below the threshold $\tau^{L}$. Then we
observe only the signals that are greater than $\tau^{L}$ in
amplitude. The number of flaws that were present with signals below
$\tau^{L}$ is unknown.  The observations recorded as being above
$\tau^{L}$ are known as left-truncated observations, or observations
from a left-truncated distribution (in the case where $\tau^{L}$ is
the same for all readings).
\end{itemize}
\end{example}


If all units below $\tau^{L}$ in a population or process are screened
out before observation, the remaining data are from a
``left-truncated'' distribution.  Depending on the application,
interest could center on either the original untruncated distribution
or on the truncated distribution.  For most problems the additional
information provided by the proportion of observations truncated
(either in the population or the sample) would lead to censoring
instead of truncation and importantly improve estimation precision of
the original (unconditional) distribution's parameters.


\begin{example}
\label{example:pretest.and.left.trun}
{\bf Life data with pre-test screening.}
Table~\ref{table:cirpack.track.data}, described in
Example~\ref{example:electronic.subsystem.data}, gives the number of
observed failures from a field-tracking study of circuit packs.  The
Vendor 2 units had already seen 1000 hours of burn-in testing at the
manufacturing plant, but no information was available on the number of
units that had failed in that test. Thus the Vendor 2 circuit packs
are left-truncated.
If the number of circuit packs that failed in the burn-in period
were known, then the data could be treated as censored.
\end{example}

\begin{example}
\label{example:brake.pad.trun}
{\bf Distribution of brake pad life from observational data.}
Kalbfleisch and Lawless (1992) give data on brake pad life from a
study of automobiles. For each automobile in the study (where $i$ is
used to index the individual automobiles), the number of kilometers
($v_{i}$) driven and a wear measurement ($w_{i}$) were taken at a
point in time when the automobiles were being serviced (for
something other than brake problems). The wear measurement is such
that $w_{i}=0$ represents no wear and $w_{i}=1$ is the level of
brake wear that requires replacement of the pads. Suppose, for a
given automobile, that wear is proportional to accumulated driven
kilometers, which suggests a brake pad life estimate of
$t_{i}=v_{i}/w_{i}$.  The important assumption being used here is
that the main source of variability in brake pad life is the
automobile-to-automobile variability in the $t_{i}$ values and that
failure time could be predicted accurately from these observed
ratios. Automobiles that had previously had brake pads replaced were
not included in the study. For this reason, high-wear-rate
automobiles are under-represented in the study. To correct for this,
Kalbfleisch and Lawless treat the life prediction $t_{i}$ as
left-truncated at $\tau^{L}_{i}=v_{i}$, the number of kilometers of
service at the time of the prediction. The idea is that if the wear
rate had been high enough to cause failure before the regularly
scheduled service call at $\tau_{i}^{L}$, the automobile would not
have been included in the study.
\end{example}



\subsection{Likelihood with left truncation}
\label{section:likemeth.le}
Following the general development in
Section~\ref{section:general.likelihood.terms}, if a random variable
$\rv_{i}$ is truncated on the left at $\tau^{L}_{i}$, then the
likelihood (probability) of an observation in the interval
$(\realrv^{L}_{i}, \realrv_{i}]$ is the conditional probability
\begin{displaymath}
\like_{i}(\thetavec)=\Pr(\realrv^{L}_{i} <  \rv_{i} \leq \realrv_{i} |
\rv_{i} > \tau^{L}_{i}) =
\frac{F(\realrv_{i};\thetavec)-F(\realrv^{L}_{i};\thetavec)}
{1-F(\tau^{L}_{i};\thetavec)},\quad \realrv_{i} > \realrv_{i}^{L} \geq \tau_{i}^{L}.
\end{displaymath}
For an observation reported as an exact failure at time $t_{i}$, the
corresponding density approximation form of the likelihood is
\begin{equation}
\label{equation:trun.den.like}
\like_{i}(\thetavec)=
\frac{f(\realrv_{i};\thetavec)}
{1-F(\tau^{L}_{i};\thetavec)}, \quad \realrv_{i} > \tau^{L}_{i}.
\end{equation}
It is possible to have either right or left censoring when sampling
from a left truncated distribution. The recorded censoring time will
exceed $\tau_{i}^{L}$.  As in
Table~\ref{table:general.likelihood.terms}, to obtain
$\like_{i}(\thetavec)$ for a censored observation, one simply replaces
the numerator in (\ref{equation:trun.den.like}) by
$F(\realrv_{i};\thetavec)-F(\tau^{L}_{i};\thetavec)$ for an observation that is left censored at
$\realrv_{i} > \tau^{L}_{i}$ and by $1-F(\realrv_{i};\thetavec)$ for an observation
that is right censored at $\realrv_{i} > \tau^{L}_{i} $.

\subsection{Nonparametric estimation with left truncation}
\label{section:cond.left.trun.prob}
Section~\ref{section:arbitrary.censoring} presents a general method,
due to Turnbull~(1976), for nonparametric estimation that can be
extended to the analysis of arbitrarily censored and truncated data.
When there is left truncation, Turnbull's method
provides a nonparametric estimate of
conditional distribution
\begin{equation}
\label{equation:cond.left.trun.prob}
F_{C}(t) = \Pr( \rv \leq \realrv |\rv > \tau^{L}_{\min}) =
       \frac{\Pr(\tau^{L}_{\min} < \rv \leq \realrv)}{\Pr( \rv >
       \tau^{L}_{\min} ) } = \frac{F(\realrv) -F(\tau^{L}_{\min})
       }{1-F(\tau^{L}_{\min})},\quad \realrv > \tau^{L}_{\min}
\end{equation}
where $\tau^{L}_{\min}$ is the smallest left truncation time in the
sample.  Without a parametric assumption, the data contain no
information about $F(t)$ below $\tau^{L}_{\min}$.

For purposes of probability plotting to assess the adequacy of a
parametric assumption for $F(\realrv;\thetavec)$, one must have,
instead, an estimate of the unconditional distribution of $\rv$.  In
this case we use a parametric model to estimate $\Pr( \rv >
\tau^{L}_{\min} )$ and then compute a parametrically-adjusted
nonparametric estimate of $F(\realrv)$. Let $\Fhat_{NPC}(t)$ denote
the nonparametric estimate of the conditional distribution
$F_{C}(t)$. Then a parametrically-adjusted unconditional
nonparametric estimate of $F(\realrv)$ will be denoted by
$\Fhat_{NPU}(t)$.  This estimate is obtained by substituting
$\Fhat_{NPC}(t)$ for $F_{C}(t)$ and
$F(\tau^{L}_{\min};\thetavechat)$ for $F(\tau^{L}_{\min})$ in
(\ref{equation:cond.left.trun.prob}) and solving for $F(\realrv)$.
This gives
\begin{equation}
\label{equation:uncond.left.trun.prob}
\Fhat_{NPU}(t) = F(\tau^{L}_{\min};\thetavechat) +  
	\Fhat_{NPC}(t)[1 - F(\tau^{L}_{\min};\thetavechat) ] , \quad
\realrv > \tau^{L}_{\min}
\end{equation}
where $\thetavechat$ is a parametric estimate of the parameters in
$F(\realrv;\thetavec)$.

\subsection{ML estimation with left truncated data}
\begin{example}
\label{example:pretest.mle}
{\bf Analysis of life data with pre-test screening.} As described in
Example~\ref{example:pretest.and.left.trun}, because they had already
seen 1000 hours of burn-in testing, the Vendor 2 data in
Table~\ref{table:cirpack.track.data} are left truncated at 1000 hours.
Figures~\ref{figure:cirpak6.cfmleprobplot.lnor.weib.ps} is a lognormal
probability plot of the Vendor 2 failure data,
parametrically-adjusted with a lognormal ML fit to the data.
The plot also shows the lognormal (straight line) and Weibull (curved
line) ML estimates, and approximate 95\% pointwise lognormal
confidence intervals.
%-------------------------------------------------------------------
\begin{figure}
\splusbookfigure{\figurehome/cirpak6.cfmleprobplot.lnor.weib.ps}
\caption{Parametrically-adjusted lognormal probability plot of Vendor 2 
circuit pack data comparing Weibull and lognormal ML estimates.}
\label{figure:cirpak6.cfmleprobplot.lnor.weib.ps}
\end{figure}
%-------------------------------------------------------------------
In this example, the estimated proportion truncated on the left of
1000 hours appears to be (through parametric estimation) very
small [$\Fhat(1000) = .95\times 10^{-5}$ for the lognormal and
$.38\times 10^{-4}$ for the Weibull distribution] and so the
parametric adjustment to the nonparametric estimate used in making the
probability plot was extremely small. If the failure mode before 1000
hours was different than that after 1000 hours, however, this kind of
extrapolation into the lower tail of the distribution would give a
misleading view of what happened up to 1000 hours (analysis of the
Vendor 1 data described in Exercise~\ref{exercise:vendor1.trun} and
Section~\ref{section:crisk.mixture.model} suggests that there well
might have been such a change). In any case, the available data can be
used to estimate failure probabilities beyond 1000 hours, conditional
on the burn-in test results.

For extrapolation beyond the range of the data, the lognormal
distribution gives much more optimistic (i.e., lower) estimates of
failure probabilities than the Weibull distribution. This
lognormal/Weibull contrast is common, and leads to the
question``which distribution is most appropriate for
extrapolation?''  The data have little to say about differentiating
between these two distributions. If there is an answer to the
question it would be in knowledge about the physics of failure for
the observed failure mode(s). In the absence of such knowledge,
reported uncertainty would have to encompass not only the sampling
variability quantified in the confidence intervals, but also the
distribution uncertainty.  As illustrated in
Section~\ref{section:ml.gen.gamma}, fitting the extended generalized
gamma will help to quantify and illustrate such distributional
uncertainty.
\end{example}


\subsection{Examples of right truncation}
Right truncation is similar to left truncation, and occurs when the
values in the upper tail of the distribution are removed. 
 
\begin{example}
\label{example:Inspection of castings to improve reliability.}
{\bf Screening out units with large flaws.} Degree of porosity is an
important quality metric in casting processes. The distribution of
pore sizes is closely related to the failure time of a component
in a particular application. If large pores or other voids occur
inside a casting, fatigue cracks will initiate more rapidly, leading
to premature failure.  In a particular manufacturing process, castings
for automobile engine mounts are inspected by x-ray to make sure that
there are no large internal voids or pores.  Pores larger than 10
microns can be detected with high probability. The casting process 
output has a
distribution of pore sizes. The inspection process truncates off the
upper tail of the pore-size distribution (i.e., units containing
pores greater than 10 microns are eliminated). Thus the distribution of
pore size in units passing inspection could be described by a
right-truncated distribution.
\end{example}


\begin{example}
\label{example:warranty.and.right.trun}
{\bf Warranty data with limited information for unfailed units.} A
particular home appliance, after purchase, is either used regularly or
not at all. The percentage of units actually put into regular use is
unknown. During a particular production period, an incorrect component
(i.e., one that did not have the specified power rating) was installed
in all of the units that were produced. When failures occur among
these regularly-used units, the units are returned to the manufacturer
for repair or replacement under a long-term warranty program. The manufacturer
learns about failures from this group of units only if the unit is
actually put into service and if the unit fails before the analysis
time.  In this case, the observed failure times can be viewed as a
sample from a distribution right truncated at a time equal to the difference
between the time of analysis and the time at which the unit was put
into service.
\end{example}


\begin{example}
\label{example:lfp.trunc}
{\bf Limited failure population data.} Similar to the situation in
Example~\ref{example:warranty.and.right.trun}, the 28 IC failure times
in Example~\ref{example:lfp.ic.data} can be viewed as a sample from a
distribution right truncated at $\tau^{U}=\censortime=1370$ hours.
Intuitively this is because, although 4156 ICs were tested, the number
of potential susceptible units in the population was unknown.
Susceptible units become known as such only if they failed within the
1370-hour long life test.
Exercise~\ref{exercise:lfp.factor.likelihood} provides technical
justification for this.
\end{example}

As shown in the analysis of the LFP data in
Section~\ref{section:using.the.lfp.model} it will be impossible to
estimate the unconditional failure-time distribution from
right-truncated data unless a parametric form is specified for the
distribution. Even with a specified failure-time distribution, there
are serious estimability problems (leading to wide confidence
intervals on quantities of interest) unless the proportion truncated
is very small (say less than 5\%). This problem is also described in
Kalbfleisch and Lawless~(1988).

%-------------------------------------------------------------------
\subsection{Likelihood with right (and left) truncation}

If the random variable $\rv_{i}$ is truncated when it lies above
$\tau^{U}_{i}$ then the likelihood (probability) of an interval
observation is
\begin{displaymath}
\like_{i}(\thetavec)=\Pr(\realrv^{L}_{i} < \rv_{i} 
\leq \realrv_{i}|\rv_{i} \leq
\tau^{U}_{i}) =
\frac{F(\realrv_{i};\thetavec)-F(\realrv^{L}_{i};\thetavec)}
{F(\tau^{U}_{i};\thetavec)}, \quad 0 \leq \realrv_{i}^{L} < 
\realrv_{i} \leq \tau_{i}^{U}.
\end{displaymath}
For an observation reported as an exact failure at time $t_{i}$, the
corresponding density-approximation form of the likelihood is
\begin{displaymath}
\like_{i}(\thetavec)=
\frac{f(\realrv_{i};\thetavec)}
{F(\tau^{U}_{i};\thetavec)}.
\end{displaymath}
As with left truncation, it is possible to have either left or right
censoring when sampling from the right truncated distribution.
With both left and right truncation, the appropriate likelihood for an
interval-censored observation is
\begin{displaymath}
\like_{i}(\thetavec) = \Pr(\realrv^{L}_{i} <  \rv_{i} 
\leq \realrv_{i}|\tau^{L} \leq \rv < \tau^{U}) =
\frac{F(\realrv_{i};\thetavec)-F(\realrv^{L}_{i};\thetavec)}
{F(\tau^{U}_{i};\thetavec) - F(\tau^{L}_{i};\thetavec)}, \quad \tau_{i}^{L} 
\leq \realrv_{i}^{L} < \realrv_{i} \leq \tau_{i}^{U}.
\end{displaymath}
The likelihood for censored observations is obtained in a manner similar
to that described in Section~\ref{section:likemeth.le}.

%-------------------------------------------------------------------
\subsection{Nonparametric estimation with right (and left) truncation}
\label{section:np.par.corr.right.trun}
Section~\ref{section:cond.left.trun.prob} showed how to parametrically
adjust a truncated-data nonparametric estimator so that it could be
used to for making a probability plot. This approach can be extended
to work in situations with right or both left and right truncation.
In particular, the nonparametric estimate is for the conditional probability
\begin{equation}
\label{equation:cond.right.trun.prob}
F_{C}(t) = \Pr(\rv\leq\realrv |  \tau^{L}_{\min} \leq\rv<\tau^{U}_{\max}) =
	\frac{F(\realrv)-F(\tau_{\min}^{L})}{  F(\tau^{U}_{\max})  -  
	F(\tau^{L}_{\min})},\quad \tau^{L}_{\min} < \realrv\leq\tau^{U}_{\max}
\end{equation}
where $\tau_{\max}$ is the largest right-censoring time in the
sample.  Then, as in (\ref{equation:uncond.left.trun.prob}) a
parametrically-adjusted unconditional nonparametric estimate of
$F(\realrv)$ is obtained from
\begin{equation}
\label{equation:uncond.right.trun.prob}
\Fhat_{NPU}(t) = F(\tau^{L}_{\min};\thetavechat) +  
	\Fhat_{NPC}(t)[F(\tau^{U}_{\max};\thetavechat) -
F(\tau^{L}_{\min};\thetavechat) ] , \quad
 \tau^{L}_{\min} < \realrv \leq \tau^{U}_{\max}.
\end{equation}


\begin{example}
\label{example:lfp.trunc.mle}
{\bf Using right truncation to estimate the failure-time distribution
from limited failure population data.} Following
Example~\ref{example:lfp.trunc},
Figure~\ref{figure:lfptrunc.uncorrprobplot.lognor.ps} is a lognormal
probability plot of the 25 pre-100 hour IC failure times from
Table~\ref{table:lfp.data}.
%-------------------------------------------------------------------
\begin{figure}
\splusbookfigure{\figurehome/lfptrunc.uncorrprobplot.lognor.ps}
\caption{Lognormal probability plot of the (unadjusted) nonparametric
estimate of the IC failure time distribution, conditional on failure
in the first 100 hours.}
\label{figure:lfptrunc.uncorrprobplot.lognor.ps}
\end{figure}
%-------------------------------------------------------------------
In contrast to the probability plot in
Figure~\ref{figure:lfp.probability.plot.ps}, the 4131 units that were
unfailed (censored) at 100 hours are ignored. The plotted estimate in
Figure~\ref{figure:lfptrunc.uncorrprobplot.lognor.ps}, as described in
Section~\ref{section:np.par.corr.right.trun}, is really an estimate of
$\Pr(\rv \leq \realrv | \rv \leq 100)$.  Treating the 25 failures as right
truncated at 100 hours (or estimating the parameters of a
lognormal distribution truncated at 100 hours) gives $\muhat=4.44$ and
$\sigmahat=3.44$.  From this, $F(100; \muhat,\sigmahat)=\Pr(\rv \leq 100)
= \Phi[(\log(100)-4.44)/3.44]= .5191$.  This can be used in
(\ref{equation:uncond.right.trun.prob}) to parametrically adjust the
nonparametric estimate of $\Pr(\rv \leq \realrv | \rv \leq 100)$ for
purposes of probability plotting. The parametrically adjusted
probability plot is shown in
Figure~\ref{figure:lfptrunc.corrprobplot.lognor.ps} along with the ML
estimate $F(t; \muhat,\sigmahat)$ based on the truncated data. 
%-------------------------------------------------------------------
\begin{figure}
\splusbookfigure{\figurehome/lfptrunc.corrprobplot.lognor.ps}
\caption{Lognormal probability plot of 
the lognormal-adjusted (unconditional) nonparametric estimate of the
IC failure time distribution with pointwise 95\%
normal-approximation confidence intervals for the lognormal $F(t)$.}
\label{figure:lfptrunc.corrprobplot.lognor.ps}
\end{figure}
%-------------------------------------------------------------------
The
pointwise confidence intervals are very wide because of the important
amount of information lost by not having direct information about
$\Pr(\rv \leq  100)$. If we knew the actual number of defectives in the
sample of 4156 ICs, and used this to determine the number of censored
susceptible ICs, the intervals would be much tighter (see
Exercise~\ref{exercise:lfp.sen.anal}).
%splus>  pnorm((log(100)-4.44)/3.44)=.5191
%splus> 
%splus> ML Estimates
%splus>            MLE       se  t.ratio 95% lower 95% upper 
%splus>    mu 4.441958 3.091394 1.436879 -1.617063 10.500980
%splus> sigma 3.442543 1.349368 2.551226  1.596744  7.422043
%splus> 
\end{example}

%----------------------------------------------------------------------
%----------------------------------------------------------------------
\section{Fitting Distributions that have a Threshold Parameter}
\label{section:threshold.dist.ml}

In many areas of application, analysts wish to fit distributions with a
``threshold parameter'' $\threshold$ that shifts the distribution of a
positive random variable (usually to the right) by an amount $\threshold$.
Recall from Section~\ref{section:intro.to.threshold.distributions}
that the three-parameter lognormal cdf and pdf can be written as
\begin{eqnarray*}
 	F(\realrv;\mu,\sigma,\threshold)&=&
		\Phi_{\nor}\left [\frac{\log(\realrv-\threshold)-\mu}{\sigma}
		\right ] \\
 f(\realrv;\mu,\sigma,\threshold)&=&	
		\frac{1}{\sigma (\realrv-\threshold)}\, \phi_{\nor}
	\left [ 
	\frac{\log(\realrv-\threshold)-\mu}{\sigma} \right], 
	\quad \realrv> \threshold.
\end{eqnarray*}
Threshold versions of any other distribution with support on
$[0,\infty)$ (e.g., exponential, Weibull, loglogistic,
gamma, inverse Gaussian, and Birnbaum-Saunders) can be defined in
this manner.  Although the discussion in the following sections
uses the three-parameter lognormal and Weibull distributions,
the basic ideas hold also for the other threshold distributions.  In
some physical applications it makes sense to constrain
$\threshold>0$, but there is no theoretical need to do this.
%----------------------------------------------------------------------
%----------------------------------------------------------------------
\subsection{Estimation with a given threshold parameter}
%----------------------------------------------------------------------
If the threshold parameter $\threshold$ is given,
one can subtract $\threshold$ from all reported failure, inspection,
and censoring times and then use the simpler methods for the base
distribution without the threshold parameter (e.g., the one-parameter
exponential distribution methods in
Chapter~\ref{chapter:parametric.ml.one.par} and the two-parameter
Weibull and lognormal distribution methods in
Chapter~\ref{chapter:parametric.ml.ls} ).  Of course one needs to
adjust inferences accordingly. For example, the given value of
$\threshold$ must be added back into estimates of quantiles and one
must subtract $\threshold$ from times before computing failure
probabilities, hazard function values, or other functions of time.
Using a specified value of $\threshold$ that is seriously incorrect
can lead to seriously incorrect conclusions.

%----------------------------------------------------------------------

%----------------------------------------------------------------------
%----------------------------------------------------------------------
\subsection{Probability plotting methods}
With three-parameter log-location-scale distributions,
$\log(\rv-\threshold)$ has a location-scale distribution with
parameters $\mu$ and $\sigma$.  For such distributions, there are two
different method of probability plotting. These methods do not
compete, but rather complement each other.
\begin{itemize}
\item
One can make log-location-scale distribution probability plots
(e.g., standard Weibull or lognormal probability plots) using
$T-\threshold$ over a range of different $\threshold$
values. Choosing a value that linearizes the probability plot (as
discussed in Section~\ref{section:probplot.weibull}) provides a
graphical estimate of $\threshold$.  Then, conditional on the fixed
value of $\threshold$, one can obtain graphical estimates of $\mu$
and $\sigma$.
\item
As illustrated in Section~\ref{section:probplot.extensions}, with a
specified value for the shape parameter $\sigma$, a
log-location-scale distribution with a threshold parameter can be
treated as a location-scale distribution with a location parameter
$\threshold$ and scale parameter $\exp(\mu)$. One can either use ML
to estimate the shape parameter $\sigma$ or try different values of
$\sigma$ to find one that provides a reasonably straight probability
plot.  Conditional on the fixed value of $\sigma$, one can obtain
graphical estimates of $\threshold$ and $\exp(\mu)$.
\end{itemize}
Both of these approaches are easy to implement with flexible computer
programs.


%----------------------------------------------------------------------
%----------------------------------------------------------------------
\subsection{Likelihood methods}
\label{section:like.methods}
The likelihood for a three-parameter log-location-scale distribution
using the ``density approximation'' for exact failures, and allowing
for right-censored observations, has the form
\begin{eqnarray*}
\like(\mu,\sigma,\threshold) &=& \prod_{i=1}^{n} 
\like_{i}(\mu, \sigma,\threshold;\data_{i}) \\
&=&
  \prod_{i=1}^{n}    
 \left\{ 
 f(\realrv_{i};\mu,\sigma,\threshold)
\right\}^{\delta_{i}} 
\left\{1- F(\realrv_{i};\mu,\sigma,\threshold) \right\}^{1-\delta_{i}}
\\
&=&  \prod_{i=1}^{n}
\left\{ \frac{1}{\sigma (\realrv_{i}-\threshold)} \, \phi
\left[\frac{ \log(\realrv_{i}-\threshold) -\mu}{\sigma}
\right]
\right\}^{\delta_{i}}
\left\{1- \Phi \left[\frac{ \log(\realrv_{i}-\threshold) -\mu}{\sigma}
\right] \right\}^{1-\delta_{i}} 
\end{eqnarray*}
where data$_{i}=(t_{i}, \delta_{i})$ is defined as in
(\ref{equation:general.parametric.den.likelihood}).  This is a
classic example for which density approximation defined in
equation~(\ref{equation:density.approximation}) can cause serious
numerical and statistical problems in the application of ML
estimation. The problem is that when this approximation is used,
there can be, for some model/data combinations, a path in the
parameter space for which the likelihood goes to $\infty$.  In
particular, when $\threshold \rightarrow t_{(1)}$ (the smallest
observation) and $\sigma \rightarrow 0$, the likelihood
$\like(\mu,\sigma,\threshold)$ approaches $\infty$.  The likelihood
approaches $\infty$ {\em not} necessarily because the probability of
the data is large in that region of the parameter space, but rather
because of a breakdown in the density approximation in
(\ref{equation:density.approximation}).  For some (but not all) data
sets there is a local maximum for $\like(\mu,\sigma,\threshold)$,
corresponding to the maximum of the correct likelihood (probability
of the data).  Although it has been suggested that one could ignore
the part of the likelihood surface where
$\like(\mu,\sigma,\threshold) \rightarrow
\infty$
and use the local maximum to provide the ML estimates, this practice
can lead to numerical difficulties and it is possible for the local
maximum to be masked by the breakdown of the density approximation. 
A better solution is to 
use the correct likelihood contributions, 
\begin{equation}
\like_{i}(\mu, \sigma,\threshold;\data_{i})
=[F(\realrv_{i}+\Delta_{i};\thetavec)-
        F(\realrv_{i}-\Delta_{i};\thetavec)]
\label{equation:threshold.interval.likelihood}
\end{equation}
(based on small intervals implied by the data's precision) instead of
the density approximation.  The correct likelihood will always be
bounded (because probabilities can be no larger than 1). Using the correct
likelihood eliminates the problem of an unbounded likelihood and helps
simplify the
process of finding the ML estimates. The values
of $\Delta_{i}$ should be chosen to reflect the round off in the data
(which often depends on the magnitude of the observations within a
data set). Generally the shape and position of the likelihood (and
thus the ML estimates) are not very 
sensitive to the value of $\Delta_{i}$ used here.

Numerical problems with fitting threshold parameter distributions can
also arise from the embedded distributions problem described in
Section~\ref{section:embedded.models}. In order to circumvent this
problem, one can fit, instead, the generalized threshold-scale (GETS)
distribution introduced in Section~\ref{section:gets.dist}.  When
$\sigmahat>0$, the ML estimates for this model are equivalent to what
one would get with the corresponding threshold parameter distribution.
When $\sigmahat < 0$, it is an indication that the ML estimates for
the corresponding threshold parameter distribution would be on the
boundary of the parameter space (i.e., the limiting embedded
distribution) and careful consideration should be given to using the
GETS or some other alternative to the corresponding
threshold-parameter distribution.

\begin{example}
{\bf Fitting the
three-parameter Weibull distributions to the Alloy-C strength data.}
Table~\ref{table:alloy.c} gives interval data
for tensile strength (in ksi) from a sample of 84 specimens of Alloy-C.
\begin{table}
\caption{Alloy-C Strength Data.}
\centering\small
\begin{tabular}{ccr}
\\[-.5ex]
\hline
\multicolumn{2} {c} {Strength (ksi)} &\multicolumn{1}{c}{Number} \\
\cline{1-2} 
\multicolumn{2} {c} {Interval Endpoint}&\multicolumn{1}{c}{of}\\
\cline{1-2} 
\multicolumn{1} {c} {Lower}&
\multicolumn{1} {c} {Upper}&
\multicolumn{1} {c} {Failures} \\
\cline{1-3} 
79  & 80  &    1\\
80  & 81  &    0\\
81  & 82  &    4\\
82  & 83  &    4\\
83  & 84  &    9\\
84  & 85  &   25\\
85  & 86  &   21\\
86  & 87  &   18\\
87  & 88  &    2\\
\hline
\end{tabular}\\
\begin{minipage}[t]{2in}
\end{minipage}
\label{table:alloy.c}
\end{table}
%-------------------------------------------------------------------
%----------
%-------------------------------------------------------------------
The test was run to obtain information on the strength of the alloy
when produced with a modified process.  Figure~\ref{figure:alcoa4.hist.ps}
gives a histogram of the data. 
%------------------------------------------------------------------
\begin{figure}
\splusbookfigure{\figurehome/alcoa4.hist.ps}
\caption{Histogram for binned 
strength readings on 84 specimens of Alloy-C.}
\label{figure:alcoa4.hist.ps}
\end{figure}
%-------------------------------------------------------------------
The distribution is skewed to the left and the lowest values of
strength seem far from the origin, relative to the spread in the
data. This figure indicates that a threshold-parameter distribution
could provide an adequate description for the data.

%-------------------------------------------------------------------
Figure~\ref{figure:alcoa4.var.profile.weib.ps} shows a sequence of
fitted three-parameter Weibull distributions in which $\threshold$ 
was fixed at a set of values between $-20$ and $79$.
%-------------------------------------------------------------------
\begin{figure}
\splusbookfigure{\figurehome/alcoa4.var.profile.weib.ps}
\caption{Profile and Weibull probability plots of the Alloy-C
strength data with $\threshold$ varying between $-20$ and 79. Also shown
are approximate 95\% pointwise confidence intervals for $F(t)$ for the
given value of $\threshold$.}
\label{figure:alcoa4.var.profile.weib.ps}
\end{figure}
%-------------------------------------------------------------------
The probability plot in which $\threshold=0$ illustrates the fitting
of a two-parameter Weibull distribution.  As $\threshold$ gets larger
than 70, the likelihood drops off rapidly and, simultaneously, the fit
in the probability plots becomes poorer.  We see, however, that as
$\threshold$ gets smaller, the profile levels off and the fit remains
good. This is an example in which the three-parameter Weibull is
approaching the embedded SEV distribution.  The scales on the
probability plot suggest an intuitive explanation for the embedding
behavior. As $\threshold$ becomes smaller and smaller, we are adding a
larger and larger number to the original strength values. With the
plot running over a range of strength values
that is relatively small, the log axis on
the strength scale is approximately linear and further shifting will
act as a location shift and thus have little effect on the fit of the
distribution to the data.  

Figure~\ref{figure:alcoa4.getssev.gmleprobplot.ps} shows a $\SEV$
probability plot of the Alloy-C strength data nonparametric estimate
of $F(t)$ along with straight lines depicting ML estimates of the
three-parameter Weibull (SEV-GETS with $\sigma>0$) and the SEV
distributions (corresponding to the limiting embedded SEV
distribution with $\threshold=-\infty$ where $\muhat=85.512$ and
$\sigmahat=1.159$).
%-------------------------------------------------------------------
\begin{figure}
\splusbookfigure{\figurehome/alcoa4.getssev.gmleprobplot.ps}
\caption{Smallest extreme value probability plot of the 
Alloy-C strength data showing the three-parameter Weibull (i.e., SEV-GETS
distribution with $\sigma>0$) (solid line)
along with the SEV distribution (dotted line).}
\label{figure:alcoa4.getssev.gmleprobplot.ps}
\end{figure}
%-------------------------------------------------------------------
Either distribution fits the data very well.  The SEV-GETS ML
estimates are $\alphahat=85.503,$ $\sigmahat=.015$, and $\varsigmahat=1.166$.
Then using the relationships in
Section~\ref{section:gets.special.cases}, with $\sigmahat >0$, yields the
three-parameter Weibull distribution ML estimates as
$\thresholdhat=\alphahat-\varsigmahat/\sigmahat=7.77$ and
$\muhat=\log(\varsigmahat/\sigmahat)=4.35$.
%splus> Orig-Parameters:
%splus>   alpha sigma varsigma 
%splus>  85.503 0.015   1.166
%splus> $thetainterp:
%splus>   gamma     mu sigma 
%splus>  7.7686 4.3533 0.015
This plot confirms that the embedded SEV distribution fits well, but
there may be an objection to the use of such a distribution because
there is a positive probability (albeit extremely small) of a negative
strength. In this example, two-parameter Weibull fit, illustrated in
the $\threshold=0$ plot in
Figure~\ref{figure:alcoa4.var.profile.weib.ps}, is indistinguishable
over the range of the data from the three-parameter Weibull fit in
Figure~\ref{figure:alcoa4.getssev.gmleprobplot.ps}.  Thus the
two-parameter Weibull fit provides a physically reasonable,
parsimonious description of the data.
\end{example}


%-------------------------------------------------------------------
\begin{example}
{\bf Fan data and the three-parameter lognormal distribution.} This
example fits the three-parameter lognormal distribution to the fan
data that were introduced in Section~\ref{example:fan.data} and are
given in Appendix Table~\ref{atable:fan.data}. The profile plot in
Figure~\ref{figure:fan.thresh.profile.lnor.d0.ps} shows the breakdown
in the density approximation for the likelihood. 
As $\threshold \rightarrow x_{(1)}$ (the smallest observation),
the profile likelihood for $\threshold$ blows up. To eliminate this
problem, use the correct likelihood. For these data we use
%-------------------------------------------------------------------
%-------------------------------------------------------------------
\begin{figure}
\splusbookfigure{\figurehome/fan.thresh.profile.lnor.d0.ps}
\caption{Density-approximation profile and three-parameter 
lognormal probability plots of the fan data with $\threshold$ varying
between $-100$ and $449.99$. Also shown are approximate 95\%
pointwise confidence intervals for $F(t)$ for the given value of
$\threshold$.}
\label{figure:fan.thresh.profile.lnor.d0.ps}
\end{figure}
%-------------------------------------------------------------------
%-------------------------------------------------------------------
\begin{figure}
\splusbookfigure{\figurehome/fan.thresh.profile.lnor.d01.ps}
\caption{Correct likelihood $(\Delta=.01)$ profile and three-parameter 
lognormal probability plots of the fan data with $\threshold$ varying
between $-100$ and $449.99$. Also shown are approximate 95\%
pointwise confidence intervals for $F(t)$ for the given value of
$\threshold$.}
\label{figure:fan.thresh.profile.lnor.d01.ps}
\end{figure}
%-------------------------------------------------------------------
the likelihood based on recognizing that data were recorded to a
precision of $\pm5$ hours and choose $\Delta_{i}=5$ in
(\ref{equation:threshold.interval.likelihood}).
Figure~\ref{figure:fan.thresh.profile.lnor.d01.ps} shows that, with
the correct likelihood, the profile plot is well behaved with a
clear maximum at a value of $\threshold$ that is a little less than
400.  Figure~\ref{figure:fan.getsnormal.gmleprobplot.ps} is a
lognormal probability plot of the fan data comparing the
three-parameter lognormal and the three-parameter Weibull
distributions with the two-parameter lognormal distribution.  The
NOR-GETS ML estimates in this example are
$\alphahat=41261.2588$, $\sigmahat =2.27164389$, and
$\varsigmahat=92853.2782$ which, because $\sigmahat>0$, yields the
three-parameter lognormal distribution ML estimates as
$\thresholdhat=\alphahat-\varsigmahat/\sigmahat=386.33$ and
$\muhat=\log(\varsigmahat/\sigmahat)=10.618$ and $\sigmahat$ as above.

%splus>$origparam:
%splus>      alpha      sigma    varzeta 
%splus> 41261.2588 2.27164389 92853.2782
%splus>
%splus> $thetainterp:
%splus>  gamma     mu  sigma 
%splus> 386.33 10.618 2.2716
%splus> 
%splus> 41261.2588 -   92853.2782/2.27164389 = 386.331788
%splus> log(92853.2782/2.27164389) =  10.618 
%splus> 
%-------------------------------------------------------------------
\begin{figure}
\splusbookfigure{\figurehome/fan.getsnormal.gmleprobplot.ps}
\caption{Lognormal probability plot of the fan data comparing
the three-parameter lognormal and Weibull distributions with the
two-parameter lognormal distribution.}
\label{figure:fan.getsnormal.gmleprobplot.ps}
\end{figure}
%-------------------------------------------------------------------
 
These analyses give a strong indication that fitting a
threshold-parameter distribution to these data would be overfitting,
unless there were strong physical reasons to suggest that such a
threshold exists. Visually, the two-parameter lognormal distribution
($\mu=10.14$, $\sigma=1.68$) provides a reasonably adequate and
parsimonious fit to the data.
\end{example}



%-------------------------------------------------------------------
\begin{example}
\label{example:alloy.t7987.gets.mle}
{\bf Fitting the
threshold-parameter distributions to the Alloy T7987 data.}
Example~\ref{example:alloy.t7987.weibull.plot.compare} 
suggested that a threshold-parameter distribution might be appropriate
for the Alloy T7987 data given in
Table~\ref{table:alloy.t7987}.
Figure~\ref{figure:at7987.sevgets.gmleprobplot.ps}
is a Weibull probability plot comparing 
the three-parameter lognormal and three-parameter
Weibull distributions for the Alloy T7987 data.
%-------------------------------------------------------------------
\begin{figure}
\splusbookfigure{\figurehome/at7987.sevgets.gmleprobplot.ps}
\caption{Lognormal probability plot comparing 
the three-parameter lognormal and three-parameter
Weibull distributions and the two-parameter
Weibull distribution for the Alloy T7987 data.}
\label{figure:at7987.sevgets.gmleprobplot.ps}
\end{figure}
%-------------------------------------------------------------------
The SEV-GETS ML estimates are $\alphahat=186.3$, $\sigmahat =.76$, and
$\varsigmahat=70.64$ 
which, because $\sigmahat>0$, yields the three-parameter Weibull
distribution ML estimates as
$\thresholdhat=\alphahat-\varsigmahat/\sigmahat=92.99$
and $\muhat=\log(\varsigmahat/\sigmahat)=4.54$.
%splus> Orig-Parameters:
%splus>     alpha     sigma  varzeta 
%splus>  186.2569 0.7574889 70.64656
%splus> $thetainterp:
%splus>     gamma       mu     sigma 
%splus>  92.99275 4.535436 0.7574889
The NORMAL-GETS ML estimates are $\alphahat=162.25$, $\sigmahat=.613$, and
$\varsigmahat=55.28$. Using the relationships in
Section~\ref{section:gets.special.cases}, with $\sigma>0$, yields the
three-parameter lognormal distribution ML estimates as
$\thresholdhat=\alphahat-\varsigmahat/\sigmahat=72.03$ and
$\mu=\log(\varsigmahat/\sigmahat)=4.50$.
%splus> $origparam:
%splus>     alpha     sigma  varzeta 
%splus>  162.2469 0.6127742 55.28317
%splus> $thetainterp:
%splus>   gamma       mu     sigma 
%splus>  72.029 4.502227 0.6127742
In this case there is strong evidence that the threshold parameter is
important for describing the failure-time distribution.  The data do
not suggest a preference for either the three-parameter lognormal or
three-parameter Weibull distribution.
\end{example}
%-------------------------------------------------------------------



%-------------------------------------------------------------------
\subsection{Summary of results  of fitting models to
skewed distributions}

The results in this section indicate some general guidelines for
fitting parametric distributions to skewed data.  We have seen that
as $\threshold \rightarrow -\infty$ and $\sigma
\rightarrow 0$, the shape of the three-parameter threshold
distribution approaches that of the underlying embedded distribution.
The Weibull approaches the form of a smallest extreme value (which is
left-skewed) and the lognormal approaches the form of a normal
distribution (symmetric).  These results indicate that, for purposes
of fitting parametric distributions to data,
\begin{itemize}
\item
If data are left-skewed, even if far from the origin, it is
generally possible to fit a three-parameter Weibull distribution and
achieve a good fit to the data. In many cases, however, it will be
possible to fit the simpler two-parameter Weibull or smallest
extreme value distributions and get, effectively, the same results.
The typical profile likelihood shape for such data is shown in
Figure~\ref{figure:alcoa4.var.profile.weib.ps}.
\item
If data are approximately symmetric, one can generally fit a
three-parameter Weibull, a three-parameter lognormal model, or
two-parameter versions of the distribution and get a reasonably good
fit to the data. In many cases, however, it will be possible to fit
the simpler two-parameter normal or logistic distributions
(depending on the heaviness of the tails) and get, effectively, the
same results.  Often, unless there is a large amount of data
(hundreds of observations), it will be difficult to distinguish
among these alternative distributions.
\item
If the data are right-skewed, it is often possible to fit either the
three-parameter Weibull or the three-parameter lognormal distribution and get
a good fit to the data.
\item
The use of a threshold parameter can be viewed from two different
directions.  Sometimes it might be viewed as a physical parameter---a
time before which probability of failure is zero or a threshold
strength. In such cases it may be important to constrain
$\threshold>0$ or some other number.  In other cases, $\threshold$ is
one of several parameters of a curve being fit to data. In such cases,
the ML estimate of $\threshold$ may be negative, yielding a positive
probability of negative failure time or strength. Generally, however,
the estimated probability of such small events is small enough to ignore.
\end{itemize}

When one of the simpler distributions (e.g., two-parameter
lognormal or two-parameter Weibull) fits ones data well, the simpler
description will be preferred to a threshold distribution,
especially when the amount of data available is limited. When one of
the simpler distributions do not fit, however, using a threshold
parameter may provide an important improvement in data
description. As usual, however, it is important to be especially
cautious when making inferences outside the range of ones data,
especially when the fitted distribution is chosen purely on the
basis of its fit to the data.


\section*{Bibliographic Notes}
Farewell and Prentice~(1977) show that a judicious choice of
parameters for the GENG model can make an important difference in
one's ability to apply ML methods.  They suggest the parameterization
given in Section~\ref{section:egeng.parameterization}.  Lawless~(1982,
Chapter~5) shows how to use likelihood-based methods with the
generalized gamma distribution to assess and compare results from the
special case distributions. Liu, Meeker, and Escobar~(1998) suggest
and illustrate the use of an EGENG parameterization that is stable
even for heavy censoring.

Engelhardt, Bain, and Wright~(1981) describe methods for ML estimation
for the Birnbaum-Saunders distribution. Cohen and Whitten~(1988)
describe estimation methods, including ML, for a wide variety of life
distributions including the Weibull, lognormal, inverse Gaussian,
gamma, and generalized gamma distributions.


Meeker~(1987) provides a more complete analysis and more technical
details for the LFP model described in
Section~\ref{section:using.the.lfp.model}.  Also, Monte Carlo
simulation showed that the likelihood-based confidence intervals
provide a much better approximation to the nominal confidence levels
over a wide range of parameter values for the LFP
model. Trindade~(1991) also uses this model in a similar
application.

In general, fitting mixture distributions with maximum likelihood
presents some difficult and challenging issues.  See, for example,
Day~(1969), Falls~(1970), Hosmer~(1973), Titterington, Smith, and
Makov~(1985), and Gelman, Carlin, Stern, and Rubin~(1995, Chapter 16).
Also Everitt and Hand~(1981) review the pertinent literature.


Turnbull~(1976) presents a generalization of the Kaplan-Meier estimate
for arbitrarily censored and  
truncated data.  Kalbfleisch and Lawless~(1988) describe examples of
field reliability data that can be analyzed using truncated data
methods.  Nelson~(1990b) describes examples of truncated reliability
data and shows how adapt the method of hazard plotting to such data.
Kalbfleisch and Lawless~(1992) provide further examples and methods.
Woodroofe~(1972, 1974),
Schneider~(1986), and Cohen~(1991) are useful references for the
theoretical aspects of truncated data and estimating the parameters of
truncated distributions.  Escobar and Meeker~(1998d) show how to
compute the Fisher information matrix and asymptotic variances for
truncated distributions and the LFP model.


Serious numerical and statistical problems can arise when estimating
the parameters of threshold-parameter distributions, especially when
using the density approximation for the likelihood contributions of
observations reported as exact failures (see Kempthorne and
Folks~1971, Giesbrecht and Kempthorne~1976, and Cheng and Iles~1987
for more details and other references).  Griffiths~(1980) and Smith
and Naylor~(1987) describe likelihood-based inferences for the
three-parameter lognormal and Weibull distributions, respectively.
Also, the asymptotic ML theory for this approach is complicated (e.g.,
see Smith~1985) and, arguably, inappropriate for data with finite
precision.  Cheng and Iles~(1990) noted that the smallest extreme
value (normal) distribution is a limiting case of the three-parameter
Weibull (lognormal) distribution when the threshold parameter
$\rightarrow -\infty$.  Hirose and Lai~(1997) use an example to
illustrate the problems in inference created by embedded models when
fitting a threshold Weibull model with binned data. They propose a
solution to the those problems by embedding the Weibull model in
SEV-$\GETS$ family and using ML methods.

\section*{Exercises}


\begin{exercise}
Wilk, Gnanadesikan, and Huyett~(1962b) give the number of weeks
until failure for a sample of 34 transistors subjected to
accelerated conditions. The reported times, with 
the number of ties shown in parentheses,
were 3, 4, 5, $ 6 (2) $, 7, $8 (2) $, $9 (3)$, $ 10 (2) $, $ 11 (3) $,
$ 13 (5) $, $ 17 (2)$, $19 (2) $, 25, 29, 33, $42 (2)$, 52.  The other 3
transistors had not failed at the end of 52 weeks.
\begin{itemize}
\item
Use ML to estimate the parameters of the BISA, IGAU, gamma, and
lognormal distributions to these data using a discrete-data
likelihood. Plot all of these estimates on lognormal probability paper
and compare the different estimates.  Describe any important
differences that you see in the estimates.
\item
Redo the gamma distribution analyses assuming that the failures
occurred at exactly the reported time.  Are the differences of
practical importance in this example?
\end{itemize}
\end{exercise}


\begin{exercise}
\label{exercise:lfp.sen.anal}
The engineers who collected the IC data from
Example~\ref{example:lfp.trunc} felt that if the life test had been
extended for another 50,000 hours (corresponding to the
technological life of the application system), only another 2 or 3
failures might have been observed.  Use this information to
construct several Weibull and lognormal probability plots for the
failure-time distribution for the defective subpopulation.
\end{exercise}



\begin{exercise}
\label{exercise:lfp.reparameterization}
For the Weibull LFP (limited failure population) model with cdf
given in (\ref{equation:lfp.cdf}),
a proportion $p$ of the units in the 
population is defective and will eventually
fail (according to a Weibull distribution)
and all other units are immune to failure.
\begin{enumerate}
\item
What is the practical interpretation of the parameter $\mu$ in this model?
\item
Generally in a life test of a limited failure population, if
the test is stopped far before all of the units in the population
have failed, the parameters $\mu$ and $p$ will be highly correlated.
Give an intuitive explanation for this.
\item
The expected proportion of units failing in a life test of length
$\censortime$ might be a more ``stable'' parameter to estimate.
Explain why. Write down the reparameterized model, and describe
the steps that you would use to find the ML estimates of this
new parameter as well as $\mu$ and $\sigma$.
\end{enumerate}
\end{exercise}


\begin{exercise}
\label{exercise:lfp.factor.likelihood}
Refer to Exercise~\ref{exercise:lfp.reparameterization}.  Show that
the likelihood for the LFP model with a single censoring time can be
factored into two components, one a binomial with parameter $\pi=p
\times \Phi\{[\log(\censortime)-\mu]/\sigma\}$ and another consisting
of a right-truncated failure time distribution with truncation time
$\tau_{i}^{U}=\censortime$.
\end{exercise}


\begin{exercise1}
Comment on the statistical implications of the factoring in
Exercise~\ref{exercise:lfp.factor.likelihood}.
\end{exercise1}


\begin{exercise}
\label{exercise:trun.dist}
Suppose that $\rv$ has a $\WEIB(\mu, \sigma)$ distribution.
Show that
\begin{enumerate}
\item
\label{exer.part:left.truncated.cdf}
The cdf of the left-truncated Weibull distribution is 
\begin{displaymath}
F(\realrv;\tau^{L}; \mu,\sigma)=
1-\exp
\left [
(\tau^{L}/\weibscale)^{\beta}-(\realrv/\weibscale)^{\beta}
\right ], \quad \realrv>\tau^{L} \ge 0
\end{displaymath}
where $\weibscale=\exp(u)$ and $\beta=1/\sigma$.
\item
\label{exer.part:right.truncated.cdf}
The cdf of the right-truncated Weibull distribution is 
\begin{displaymath}
F(\realrv; \tau^{U}; \mu,\sigma)=
\frac{
1-\exp \left [-(\realrv/\weibscale)^{\beta} \right ]
    }
    {
1-\exp \left [-(\tau^{U}/\weibscale)^{\beta} \right ]
    },\quad 0< \realrv \le \tau^{U}.
\end{displaymath}
\item
\label{exer.part:left.right.truncated.cdf}
The cdf of the left-truncated and right-truncated Weibull
distribution is
\begin{displaymath}
F(\realrv; \tau^{L},\tau^{U}; \mu,\sigma)=
\frac{
 \exp \left [-(\tau^{L}/\weibscale)^{\beta} \right ]
-\exp \left [-(\realrv/\weibscale)^{\beta} \right ]
    }
    {
 \exp \left [-(\tau^{L}/\weibscale)^{\beta} \right ]
-\exp \left [-(\tau^{U}/\weibscale)^{\beta} \right ]
    },\quad \tau^{L}< \realrv \le \tau^{U}.
\end{displaymath}
\end{enumerate}
\end{exercise}

\begin{exercise}
Refer to Exercise~\ref{exercise:trun.dist},
part~\ref{exer.part:left.truncated.cdf}.  What is the cdf of the
truncated distribution when $\sigma=1$?
\end{exercise}

\begin{exercise1}
Derive the expressions given in
parts~\ref{exer.part:left.truncated.cdf} and
\ref{exer.part:right.truncated.cdf} of
Exercise~\ref{exercise:trun.dist} as limiting cases of the
expression in part~\ref{exer.part:left.right.truncated.cdf} when
$\tau \to 0$ and $\tau \to \infty$, respectively.
\end{exercise1}
%----------------------------------------------------------------------
%----------------------------------------------------------------------

%----------------------------------------------------------------------
%----------------------------------------------------------------------
\begin{exercise1}
\label{exercise:trunc.releff.avar}
Consider ML estimation from a random sample of size $n$
from 
an $\EXP(\expmean)$ distribution. Denote by $\thetahat_{n}$, $\thetahat_{c}$
and $\thetahat_{t}$ the ML estimates of $\theta$ from 
complete, right-censored, and right-truncated samples,
respectively. 
Define
the asymptotic relative efficiency of the
estimator $\thetahat_{a}$  with respect to the 
estimator $\thetahat_{b}$ as
$\mbox{RE}(\thetahat_{a},\thetahat_{b})=\avar(\thetahat_{b})/
\avar(\thetahat_{a})$.
\begin{enumerate}
\item
\label{exer.part:avar.com.cen.tru}
Write the Fisher information matrix for the estimators and show that
\begin{eqnarray*}
\avar(\thetahat_{n})&=&\theta^{2}/n
\\
\avar(\thetahat_{c})&=&\avar(\thetahat_{n})/(1-\censorprop)
\\
\avar(\thetahat_{t})&=&\frac{\avar(\thetahat_{n})}
		            {
(1-\truncatedprop) \left \{
                   1-\truncatedprop [\log(\truncatedprop)]^{2}/
	           (1-\truncatedprop)^{2}
		   \right \}
                            }
\end{eqnarray*}
where $\censorprop=\exp(-\censortime/\theta)$ is the
proportion of right censoring and 
$\truncatedprop=\exp(-\tau^{U}/\theta)$ is the proportion
of right truncation.
\item
Use the results in part~\ref{exer.part:avar.com.cen.tru} 
with $\truncatedprop=\censorprop$ to
show that
\begin{eqnarray*}
\mbox{RE}(\thetahat_{c},\thetahat_{n})&=&(1-\censorprop)
\\
\mbox{RE}(\thetahat_{t},\thetahat_{n})&=&(1-\truncatedprop) \left \{
                   1-\truncatedprop [\log(\truncatedprop)]^{2}/
	           (1-\truncatedprop)^{2}
		   \right \}
\\
\mbox{RE}(\thetahat_{t},\thetahat_{c})&=&
                   1-\truncatedprop [\log(\truncatedprop)]^{2}/
	           (1-\truncatedprop)^{2}.
\end{eqnarray*}
When $\truncatedprop=\censorprop=.1$ evaluate the
relative efficiencies and show that
$\mbox{RE}(\thetahat_{t},\thetahat_{n})=.31$
and
$\mbox{RE}(\thetahat_{c},\thetahat_{n})=.90$.
Comment on these efficiencies.
\item
Again suppose that $\truncatedprop=\censorprop$.
In the same plot draw 
$\mbox{RE}(\thetahat_{c},\thetahat_{n})$ and 
$\mbox{RE}(\thetahat_{t},\thetahat_{n})$ as
a function of $\truncatedprop$. Comment
on the effect of right censoring a proportion 
$p$ of units when compared with right truncating
the same proportion of units.
\end{enumerate}
\end{exercise1}
%----------------------------------------------------------------------
%----------------------------------------------------------------------

%----------------------------------------------------------------------
%----------------------------------------------------------------------


\begin{exercise}
\label{exercise:vendor1.trun}
Consider the data in Examples~\ref{example:electronic.subsystem.data},
\ref{example:pretest.and.left.trun},
and \ref{example:pretest.mle}. Truncate the Vendor 1 data at 1000
hours and compare the resulting fit with that for Vendor 2. Can you
detect any difference that might be considered to be important?
\end{exercise}


\begin{exercise}
Consider the results from either
Exercise~\ref{exercise:trunc.releff.avar} or from
Exercise~\ref{exercise:trun.cens.sim}. Provide an intuitive
explanation for the reason that precision from the censored
distribution is much better than from the truncated distribution.
\end{exercise}

\begin{exercise}
During a single month a company sold 2341 modems.  These modems have
a 36-month warranty. During the first 24 months, 75 of these modems
had been returned for warranty service.  Suppose that it is
reasonable to assume that this is the number failing out of the 2341
modems that were sold.  From previous experience with similar
products, it is known that a Weibull distribution with a shape
parameter of $\beta=.85$ provides an adequate description of the
failure time distribution.
\begin{enumerate}
\item
Show that the conditional probability of failing in the third year
of life, given survival up to two years, can be expressed as a
truncated distribution.
\item
\label{exer.part:modem.warranty1}
Although the times to failure for the returned modems were not
available, it is still possible to compute an estimate of the
Weibull cdf using the given value of $\beta=.85$.  Show how to do
this.
\item
Use the estimate from part~\ref{exer.part:modem.warranty1}
to compute an estimate for the number of units that will fail
in the third year of operation.
\item
Suppose that you had learned that two years after being sold,
between 5\% and 10\% of the purchased modems had never been put into service.
How would you do part\ref{exer.part:modem.warranty1}?
\end{enumerate}
\end{exercise}

\begin{exercise1}
Consider the discrete data likelihood
\begin{displaymath}
\like(\thetavec)=\prod_{i=1}^{m} [F(\realrv_{i}+\Delta_{i};\eta,\sigma,\zeta)-
        F(\realrv_{i}-\Delta_{i};\eta,\sigma,\zeta)]^{\deadin_{i}}
\end{displaymath}
where $F(\realrv; \eta,\sigma,\zeta)$ is the $\GETS$ cdf 
in (\ref{equation:gets.cdf}).
\begin{enumerate}
\item
Suppose that the $\GETS$ cdf is Weibull based (i.e.,
$\Phi=\Phi_{\sev}$) and that $\sigma>0$. 
Show that the derivatives of the
log likelihood with respect to the parameters are
discontinuous at $\realrv_{0}$ when 
$\realrv_{0}=(\eta/\zeta-1/\sigma)$ and $\sigma>1$.
\item
Suppose that the $\GETS$ cdf is Weibull based with
$\sigma<0$. 
Show that the derivatives of the
log likelihood with respect to the parameters are
discontinuous at $\realrv_{m}$ when 
$\realrv_{m}=(\eta/\zeta-1/\sigma)$ and $|\sigma|>1$.
\item
Show that if the $\GETS$ cdf is lognormal based,
then the derivatives of the log likelihood
with respect to the parameters are
continuous everywhere.
\end{enumerate}
\end{exercise1}
%----------------------------------------------------------------------
%----------------------------------------------------------------------

\begin{exercise2}
\label{exercise:trun.cens.sim}
Conduct the following simulation to compare 
the effects of censoring and truncation on estimation precision.
\begin{enumerate}
\item
Generate a sample from a Weibull distribution with $\weibscale=100$
hours and $\beta=2$.
\item
Find the ML estimates of the
parameters and $t_{.1}$, treating any
observations beyond 150 hours as right censored.
\item
Find the ML estimates of the
parameters and $t_{.1}$, truncating observations beyond 150 hours.
\item
Repeat the simulation 500 times.  Make appropriate plots of the
sample estimates (including scatter plots to see correlation).
Compute and use histograms or other graphical displays to compare
the estimates from the censored and the truncated samples.
Also compute the sample variances for the parameter estimates
and for the estimates of $t_{.1}$.
\item
What can you conclude from this simulation experiment?
\end{enumerate}
\end{exercise2}
