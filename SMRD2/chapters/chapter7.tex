%chapter 7
%original by wqmeeker  12 Jan 94
%edited by wqmeeker  27/31 Jan 94
%edited by wqmeeker  6/7 feb 94
%edited by driker 2/17/94
%edited by driker 3/11/94
%edited by wqmeeker  11 mar 94
%edited by driker 3/18/94
%edited by wqmeeker  19 mar 94 removed censoring contribution to chap2
%edited by wqmeeker  26 mar 94
%edited by wqmeeker  1 june 94
%edited by driker 18 june 94  europe changes
%edited by wqmeeker  2 aug 94
%edited by wqmeeker  7 aug 94
%edited by wqmeeker  5 oct 94  smoothing
%edited by wqmeeker  22 oct 94  more smoothing
%edited by wqmeeker  01/04 jan 94 bootstrap
%edited by wqmeeker  07 jan 94  zero failures
%edited by driker 3 feb 95
%edited by driker 7 feb 95
%edited by wqmeeker 12 feb 94  ci comparison
%edited by driker 15 feb 94 
%edited by wqmeeker  17 june 95 moving bootstrap
%edited by driker 20 june 95
%edited by driker 27 june 95
%edited by driker 4 nov 95
%edited by driker 11 nov 95
%edited by wqmeeker 3 jan 96 likelihood ratio
%edited by driker 5 jan 96
%edited by driker 1 july 96
%edited by driker 20 nov 96
\setcounter{chapter}{6}

\chapter{Parametric Likelihood Fitting Concepts: Exponential Distribution}
\label{chapter:parametric.ml.one.par}


%{\Large {\bf William Q. Meeker and Luis A. Escobar}}\\
%Iowa State University and Louisiana State University\\[5ex]
{\large {\bf \today}}\\[2ex]
Part of 
{\em Statistical Methods for Reliability Data}\\
Copyright 1997 W. Q. Meeker and L. A. Escobar. \\[1ex]
To be published by John Wiley \& Sons Inc. in 1998.



%----------------------------------------------------------------------
%----------------------------------------------------------------------
\section*{Objectives}
This chapter explains
\begin{itemize} 
\item 
Likelihood for a parametric model using discrete
data.
\item
Likelihood for samples containing right 
and left censored observations.
\item 
Use of parametric likelihood as a tool for data
analysis and inference about a single population or process.
\item
The use of likelihood and normal-approximation
confidence intervals for model parameters and
other quantities of interest.
\item 
The density approximation to the 
likelihood for observations reported as exact failures.
\end{itemize}

%----------------------------------------------------------------------
%----------------------------------------------------------------------
\section*{Overview}
This chapter introduces some basic ideas of parametric maximum
likelihood (ML) methods.  The ideas presented here are used
throughout the rest of the book.  Section~\ref{section:param.like}
shows how to construct the likelihood (probability of the data)
function and describes the basic ideas behind using this function to
estimate a parameter.  Sections~\ref{section:one.par.ci.expmean} to
\ref{section:exponential.ci.comparison} describe methods for computing
confidence intervals for parameters and functions of parameters.
Section~\ref{section:exact.fail.like} describes the commonly used
probability density in the construction of a likelihood function.
Section~\ref{section:expon.zero.fail} shows how to get a confidence
bound on the exponential distribution parameter even if there are no
failures.
%----------------------------------------------------------------------
%----------------------------------------------------------------------

\section{Introduction}
\label{section:ml.intro}
As explained in Chapter 4, parametric distributions, when used 
appropriately, can provide a simple, parsimonious, versatile, visually
appealing failure-time model.
ML is perhaps the most versatile method for
fitting statistical models to data.  The appeal of ML stems from the
fact that it can be applied to a wide variety of statistical models
and kinds of data (e.g., continuous, discrete, categorical, censored,
truncated, etc.), where other popular methods, like least squares, are 
not, in general, satisfactory. In
typical applications, the goal is to use a parametric statistical
model to describe a set of data or a process or population that
generated a set of data. Modern computing hardware and software have
tremendously expanded the feasible areas of application for ML
methods.

Statistical theory (see the bibliographic notes at the end of this
chapter and Appendix Section~\ref{asection:asymptotic.theory.mle}
for some references) shows that, under standard regularity
conditions, ML estimators are ``optimal'' in large samples. More
specifically, this means that
ML estimators are consistent and asymptotically (as the
sample size increases) efficient. That is, among consistent
competitors to ML estimators, none has a smaller asymptotic variance.

Chapter~\ref{chapter:singledist.bayes} 
describes and illustrates the use of the closely related Bayesian
methods that allow one to incorporate prior information into the
model fitting and estimation process. Besides the Bayesian methods
(which require specification of a prior distribution for the unknown
parameters), there is no general theory that suggests alternatives to
ML that will be optimal in finite samples. Comparisons in specific
cases have shown that, for practical purposes, and without
incorporating prior information, it is difficult to improve on ML
methods.


This chapter emphasizes methods, concepts, examples, and
interpretation of data.  Appendix Section~\ref{asection:general.theory}
outlines the general theory.

%----------------------------------------------------------------------
\begin{example}
{\bf Time between $\alpha$-particle emissions of americium-241.}
\label{example:alpha.particle.data}
Although not from the area of reliability, this example is analogous
to certain special reliability applications in which the distribution
of time between events can be described with an exponential
distribution.

Berkson~(1966) investigates the randomness of $\alpha$-particle
emissions of americium-241 (which has a half-life of about 458 years).
Physical theory suggests that, over a short period of time, the
interarrival times of observed particles would be independent and
come from an exponential distribution
\begin {equation}
\label{equation:exponential.cdf}
F(\realrv;\expmean) = 1 - \exp 
\left( - \frac{\realrv}{\expmean} \right)
\end {equation}
where $\expmean$ is the mean time between arrivals. The corresponding 
homogeneous Poisson process that counts the number of emissions on the
real-time line (see Section~\ref{section:exponential.distribution} for
more information on the exponential distribution and
Chapter~\ref{chapter:repairable.system} for more information on the
Poisson process) has arrival rate or the intensity $\lambda=1/\expmean$. 
For the interarrival times of $\alpha$ particles,
$\lambda$ is proportional to the americium-241 decay rate, size of the 
sample, the counter size and efficiency, etc.

The data consisted of 10,220 observed interarrival
times of $\alpha$ particles (time unit equal to 1/5,000 second). The
observed interarrival times were put into intervals (or bins) running
from 0 to 4,000 time units with interval lengths ranging from 25 to
100 time units, with one additional interval for observed times exceeding
4,000 time units.  To save space in our analysis, this example uses a
smaller number of larger bins; reducing the number of bins in this way
will not seriously affect the precision of ML estimates.  These data
are shown in Table~\ref{table:berkson.alpha.particle}.

To illustrate the effects of sample size on the inferences, simple
random samples (i.e., each interval-censored interarrival time having
equal probability) of sizes $n$ = 2,000, 200, and 20 were drawn 
with replacement from
these interarrival times.  The following examples compare the results
that one obtains with these different sample sizes.  When focusing
on just one sample, the sample of size $n=200$ interarrival times is
used.
%----------------------------
\begin{table}
\caption{Binned $\alpha$-particle interarrival time data in 1/5,000 seconds.}
\centering\small
\begin{tabular}{*{6}{r}}
\\[-.5ex]
\hline
\multicolumn{2} {c} {}&
\multicolumn{4} {c} {Interarrival Times} \\
\multicolumn{2} {c} {Time}&
\multicolumn{4} {c} {Frequency of Occurrence} \\
\cline{1-2} \cline{3-6}
\multicolumn{2} {c} {Interval Endpoint}&
\multicolumn{1} {c} {All Times}&
\multicolumn{3} {c} {Random Samples of Times} \\
\cline{1-2} \cline{4-6}
\multicolumn{1} {c} {lower}&
\multicolumn{1} {c} {upper}&
\multicolumn{1} {c} {$n=10220$}&
\multicolumn{1} {c} {$n=2000$}&
\multicolumn{1} {c} {$n=200$}&
\multicolumn{1} {c} {$n$=20}\\
\cline{1-2} \cline{3-6}
 0 & 100 & 1609 & 292 & 41 & 3 \\ 100 & 300 & 2424 & 494 & 44 & 7 \\ 300 & 500
& 1770 & 332 & 24 & 4 \\ 500 & 700 & 1306 & 236 & 32 & 1 \\ 700 & 1000 & 1213
& 261 & 29 & 3 \\ 1000 & 2000 & 1528 & 308 & 21 & 2 \\ 2000 & 4000 & 354 & 73
& 9 & 0 \\ 4000& $ \infty$ & 16 & 4 & 0 & 0
\\
\hline
\end{tabular}
\label{table:berkson.alpha.particle}
\end{table}
\end{example}


Statistical modeling, in practice, is an iterative procedure of fitting
proposed models in search of a model that provides an adequate
description of the population or process of interest,
without being unnecessarily complicated.  Application of
ML methods generally starts with a set of data and a tentative
statistical model for the data.  The tentative model is often
suggested by the initial graphical analysis
(Chapter~\ref{chapter:probability.plotting}), physical theory,
previous experience with similar data, or other expert knowledge.

%----------------------------------------------------------------------
\begin{example}
{\bf Probability plot for the $\alpha$-particle data.}
Figure~\ref{figure:alpha.nppp.exp.ps} shows an exponential
probability plot of the sample with $n=200$.  The approximate
linearity of the plot indicates that the exponential distribution
provides a good fit to these data. This is reinforced by the
simultaneous nonparametric confidence bands.
%-------------------------------------------------------------------
\begin{figure}
\splusbookfigure{\figurehome/alpha.nppp.exp.ps}
\caption{Exponential probability plot of
the $n = $ 200 sample of $\alpha$-particle interarrival time
data. The plot also shows 
simultaneous nonparametric approximate 95\% confidence bands.}
\label{figure:alpha.nppp.exp.ps}
\end{figure}
%----------------------------------------------------------------------
\end{example}

%----------------------------------------------------------------------
%----------------------------------------------------------------------
\section{Parametric Likelihood}
\label{section:param.like}
%----------------------------------------------------------------------
\subsection{Probability of the data}

Proceeding from the ideas introduced in
Section~\ref{section:probability.of.data}, but now using a parametric
model (as described in
Chapter~\ref{chapter:ls.parametric.models}), the likelihood function can be
viewed as the {\em probability of the observed data}, written as a
function of the model's parameters.  For a parametric model, the
number of parameters is usually small relative to the
nonparametric models described and used in
Chapters~\ref{chapter:np.models.censoring.likelihood} and
\ref{chapter:nonparametric.estimation}. The exponential distribution
in (\ref{equation:exponential.cdf}) has only one parameter.

For a set of $n$ independent observations, the likelihood function
can be written as the following joint probability
\begin{equation}
\label{equation:general.parametric.likelihood}
\like(\theta)=\like(\theta;\DATA)= \likeconstant \prod_{i=1}^{n}
\like_{i}(\theta;\data_{i}).
\end{equation}
As described in Section~\ref{section:form.of.constant}, the quantity
$\likeconstant$ in (\ref{equation:general.parametric.likelihood}) is a
constant term that does not depend on the data or on $\theta$
(in general $\theta$ can be a vector, but in this chapter it is a 
scalar).  As
in Chapter~\ref{chapter:nonparametric.estimation}, for computational
purposes, let $\likeconstant=1$.  The likelihood contribution terms
$\like_{i}(\theta;\data_{i})$ were explained in detail in
Section~\ref{section:likelihood.contributions}. For example, if a
failure time is known to have occurred between times $\realrv_{i-1}$
and $\realrv_{i}$, the probability of this event is
\begin{equation}
\label{equation:parametric.interval.likelihood}
\like_{i}(\theta;\data_{i})=\like_{i}(\theta)=
\int_{t_{i-1}}^{t_{i}} f(t;\theta) \, dt = 
F(\realrv_{i};
\theta)-F(\realrv_{i-1};\theta).
\end{equation}
For a given set of data, $\like(\theta)$ can be viewed as a function
of $\theta$.  The dependence of $\like(\theta)$ on the data will
be understood and is usually suppressed in notation.  The values
of $\theta$ for which $\like(\theta)$ is relatively large are
more plausible than values of $\theta$ for which the probability of
the data is relatively small.  There may or may not be a unique value
of $\theta$ that maximizes $\like(\theta)$.  Regions in the
space of $\theta$ with relatively large $\like(\theta)$ can be
used to define confidence regions for $\theta$.  One can also use ML
to estimate {\em functions} of $\theta$.
The rest of this chapter shows how to make these concepts
operational for the single-parameter exponential distribution,
using simple examples for illustration. Subsequent chapters treat
models with two or more parameters.


%----------------------------------------------------------------------
\subsection{Likelihood function and its maximum}
\label{section:loglikelihood}
Given a sample of $n$ independent observations, denoted generically by
$\data_{i}$, $i=1, \ldots , n$, and a specified model, the total
likelihood $\like(\theta)$ for the sample is given by
equation~(\ref{equation:general.parametric.likelihood}).  
For some purposes, it is convenient to use the log
likelihood $\loglike_{i}(\theta) = \log[\like_{i}(\theta)]$.
For all practical problems $\loglike(\theta)$ will be
representable in computer memory without special scaling (which is not
so for $\like(\theta)$ because of possible extreme exponent
values), and some theory for ML is developed more naturally in terms of
sums like
\begin{displaymath}
\loglike(\theta) = \log[\like(\theta)] = \sum_{i=1}^{n}
\loglike_{i}(\theta)
\end{displaymath}
rather than in terms of the products
in equation~(\ref{equation:general.parametric.likelihood}). Note that the
maximum of $\loglike(\theta)$, if one exists, occurs at
the same value of $\theta$ as the maximum of
$\like(\theta)$.


%----------------------------------------------------------------------
\begin{example}
\label{example:alpha.likelihood}
{\bf Likelihood for the $\alpha$-particle data.}
In this example, the unknown parameter $\expmean$ is a scalar and this
makes the analysis particularly simple and provides a useful first
example to illustrate basic concepts.  Substituting
equation~(\ref{equation:exponential.cdf}) into
(\ref{equation:parametric.interval.likelihood}), and
(\ref{equation:parametric.interval.likelihood}) into
(\ref{equation:general.parametric.likelihood}) gives the exponential
distribution likelihood function (joint probability) for interval
data (e.g., Table~\ref{table:berkson.alpha.particle}) as
\begin{eqnarray}
\label{equation:grouped.exponential.likelihood}
\like(\expmean)&=& \prod_{i=1}^{n} \like_{i}(\expmean)= \prod_{i=1}^{n} 
\left [ F(\realrv_{i};\expmean)-F(\realrv_{i-1};\expmean)
 \right ] \\
             &=& \prod_{j=1}^{8} 
\left [ F(\realrv_{j};\expmean)-F(\realrv_{j-1};\expmean)
 \right ]^{d_{j}} = \prod_{j=1}^{8} \left[
\exp \left(- \,\frac{\realrv_{j-1}}{ \expmean} \right)-
\exp \left( -\,\frac{\realrv_{j}}{ \expmean} \right)
\right]^{d_{j}} 
\nonumber
\end{eqnarray}
where $d_{j}$ is the number of interarrival times in
interval $j$. Notice that in the first line of
(\ref{equation:grouped.exponential.likelihood}), the product is over
the $n$ observed times and in the second line, it is over the 8 bins into
which the data have been grouped.
\end{example}

The ML estimate of $\theta$ is found by
maximizing $\like(\theta)$.  When there is a unique global maximum,
$\thetahat$ denotes the value of $\theta$ that maximizes
$\like(\theta)$.  In general, however, the maximum may not be
unique. The function $\like(\theta)$ may have multiple local maxima or can
have flat spots along which $\like(\theta)$ changes slowly, if at all.
Such flat spots may or may not be at the maximum value of
$\like(\theta)$. The shape and magnitude of $\like(\theta)$ relative
to $\like(\thetahat)$ over all possible values of $\theta$ describe
the information on $\theta$ that is contained in $\data_{i}, i=1,
\ldots , n$.

%----------------------------------------------------------------------
%----------------------------------------------------------------------
\begin{example}
{\bf Relative likelihood for the $\alpha$-particle data.}
%-------------------------------------------------------------------
\begin{figure}
\splusbookfiguresize{\figurehome/alpha.exp.thetapro.ps}{5in}
\caption{Relative likelihood functions
$R(\expmean)=\like(\expmean)/\like(\expmeanhat)$ for the $n= 20$, 200, 
and 2,000 samples and ML estimate for the $n=$10,220 sample of the
$\alpha$-particle data. Vertical lines give corresponding
approximate $95\%$ likelihood confidence intervals.}
\label{figure:alpha.exp.thetapro.ps}
\end{figure}
%----------------------------------------------------------------------
Figure~\ref{figure:alpha.exp.thetapro.ps} shows the {\em relative
likelihood functions}
\begin{displaymath}
R(\expmean)=\frac{\like(\expmean)}{\like(\expmeanhat)}
\end{displaymath}
for the samples of size $n=$ 2,000, 200 and 20 and a vertical line at
the mean of all $n=$10,220 times.  These functions allow one to judge the
probability of the data for values of $\expmean$, {\em relative} to
the probability at the ML estimate.  For example, 
$R(\expmean_{0})=.1$
implies that the probability 
of the data is 10 times larger at $\expmeanhat$ than
at $\expmean_{0}$. The next section explains 
how to use $R(\expmean)$
to compute confidence intervals for $\expmean$.



Figure~\ref{figure:alpha.exp.thetapro.ps} indicates that the spread of
the likelihood function tends to decrease as the sample size increases.
The relative
likelihood functions for the larger samples are much tighter
than those for the smaller samples, indicating that the larger samples
contain more information about $\expmean$.  The $\expmean$ value at
which the different likelihood functions are maximized is random and
depends, in this comparison, on the results of the sampling described
in Example~\ref{example:alpha.particle.data}. The four $\expmeanhat$'s
differ, but they are consistent with the variability that one would expect
from random sampling using the corresponding sample sizes.  
\end{example}

%----------------------------------------------------------------------
\subsection{Comparison of 
$\alpha$-particle data analyses}
%-------------------------------------------------------------------
%-------------------------------------------------------------------
\begin{figure}
\splusbookfiguresize{\figurehome/alpha.mlep.exp.ps}{4in}
\caption{Exponential probability plot of
the $n = $ 200 sample of $\alpha$-particle interarrival time data. The
plot also shows the parametric exponential ML estimate and approximate
95\% confidence intervals for $F(\realrv;\theta)$.}
\label{figure:alpha.mlep.exp.ps}
\end{figure}
%-------------------------------------------------------------------
Figure~\ref{figure:alpha.mlep.exp.ps} shows another exponential
probability plot of the $n=200$ sample.  The solid line on this graph
is the ML estimate of the exponential cdf $F(\realrv;\theta)$.  The dotted
lines are drawn through a set of pointwise parametric
normal-approximation 95\% confidence 
intervals for $F(\realrv;\theta)$; these parametric intervals will be
explained in Section~\ref{section:one.par.ci.expmean}.
Table~\ref{table:berkson.results} summarizes the results of fitting
exponential distributions to the four different samples in
Table~\ref{table:berkson.alpha.particle}; it includes ML estimates,
standard errors, and confidence intervals.
Section~\ref{section:one.par.ci.expmean} provides results specifically
for $\expmean$, the mean (which is also the .632 quantile, $t_{.632}$) of the
exponential distribution.
Section~\ref{section:ci.on.lambda} shows how to obtain
similar results for $\lambda=1 / \expmean$, the arrival intensity rate
(per unit of time).
%-------------------------------------------------------------------
\begin{table}
\caption{Comparison of $\alpha$-particle ML results}
\centering\small
\begin{tabular}{*{6}{r}}
\\[-.5ex]
\hline
\multicolumn{2} {c} {}&
\multicolumn{1} {c} {All Times}&
\multicolumn{3} {c} {Sample of Times} \\
\cline{4-6}
\multicolumn{1} {c} {}&
\multicolumn{1} {c} {}&
\multicolumn{1} {c} {$n=10$,220}&
\multicolumn{1} {c} {$n=2$,000}&
\multicolumn{1} {c} {$n=200$}&
\multicolumn{1} {c} {$n=20$}\\
\cline{1-2} \cline{3-6} \\ [-1.5ex]
\multicolumn{1} {l} {ML Estimate $\expmeanhat$}
&&  596.3 &  612.8 & 572.3  & 440.2  \\[1ex]
\multicolumn{1} {l} {Standard Error $ \sehat_{\expmeanhat}$}
&&  6.084 & 14.13 & 41.72 & 101.0  \\[1ex]
\multicolumn{1} {l} {Approximate 95\%}\\
\multicolumn{1} {l} {Confidence Intervals}
&&&&& \\
\multicolumn{1} {l} {for $\expmean$ Based on}
&&&&& \\
\multicolumn{1} {l} {\hspace{1em}  Likelihood}
&& [584, 608] & [586, 641] & [498, 662] & [289, 713]  \\[.51ex]
\multicolumn{1} {l} {\hspace{1em}
        $Z_{\log(\expmeanhat)} \approxdist \NOR(0,1)$ }
&& [584, 608] & [586, 641] & [496, 660] & [281, 690]  \\[.51ex]
\multicolumn{1} {l} {\hspace{1em}
        $Z_{\expmeanhat} \approxdist \NOR(0,1)$ }
&& [584, 608] & [585, 640] & [490, 653] & [242, 638]  \\[3.1ex]
\hline
\multicolumn{1} {l} {ML Estimate $\lambdahat \times 10^{5}$}
&& 168 & 163& 175 & 227  \\[1ex]
\multicolumn{1} {l} {Standard Error $ \sehat_{\lambdahat \times 10^{5} }$}
&& 1.7 & 3.8& 13 & 52  \\[1ex]
\multicolumn{1} {l} {Approximate 95\%}\\
\multicolumn{1} {l} {Confidence Intervals}
&&&&& \\
\multicolumn{1} {l} {for $\lambda\times 10^{5}$ Based on}
&&&&& \\
\multicolumn{1} {l} {\hspace{1em} Likelihood}
&& [164, 171] & [156, 171] & [151, 201]& [140, 346] \\[.51ex]
\multicolumn{1} {l} {\hspace{1em}
        $Z_{\log(\lambdahat)} \approxdist \NOR(0,1)$ }
&& [164, 171] & [156, 171] & [152, 202]& [145, 356] \\[.51ex]
\multicolumn{1} {l} {\hspace{1em}
        $Z_{\lambdahat} \approxdist \NOR(0,1)$ }
&& [164, 171] & [156, 171] & [149, 200]& [125, 329] \\[.51ex]
\hline
\end{tabular}
\label{table:berkson.results}
\end{table}
%-------------------------------------------------------------------



%----------------------------------------------------------------------
%----------------------------------------------------------------------
\section{Confidence Intervals for $\expmean$}
\label{section:one.par.ci.expmean}
%----------------------------------------------------------------------
\subsection{Likelihood confidence intervals for $\theta$}
\label{section:lr.ci.expmean}
The likelihood function provides a versatile method for assessing the
information that the data contains on parameters, or functions of
parameters.  Specifically, the likelihood function provides a
generally useful method for finding approximate confidence 
intervals for parameters and functions of parameters.

An approximate $100(1-\alpha)\%$ 
likelihood-based confidence interval for $\expmean$
is the set of all values of $\expmean$ such that
\begin{displaymath}
-2 \log[R(\expmean)] \leq \chisquare_{(1-\alpha;1)}
\end{displaymath}
or, equivalently, the set defined by
\begin{displaymath}
R(\expmean) \geq \exp \left [-\chisquare_{(1-\alpha;1)}/2 \right ].
\end{displaymath}
The theoretical justification for this interval is given in
Appendix Section~\ref{section:profile.on.theta}.
\begin{example}
{\bf Likelihood confidence intervals for the mean time between
arrivals of $\alpha$ particles.}
Figure~\ref{figure:alpha.exp.thetapro.ps} illustrates likelihood
confidence intervals.  The horizontal line at
$\exp[-\chisquare_{(.95;1)}/2] = .147$, corresponds to approximate
95\% confidence intervals. The vertical lines dropping from the
respective curves give the endpoints of the confidence intervals for
the different samples.  Table~\ref{table:berkson.results} gives
numerical values of likelihood-based approximate 95\%
confidence intervals (as well as intervals based on other methods to
be explained subsequently).
Figure~\ref{equation:general.parametric.likelihood} shows that
increasing sample size tends to reduce confidence interval
length. Approximate (large-sample) theory shows that confidence
interval length under standard regularity conditions is
approximately proportional to $1/\sqrt{n}$ (see
Appendix Section~\ref{asection:asymptotic.theory.mle} for an outline of this
theory and Chapter~\ref{chapter:test-planning} for methods of
choosing the sample size to control the width of a confidence
interval).
\end{example}

A one-sided approximate $100(1-\alpha)\%$ confidence bound can be
obtained by drawing the horizontal line at
$\exp[-\chisquare_{(1-2\alpha;1)}/2]$ and using the appropriate end
point of the resulting two-sided confidence interval.

\begin{example}
{\bf One-sided likelihood-based confidence bounds for the mean time
between arrivals of $\alpha$ particles.} 
\label{example:one.sided.lr.exponential}
Referring to Figure~\ref{figure:alpha.exp.thetapro.ps}, the horizontal
line at $\exp[-\chisquare_{(.95;1)}/2] = .147$ would provide one-sided
approximate 97.5\% confidence bounds for $\theta$. For
one-sided approximate 95\% confidence bounds the line would be drawn
at $\exp[-\chisquare_{(.90;1)}/2]$ = .259 (corresponding to .90 on the
right-hand scale on Figure~\ref{figure:alpha.exp.thetapro.ps}).
\end{example}

%----------------------------------------------------------------------
\subsection{Relationship between confidence 
intervals and significance tests}
\label{section:one.par.ci.hyp.relat}

Significance testing (sometimes called hypothesis testing) is a
statistical technique widely used in many areas of science. The basic
idea is to assess the reasonableness of a claim or hypothesis
about a model or parameter value,
relative to observed data. One can test a hypothesis by first constructing a
$100(1-\alpha\%)$ confidence interval for the quantity of interest and
then checking to see if the interval encloses the hypothesized value 
or not. If not, then the hypothesis is rejected ``at the
$\alpha$ level of significance.'' If the interval encloses the 
hypothesized value, then the appropriate conclusion is that the data are
consistent with the hypothesis (it is important, however, to
note that failing to reject a hypothesis is not the same as saying
that the hypothesis is true---see the following examples).  Most
practitioners find confidence intervals much more informative than the
yes-no result of an significance test. See pages 39-40 of Hahn and
Meeker~(1991) and other references given there for further discussion
of this subject.

To be more formal, a likelihood ratio test for a single-parameter
model can be done by comparing the maximum of the likelihood under the
``null hypothesis'' to the maximum of the likelihood over all possible
values for the parameter.  A likelihood much smaller under the null
hypothesis provides evidence to refute the hypothesis. Specifically,
for the exponential distribution, the single-point null hypothesis
$\theta=\theta_{0}$ should be rejected if
\begin{equation}
\label{equation:lr.test.exp}
-2\log \left [ { \like(\theta_{0})}/{ \like(\thetahat) } \right ] 
> \chisquare_{(1-\alpha;1)}
\end{equation}
where $\thetahat$ is the ML estimate of $\theta$.  Rejection implies
that the data are not consistent with the null hypothesis.  Using the
definition given in Section~\ref{section:lr.ci.expmean}, it is easy to
see that a likelihood-based confidence interval is the set of
all values of $\theta$ that would not be rejected under the likelihood
ratio test defined in (\ref{equation:lr.test.exp}).


\begin{example}
{\bf Likelihood-ratio test for the mean time between arrivals of
$\alpha$ particles.} Suppose that investigators conducted the
$\alpha$-particle experiment to test the hypothesis that the mean time
between arrivals of $\alpha$ particles is $\theta=650$. Based on the
confidence intervals for $n=$2000 in Table~\ref{table:berkson.results}, we
would have to conclude that there is not enough evidence to reject
this hypothesis. Correspondingly,
\begin{displaymath}
-2\log \left [{\like(650)}/{\like(572.3)}\right] 
= 2.94  < \chisquare_{(.95;1)} = 3.84
\end{displaymath} 
again showing that there is not sufficient evidence in the $n=200$
sample to reject the hypothesis.  Using the $n=$2000 sample, however, does
provide sufficient evidence to reject the hypothesis that
$\theta=650$ at the 5\%
level of significance, as 650 is not in the 95\% confidence
interval.
\end{example}

%----------------------------------------------------------------------
\subsection{Normal-approximation confidence intervals 
for $\theta$}
\label{section:normal.theory.exponential}
A $100(1-\alpha)\%$ normal-approximation confidence
interval for $\expmean$ is
\begin{equation}
\label{equation:exponential.interval.nolog}
 [ \undertilde{\expmean}, \quad \tilde{\expmean}] =
\expmeanhat \pm \norquan_{(1-\alpha/2)} \sehat_{\expmeanhat}.
\end{equation}
A one-sided approximate $100(1-\alpha)\%$
confidence bound can be obtained by replacing
$\norquan_{(1-\alpha/2)}$ with $\norquan_{(1-\alpha)}$ 
in (\ref{equation:exponential.interval.nolog}) and using the
appropriate endpoint of the resulting two-sided confidence interval.

An estimate of the standard error of 
$\expmeanhat$ is typically computed from the
``observed information'' as
\begin{equation}
\label{equation:scalar.hessian}
 \sehat_{\expmeanhat} = 
\sqrt{ \left[-\frac{d^{2} \loglike(\expmean)}
{d \expmean^{2}} \right]^{-1}}
\end{equation}
where the second derivative is evaluated at $\expmeanhat$.  This
computation is a special case of (\ref{equation:gcovariance.est}) in
Appendix Section~\ref{asection:observed.information}.  The second derivative
measures curvature of $\loglike(\expmean)$ at $\expmeanhat$. If
$\loglike(\expmean)$ is approximately quadratic, large curvature
implies a narrow likelihood and thus a small estimate of the standard
error of $\expmeanhat$.

The approximate confidence interval in
(\ref{equation:exponential.interval.nolog}) is based on the 
assumption that the distribution of
\begin{equation}
\label{equation:z.theta.exp}
Z_{\thetahat} = \frac{\expmeanhat-\expmean}{\sehat_{\expmeanhat}}
\end{equation}
can be approximated by a $\NOR(0,1)$ distribution.
Then 
\begin{equation}
\label{equation:zthetahat.normal}
\Pr \left[ \norquan_{(\alpha/2)} < Z_{\thetahat}  \leq
	\norquan_{(1-\alpha/2)} \right]	\approx 1-\alpha
\end{equation}
which implies
\begin{equation}
\label{equation:theta.ci.normal}
\Pr \left[  \expmeanhat - \norquan_{(1-\alpha/2)} \sehat_{\expmeanhat}
	 <
	\expmean \leq
	 \expmeanhat + \norquan_{(1-\alpha/2)} \sehat_{\expmeanhat}\right]
	\approx 1-\alpha
\end{equation}
because $z_{(1-\alpha/2)}=-z_{(\alpha/2)}$.
The approximation is usually better for large samples, but may be poor
for small samples.  See Appendix
Section~\ref{asection:convergence.in.distribution} for more
information on such large-sample approximations.

\begin{example}
{\bf Normal-approximation confidence intervals for the
mean time between arrivals of $\alpha$ particles.}
For the $n=200$ $\alpha$-particle data 
$\sehat_{\expmeanhat} = 41.72$ and an approximate 95\% confidence 
interval for $\expmean$ based on the assumption that  
$Z_{\expmeanhat} \approxdist \NOR(0,1)$ is
\begin{displaymath}
 [ \undertilde{\expmean}, \quad \tilde{\expmean}] =
572.3 \pm 1.960 (41.72) = [490, \quad 653].
\end{displaymath}
Thus we are 95\% confident that $\theta$ is in this interval.
\end{example}

An alternative approximate confidence interval for positive quantities
like $\theta$ is
\begin{equation}
\label{equation:exponential.interval.log}
[ \undertilde{\expmean}, \quad \tilde{\expmean}]=
	[\expmeanhat/w, \quad \expmeanhat \times w]
\end{equation}
where $w=\exp(\norquan_{(1-\alpha/2)}\sehat_{\expmeanhat}/\expmeanhat)$.
This interval is based on the assumption that the distribution of
\begin{equation}
\label{equation:expmean.pivotal.like}
Z_{\log(\expmeanhat)} = \frac{\log(\expmeanhat)-
	\log(\expmean)}{\sehat_{\log(\expmeanhat)}}
\end{equation}
can be approximated by a $\NOR(0,1)$ distribution
where $\sehat_{\log(\expmeanhat)}=\sehat_{\expmeanhat}/\expmeanhat$
is obtained by using the delta method in
Appendix Section~\ref{asection:delta.method}. The confidence interval in 
(\ref{equation:exponential.interval.log})
follows because an approximate $100(1-\alpha)$\% confidence
interval for $\log(\expmean)$ is
\begin{displaymath}
[\undertilde{\log(\expmean)}, \quad  \tilde{\log(\expmean)}] =
\log(\expmeanhat) \pm \norquan_{(1-\alpha/2)}
\sehat_{\log(\expmeanhat)} .
\end{displaymath}


For a parameter,
like $\expmean$, that must be positive,
(\ref{equation:exponential.interval.log}) is often suggested as
providing positive interval endpoints and probably
a more accurate approximate interval than
(\ref{equation:exponential.interval.nolog}).  Although there is no
guarantee that (\ref{equation:exponential.interval.log}) will be more
accurate than (\ref{equation:exponential.interval.nolog}) in a
particular setting, the sampling distribution of
$Z_{\log(\expmeanhat)}$ is usually more symmetric than that of
$Z_{\expmeanhat}$ and the log transformation ensures that the
lower endpoint of the confidence interval will be positive [which is
not always so for confidence intervals based on
(\ref{equation:exponential.interval.nolog})].

\begin{example}
{\bf Normal-approximation confidence intervals for the mean time
between arrivals of $\alpha$ particles.} For the $\alpha$-particle
data, an approximate 95\% confidence interval for $\expmean$ 
based on the assumption that $Z_{\log(\expmeanhat)}\approxdist \NOR(0,1)$
is
\begin{displaymath}
 [ \undertilde{\expmean},\quad  \tilde{\expmean}] =
[572.3/1.1536, \quad
	 572.3 \times 1.1536] = [496, \quad 660]
\end{displaymath}
where $w=\exp\{1.960 \times 41.72/572.3\}=1.1536$.
\end{example}



%----------------------------------------------------------------------
%----------------------------------------------------------------------
\section{Confidence Intervals for Functions of $\expmean$}
\label{section:one.par.ci.functions}

For one-parameter distributions like the exponential, confidence
intervals for $\expmean$ can be translated directly into confidence
intervals for monotone functions of $\expmean$.

%-------------------
\subsection{Confidence intervals for the arrival rate}
\label{section:ci.on.lambda}
The arrival rate $\lambda=1/\expmean$ is a {\em decreasing} function
of $\expmean$. Thus the upper limit $\tilde{\expmean}$ is
substituted for $\expmean$ to get a lower limit for $\lambda$ and
vice versa.  The confidence interval for $\lambda$ obtained in this
manner will contain $\lambda$ if and only if the corresponding
interval for $\expmean$ contains $\expmean$. Thus the confidence
interval for $\lambda$ has the same confidence level as the interval
for $\expmean$.
\begin{example} 
{\bf Likelihood-based confidence intervals for the arrival rate
of $\alpha$ particles.} Using the 
likelihood-based confidence interval for the $n=200$ sample in
Table~\ref{table:berkson.results},
\begin{displaymath}
 [ \undertilde{\lambda}, \quad  \tilde{\lambda} ] = 
 [1/ \tilde{\expmean}, \quad  1/ \undertilde{\expmean} ] =  [.00151,
\quad  .00201].
\end{displaymath}
Also, the ML estimate of $\lambda$ is obtained as
$\lambdahat  =  1/\expmeanhat = .00175 $.
\end{example}

%------------------------
\subsection{Confidence intervals for $F(\realrv;\theta)$}
\label{section:exp.ci.f}
Because $F(\realrv;\expmean)$ is a decreasing
function of $\expmean$, confidence intervals for the exponential
distribution $F(\estimtime;\expmean)$ for a particular
$\estimtime$ is
\begin{displaymath}
[\Flower(\estimtime), \quad \Fupper(\estimtime)] = 
[ F(\estimtime;\tilde{\expmean}), \quad 
F(\estimtime;\undertilde{\expmean})].
\end{displaymath}
One can compute a set of pointwise confidence intervals
for a range of values of $\realrv$. The set can, in this case
(unlike in Section~\ref{section:np.simultaneous.cb}), also be
interpreted as simultaneous confidence bands for the entire
exponential cdf $F(\realrv;\expmean)$.  This is because $\expmean$
is the only unknown parameter for this model and the bands will contain
the unknown exponential cdf $F(\realrv;\expmean)$ if and only if the
corresponding confidence interval for $\expmean$ contains the unknown
true $\expmean$.


\begin{example}
{\bf Confidence intervals for the $\alpha$-particle time between
arrivals cdf.}
The dotted lines in Figure~\ref{figure:alpha.mlep.exp.ps} are drawn
through a set of pointwise normal-approximation  95\%
confidence intervals for the exponential $F(\realrv;\expmean)$.
\end{example}

In subsequent chapters  where
models have more than one parameter (as with the nonparametric simultaneous
confidence bands in Section~\ref{section:np.simultaneous.cb}),
a collection of intervals must be
handled differently because the confidence level applies only to the
process of constructing an interval for a single point in time
$\estimtime$.  Generally, making a simultaneous statement would
require either a wider set of bands or a lower level of confidence.
%(see Section~\ref{section:locscale.simultaneous} for further
%discussion of parametric simultaneous confidence bands).


%----------------------------------------------------------------------
\section{Comparison of Confidence Interval Procedures}
\label{section:exponential.ci.comparison}
For the particle arrival data, Table \ref{table:berkson.results}
compares approximate 95\% confidence intervals for $\expmean$
based on the likelihood, the normal approximation
$Z_{\log(\expmeanhat)} \approxdist
\NOR(0,1)$, and the normal approximation
$Z_{\expmeanhat} \approxdist \NOR(0,1)$.  Statistical theory suggests
that, in large samples, the log likelihood will be approximately
quadratic with the approximation improving as the sample size
increases and that all of the different procedures for computing
confidence intervals will give similar answers.  This is consistent
with the results in Table~\ref{table:berkson.results}.  For the sample
with $n=20$, however, there are some rather large differences among
the procedures.

Simulation studies have shown that the computationally demanding
likelihood procedure can be expected to provide better intervals (i.e.,
an actual coverage probability closer to the nominal confidence
level).  Also, between the other two simple-to-compute procedures, the
normal-approximation procedure based on $\log(\expmeanhat)$ provides a
better approximation. 

It is possible to improve slightly the normal-approximation
procedure in (\ref{equation:z.theta.exp}) by using the $p$ quantile
of the Student's $t$ distribution, $\Tquant_{(p;\nu)}$, in place of
the standard normal quantile, $z_{(p)}$ in
(\ref{equation:exponential.interval.nolog}) or
(\ref{equation:exponential.interval.log}). For complete data, $n-1$
is an obvious choice for the degrees of freedom $\nu$. This is also
a reasonable choice for censored data (a correction for censoring
might be contemplated, but no generally useful rule is known). The
improvement afforded by using $\Tquant_{(p;\nu)}$ instead of
$z_{(p)}$ is negligible in samples larger than 30 or so because
$\Tquant_{(p;\nu)}$ approaches $z_{(p)}$ for large $\nu$. In samples
with fewer than 30 or even 50 observations, any normal-approximation
procedure can be rather crude (especially for the single-sided
coverage probabilities, which are important in the common situation
where the cost of being outside the interval differs from one
side to the other).  Usually, however, the normal-approximation
procedures are quick, useful, and adequate for exploratory
work. When more accurate confidence interval approximations are
required (e.g., for reporting final results), one should use
likelihood procedures or other procedures based on simulation (to be
described in Chapter~\ref{chapter:bootstrap}).


%----------------------------------------------------------------------
%----------------------------------------------------------------------
\section{Likelihood for Exact Failure Times}
\label{section:exact.fail.like}
%----------------------------------------------------------------------
\subsection{Correct likelihood for observations 
reported as exact failures}

Consider the diesel generator fan data in Appendix
Table~\ref{atable:fan.data}.  Although time is a continuous variable
and the failure times were
initially reported as exact times, these data (as with most data)
are actually discrete.  In this case, the reported
failure times were rounded to the nearest 10 hours.  Thus the
``correct likelihood'' is one for interval-censored data
(\ref{equation:parametric.interval.likelihood}).  For example, with
the exponential distribution, the likelihood contribution
(probability) of the failure recorded at 450 hours is
\begin{displaymath}
\like_{1}(\theta)=F(455;\expmean)-F(445;\expmean).
\end{displaymath}

\subsection{Using the density approximation for 
observations reported as exact values}

The traditional and commonly used form of the likelihood for an
observation, say the $i$th,
reported as an ``exact'' failure
at time $\realrv_{i}$, is
\begin{equation}
\label{equation:density.approximation}
\like_{i}(\theta)=f(\realrv_{i};\theta)
\end{equation}
where $f(\realrv_{i};\theta)= dF(\realrv_{i};\theta)/d\realrv$ is the
assumed pdf for the random variable
$\rv$.  The density approximation in
(\ref{equation:density.approximation}) is convenient, easy-to-use,
and, in some simple special cases, yields closed
form equations for ML estimates.

The use of the density approximation
(\ref{equation:density.approximation})
instead of the correct discrete likelihood can be
justified as follows. For most statistical models, the contribution to
the likelihood (i.e., probability of the data) of observations
reported as exact values can, for small $\Delta_{i}>0$, be
approximated by
\begin{equation}
\label{equation:why.density.approximation}
        [F(\realrv_{i};\theta)-
        F(\realrv_{i}-\Delta_{i};\theta)]
        \approx f(\realrv_{i};\theta)\Delta_{i}
\end{equation}
where $\Delta_{i}$ does not depend on $\theta$.  Because the
right-hand sides of (\ref{equation:why.density.approximation}) and
(\ref{equation:density.approximation}) differ by a factor of
$\Delta_{i}$, when the density approximation is used, the
approximate likelihood in (\ref{equation:density.approximation})
differs from the probability in
(\ref{equation:why.density.approximation}) by a constant scale
factor.  As long as the approximation in
(\ref{equation:why.density.approximation}) is adequate and because
$\Delta_{i}$ does not depend on $\theta$, however, the general
character (i.e., the shape and the location of the maximum) of the
likelihood is not affected.

%----------------------------------------------------------------------
\subsection{ML estimates for the exponential $\theta$
based on the density approximation} 
\label{section:exponential.density.approx.mle}
The density approximation to the likelihood generally provides an
adequate approximation for the exponential distribution.  For a
sample consisting of only right censored observations and
observations reported as exact failure times (and no left-censored
or interval-censored observations), it is easy to show that the ML
estimate of $\expmean$ is computed as
\begin{equation}
\label{equation:exp.exact.ml}
\expmeanhat=\frac{\ttt}{r}
\end{equation}
where $\ttt=\sum_{i=1}^{n}\realrv_{i}$ is known as the ``total time on test''
and where $\realrv_{i}$, $i=1,n$ are the reported failure times for units 
that failed and
the running (or censoring) time for the right-censored observations.
Note that the sum runs over {\em all} of the failures {\em and}
each of the censoring times. In this case, an estimate of standard error 
of $\expmeanhat$ is a special case of (\ref{equation:scalar.hessian}),
and is computed as
\begin{equation}
\label{equation:sehat.exp.mean}
 \sehat_{\expmeanhat} = 
\sqrt{ \left[-\frac{d^{2} \loglike(\expmean)}
{d \expmean^{2}} \right]^{-1}} = \sqrt{\frac{\expmeanhat^{2}}{r}}
= \frac{\thetahat}{\sqrt{r}} 
\end{equation}
where $r$ is the number of failures.

%----------------------------------------------------------------------
\begin{example}
{\bf Comparison of ML estimates for the fan data
based on the correct likelihood and
the density approximation.} Fitting the exponential distribution to
the diesel generator fan data in Appendix Table~\ref{atable:fan.data}
using the ``correct'' likelihood contribution in
(\ref{equation:parametric.interval.likelihood}) and the density
approximation in (\ref{equation:density.approximation}) give, to 7
decimal places, the same answers [$\log(\expmeanhat)=10.26476$ and
$\sehat_{\log(\expmeanhat)}=.2886757$]. Although the agreement is
nearly exact in this case, with other distributions, the degree of
agreement will not be so good unless the $\Delta_{i}$ are small.
\end{example}


%----------------------------------------------------------------------
\subsection{Confidence intervals for the exponential
with complete data or failure censoring} Suppose that a life test
starts at time 0 and that all failures are reported as exact
failures. If the test continues until all units have failed or if the
test is terminated after a prespecified number $r$ failures
(Type~II censoring), then the ML estimate of $\expmean$ is
$\expmeanhat=\ttt/r$ and an exact $100(1-\alpha)\%$ confidence
interval for $\expmean$ can be computed from
\begin{displaymath}
 [ \undertilde{\expmean}, \quad \tilde{\expmean}] =
\left[ \frac{2(\ttt)}{\chisquare_{(1-\alpha/2;2r)}}, \quad
\frac{2(\ttt)}
{\chisquare_{(\alpha/2;2r)}} \right].
\end{displaymath}
This interval is based on the fact that $2(\ttt/ \expmean) \sim
\chisquare_{(2r)}$. This confidence interval
procedure has exactly the specified confidence level
$100(1-\alpha)\%$.  Lawless~(1982, page 127) and Bain and
Engelhardt~(1991, page 122) provide technical details and
justification for this method. With time (Type~I) censoring, the
procedure still provides a useful approximation.
%----------------------------------------------------------------------
\begin{example}
\label{example:insul.type2}
{\bf Confidence interval for the mean life of a new insulating
material.} A life test for a new insulating material used 25
specimens.  The specimens were tested simultaneously at 30 kV
(considerably higher than the rated voltage of 20 kV). The test was
run until 15 of the specimens failed (failure or Type~II censoring).
The failure times were recorded as 1.08, 12.20, 17.80, 19.10, 26.00,
27.90, 28.20, 32.20, 35.90, 43.50, 44.00, 45.20, 45.70, 46.30, and
47.80 hours.  The total time on test for these data is $1.08 + 12.20 +
\cdots + 47.80 + 10\times 47.80 = 950.88$ hours and thus the ML
estimate of $\expmean$ is 950.88/15=63.392 hours. A 95\%
confidence interval for $\expmean$ is
\begin{displaymath}
 [ \undertilde{\expmean}, \quad \tilde{\expmean}] =
\left [ \frac{2(950.88)}{\chisquare_{(.975;30)}}, \quad
\frac{2(950.88)}{\chisquare_{(.025;30)}} \right ] =
\left [ \frac{1901.76}{46.98}, \quad
\frac{ 1901.76 }{ 16.79 } \right ] = [ 40.48, \quad 113.26 ].
\end{displaymath}
An estimate of the standard error of $\expmeanhat$ 
using (\ref{equation:sehat.exp.mean}) is $\sehat_{\thetahat}=
\sqrt{(63.392)^{2}/15}=16.37$.
%splus  signif(sort(-74*log(1-runif(25))),digits=3)
%splus
%splus (sum(c(1.08, 12.20, 17.80, 19.10, 26.00, 27.90,28.20, 32.20, 35.90, 43.50, 44.00, 45.20, 45.70, 46.30, 47.80))+10* 47.80)
%splus    [1] 950.88
%splus
%splus   (2*950.88)/qchisq(.975,30)
%splus    [1] 40.48086
%splus
%splus  (2*950.88)/qchisq(.025,30)
%splus   [1] 113.2622
%splus
%splus    950.88/15
%splus   [1] 63.392
%splus   sqrt(63.392^2/15)
%splus  [1] 16.36774
%splus
\end{example}


%----------------------------------------------------------------------
\section{Data Analysis with No Failures}
\label{section:expon.zero.fail}
For data on high reliability components, it is possible that
there will be no failures. The ML estimate of the exponential
distribution mean is then $\thetahat=\infty$ (and
sometimes it is said that the ML estimate ``does not exist'').  This
is not a useful answer because, generally, it is known that there would
have been failures if the period of testing had been extended. With
zero failures and an assumed exponential distribution it is, however,
possible to obtain a {\em lower} confidence bound for $\expmean$.  In
particular, if there are no failures in a life test with 
total time on test $\ttt$ (defined in
Section~\ref{section:exponential.density.approx.mle}) a conservative
$100(1-\alpha)$\% lower confidence bound on $\expmean$ is
\begin{equation}
\label{equation:zero.failures.exp.ci}
\undertilde{\expmean} = \frac{2 (\ttt)}{ \chisquare_{(1-\alpha;2)}}
	 = \frac{\ttt}{ -\log(\alpha) }
\end{equation}
because $ \chisquare_{(1-\alpha;2)} = - 2 \log(\alpha) $.
This bound is based on the fact that under the exponential
failure-time distribution, with immediate replacement of failed
units, the number of failures observed in a life test with a fixed
total time on test has a Poisson distribution.

As in Section~\ref{section:one.par.ci.functions}, this confidence
bound can be translated into an lower confidence bound for functions
of $\expmean$ like $\rvquan_{p}$ for specified $p$ or a upper
confidence bound for $F(\estimtime)$ for a specified $\estimtime$.
Unless $\ttt$ is large, however, the resulting bound may not be very
informative.  See Nelson~(1985) for justification and further
discussion of this method.

\begin{example}
{\bf Analysis of the diesel generator fan data assuming removal after
200 hours of service.} For this example, suppose that each of the
diesel generator fans described in Example~\ref{example:fan.data} had
been removed unfailed after 200 hours of service. There was a total of
70 fans in the study. None failed before 200 hours of service.  Thus
$\ttt$=14,000 hours. A conservative 95\% lower confidence bound on
$\expmean$ is
\begin{displaymath}
%splus 28000/5.991 = 4673.677
\undertilde{\expmean} = 
\frac{2 (\ttt)}{ \chisquare_{(.95;2)}} =
\frac{28000}{ 5.991} =  4674.
\end{displaymath}
Using the entire data set, the point estimate of $\expmean$ was 28,701
with a likelihood-based approximate 95\% lower confidence bound of
$\undertilde{\expmean}=$ 18,485 hours.  This shows how little
information comes from a short test with zero or few failures.  

A conservative $95\%$ upper confidence bound on $F(10000;\theta)$,
the probability of failing before 10,000 hours, is
$\Fupper(10000)=F(10000;
\undertilde{\theta})=1-\exp(-10000/4674) = .882$. Again, due to the
%splus  1-exp(-10000/4674)
limited amount of information from the short test, this upper bound is
not very useful for making a statement on the fan's 10,000-hour
reliability (of course, the only lower bound that can be computed for
$F(10000;\theta)$ is $0$ which also is not very useful).
\end{example}

%----------------------------------------------------------------------
%----------------------------------------------------------------------
\section*{Bibliographic Notes}

The ML method dates back to early work by Fisher~(1925) and has been
used as a method for constructing estimators ever since.
Statistical theory for ML is briefly covered in most textbooks on
mathematical statistics. Some particularly useful references for the
approach used here are Kulldorff~(1961), Kempthorne and
Folks~(1971), Rao~(1973), and Cox and Hinkley~(1974),
Lehmann~(1983), Nelson~(1982), Lawless~(1982), Casella and
Berger~(1990), and Lindsey~(1996).  Epstein and Sobel~(1953)
described the fundamental ideas of using the exponential
distribution as a model for life data.

Anscombe~(1964) and Sprott~(1973) suggest using the transformed
parameter $\nu=\expmean^{-1/3}$ for the exponential parameter
because it will make the third derivative of exponential log
likelihood function nearly 0. The important implication is that this
transformation makes the exponential log likelihood function nearly
symmetric and approximately quadratic in $\nu$. In this scale, the
normal-approximation inferences provide an excellent approximation to
the likelihood-based inferences.  Meeker and Escobar~(1995) show that
the normal-approximation intervals can be viewed as approximations to
likelihood intervals obtained by using a quadratic approximation to
the log likelihood function.

Meeker~(1986) uses asymptotic variances to show that binning data will
not seriously affect the precision of ML estimates as long as there
are a reasonable number of bins (say more than 5 or 6) and that the
bins are chosen so that the number of observations in the different
bins is not too uneven.

For examples of failures of the ``density approximation'' to the
likelihood, see Le Cam~(1990), Friedman and Gertsbakh~(1980) and
Meeker and Escobar~(1994). Each of these references describes a
model or models for which the ``density approximation'' likelihood
has a path or paths in the parameter space along which
$\like(\theta)$ approaches $\infty$. In these cases, using the
``correct'' likelihood based on discrete data eliminates the
singularity in $\like(\theta)$.

\section*{Exercises}

\begin{exercise}
\label{exercise:exponential.ml.complete}
Consider the case of $n$ observations reported as ``exact''
failure times from an $\EXP(\expmean)$ distribution.  	 
\begin{enumerate}
\item 
Show that the ML estimate, $\expmeanhat$, of $\expmean$ is
the sample mean, say $\bar{\realrv}=\sum_{i=1}^{n}t_{i}/n$.  	 
\item 
\label{exercise.part:complete.data.exponential.profile}
Show that the relative likelihood has the simple expression
\begin{displaymath}
R(\expmean)=\exp(n) \left (\frac{\bar{\realrv}}{\expmean}\, \right )^{n}
\exp \left ( - \frac{n\bar{\realrv}}{\expmean} \,\right ).
\end{displaymath} 
\item
\label{exercise.part:chisquare.exponential.lr.interval}
Explain how to use $R(\expmean)$ to obtain an approximate confidence
interval
for $\expmean$ based on inverting a likelihood ratio test
[i.e., assume that when evaluated at the
true value of $\expmean$, $R(\expmean) \approxdist \chisquare_{1}$].
\item 
\label{exercise.part:chisquare.exponential.lr.interval.ex}
Suppose that $n=4$ and $\bar{\realrv}=0.87$. Obtain 
the ML estimate and an
approximate $90\%$ confidence interval for $\expmean$ using the
method in part~\ref{exercise.part:chisquare.exponential.lr.interval}.
Plot $R(\expmean)$ to facilitate your work.
\end{enumerate}
\end{exercise}





\begin{exercise}
\label{exercise:exponential.ml.type1.censored}
Let $\realrv_{1}, \ldots, \realrv_{r}$ be the failure times in a
singly time-censored sample of size $n$ from an $\EXP(\expmean)$
distribution. Suppose
that the prespecified censoring time is $\censortime$.
The total time on test is $\ttt=\sum_{i=1}^{r}\realrv_{i}+(n-r)\censortime$.
\begin{enumerate}
\item 
\label{exercise.part:exponential.ml.density}
Write an expression for the log likelihood using the density
approximation for the observations
reported as exact failures.
\item 
Show that the ML estimate is $\expmeanhat=\ttt/r$.  		
\item 
\label{exercise.part:exponential.ml.exist.limit.allc}
Use the result of part~\ref{exercise.part:exponential.ml.density} to show
that the ML estimate of $\expmean$ is equal to $\infty$ (or ``does not
exist'') when all the observations are censored.
\item 
Derive an expression for the relative likelihood similar to the one
given in part~\ref{exercise.part:complete.data.exponential.profile} of
Problem~\ref{exercise:exponential.ml.complete} (note that in this case
$\ttt/r$ plays the role of $\bar{\realrv}$ and $r$ the role of $n$).
\end{enumerate}
\end{exercise}

\begin{exercise}
Refer to Exercise~\ref{exercise:exponential.ml.type1.censored}. Show that 
if there are no failures by time
$t_{c}$, then $L(\theta)$ always increases as a function of 
$\theta$ and thus $L(\theta)$ does not have a maximum.
\end{exercise}

\begin{exercise}
\label{exercise:elec.sys}
A large electronic system contains thousands of copies of a particular
component, which we will refer to as Component-A (each system is
custom-made to order and the actual number of components varies with
the size of the particular system). The failure rate for Component-A
is small, but because there are so many of the components in
operation, failures are reported from time to time. A field tracking
study was conducted to estimate the failure rate of a new design for
Component-A (which was intended to provide better reliability at lower
cost), to see if there was any real improvement. A number of systems
were put into service simultaneously and monitored for 1000 hours. The
total number of copies of Component-A in all of the systems was 9432.
Failures of Component-A were observed at 345, 556, 712, and 976 hours.
Failure mode analysis suggested that the failures were due to random
shocks rather than any kind of quality defects or wearout. This,
along with past experience with simpler components, suggested that
an exponential distribution might be an appropriate
model for the lifetime of Component-A.
\begin{enumerate}
\item
Compute the ML estimate for the exponential mean $\theta$
for Component-A.
\item
\label{ex.pt:ci.theta}
Compute an approximate 95\% confidence interval for $\theta$.
\item
Explain the interpretation of the confidence interval
obtained in part~\ref{ex.pt:ci.theta}.
\item
Explain the interpretation of the hazard function 
$\lambda=1/\theta$. In what way can this quantity be used to compute
failure rates for Component-A or for the overall system?
\item
Use the results of part~\ref{ex.pt:ci.theta} to obtain an approximate
95\% confidence interval for $\lambda$. Express this in units of FITs
(failures per $10^{9}$ hours of operation).
\item
The hazard for the old part was $\lambda_{\mbox{old}}=8.5 \times
10^{-7}$
(or 850 FITs). How strong is the evidence that reliability has improved
with the new design? Describe how you would phrase a statement about
the relative improvement.
\item
Compute and plot the exponential $\Fhat(t)$ from 0 to 10,000
hours on Weibull paper. Comment on the usefulness of this plot.
\end{enumerate}
\end{exercise}

\begin{exercise}
The manufacturer of computer hard disks reports in its promotional
literature
an ``MTBF'' figure. This figure is obtained by taking a sample of units
from each day's production, putting the units on test for a period
$\censortime$ (perhaps one week), pooling the available data over
several months, and computing
\begin{displaymath}
\mtbfhat = \frac{\ttt}{r}
\end{displaymath}
where $\ttt$ is the total time on test of all units that were tested
and $r$ is the observed number of failures (typically a very small
number of drives).  Reported figures are typically numbers like MTBF=30
years.  Comment on the usefulness and validity of the use of this
figure for characterizing the reliability of disk drives.
\end{exercise}


\begin{exercise}
Nelson~(1982, page 105) provides data on minutes to breakdown for an
insulating fluid. There were 11 tests at 30 kV.  After 100
minutes, there were $7$ breakdowns (failures) at the following times
(in minutes): 7.74, 17.05, 20.46, 21.02, 22.66, 43.40, 47.30. The
other $4$ units had not failed.
\begin{enumerate}
\item
Make a Weibull probability plot and an exponential probability plot for these data.
\item 
What do you think about the suggestion of using 
an exponential distribution, $\EXP(\theta)$, to model
the data? 
\item
Assuming an exponential distribution, obtain the ML estimate of 
$\theta$ and give an estimate of $\se_{\thetahat}$.
\item
Compute an approximate $95\%$ confidence interval for $\theta$.
\item
Compute the ML estimate of $\rvquan_{.1}$, the $.1$ quantile of the
time
to breakdown distribution.
\item
Compute an approximate $95\%$ confidence interval for $\rvquan_{.1}$.
\end{enumerate}
\end{exercise}


\begin{exercise}
\label{exercise:insul.material.40kv}
 A life test for a new insulating material used 50 specimens.  The
specimens were tested simultaneously at 40 kV (considerably higher
than the rated voltage of 20 kV). The test was run until 10 of the
specimens failed (this is known as ``failure'' 
or Type~II censoring). The failure
times were recorded as 8, 11, 12, 13, 19, 21, 28,
34, 36, and 44 hours. The engineers responsible for the reliability
believe, based on previous experience with similar materials, tested
under similar conditions, that the failure-time distribution at 40
kV can be described by an $\EXP(\expmean)$ distribution.
\begin{enumerate}
\item	
Construct an exponential probability plot of the data.
Does the plot provide any evidence that the exponential distribution
is inadequate?
\item	
Compute $\ttt$, the total time on test, 
and $\expmeanhat$, the ML estimate for $\expmean$.
\item	
Compute an estimate of the standard error of $\expmeanhat$.
\item	
Compute 95\% confidence intervals for $\expmean$ based on,
$Z_{\expmeanhat} \approxdist \NOR(0,1)$,
$Z_{\log(\expmeanhat)} \approxdist \NOR(0,1)$, and
the exact distribution of $2(\ttt/\expmean)$.
Which of these intervals would you feel comfortable using? Why?
\item	
For this problem, is there any extrapolation involved in 
estimating $\expmean$? Explain.
\item	
Compute 95\% confidence intervals for $\rvquan_{.1}$,
$h(50;\expmean)$, and $F(50;\expmean)$
based on the exact distribution of $2(\ttt/\expmean)$. Is there any 
extrapolation involved in these intervals? Explain.
\end{enumerate}
\end{exercise}

\begin{exercise}
Use the results from Exercise~\ref{exercise:insul.material.40kv} to
compute and plot (use an exponential plot) the ML estimate of
$F(\realrv;\expmean)$ and 95\% simultaneous parametric
confidence bands for $F(\realrv;\expmean)$, based on the exact
distribution of $2(\ttt/\expmean)$.
\end{exercise}

\begin{exercise}
A life test was conducted for the same insulating material described
in Exercise~\ref{exercise:insul.material.40kv}.  Again, 50 specimens
were tested, but at 25 kV. The test ran for 20 hours without any
failures. The test had to be terminated at this time so that the
test equipment could be used for other experiments.
\begin{enumerate}
\item	
Compute the $\ttt$ and show why the ML estimate for $\expmean$
is equal to $\infty$.
\item
\label{exercise.part:insulation.ci.for.expmean}
Compute a conservative 95\% lower confidence bound for $\expmean$.
\item	
For this problem, is there any extrapolation involved in 
computing the lower confidence bound for $\expmean$? Explain.
\item	
Use the result in part~\ref{exercise.part:insulation.ci.for.expmean}
to compute a conservative 95\% lower confidence bound for
$\rvquan_{.1}$, and conservative 95\% upper confidence bounds
for $h(50;\expmean)$ and $F(50;\expmean)$.
\end{enumerate}
\end{exercise}

\begin{exercise1}
Refer to Exercise~\ref{exercise:exponential.ml.complete} to show that
in this case $2 n \expmeanhat /\expmean$ has a $\chisquare$
distribution with $2n$ degrees of freedom and use this to obtain an
expression for a $100(1-\alpha)\%$ confidence interval for $\expmean$.
\end{exercise1}

\begin{exercise1}
Refer to Exercise~\ref{exercise:exponential.ml.type1.censored}.
Derive an expression for the bias of $\expmeanhat$ [i.e.,
$\E(\expmeanhat) -\expmean$], conditional on the existence of the
estimator (i.e., conditional on $r>0$) and show that the bias is
negligible when $\censortime$ is large.
\end{exercise1}

\begin{exercise1}
Refer to Exercise~\ref{exercise:exponential.ml.type1.censored}.
Derive (\ref{equation:sehat.exp.mean}).
\end{exercise1}

\begin{exercise1}
\label{exercise:exponential.ml.exist}
Consider $n$ observations with inspection data from an
$\EXP(\expmean)$ distribution with $0<\expmean<\infty$.  Show that
\begin{enumerate}
\item	
\label{exercise.part:exponential.ml.exist.allf}
The ML estimate of $\expmean$ does not 	exist (i.e., there is not a
unique maximum of the likelihood function) when
$\deadin_{1}=n$ or $\deadin_{m+1}=n$.
\item 
\label{exercise.part:exponential.ml.exist.allc}
When $\deadin_{i}=n$ ($i \ne 1, m+1$), the ML estimator is
\begin{displaymath} 
\expmeanhat=\frac{\realrv_{i}-\realrv_{i-1}}
{\log(\realrv_{i})-\log(\realrv_{i-1})}.
\end{displaymath} 	
\end{enumerate}
\end{exercise1}


