%%chapter 15
%\batchmode
%original by wqmeeker   7 aug 94
%edited by wqmeeker  8 aug 94
%edited by wqmeeker  15 sept 94 morgan data table
%edited by wqmeeker  17 june 95 breaking out system ideas
%edited by wqmeeker  15 June 96 bringing in slides
%edited by driker 28 June 96
%edited by wqmeeker 28-29 June 96 tuning
%edited by wqmeeker 1-8 July 96 adding examples and exercises
%edited by driker 9 july 96
%edited by wqmeeker 9 July 96 minor changes
%edited by wqmeeker 28 nov 96 new exercise
%edited by driker 21 july 97

\setcounter{chapter}{14}

\chapter{System Reliability Concepts and Methods}
\label{chapter:system.reliability}


%{\Large {\bf William Q. Meeker and Luis A. Escobar}}\\
%Iowa State University and Louisiana State University\\[5ex]
{\large {\bf \today}}\\[2ex]
Part of 
{\em Statistical Methods for Reliability Data}\\
Copyright 1997 W. Q. Meeker and L. A. Escobar. \\[1ex]
To be published by John Wiley \& Sons Inc. in 1998.



%----------------------------------------------------------------------
%----------------------------------------------------------------------

\section*{Objectives}
This chapter explains
\begin{itemize} 
\item
Important system reliability concepts like system structure;
redundancy; nonrepairable and repairable systems; maintainability and
availability.
\item
Basic concepts of system reliability modeling.
\item 
The distribution of system failure time as a function of individual
component failure-time distributions.
\item 
Simple methods for using
component test data to estimate system reliability.
\item
Analysis of data with more than one failure mode.
\end{itemize}

%----------------------------------------------------------------------
%----------------------------------------------------------------------
\section*{Overview}
This chapter describes and illustrates some basic ideas behind system
reliability analysis.  Section~\ref{section:system.structure} describes some
simple system structures and shows how to compute system reliability as a
function of component reliability. These simple structures can be used
as building blocks to compute system reliability for more complicated
systems. Section~\ref{section:est.sys.rel} shows how to estimate and
compute confidence intervals for system reliability from limited
component data. Section~\ref{section.multiple.causes} explains
methods of analyzing failure time data with more than one cause of
failure and how to use such data to estimate the distribution
corresponding to the individual failure modes or the overall system.
Section~\ref{section:other.sys.rel.topics} provides a brief overview
of other topics and references related to system reliability.

%----------------------------------------------------------------------
%----------------------------------------------------------------------
%----------------------------------------------------------------------
%----------------------------------------------------------------------
\section{Introduction}

A system is a collection of components interconnected to
perform a given task.  Component state (e.g., working or not working)
and system structure determine whether a system is working or not.
System structure is described by a logic diagram illustrating the
relationship between components and satisfactory system performance.
Ultimately, interest centers on the reliability of specific systems.
Assessing and improving system reliability generally requires
consideration of system structure and component reliability. Some
systems are replaced upon failure.  Many systems, however, are
maintained (e.g., replacing worn components before they fail) and/or
repaired after failure.  For repairable systems, availability (the
fraction of time that a system is available for use) may be the
appropriate metric.  This leads to consideration of maintainability
(e.g., improvement of reliability through inspection and/or preventive
maintenance) and repairability (characterized by the distribution of time
to do a repair).  In general, availability can be increased by
increasing reliability or by improving maintainability and
repairability.

%----------------------------------------------------------------------
%----------------------------------------------------------------------
\section{System Structures and System Failure Probability}
\label{section:system.structure}
System failure probability, $F_{\rv}(\realrv; \thetavec)$, is 
the probability that the system fails before $\realrv$.
The failure probability of the system is a function of time in
operation $t$ (or other measure of use), the system structure,
reliability of system components, interconnections, and interfaces
(including, for example, human operators).
%\item
%Repair time of components in the case of repairable systems.

This section describes several simple system structures. Not all
systems fall into one of these categories, but the examples provide
a collection of building blocks to illustrate the basics of system
structure. Complicated system structures can generally be decomposed
into collections of the simpler structures presented here. The
methods for evaluation of system reliability can be adapted to more
complicated structures.

%----------------------------------------------------------------------
\subsection{Time dependency of system reliability}
\label{F.as.fun.of.time}
For a new system (i.e., all components starting
a time 0) with $\ncomponents$ independent components, the cdf for
component $i$ is denoted by $F_{i}=F_{i}(t;\thetavec_{i})$. The
corresponding survival probability (reliability) for component $i$ is
$S_{i}=S_{i}(t;\thetavec_{i})= 1-F_{i}(t;\thetavec_{i})$.  The
$\thetavec_{i}$s may have some elements in common. We let $\thetavec$
denote the unique elements in
$(\thetavec_{1},\ldots,\thetavec_{\ncomponents})$.  The cdf for the
system is denoted by $F_{\rv}=F_{\rv}(\realrv ; \thetavec)$.  This cdf
is determined by the $F_{i}$'s and the system structure.  Then
$F_{\rv}(\realrv; \thetavec)=g[F_{1}(\realrv; \thetavec_{1}),
\ldots,F_{\ncomponents}(\realrv; \thetavec_{\ncomponents})]$.
To simplify the presentation, time (and parameter) dependency will
usually be suppressed in this chapter.  Then this function can also be
expressed in one of the simpler forms $F_{\rv}( \thetavec)=g[F_{1}(
\thetavec_{1}),
\ldots,F_{\ncomponents}(\thetavec_{\ncomponents})]$ or
$F_{\rv}=g(F_{1}, \ldots,F_{\ncomponents})$.  

%Probability that component $i$ will fail before time $t$ is
%$F_{i}=F_{i}(t)$. The corresponding survival probability is
%$S_{i}=S_{i}(t)= 1-F_{i}(t)$. To simplify the formulas, 
%time-dependency will usually be suppressed in this chapter.

%----------------------------------------------------------------------
\subsection{Systems with components in series}
\label{section:system.series}
%----------------------------------------------------------------------
A series system structure with $\ncomponents$ components works if
and only if all the components work. 
%-------------------------------------------------------------------
\begin{figure}
\xfigbookfiguresize{\figurehome/system.seriesfig.ps}{1.5in}
\caption{A system with two components in
series.}
\label{figure:system.seriesfig.ps}
\end{figure}
%-------------------------------------------------------------------
Examples of systems with components in series include chains,
high-voltage multicell batteries, inexpensive computer systems,
inexpensive decorative tree lights using low-voltage bulbs, etc.  For
a system with 
two independent components in series, illustrated in
Figure~\ref{figure:system.seriesfig.ps}, the cdf is
\begin{eqnarray}
\label{equation:two.comp.series}
F_{\rv}(\realrv)&=&\Pr( \rv \le \realrv)
=1-\Pr( \rv > \realrv)=1-\Pr(\rv_{1} > \realrv \cap \rv_{2} > \realrv)
\\ \nonumber
&=&1-\Pr(\rv_{1} > \realrv) \Pr(\rv_{2} > \realrv)
=1-(1-F_{1})(1-F_{2})=  F_{1} + F_{2} -  F_{1}F_{2} .
\end{eqnarray}
For $\ncomponents$ independent components $F_{\rv}(\realrv)=
1-\prod_{i=1}^{\ncomponents} \left (1-F_{i} \right )$ and for
$\ncomponents$ ${\rm iid}$ components ($F=F_{i}, \, i=1, \ldots,
\ncomponents$), $F_{\rv}(\realrv)= 1-(1-F)^{\ncomponents}$.
This is the same as the minimum-type distributions discussed in
Section~\ref{section:power.dist}.  For a series system of independent
components, the system hazard function is the sum of the
component hazard functions
\begin{equation}
\label{equation:sum.hazard}
h_{T}(t) = \sum_{i=1}^{\ncomponents} h_{i}(t).
\end{equation}

Figure~\ref{figure:series.effect.ps} shows the relationship between
system reliability $1-F_{T}(t)$ and individual component reliability
$1-F(t)$ for different
numbers of identical independent components in series.  This figure
shows that extremely high component reliability is needed to maintain
high system reliability, particularly if the system has many
components in series.
%-------------------------------------------------------------------
\begin{figure}
\splusbookfigure{\figurehome/series.effect.ps} 
\caption{Reliability of a system with $\ncomponents$ identical
independent components in series.}
\label{figure:series.effect.ps}
\end{figure}
%-------------------------------------------------------------------

{\bf Importance of part count in product design.} An important rule
of thumb in reliability engineering design practice is ``keep the
part count small,'' meaning keep the number of individual parts (or
components) in a system to a minimum. Besides the cost of purchase
and handling of additional individual parts, there is also an
important reliability motivation for having a smaller number of
parts at risk of failure in a product. Of course, this rule of thumb
holds when the reliability of the individual parts in the design
with a smaller number of parts is the same or similar to the
reliability of the parts in the design with a larger number of
parts.

\begin{example}
{\bf Effect of part-count reduction on modem reliability.}
The design for a new computer modem uses a higher level of
microelectronic integration and requires only 20 discrete parts
instead of the 40 parts required in the previous generation.  
For a series system of parts with independent failure times, the hazard
function of the system can be obtained by summing the hazards for
the individual parts.  This is particularly simple if 
a constant hazard rate (or equivalently, an exponential time to
failure distribution) provides an adequate model for part life.  As a
rough approximation, suppose that all failures are due to part
failures, and that all of the parts have the same hazard function.
Then the population of modems produced with the new design with only 20
parts will experience only half of the failures when compared to the
old design. Allowing that failures can occur at interfaces and
interconnections between parts with the same frequency in the new and
old designs would widen the reliability gap because of the larger
number of such interfaces with a higher number of parts.  With a
nonconstant hazard function (more common in practice) the idea is
similar.
\end{example}

\noindent
{\bf Series system of independent components having Weibull distributions with the
same shape parameter.}
Recall from Section~\ref{section:weibull.distribution} that the
Weibull hazard function can be written as
\begin{displaymath}
 h(\realrv) = \frac{\beta}{\weibscale}
\left (\frac{\realrv}{\weibscale} \right )^{\beta-1},  \quad \realrv > 0.
\end{displaymath}
For a series system of $\ncomponents$ independent components having a Weibull
distribution with the same shape parameter
$\beta$
but possibly differing $\weibscale$ values, the system 
failure time distribution is also Weibull. The system hazard function is
\begin{displaymath}
 h_{T}(\realrv) = \sum_{i=1}^{\ncomponents}\frac{\beta}{\weibscale_{i}}
\left (\frac{\realrv}{\weibscale_{i}} \right )^{\beta-1}=\frac{\beta}{\weibscale_{T}}
\left (\frac{\realrv}{\weibscale_{T}} \right )^{\beta-1} ,  \quad \realrv > 0
\end{displaymath}
where
\begin{displaymath}
\weibscale_{T} = \left( \sum_{i=1}^{\ncomponents} 
	\frac{1}{\weibscale_{i}^{\beta}}\right)^{-1/\beta}.
\end{displaymath}
If all of the $\ncomponents$ components have the same $\weibscale$,
then this simplifies to $\weibscale_{T} =
\weibscale/\ncomponents^{1/\beta}$.

\begin{example}
{\bf Reliability of a chain.}  A particular kind of chain link can
fail from growth of fatigue cracks and eventual fracture. The life
distribution of a single link has a Weibull distribution with
$\weibscale=100$ thousand use cycles and a shape parameters
$\beta=2.3$. A chain of 75 links can be viewed as a series
system. When the first link breaks, the chain fails and has to be
replaced. In the application, all links in the chain are subject to
the same level of stress. If the cycles to failure for the
individual links are independent, then the chain has a Weibull time
to failure distribution with $\weibscale_{T} =
100/(75)^{1/2.3}=15.30$ thousand cycles and a shape parameter
$\beta=2.3$.
%splus> 100/(75)^(1/2.3)
\end{example}

\noindent
{\bf Effect of positive dependency in a two-component
series system.}
If a series system contains two components with dependent failure
times, then the first line of $(\ref{equation:two.comp.series})$ still
gives $F_{\rv}(\realrv)$, but the evaluation has to be done with
respect to the bivariate distribution of $\rv_{1}$ and $\rv_{2}$. More
generally, for a system with $\ncomponents$ components in series, the system
$F_{\rv}(\realrv)$ would have to be computed with respect to the
underlying $\ncomponents$-variate distribution.  Such computations
are, in general difficult. If the correlation among the $\ncomponents$
series components is positive, then the assumption of independence is
conservative in the sense that the actual $F_{\rv}(\realrv)$ is
smaller than that predicted by the independent-component model.
%-------------------------------------------------------------------
\begin{figure}
\splusbookfigure{\figurehome/series.dep.effect.ps} 
\caption{Reliability of a system with 2 dependent
components in series.}
\label{figure:series.dep.effect.ps}
\end{figure}
%-------------------------------------------------------------------
For a simple two-component series system,
Figure~\ref{figure:series.dep.effect.ps} shows the reliability
$1-F_{T}(t)$ of a two-component series system as a function of the
reliability $1-F(t)$ of the individual components that have positive
correlation.  For this example, the distributions of log failure
times for the individual components is bivariate normal with the
same (arbitrary) mean and standard deviation for both components and
correlation $\rho$.  When $\rho=1$ (so the two components are
perfectly dependent and will fail at exactly the same time), the
curve is the same as the $s=1$ (single component) system shown in
Figure~\ref{figure:series.effect.ps}. When $\rho=0$ (so the two
components are independent), the curve would correspond to an $s=2$
curve in Figure~\ref{figure:series.effect.ps}.
Figure~\ref{figure:series.dep.effect.ps} shows that when there is
positive correlation between the failure times of the individual
components, the actual reliability of the system exceeds that
predicted by the independent-component series system. The
multivariate generalization of this result is important in
reliability modeling applications.

\begin{example}
{\bf Reliability of a jet engine turbine disk.}  The primary threat
for a jet engine turbine disk failure is the initiation and growth
of a fatigue crack. Generally it is not economically desirable to
test more than one or two jet engine turbine disks.  Additionally,
even with under realistic continuous accelerated testing, no
failures would be expected for years. Instead, disk reliability is
predicted by using a model.  The reliability model for the disk is
obtained by dividing the disk into a large number of small
``elements'' that are first modeled individually.  Accelerated tests
on materials specimens provide information to predict the life of an
element as a function of temperature and stress. The overall
reliability of the system can then be modeled as a series system of
independent components. Modeling the individual elements' failure
time distribution as a function of temperature and stress (which
depend on position within the disk) improves the adequacy of the
independence assumption. Still, however, one would expect the
initiation and growth of cracks to be positively correlated from
element to element within a disk, especially among elements that are
close together.  When the failure times of a series system's
components have positive association (nonnegative correlation
between all pairs), the independence model provides a conservative
prediction of the system's overall reliability.  Theoretical
justification for this result is given in Chapter 2 of Barlow and
Proschan~(1975).
\end{example}

Of course, if the lifetimes of components in a series system have
negative association, then the reliability predicted with the independence
model will be anticonservative.  We do not give details for this
situation because it is not common in physical systems.

%----------------------------------------------------------------------
\subsection{Systems with components in parallel}
A parallel system structure with $\ncomponents$ components works if at
least one of the components works.  Examples of systems with
components in parallel include automobile headlights, RAID computer
disk array systems, stairwells with emergency lighting, overhead
projectors with backup bulb, and multiple light banks in classrooms.
For two independent parallel components,
illustrated in Figure~\ref{figure:system.parallelfig.ps},
%-------------------------------------------------------------------
\begin{figure}
\xfigbookfiguresize{\figurehome/system.parallelfig.ps}{1.5in}
\caption{A system with two components in parallel.}
\label{figure:system.parallelfig.ps}
\end{figure}
%-------------------------------------------------------------------
\begin{eqnarray}
\label{equation:two.comp.parallel}
F_{\rv}(\realrv)&=&\Pr( \rv \le \realrv)
=\Pr(\rv_{1} \le  \realrv \cap \rv_{2} \le \realrv)
\\ \nonumber
&=&\Pr(\rv_{1} \le \realrv) \Pr(\rv_{2} \le \realrv)
=F_{1} F_{2}.
\end{eqnarray}
For $\ncomponents$ independent components
$F_{\rv}(\realrv)=\prod_{i=1}^{\ncomponents} F_{i}$ and for
$\ncomponents$ iid components $(F_{i}=F, \, i=1, \ldots,
\ncomponents)$, $F_{\rv}(\realrv)=F^{\ncomponents}$.

%-------------------------------------------------------------------
\begin{figure}
\splusbookfigure{\figurehome/parallel.effect.ps} 
\caption{Reliability of a system with $s$ iid
components in parallel.}
\label{figure:parallel.effect.ps}
\end{figure}
%-------------------------------------------------------------------
Figure~\ref{figure:parallel.effect.ps} shows the relationship between
system reliability $1-F_{T}(t)$ and individual component reliability
$1-F(t)$ for different numbers of identical independent components in
parallel.  The figure shows the dramatic effect that parallel
redundancy can have on the reliability of the system or subsystem.
If the components are not independent, then the first line of
$(\ref{equation:two.comp.parallel})$ still gives $F_{\rv}(\realrv)$,
but the evaluation has to be done with respect to the bivariate
distribution of $\rv_{1}$ and $\rv_{2}$.

\noindent
{\bf Effect of positive dependency in a two-component
parallel-redundant system.}
%-------------------------------------------------------------------
\begin{figure}
\splusbookfigure{\figurehome/parallel.dep.effect.ps} 
\caption{Reliability of a system with 2 dependent
components in parallel.}
\label{figure:parallel.dep.effect.ps}
\end{figure}
%-------------------------------------------------------------------
For a simple two-component parallel system,
Figure~\ref{figure:parallel.dep.effect.ps} shows the effect that
positive dependency between the failure times of the two components
has on system reliability $1-F_{\rv}(\realrv)$.
For this example, the distributions of log failure times for the
individual components is bivariate normal with the same (arbitrary)
mean and standard deviation for both components and correlation
$\rho$. With $\rho=0$ (so the components are independent), the curve
is the same as the $s=2$ curve shown in
Figure~\ref{figure:parallel.effect.ps}. When $\rho=1$ (so the
components are perfectly dependent and will fail at exactly the same
time), the curve is the same as the $s=1$ curve (single component) shown in
Figure~\ref{figure:parallel.effect.ps}. The advantages of redundancy
can be degraded seriously when the failure times of the individual
components have positive dependence.

%----------------------------------------------------------------------
\subsection{Systems with components in series-parallel}
\label{section:series.parallel}
Methods for evaluating the reliability of structures with components
in both series and parallel provide the basis for evaluating more
complicated structures that use redundancy to increase system
reliability.  There are two types of simple (i.e., rectangular)
series-parallel structures: series-parallel with system-level
redundancy and series-parallel with component-level redundancy.

\mbox{ }\\
\noindent
{\bf Series-parallel systems with system-level redundancy.}
%-------------------------------------------------------------------
\begin{figure}
\xfigbookfiguresize{\figurehome/system.sysredunfig.ps}{2.5in}
\caption{A 
series-parallel system structure with system-level redundancy.}
\label{figure:system.sysredunfig.ps}
\end{figure}
%-------------------------------------------------------------------
In some applications it is more cost effective to achieve higher
reliability by using two or more copies of a series system rather
than having to improve the reliability of the single system itself.
Series-parallel system structures with system-level redundancy are
used in applications like parallel central processors for a
system-critical communications switching system, space-craft or
aircraft fly-by-wire computer control systems, automobile brake
system (hydraulic and mechanical), and multiple trans-Atlantic
transmission cables.

A $k \times r$ series-parallel system-level redundancy structure has
$r$ parallel sets each of $k$ components in series.  For a $2 \times
2$ structure with independent components, illustrated in
Figure~\ref{figure:system.sysredunfig.ps},
\begin{eqnarray}
\label{equation:system.sysredunfig}
F_{\rv}(\realrv)&=&\Pr( \rv \le \realrv)
=\Pr [\mbox{``series 1 failed''} \cap 
\mbox{``series 2 failed''} ]
\\ \nonumber
&=&[1-(1-F_{11})(1- F_{12})][1-(1-F_{21})(1- F_{22})]
\\ \nonumber
&=& [F_{11} + F_{12} - F_{11} F_{12}   ] [F_{21} + F_{22} - F_{21} F_{22}  ] 
\end{eqnarray}
where $F_{ij}$, $j=1,2$, are the cdfs for the series
subsystem $i$.  For a $r
\times k$ structure with independent components
$F_{\rv}(\realrv)=\prod_{i=1}^{r} \left [ 1- \prod_{j=1}^{k} \left
(1-F_{ij} \right )\right ]$ and for a $r \times k$ parallel-series
structure with ${\rm iid}$ components $F_{\rv}(\realrv)=
\left [1- (1-F)^{k}\right ]^{r}$.
If the system components are not independent, then the first line of
$(\ref{equation:system.sysredunfig})$ still gives $F_{\rv}(\realrv)$, but
the evaluation has to be done with respect to the multivariate
distribution of $\rv_{1}, \dots, \rv_{4}$.

\mbox{ }\\
\noindent
{\bf Series-parallel system structure with component-level
redundancy.} 
%-------------------------------------------------------------------
\begin{figure}
\xfigbookfiguresize{\figurehome/system.compredunfig.ps}{2.5in}
\caption{A 
simple series-parallel system structure with component-level
redundancy.}
\label{figure:system.compredunfig.ps}
\end{figure}
%-------------------------------------------------------------------
Component redundancy is an important method for improving
system reliability. Series-parallel system structures with
component-level redundancy are found in numerous applications
including parallel dual repeaters in under-sea fiber-optic data
transmission systems, and the human body (lungs, kidneys).  A $k \times r$
component-level redundant structure has $k$ series structures each one
made of $r$ components in parallel. If it is necessary to have only one
path through the system, such a structure is, for a given number of
identical components, more reliable than the series-parallel
system-level redundancy.  For a $2 \times 2$ series-parallel system with
independent components, illustrated in
Figure~\ref{figure:system.compredunfig.ps},
\begin{eqnarray}
\label{equation:system.compredunfig}
F_{\rv}(\realrv)&=&1-\Pr( \rv > \realrv)
=1-\Pr [\mbox{``parallel 1 works''} \cap 
\mbox{``parallel 2 works''} ]
\\ \nonumber
&=&1-(1-F_{11} F_{21})(1-F_{12} F_{22})
\\ \nonumber
&=& F_{11} F_{21} + F_{12} F_{22} - F_{11} F_{21} F_{12} F_{22}
\end{eqnarray}
where $F_{ij}$, $i=1,2$ are the cdfs for parallel
subsystem $j$.  For a $k
\times r$ series-parallel system with independent components
\begin{eqnarray*}
F_{\rv}(\realrv)&=&
1- \prod_{i=1}^{k} \left (1-\prod_{j=1}^{r} F_{ij} \right).
\end{eqnarray*}
When all of the system's components are ${\rm iid}$ 
$F_{\rv}(\realrv)=1- (1-F^{r})^{k}$.
If the system components are not independent, then the first line of
$(\ref{equation:system.compredunfig})$ still gives $F_{\rv}(\realrv)$,
but the evaluation has to be done with respect to the multivariate
distribution of $\rv_{1}, \dots, \rv_{4}$.

\subsection{Bridge-system structure}

Bridge-system structures provide another useful structure for
improving the reliability of certain systems. Bridge-system structures are
common in computer and electric power-distribution networks.
Figure~\ref{figure:system.bridgefig.ps} illustrates a simple
bridge-system structure. Note that if component 3 is {\em not}
working, the bridge system has the same structure as
Figure~\ref{figure:system.sysredunfig.ps}.  If component 3 is working,
the bridge system has the same structure as
Figure~\ref{figure:system.compredunfig.ps}.  In many practical
situations a bridge such as the one at component 3 in
Figure~\ref{figure:system.bridgefig.ps} can be installed at little
extra cost, but provides a potentially important improvement in
reliability when compared to a simple series-parallel system (see
Exercise~\ref{exercise:sp.bridge.comp}).
%-------------------------------------------------------------------
\begin{figure}
\xfigbookfiguresize{\figurehome/system.bridgefig.ps}{2.5in}
\caption{A bridge-system structure.}
\label{figure:system.bridgefig.ps}
\end{figure}
%-------------------------------------------------------------------

The relationship between the bridgesystem structure and the two different
series-parallel structures provides a method to compute the
bridge-system structure reliability.  We use $A_{3}$ ($A_{3}^{c}$)
to denote the event that component $3$ is working (is not working).
Then similar to (\ref{equation:system.compredunfig}) $\Pr( \rv \le
\realrv | A_{3})= F_{1} F_{4} + F_{2} F_{5} - F_{1} F_{2}
F_{4} F_{5}$ and similar to (\ref{equation:system.sysredunfig})
$\Pr(\rv \le
\realrv | A_{3}^{c})= [F_{1} + F_{2} - F_{1} F_{2} ] [F_{4} +
F_{5} - F_{4} F_{5} ] $ and thus
\begin{eqnarray*}
F_{\rv}(\realrv)&=& 
\Pr( \rv \le \realrv \cap A_{3})
+
\Pr(\rv \le \realrv \cap A_{3}^{c})
\\[1ex]
&=&
\Pr(A_{3}) \Pr( \rv \le \realrv | A_{3}) 
+
\Pr(A_{3}^{c}) \Pr(\rv \le \realrv | A_{3}^{c}) 
\\[1ex]
&=&
(1-F_{3}) \left (
F_{1}F_{4}+ F_{2}F_{5}
-
F_{1}F_{2}F_{4}F_{5}
          \right )
+
F_{3}     \left (
F_{1}+F_{2}-F_{1}F_{2}
          \right )
          \left (
F_{4}+F_{5}- F_{4}F_{5}
          \right ).
\end{eqnarray*}

\subsection{$\boldsymbol{k}$-out-of-$\boldsymbol{\ncomponents}$ 
\label{section:k.of.n}
system structures}

Some systems work if at least $k$ out of $\ncomponents$ components
work, but not otherwise. Special cases include the 1 of $\ncomponents$
parallel structure and the $\ncomponents$ of $\ncomponents$ series
structure.  Other examples of $k$-out-of-$\ncomponents$ system structures
include a satellite battery system in which the system will continue
to operate as long as 6 of 10 batteries continue to operate correctly,
or computer storage disks which continue to provide service by
blocking out bad sectors up to a certain limit, and heat exchangers
that continue to operate even when a certain small proportion of their tubes
have been plugged (see Example~\ref{example:heat.exchanger.data}).

%-------------------------------------------------------------------
\begin{figure}
\xfigbookfiguresize{\figurehome/system.kofnfig.ps}{2.5in}
\caption{A $k$-out-of-$\ncomponents$ system structure.}
\label{figure:system.kofnfig.ps}
\end{figure}
%-------------------------------------------------------------------
Figure~\ref{figure:system.kofnfig.ps} illustrates
a system requiring that at least two out of three independent
components work.  Note that the system structure diagram does not
reflect physical layout, but rather paths through the system that will
allow operation of the system. Computationally, for a two-out-of-three
system of independent components,
\begin{eqnarray*}
F_{\rv}(\realrv)&=&\Pr( \rv \le \realrv)
\\
&=&
  \Pr(\mbox{``exactly two fail''} )
 +\Pr(\mbox{``exactly three fail''})
\\
&=&
 [F_{1} F_{2}(1-F_{3})
+F_{1} F_{3}(1-F_{2})
+F_{2} F_{3}(1-F_{1})]
+F_{1} F_{2}F_{3}
\\
&=&
F_{1} F_{2}
+F_{1} F_{3}
+F_{2} F_{3}
-2F_{1} F_{2}F_{3}.
\end{eqnarray*}

For $k$ out of $\ncomponents$ independent components
\begin{equation}
\label{equation:system.kofsfig}
F_{\rv}(\realrv)=
\sum_{j=k}^{\ncomponents}
\left \{
\sum_{\deltavec \in A_{j} } \left [
\prod_{i=1}^{\ncomponents}
F_{i}^{\delta_{i}}(1-F_{i})^{(1-\delta_{i})}\right ]\right \}
\end{equation}
where $\deltavec =
(\delta_{1}, \ldots, \delta_{\ncomponents})$ with
$\delta_{i}=1$ indicating failure of
component $i$ by time $\realrv$ and 
$\delta_{i}=0$ otherwise and $A_{j}$ is the set of all
$\deltavec$
such that $\sum_{i=1}^{\ncomponents} \delta_{i}=j$.
For identically distributed components ($F=F_{i}, i=1, \dots,s$)
$F_{\rv}(\realrv)=\sum_{j=k}^{\ncomponents} {\ncomponents \choose j}
F^{j}(1-F)^{\ncomponents-j}$, a binomial distribution.

\begin{example} {\bf Spacecraft power system.}
A spacecraft power system uses 10 rechargable batteries in parallel.
The capacity of the power system is equal to the sum of
the power provided by each of the batteries. The system can continue
to operate at design specifications as long as seven of the ten
batteries are functional.
\end{example}

%----------------------------------------------------------------------
%----------------------------------------------------------------------
\section{Estimating System Reliability from Component Data}
\label{section:est.sys.rel}
%----------------------------------------------------------------------
%----------------------------------------------------------------------
\subsection{Computing system reliability from component reliability}

To compute the system cdf, one can use $F_{\rv}=g(F_{1}, \ldots,
F_{\ncomponents})$ when $g$ is known from the system structure.
If $g$ cannot be expressed in closed form or is otherwise difficult to
compute, one can use a computer simulation of the system based on the
$F_{i}$ and the system structure.  When the $F_{i}$ are unknown, an
estimate of the system $F_{\rv}$ can be obtained by evaluating
$F_{\rv}$ at the ML estimates of the needed $F_{i}$ values.

%----------------------------------------------------------------------
\subsection{Sources of reliability data}

Laboratory tests are used widely, especially to test new materials and
components where there is little past experience.  Such testing is
generally expensive and may have limited ability to predict product
field reliability. Special care must be taken to assure that test
conditions can be accurately related to actual field conditions (as
described in Chapters~\ref{chapter:accelerated.test.models} through
\ref{chapter:accelerated.degradation}, laboratory tests are often
accelerated with the goal of getting component reliability
information more quickly).  Carefully collected field data, when
available, provides the most accurate information about how
components and systems behave in the field.  Field data collection,
however, is also expensive.  Warranty data often have serious
deficiencies. For example, warranty often contain no information on
units that do not fail (see Robinson and McDonald~(1991) for further
discussion of this and other related issues). Ireson~(1996) 
describes some general issues relating to the collection
and storage of reliability data. Reliability hand
books and data banks can be useful (e.g., Klinger, Nakada, and
Menendez~1990 and MIL-HDBK~217E).  One common complaint about such
hand books, however, is that data become obsolete by the time they
are published and that reported hazard rates and failure
probabilities may be off by an order of magnitude or more.
Technology, in many areas, is moving faster than accurate
traditional reliability data can be obtained. Expert knowledge is
often used when no other source of information is available. Unless
data are collected from carefully conducted statistical studies,
quantifying uncertainty may be impossible.

%----------------------------------------------------------------------
\subsection{Maximum likelihood estimation of system reliability}
Suppose that sample data are available to estimate the failure-time
distributions of the system's individual components.  For example,
data on component $i$ for $i=1,\dots, \ncomponents $ can be used to
estimate $\thetavec_{i}, i=1,\dots,\ncomponents$, providing
estimates $\Fhat_{1}, \ldots, \Fhat_{\ncomponents}$,
respectively. These cdf estimates are functions of time, as
described in Section~\ref{F.as.fun.of.time}. Then the system cdf (or
other related functions) can be estimated as functions of
$\Fhat_{1},
\ldots, \Fhat_{\ncomponents}$, the function being 
determined from the system's structure, as described in
Sections~\ref{section:system.series} through
\ref{section:k.of.n}.  Let $\thetavechat$ be the ML estimate of
$\thetavec$ (the unique parameters describing
the components' cdfs) and $\vcvmathat_{\thetavechat}$ the ML
estimate of $\vcvmat_{\thetavechat}$ obtained from the component
data. Then using the same methods as in previous chapters,
$\Fhat_{\rv}=F_{\rv}(\thetavechat) =g[F_{1}(\thetavechat_{1}),
\ldots, F_{\ncomponents}(\thetavechat_{\ncomponents})]$. The
variance of $\Fhat_{\rv}$ can be computed by using the delta method
(Appendix Section~\ref{asection:delta.method}) as
$\varhat(\Fhat_{\rv})= \left (\frac{\partial F_{\rv}}{\partial
\thetavec } \right ) \transpose
\vcvmathat_{\thetavechat} 	\left (\frac{\partial
F_{\rv}}{\partial \thetavec } \right )$, where the derivatives are
evaluated at $\thetavechat$.

\begin{example}
{\bf 
Maximum likelihood estimation for a simple parallel system structure.
}For a parallel structure with 
$\ncomponents$ ${\rm iid}$ components
\begin{eqnarray*}
\Fhat_{\rv}&=& [\Fhat]^{\ncomponents} = [F(\thetavechat)]^{\ncomponents}
\\[2ex]
\varhat(\Fhat_{\rv})&=&
	\left (\frac{\partial F_{\rv}}{\partial \thetavec }
 	\right ) \transpose
 	\vcvmathat_{\thetavechat}
	\left (\frac{\partial F_{\rv}}{\partial \thetavec } \right )=
 \left (\frac{\partial F_{\rv}}{\partial F} \, \frac{\partial F}{\partial \thetavec }
 	\right ) \transpose
 	\vcvmathat_{\thetavechat}
        \left (
\frac{\partial F_{\rv}}{\partial F} \, \frac{\partial F}{\partial \thetavec }	
        \right )\\[1ex]
&=&
	\left (\ncomponents \Fhat^{\ncomponents-1}\frac{\partial F}{\partial \thetavec }
 	\right ) \transpose
 	\vcvmathat_{\thetavechat}
        \left (
\ncomponents \Fhat^{\ncomponents-1}\frac{\partial F}{\partial \thetavec }	
        \right ).
\end{eqnarray*}
Then $\sehat_{\Fhat_{\rv}}= \sqrt{\varhat(\Fhat_{\rv})}$.
\end{example}

%----------------------------------------------------------------------
\subsection{Normal-approximation confidence intervals for system
reliability}
\label{section:system.rel.not.approx}
A normal approximation 100$(1-\alpha)$\% confidence interval for $F_{\rv}(t;
\thetavec)$ based on
$Z_{\logit(\Fhat_{\rv})} \approxdist \NOR(0,1)$ is
\begin{eqnarray*}
[\undertilde{F_{\rv}}, \quad \tilde{F_{\rv}}] = 
\left[\frac{\Fhat_{\rv}}{\Fhat_{\rv}
+(1-\Fhat_{\rv}) \times w },  \quad \frac{\Fhat_{\rv}}{\Fhat_{\rv}+
(1-\Fhat_{\rv})/w }
\right]
\end{eqnarray*}
where
$w=\exp\{\norquan_{(1-\alpha/2)}\sehat_{\Fhat_{\rv}}/
[\Fhat_{\rv}(1-\Fhat_{\rv})]\}$.

%----------------------------------------------------------------------
\subsection{Bootstrap approximate confidence intervals for system reliability}
Bootstrap confidence intervals can be used to improve upon the simple
normal-approximation method in
Section~\ref{section:system.rel.not.approx}. One
iteration of the bootstrap procedure requires the computation of
bootstrap estimates $\Fhatboot_{i}$, $i=1, \ldots, \ncomponents$ (as
in Chapter~\ref{chapter:bootstrap}) for each of the $\ncomponents$ components.
These lead
to the system bootstrap estimate $\Fhatboot_{\rv}=g(\Fhatboot_{1},
\ldots,
\Fhatboot_{\ncomponents})$.
The procedure is repeated $B$ times and, as in
Section~\ref{section:par.boot.ci.f}, an approximate $100(1-\alpha)\%$ 
confidence interval for $F_{T}(t)$ based on $Z_{\logit(\Fhat_{\rv})}
\approxdist Z_{\logit(\Fhat_{\rv}^{*})}$ and the $B$ bootstrap
samples is
\begin{displaymath}
[\undertilde{F_{\rv}}, \quad \tilde{F_{\rv}}] = 
\left[\frac{\Fhat_{\rv}}{\Fhat_{\rv}
+(1-\Fhat_{\rv}) \times \undertilde{w}},  \quad \frac{\Fhat_{\rv}}{\Fhat_{\rv}+
(1-\Fhat_{\rv}) \times \tilde{w}}
\right].
\end{displaymath}
Here $\undertilde{w}=\exp\{ z_{\logit(\Fhat^{*})_{(1-\alpha/2)}}
\sehat_{\Fhat_{\rv}}/ [\Fhat_{\rv}(1-\Fhat_{\rv})]\}$ and
$\tilde{w}=\exp \{ z_{\logit(\Fhat^{*})_{(\alpha/2)}}/
[\Fhat_{\rv}(1-\Fhat_{\rv}) ] \}$ where
$z_{\logit(\Fhat^{*})_{(1-\alpha/2)}}$ and
$z_{\logit(\Fhat^{*})_{(\alpha/2)}}$ are obtained from the quantiles
of the $B$ bootstrap estimates $\Fhatboot_{\rv}$,
as described in Section~\ref{section:par.boot.ci.f}.

%----------------------------------------------------------------------
%----------------------------------------------------------------------
\section{Estimating Reliability with Two or More Causes of Failure}
\label{section.multiple.causes}
%----------------------------------------------------------------------
\subsection{Products with two or more causes of failure}

Many systems, subsystems, and components (which we generically refer
to as ``units'') have more than one cause of failure.  In some
applications and for some purposes it is important to distinguish
between the different failure causes (sometimes referred to as
``failure modes''). For purposes of improving reliability, it is
essential to identify the cause of failure down to the component
level and, in many applications, down to the actual physical cause
of failure.

Multiple failure modes should be distinguished from population
mixtures (e.g.,
Section~\ref{section:discrete.mix.intro}). Population mixtures
divide a population into different mutually exclusive groupings of
units.  Such subpopulations result from differences in the
manufacture or use of the product.  Multiple failure modes, on the
other hand, are the different ways in which a particular unit might
fail.

Failure time of a system with two or more failure modes can be
modeled with a series-system or competing risk model.  Each risk is
like a component in a series system.  When one component fails, the
system (i.e., product) fails.  Each unit has a potential failure
time associated with each failure mode. The observed  failure time
is the minimum of these individual potential failure times.

%----------------------------------------------------------------------
\subsection{Estimation with two or more causes of failure}
\label{section:competing.risk}
This section explores applications in which failed units are
replaced rather than repaired after failure (repairable system data
analysis is described in Chapter~\ref{chapter:repairable.system}).
Some life tests result in failure-time data that have more than one
cause of failure. Most field data could have both failure-time and
failure-cause information reported for each failure (although
failure-cause information is often expensive or otherwise
difficult to obtain and is therefore often {\em not}
reported). Warranty data have potential problems of bias and limited
information about surviving units. As described in
Example~\ref{example:warranty.and.right.trun}, it may be necessary
to conduct a survey to get information about the status of units
that have not been reported as failing (e.g., if and when units have
been retired and the number of use-cycles for units still in
service).  Typically, many of units reported as failing in the
warranty period are units that have been subjected to the harshest
use conditions.

Field tracking studies will follow, more carefully, a group of units
in service (or simulated service). Such studies are more expensive,
but provide better information about field reliability. See Amster,
Brush, and Saperstein~(1982) for more information on planning field
tracking studies. For some applications it is possible to test in an
``accelerated'' field environment where failures could be expected
to occur more rapidly than in typical service applications
(providing time to make corrections in customer units before serious
problems might arise).

\begin{example}
{\bf Estimation of Device-G $\boldsymbol{F_{T}(t)}$ 
using failure mode information.}
\label{example:deviceg}
\begin{table}
\caption{Device-G failure times and cause of failure  
for devices that failed and running times for units that did not
fail.}
\centering\small
\begin{tabular}{rcrcrc} 
\\[-.5ex]
\hline
\multicolumn{1}{c}{Thousands}&{Failure}&\multicolumn{1}{c}{Thousands}&{Failure}&\multicolumn{1}{c}{Thousands}&{Failure}\\
\multicolumn{1}{c}{of Cycles}&{Mode}&\multicolumn{1}{c}{of
Cycles}&{Mode}&\multicolumn{1}{c}{of Cycles}&{Mode}\\
\hline
275  & W  &  106  &  S  &   88  & S \\
 13  &  S &   300 &   -- &   247 &  S  \\
147  &  W &   300 &   -- &    28 &  S  \\
 23  &  S &   212 &   W &   143 &  S  \\
181  &  W &   300 &   -- &   300 &  --  \\
 30  &  S &   300 &   -- &    23 &  S  \\
 65  &  S &   300 &   -- &   300 &  --  \\
 10  &  S &     2 &   S &    80 &  S  \\
300  &  -- &   261 &   S &   245 &  W  \\
173  &  S &   293 &   W &   266 &  W  \\
\hline
\end{tabular}
\begin{minipage}[t]{4in}
W indicates a wearout failure, S indicates an electrical surge
failure, and -- indicates a unit still operating after 300 thousand
cycles.
\label{table:deviceg.data}
\end{minipage}
\end{table}
Table~\ref{table:deviceg.data} gives times of failure and running
times for a sample of devices from a field tracking study of a
larger system. At a certain point in time, 30 units were installed
in typical service environments. Cause of failure information was
determined for each unit that failed. Mode S failures were caused by
an accumulation of randomly occurring damage from power-line voltage
spikes during electric storms resulting in failure of a particular
unprotected electronic component. These failures predominated early
in life. Mode W failures, caused by normal product wear, began to
appear after 100 thousand cycles of use.
%-------------------------------------------------------------------
\begin{sidewaysfigure}
\splusbookfiguresize{\figurehome/deviceg.crisk.comp.ps}{7in}
\caption{Weibull analyses of Device-G data 
estimating time to failure Mode S only, failure Mode W
only, and ignoring the cause of failure.}
\label{figure:deviceg.crisk.comp.ps}
\end{sidewaysfigure}
%-------------------------------------------------------------------  
The NW corner of Figure~\ref{figure:deviceg.crisk.comp.ps} displays
the results of a Weibull analysis of the Mode S failures only
(failures due to Mode W were treated as censored at the time of the
Mode W failure---all we know is that the unobserved Mode S failure
time would have been sometime after the observed Mode W failure).
Similarly, the NE corner of
Figure~\ref{figure:deviceg.crisk.comp.ps} displays the results of a
Weibull analysis of the Mode W failures only. The results for these
two analyses are also summarized in Table~\ref{table:deviceg.mles}.
In both cases, the Weibull distribution provides a good fit to the
data. The SW corner shows the results of a Weibull analysis ignoring
the cause of failure information. Looking carefully shows evidence
of a change in the slope of the plotted points, indicating a gradual
shift from one failure mode to another.
\begin{table}
\caption{Device-G field-tracking data ML estimation results for the
electric surge (S) and
wearout (W) failure modes.}
\centering\small
\begin{tabular}{lcrrrr} 
\\[-.5ex]
\hline
& & & & \multicolumn{2}{c}{95\% Approximate}\\
& &\multicolumn{1}{c}{ML} &Standard & 
\multicolumn{2}{c}{Confidence Interval}\\
Mode & Parameter &
 \multicolumn{1}{c}{Estimate}& \multicolumn{1}{c}{Error} & Lower & Upper \\
\hline 
S&$\mu_{\rm S} $ & 6.11 & .427 &  5.27  & 6.95 \\[.7ex] 
&$\sigma_{\rm S}$ & 1.49 & .35 &  .94 &  2.36\\[1.2ex]
\hline 
W&$\mu_{\rm W} $ & 5.83 & 0.11 & 5.62 & 6.04 \\[.7ex] 
&$\sigma_{\rm W}$ &.23 &.08 & .12 & .44 \\[.7ex]\hline 
S and W&$\mu_{\rm SW} $ & 5.49 & 0.23 & 5.04 & 5.94 \\[.7ex] 
&$\sigma_{\rm SW}$ &1.08 &.21 & .74 & 1.57 \\[.7ex]\hline 
\end{tabular}
\begin{minipage}[t]{4in}
For Mode S alone, $\loglike_{\rm S}=-101.36$
for Mode W alone, $\loglike_{\rm W}=-47.16$, and for both modes
together,  $\loglike_{\rm SW}= -142.62$.
\label{table:deviceg.mles}
\end{minipage}
\end{table}
The dotted lines in Figure~\ref{figure:deviceg.crisk.comp2.ps} show
the estimated ML lines for the two individual failure modes.  The
thin, straight, solid line is the ML line estimating the Weibull
$F_{T}(t)$ obtained from ignoring the cause of failure information
(i.e., using both failure modes together in the analysis). The
curved line is the series-system estimate of $F_{T}(t)$ for the two
failure modes acting together. This estimate was computed, under the
assumption of independence of $T_{1}$ and $T_{2}$ as
$\Fhat_{T}(t)=1-[1-\Fhat_{1}(t)]\times [1-\Fhat_{2}(t)]$. The two
estimates diverge rapidly after 100 thousand cycles.
%-------------------------------------------------------------------
\begin{figure}
\splusbookfigure{\figurehome/deviceg.crisk.comp2.ps}
\caption{Weibull analyses of Device-G data 
estimating time to failure Mode S only, failure Mode W
only, and distribution to the minimum of Mode S and Mode W.}
\label{figure:deviceg.crisk.comp2.ps}
\end{figure}
%-------------------------------------------------------------------
Estimates of the mean time to failure were computed from
%\ref{equation:mttf} later
$\mttfhat=\int_{0}^{\infty}[1-\Fhat_{T}(t)]dt$ and were 251.3 and 196.0
thousand cycles, respectively, for the models ignoring and using
the failure mode information. The difference between these estimates
would have been greater if the censoring had been heavier (implying
more extrapolation in time).
\end{example}

%----------------------------------------------------------------------
\subsection{Estimation of multiple failure mode distributions
when only some failure modes are identified in the data}

When the failure modes are not identified or are only partially
identified, it is sometimes possible to estimate the individual
$F_{i}(t)$ distributions by using maximum likelihood.  This may,
however, be difficult because the analyses are no longer separable,
and failure-time distributions must be estimated simultaneously
even if the modes act independently.  Also, the parameter estimates
for the distribution for one mode will be correlated with those of the
other modes (Friedman and Gertsbakh~1980).  In practice, one is
likely to analyze the data as if there were only a single mode.  This
can result in the pitfalls described earlier, especially when the
shapes of the distributions of the individual failure modes are not
similar or if censoring is somehow linked to one or more of the
failure modes. Guess, Usher,  and Hodgson~(1991), for example,
describe maximum likelihood methods. Determining the
appropriate likelihood is straight forward. Determining if
there is enough information in the data to estimate
all of the model parameters can be problematic. Also see the example
in Section~\ref{section:crisk.mixture.model}.

%----------------------------------------------------------------------
%----------------------------------------------------------------------
%----------------------------------------------------------------------
\section{Other Topics in System Reliability}
\label{section:other.sys.rel.topics}
%----------------------------------------------------------------------
\subsection{Other system structures}
For standby redundancy (also known as passive redundancy), a redundant
unit is activated only when another unit fails and the redundant unit
is needed to keep the system working.
There are many variations of this including cold standby and
partially loaded redundancy. Also it is necessary, in some systems,
to consider the reliability of component and subsystem interfaces
as well as the switching mechanism that activates the 
standby units. This can be done by including such interfaces and
switching mechanism into the overall system structure.

%----------------------------------------------------------------------
\subsection{Dependency among components}
%----------------------------------------------------------------------
The common assumption of components with independent failures is
sometimes unrealistic. It is, for example, possible that failure of
one component either improves or degrades the reliability of other
system components (leading to either negative or positive
correlation between failure times in different components).  Another
common source of dependency is ``common cause of failure'' which
occurs when an external force causes failure of more than on
component.  Figure~\ref{figure:series.dep.effect.ps}
(Figure~\ref{figure:parallel.dep.effect.ps}) showed the effect of
dependency on a simple two-component series (parallel) system. The
same effect would be amplified in a multicomponent redundant
system.

%----------------------------------------------------------------------
\subsection{Systems with repair}
%----------------------------------------------------------------------
Many systems are repaired after failure. Questions of repairability,
maintainability, and availability generally depend on knowledge of
a repair time distribution. The methods used in this book can
also be used to estimate such distributions.

Questions involving time-dependence of system failure/repair cycles
generally use models different from those used in the previous
chapters in this book.  Chapter~\ref{chapter:repairable.system}
describes simple methods for analyzing system repair-history 
and other recurrence-type data.

%----------------------------------------------------------------------
\subsection{FMEA/FMECA}
\label{section:fmea.fmeca}
%----------------------------------------------------------------------
Products and systems often have complicated designs that are
the result of efforts of one or more design teams. Management for system
reliability requires a global process to assure that the
product/system reliability will meet customer requirements.

Failure Modes and Effect Analysis (FMEA) is a systematic, structured
method for identifying system failure modes and assessing the effects
or consequences of the identified failure modes.  Failure Modes and
Effect Criticality Analysis (FMECA) considers, in addition, the
criticality (or importance) of identified failure modes, with respect
to safety, successful completion of system mission, or other
criteria.  The goal of FMEA/FMECA is to identify all possible failure
modes at a specified level of system architecture. These methods are
typically used in product/system design review processes.  The use of
FMEA/FMECA typically begins in the early stages of product/system
conceptualization and design.  Then the FMEA/FMECA evolves over time
along with changes in the product/system design and accumulation on
information about product/system performance in pre-production
testing and field experience. FMEA/FMECA is used during the
product/system design phase to help guide decision making. FMEA/FMECA
is also used to develop product/system guidelines for system repair
and maintenance procedures, to make statements about system safety,
and to provide direction for reliability improvement efforts.

Operationally, FMEA/FMECA begins by defining the scope of the
analysis, specified by the system level at which failures are to be
considered.  FMEA/FMECA can be conducted at various different levels
in a product or a system. FMEA/FMECA might be done
initially for individual subsystems.  Then the results can be
integrated to provide an FMEA/FMECA for a entire system comprised of
many subsystems. For example, an FMEA to study the reliability of
telecommunications relay repeater might consider, as basic
components, each discrete device in the electronic circuit (e.g.,
ICs, capacitors, resistors, diodes, etc.).  At another level, an FMEA
for a large telecommunications network might consider as components
all of the network nodes and node interconnections (ignoring the
electronic detail within each node). 

The next step in the FMEA/FMECA process is the identification of all
components that are subject to failure.  This is followed by
identification of all component interfaces or connections between
components that might fail. In many applications, environmental and
human-factors related failures are considered in defining failure
modes. Finally the effects of the identified failure modes are
delineated. Determining the effect of failure modes and combinations
of failure modes uses the detailed specification of the relationship
among the product/system components (system structure). Special
worksheets and/or computer software can be used to organize all of
the information. 

%----------------------------------------------------------------------
\subsection{Fault trees}
\label{section:fault.trees}
%----------------------------------------------------------------------
The FMEA/FMECA process described above is sometimes referred to as the
``bottom-up'' approach to reliability modeling. Fault tree analysis,
on the other hand, quantifies system failure using a ``top-down''
approach. First, one or more critical ``top-events'' (such as loss of
system functionality) are defined. Then in a systematic manner, the
combination (or combinations) of conditions required for that event to
occur are delineated.  Generally this is done by identifying how
failure-related events at a higher level are caused by lower level
``primary-events'' (e.g., failure of an individual component) and
``intermediate-events'' (e.g., failure of a subsystem).  Information
from an FMEA analysis might be used as input to this step. The
information is organized in the form of a ``fault-tree diagram'' with
the top event at the top of the diagram. Events at different levels of
the tree are connected by logic gates defined on a system of Boolean
logic (e.g., AND, OR, Exclusive OR, etc. gates).

A complete fault tree can be used to model the probability of critical
system events. Additional inputs required for this analysis include
probability or conditional probabilities of the primary events.  With
this information and the detailed system structure specification
provided by the fault tree, it is possible to compute critical event
probabilities. 

Fault tree diagrams are, in one sense, similar to the reliability
block diagrams presented earlier in this section. It is generally
possible to translate from one to the other. Fault tree analysis,
differs in its basic approach to system reliability.  Reliability
block diagrams are structured around the event that the system does
{\em not} fail. Fault tree analysis, however, provides focus on the
critical failure-causing top-events like loss of system
functionality or other safety-critical events. The tree shows,
directly, the root causes of these top-events, and other
contributing events, at all levels within the scope of the
analysis. The structure and logic of the fault tree itself provides
not only a mechanism for quantitative reliability assessment, but
also clearer insight into possible approaches for reliability
improvement.

%----------------------------------------------------------------------
\subsection{Component importance}
%----------------------------------------------------------------------
A component's importance to overall system reliability depends on
the reliability of the component and the component's position in the
system structure.  Measures of component importance with respect to
reliability provide information that is needed to develop effective
strategies to improve system reliability. In particular, such
measures suggest which components should get attention in
reliability improvement efforts.  For example, one particularly
simple measure of component reliability, motivated by traditional
sensitivity analysis, is the partial derivative of overall system
reliability with respect to the individual component's reliability.
Chapter 5 of H\o yland and Rausand~(1994) provides details on a
number of other useful measures of component importance.

%----------------------------------------------------------------------
\subsection{Markov and other state-space reliability models}

System models described up to this point have had only two states of
interest: failed or not failed. A state-space model can be used to
allow for a richer formulation of system behavior. For example, in a
parallel system, the system state might describe the number of
failed components. A state-space model would describe the
different system states, possible transitions from one state to
another, and probability distributions describing how the system
goes from one state to another (transition probabilities).
 
A Markov model is a special case of a state-space model requiring
that a) a memoryless property that future events depend only on the
current state and not on the manner in which the system arrived at
that state and b) a stationarity property that transition
probability distributions do not change with time. Although somewhat
restrictive, it is possible to use Markov models to describe many
kinds of practical systems with useful approximations. If a
model with a limited number of states does not have the memoryless
property, it might be possible to reformulate the state definition,
add some more states, and find a structure that does meet (at least
approximately) the required conditions.

Markov models are useful for handling dependencies among system
components, complicated repair policies, common-cause failures,
and other system complexities.  With large, complicated systems,
however, the number of states can be large, leading to
computational difficulties. Also, because of the memoryless
property, Markov models are limited to exponential distributions for
life and repair distributions.

Non-Markovian generalizations of state-space models are possible, but
there are few analytical results available and numerical
computations become exceedingly difficult when dealing with
nontrivial system structures. Analyses of such models are generally
done by using simulation methods.

\section*{Bibliographic Notes}

Barlow and Proschan~(1975) is the classic reference outlining the
mathematical theory of system reliability. Kozlov and Ushakov~(1970),
H\o yland and Rausand~(1994), Ushakov~(1994), and Gnedenko and Ushakov
(1995) provide detailed coverage of many different kinds of system
reliability models. O'Connor~(1985), and Lewis~(1996) provide
engineering-oriented descriptions of system reliability
concepts.  Gertsbakh~(1989) also describes a number of important
system reliability concepts and methods.

Chapter 5 of Nelson~(1982) provides theory and applications of multiple
failure mode (competing risk) methods for series systems of
independent components.  David and Moeschberger~(1978) and Birnbaum
(1979) provide theory, some applications, and numerous important
references for such models (for a single subpopulation).

O'Connor~(1985), Sundararajan~(1991), H\o yland and Rausand~(1994),
and Lewis~(1996) provide more details, examples, and references for
fault-tree methods.  MIL-STD-1629A and books like H\o yland and
Rausand~(1994), Klion~(1992), and Sundararajan~(1991) outline in
more detail and provide examples for the procedures for performing
an FMEA/FMECA analysis.  H\o yland and Rausand~(1994) also list
several computer programs designed to facilitate FMEA/FMECA
analyses.
%----------------------------------------------------------------------
\section*{Exercises}

%----------------------------------------------------------------------
%----------------------------------------------------------------------
\begin{exercise}
\label{exercise:sp.bridge.comp}
Consider the bridge system structure in
Figure~\ref{figure:system.bridgefig.ps}. Assume that components 1,
2, 4, and 5 all have the same cdf $F(t)$. Plot $F_{T}(t)$ versus $0
< F(t) < 1 $ with a separate line for each value of $\Pr( A_{3})= 0,
.25, .5, .75$, and 1. Comment on the results relative to the
reliability of the two different series-parallel structures in
Section~\ref{section:series.parallel}.
\end{exercise}

%----------------------------------------------------------------------
%----------------------------------------------------------------------
\begin{exercise}
Revisit the shock absorber data introduced in
Example~\ref{example:shock.absorber.data} and given in
Appendix Table~\ref{atable:shockabs.data}.  There were actually two different
causes of failure for this product. The analyses in
Chapters~\ref{chapter:nonparametric.estimation} and
\ref{chapter:parametric.ml.ls} ignored the different
``failure modes'' in the analysis. An alternative analysis (important
for some purposes) would take into consideration the different
failure modes. Such an analysis is straightforward when
the failure times to the two different
failure modes are statistically independent.
     \begin{enumerate}
\item
\label{part:shock.separate.modes}
Assuming that the failure modes are independent, fit Weibull
distributions separately to estimate the time to failure for each
failure mode.  Also fit other parametric distributions to the
individual failure modes to see if there are other distributions that fit
better than the Weibull.
\item
\label{exer.part:shockabs.crisk}
Suppose that the shock absorber is a series system that fails as soon
as there is a failure due to one mode of the other.  Combine the
results from part~\ref{part:shock.separate.modes} to obtain an
estimate of the cdf for the shock absorbers with both modes acting
together.  Plot these on Weibull probability paper and compare the
results with the analysis that ignores the differences between the two
different failure modes.
\item
Provide an intuitive explanation for the result from
part~\ref{exer.part:shockabs.crisk}.
\item
Explain why the agreement between the two methods of analysis is so
good in this case (as compared with Example~\ref{example:deviceg}). In
general, when would you expect to see more important differences
between the analysis that accounts for the different failure modes and
the analysis that does not account for the different failure modes?
\end{enumerate}
\end{exercise}

%----------------------------------------------------------------------
%----------------------------------------------------------------------
\begin{exercise}
Consider the following system diagram. Derive an expression for the
cdf of the system as a function of $F_{1}$, $F_{2}$, and $F_{3}$.\\
\xfigbookfiguresize{\figurehome/system.threefig.ps}{1.5in}
\end{exercise}

%----------------------------------------------------------------------
%----------------------------------------------------------------------
\begin{exercise}
Consider the following system diagram. Derive an expression for the
cdf of the system as a function of $F_{1}, \dots, F_{6}$.\\
\xfigbookfiguresize{\figurehome/system.sixafig.ps}{2.2in}
\end{exercise}

%----------------------------------------------------------------------
%----------------------------------------------------------------------
\begin{exercise}
Consider the following system diagram. Derive an expression for the
cdf of the system as a function of $F_{1}, \dots, F_{6}$.\\
\xfigbookfiguresize{\figurehome/system.sixbfig.ps}{2.2in}
\end{exercise}

%----------------------------------------------------------------------
%----------------------------------------------------------------------
\begin{exercise1}
Show that for a series system of $\ncomponents$ independent
components, the system hazard function is the sum of the component
hazard functions.  That is $h_{T}(t) = \sum_{i=1}^{\ncomponents}
h_{i}(t)$.
\end{exercise1}

%----------------------------------------------------------------------
%----------------------------------------------------------------------
\begin{exercise}
\label{exercise:ram.option}
A new computer-based home entertainment system will require 16
megabytes of RAM to run the operating system, store needed information
for quick access, and other tasks.  The product designers can use a
single 16 megabyte chip or four 4 megabyte chips, the latter option
being 30\% less expensive. The manufacturers of the memory claim that
the average fit rate for 2 years at normal operating conditions is 10
fits/per chip for all of their memory chips. Compare the reliability
of the memory system for the two different design options.
\end{exercise}

%----------------------------------------------------------------------
%----------------------------------------------------------------------
\begin{exercise2}
Consider the model used in Example~\ref{example:deviceg}, with the
generalization
that only a proportion $\xi$ of the units in the population are
susceptible to
failure Mode S.
\begin{enumerate}
\item
Write down expressions for the cdf and pdf of the overall failure time
distribution.
\item
Write a computer program to compute and plot the hazard function
for this model. Find combinations of the five parameters that
give a bathtub shaped hazard function.
\end{enumerate}
See Section~\ref{section:crisk.mixture.model} for an example using
this model.
\end{exercise2}

%----------------------------------------------------------------------
%----------------------------------------------------------------------
\begin{exercise1}
Consider a parallel system with 
$\ncomponents$ independent components. Show that
\begin{enumerate}
\item
The pdf for the system is 
\begin{displaymath}
f_{\rv}(\realrv)=
F_{\rv}(\realrv) \times \sum_{j=1}^{\ncomponents} 
\frac{
    f_{j}(\realrv)
     }
     {
    F_{j}(\realrv)
     }.
\end{displaymath}
\item
The hazard function is
\begin{displaymath}
h_{\rv}(\realrv)=
\frac{F_{\rv}(\realrv)}
     {1- F_{\rv}(\realrv)}
\times \sum_{j=1}^{\ncomponents} 
\left [
\frac{
    1-F_{j}(\realrv)
     }
     {
    F_{j}(\realrv)
     }
\right ] h_{j}(\realrv).
\end{displaymath}
\end{enumerate}
\end{exercise1}

%----------------------------------------------------------------------
%----------------------------------------------------------------------
\begin{exercise1}
Suppose that a series system has $\ncomponents$
$\rm{iid}$ components. If the life of each component
is modeled with a $\WEIB(\mu, \sigma)$ then
\begin{enumerate}
\item
\label{exer.part:cdf.series.weibull}
Show that the cdf for the
system is $\WEIB[\mu-\sigma \log(\ncomponents), \sigma]$.
\item
Using the cdf obtained in part~\ref{exer.part:cdf.series.weibull}
compute the hazard function for the system. Verify that
the same answer can be obtained using the formula 
for the hazard function given in Section~\ref{equation:sum.hazard}.
\item
\label{exer.part:components.reli.haz}
Compute the reliability and hazard of 
the components at $\realrv=1$ month when $\sigma=.5$ and
$\mu=2.3$. 
\item
Use the information in part~\ref{exer.part:components.reli.haz}
to compute the
reliability and hazard of the system when $\ncomponents=10$.
\end{enumerate}
\end{exercise1}

%----------------------------------------------------------------------
%----------------------------------------------------------------------
\begin{exercise1}
For a series system with 
$\ncomponents$ ${\rm iid}$ components each having
a failure-time cdf $F=F(t)$, show that
\begin{displaymath}
\varhat(\Fhat_{\rv}) =
	\left [\ncomponents (1-\Fhat)^{\ncomponents-1}\frac{\partial F}{\partial \thetavec }
 	\right ] \transpose
 	\vcvmathat_{\thetavechat}
        \left [
\ncomponents (1-\Fhat)^{\ncomponents-1}\frac{\partial F}
{\partial \thetavec }	
        \right ].
\end{displaymath}
\end{exercise1}
%----------------------------------------------------------------------
%----------------------------------------------------------------------
\begin{exercise1}
Beginning with the general formula given 
in (\ref{equation:system.kofsfig}) for 
$F_{\rv}$ in a $k$-out-of-$\ncomponents$ system
\begin{enumerate}
\item
\label{exer.part:cdf.koutofs}
Show that with identically distributed components,
$F_{\rv}(\realrv)=\sum_{j=k}^{\ncomponents} {\ncomponents \choose j}
F^{j}(1-F)^{\ncomponents-j}$.
\item
Use the result in part~\ref{exer.part:cdf.koutofs} to 
obtain $F_{\rv}(\realrv)$ when $k=2$ and $\ncomponents=3$.
%Compare your result with the special case of 
%$F_{\rv}$ given
%in (\ref{equation:system.kofsfig})
%using $F_{1}=F_{2}=F_{3}$.
\end{enumerate}
\end{exercise1}
