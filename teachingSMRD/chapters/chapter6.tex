%chapter 6
%original by wqmeeker  12 Jan 94
%edited by wqmeeker 19 mar 1994
%edited by wqmeeker 28 mar 1994
%edited by driker 29 mar 1994
%edited by wqmeeker 29/31 mar 1994
%edited by wqmeeker  1 june 94
%edited by driker 13 july 94
%edited by wqmeeker  2 aug 94
%edited by wqmeeker  7 aug 94
%edited by wqmeeker  21 sept 94
%edited by wqmeeker  2/4 oct 94 major changes switched justification
%                    of linearization and reorganized
%                    added alloy t7987 and turbine wheel examples
%edited by riker  21 oct 94 luis' comments
%edited by wqmeeker  22 oct 94 further stuff
%edited by wqmeeker  26/27/29 nov 94 
%edited by driker    14 dec 94
%edited by wqmeeker    14 dec 94
%edited by driker 20 dec 94
%edited by driker 2 feb 95
%edited by wqm    19 feb 95 luis' comments
%edited by driker 21 june 95
%edited by driker 27 june 95
%edited by driker 30 august 95
%edited by wqm    30 sept 95 gng probplot and minor stuff
%edited by driker 5 oct 95
%edited by driker 1 july 96
%edited by wqmeeker    9 mar 97
\setcounter{chapter}{5}

\chapter{Probability Plotting}
\label{chapter:probability.plotting}


%{\Large {\bf William Q. Meeker and Luis A. Escobar}}\\
%Iowa State University and Louisiana State University\\[5ex]
{\large {\bf \today}}\\[2ex]
Part of 
{\em Statistical Methods for Reliability Data}\\
Copyright 1997 W. Q. Meeker and L. A. Escobar. \\[1ex]
To be published by John Wiley \& Sons Inc. in 1998.



%----------------------------------------------------------------------
%----------------------------------------------------------------------
\section*{Objectives}
This chapter explains
\begin{itemize}
\item
Applications of probability plots.
\item
Basic  probability plotting concepts
for both complete and censored data. 
\item 
How to analytically linearize a cdf on special plotting scales.
\item 
How to plot a modified nonparametric estimate of $F(t)$
on probability paper and
how to use such plot to judge the adequacy of a particular
parametric distribution.
\item 
Analytical and simulation methods of separating useful
information from ``noise'' when using a probability plot to assess
the reasonableness of a particular distributional model.
\item 
Graphical
estimates of important reliability characteristics like failure
probabilities and distribution quantiles.

\end{itemize}

%----------------------------------------------------------------------
%----------------------------------------------------------------------
\section*{Overview}
This chapter presents the important
topic of probability plotting. Probability plots are used throughout
this book to present data, guide modeling, and present the results of
analyses.  Sections~\ref{section:probplot.intro} through
\ref{section:plotting.pos} 
explain the basic concepts of probability plotting.
Section~\ref{section:probplot.extensions} describes some useful
extensions to the standard probability plots while
Section~\ref{section.probplot.notes} explains additional aspects of the
practical application of probability plots, including the use of
simulation to help interpret such plots.

%----------------------------------------------------------------------
%----------------------------------------------------------------------
\section{Introduction}
\label{section:probplot.intro}
Probability plots are an important tool for analyzing data and have
been particularly popular in the analysis of life data.

%-------------------------------------------------------------------
\subsection{Purposes of probability plots}
\label{section:prob.plot.purposes}
In practical applications, probability plots are used to
\begin{itemize}
\item
Assess the adequacy of a particular distributional model.
\item
Provide {\em nonparametric} graphical estimates of probabilities and
distribution quantiles. 
\item
Obtain graphical estimates of {\em parametric} model parameters (e.g., by 
fitting a straight line through the points on a probability plot).
\item
Displaying the {\em results} of a parametric
maximum likelihood fit along with the data.

\end{itemize}
In addition, probability plots often reveal information about a population,
process, or data, that might otherwise escape detection.
%-------------------------------------------------------------------
\subsection{Probability plotting scales: linearizing a cdf}

The figures in
Chapter~\ref{chapter:ls.parametric.models} 
show that cdfs from different
distributions have similar shapes.  Thus, distinguishing, by eye,
among cdfs from different distributions is not easy. Probability
plots use special scales on which a cdf of a particular
distribution plots as a straight line.

The plot of \{$\realrv$ versus $F(\realrv)$\} can be linearized
by finding transformations of $F(\realrv)$ and  $\realrv$
such that the relationship between the transformed variables is
linear.  Then the transformed axes can be relabeled in terms of the
original probability and time
variables. The resulting probability scale
is generally nonlinear and is called the ``probability scale.'' The
data scale is usually a log scale or a linear scale, depending on the
particular distribution and type of probability plot.

Probability paper was developed initially to allow data analysts
to plot data, obtain estimates,
and assess fit of a particular model by
comparing with a straight line.  Normal and Weibull probability papers
have been widely used in practice.
However, because there are many different
combinations of possible probability and data axes that might be
needed, it is useful to have a computer implementation of probability
plotting methods like those described in this chapter.

%-------------------------------------------------------------------
%-------------------------------------------------------------------
\section{Linearizing Location-Scale-Based Distributions}
\label{section:linearize.loc.scale}

The quantile function for $F(\realrv)$ provides a convenient
starting point for finding the transformation needed for linearizing a
cdf.  We illustrate the ideas in this section with a subset of the
location-scale-based distributions from
Chapter~\ref{chapter:ls.parametric.models}.  The approach, however, is
similar for other location-scale-based distributions. 
In Section~\ref{section:probplot.extensions}, we extend the method to
non-location-scale distributions.
%-------------------------------------------------------------------
\subsection{Linearizing the exponential cdf}
\label{section:linearizing.exponential.probplot}
The quantile function for the  two-parameter
exponential distribution 
(see Section~\ref{section:exponential.distribution}) is
\begin{displaymath}
\rvquan_{p}= \gamma - \log(1-p) \expmean. 
\end{displaymath}
This
implies that \{$\rvquan_{p}$ versus $-\log(1-p)$\} plots as a
straight line, as illustrated in Figure~\ref{figure:linear.exponential.ps}.
Because we plot time on the horizontal
axis and $p$ on the
vertical axis (corresponding to the traditional cdf plots used in
Chapters~\ref{chapter:reliability.data} through
\ref{chapter:other.parametric.models}) $\gamma$ is the intercept on
the time scale (because $-\log(1-p)=0$ when $p=0$) and the slope 
on the time versus quantile scales is
equal to $1/\expmean.$ In this case, $\expmean$ determines the
slope of the line and $\gamma$ determines the horizontal position of the
line. 

The linear scale on the right-hand side of the plot,
corresponding to $-\log(1-p)$, is useful for graphically estimating
the slope. Because this scale is unnecessary for the other 
applications listed in Section~\ref{section:prob.plot.purposes}
and because it is bad style
to encumber graphs with unnecessary scales, such scales will be
displayed only in selected examples in this book.
%-------------------------------------------------------------------
\begin{figure}
\splusbookfigure{\figurehome/linear.exponential.ps}
\caption{Exponential probability plot (exponential distribution probability 
scale) showing exponential cdfs as straight lines for
$\expmean=50$, 200 and $\gamma=0$, 200.}
\label{figure:linear.exponential.ps}
\end{figure}
%-------------------------------------------------------------------



%----------------------------------------------------------------------
\subsection{Linearizing the normal cdf}
As shown in Section~\ref{section:normal.distribution.definition}, the
quantile function for the normal distribution is
\begin{displaymath}
\grvquan_{p}= \mu  + \Phi_{{\rm nor}}^{-1}(p) \sigma
\end{displaymath}
where
$\Phi^{-1}_{{\rm nor}}(p)$ is the $p$ quantile of the standard normal
distribution.
As illustrated in Figure~\ref{figure:linear.normal.ps}, this implies
that \{$\grvquan_{p}$ 
versus $\Phi_{{\rm nor}}^{-1}(p)$\} plots as a straight
line. 
%-------------------------------------------------------------------
\begin{figure}
\splusbookfigure{\figurehome/linear.normal.ps}
\caption{Normal probability plot (normal distribution probability scale)
showing normal cdfs as straight lines for $\mu=40$, 80 and
$\sigma=10$, 20.}
\label{figure:linear.normal.ps}
\end{figure}
%-------------------------------------------------------------------
The normal mean $\mu$ (location parameter)
can be read from the time scale at the point where the cdf
intersects the $\Phi^{-1}_{{\rm nor}}(p)=0$ line
(the right-hand scale is $\Phi^{-1}_{\nor}(p)$ and
the horizontal dashed line
shows that $\Phi^{-1}_{{\rm nor}}(p)=0$ at $p =.5$).
The slope of the line on the time versus quantile scales is $1/\sigma$.
Any normal cdf plots as a straight line with
positive slope.  Correspondingly, any straight line with positive
slope corresponds to a normal distribution.
Note the symmetry of the probability 
scale above and below .5,  following from 
the symmetry of the normal distribution pdf (shown in
Figure~\ref{figure:distplot.nor.ps}).

%----------------------------------------------------------------------
\subsection{Linearizing the lognormal cdf}
Lognormal probability plots are closely related to
normal probability plots.
As shown in Section~\ref{section:lognormal.distribution.def}, the
quantile function for the lognormal distribution is
$\rvquan_{p}=\exp[\mu + \Phi_{{\rm nor}}^{-1}(p) \sigma ]$ where
$\Phi^{-1}_{{\rm nor}}(p)$ is the $p$ quantile of the standard normal
distribution.  This leads to
\begin{displaymath}
\log(\rvquan_{p})= \mu  + \Phi_{{\rm nor}}^{-1}(p) \sigma.
\end{displaymath}
As illustrated in Figure~\ref{figure:loglinear.lognormal.ps}, this implies
that \{$\log(\rvquan_{p})$ 
versus $\Phi_{{\rm nor}}^{-1}(p)$\} plots as a straight
line. 
%-------------------------------------------------------------------
\begin{figure}
\splusbookfigure{\figurehome/loglinear.lognormal.ps}
\caption{Lognormal probability plot (normal distribution probability 
scales) showing lognormal cdfs as straight lines for $\exp(\mu)=50$, 500 and
$\sigma=1$, 2.}
\label{figure:loglinear.lognormal.ps}
\end{figure}
%-------------------------------------------------------------------
The lognormal scale parameter (median) $\rvquan_{.5}=\exp(\mu)$ can be
read from the time at the point 
where the cdf intersects the $\Phi^{-1}_{{\rm nor}}(p)=0$
line (the
horizontal dashed line shows that $\Phi^{-1}_{{\rm nor}}(p)=0$ at
$p =.5$).  The slope of the line on the time versus
quantile scales is $1/\sigma$.  Any
lognormal cdf plots as a straight line with positive slope.
Correspondingly, any straight line with positive slope corresponds to
a lognormal distribution.

The lognormal data scale is a logarithmic scale.  The lognormal
probability scale is the same as that on normal probability plots
(Figure~\ref{figure:linear.normal.ps}). The base-10 log-time scale
on the top of the graph is preferred by some engineers.  This scale,
along with the linear $\Phi_{\nor}^{-1}(p)$ scale on the right-hand
side of the plot facilitate the computation of $\sigma$ for the
base-10 lognormal distribution (see
Section~\ref{section:lognormal.distribution.def}).

\begin{example}
{\bf Reading parameter values from a probability plot.}  To
illustrate the process of reading parameter values from a
probability plot, refer to
the $\exp(\mu), \sigma= 50, 2$ line
on Figure~\ref{figure:loglinear.lognormal.ps}. The value of the
median $\exp(\mu)$ of the lognormal distribution corresponding to this line
can be read from the Time axis where the line crosses the dotted
line. Then $\exp(\mu)=50$, which corresponds to
$\mu=\log_{10}(50)=1.69897$ for the base-10 lognormal
distribution
and $\mu=\log(50)=3.912023$ for the base-$e$ lognormal distribution.
To find sigma, one needs to find the reciprocal of the slope 
of the line. Start with the base-10
log-time
scale and for best resolution, use the extreme endpoints (in this
case $\log_{10} = 0$ and  $\log_{10} = 3$). The corresponding
standardized quantile values can be read 
from the right-hand axis as approximatley  -1.97 and 1.5.
% splus> log(10)*(3-0)/(1.5-(-1.97))
% splus> [1] 1.990708
% splus> > log(50)  [1] 3.912023
% splus> > log10(50)  [1] 1.69897
Then for the base-10 lognormal distribution
\begin{displaymath}
 \sigma \approx \left( \frac{3-0}{1.5-(-1.97)}
\right) = .865 
\end{displaymath}
and for the base-$e$ lognormal distribution
$\sigma \approx \log(10) \times .865=1.99$
where the factor $\log(10)=2.302585$ converts logarithms from 
base-10 to logarithms in base-$e$.

\end{example}

%-------------------------------------------------------------------
\subsection{Linearizing the Weibull cdf}
\label{section:probplot.weibull}
As shown in Section~\ref{section:weibull.distribution},
the quantile function for the Weibull distribution 
can be expressed (showing both the location-scale and the 
common parameterization) as
$\rvquan_{p}=\exp[\mu  + \Phi_{{\rm sev}}^{-1}(p) \sigma ]=
	\weibscale[-\log(1-p)]^{1/\beta}$
where $\Phi^{-1}_{{\rm sev}}(p)=\log[-\log(1-p)]$.
This leads to
\begin{displaymath}
\log(\rvquan_{p})= \mu  + \log[-\log(1-p)] \sigma=
	\log(\weibscale) + \log[-\log(1-p)]\frac{1}{\beta}.
\end{displaymath}
As illustrated in Figure~\ref{figure:loglinear.weibull.ps}, this implies
that \{$\log(\rvquan_{p})$ 
versus $ \log[-\log(1-p)]$\} plots as a straight line. 
% [log(t_{p_{1}})-log(t_{p_{2}})]/[\Phi_{-1}(p_{1})-\Phi_{-1}(p_{2})].
%-------------------------------------------------------------------
\begin{figure}
\splusbookfigure{\figurehome/loglinear.weibull.ps}
\caption{Weibull probability plot (smallest extreme value distribution 
probability scale) showing Weibull cdfs as straight lines for
$\weibscale=50$, 500 and $\beta=.5$, 1.}
\label{figure:loglinear.weibull.ps}
\end{figure}
%------------------------------------------------------------------- 
The Weibull scale parameter $\weibscale=\exp(\mu)$ can be read from
the time scale at the point where the cdf intersects the
$\log[-\log(1-p)]=0$ line (indicated by the horizontal dashed line at
$p \approx 0.632$).  The slope of the line on the log time versus
quantile scales is $\beta=1/\sigma$.  Any Weibull cdf plots as a
straight line with positive slope.  Correspondingly, any straight line
with positive slope corresponds to a Weibull distribution. Exponential
cdfs plot as straight lines with slopes equal to 1. Note the log scale
for time (with a linear time scale, this would be a smallest extreme
value distribution probability plot).
%
%
\section{Graphical Goodness of Fit}
\label{section:graphical.gof}
Assessment of distributional adequacy is an important application of
probability plots. As shown in Section~\ref{section:plotting.pos},
this is done by plotting the nonparametric estimate $\Fhat(t)$ on the
linearizing probability scales and assessing departures from a
straight line.  Such probability plots can be made even more useful by
plotting, in addition, simultaneous confidence bands like those
presented in Section~\ref{section:np.simultaneous.cb}. Based on the
available data, any possible $F(t)$ within these bands is,
statistically, consistent with the data. On probability paper for a
particular distribution, if it is possible to draw a straight line all
the way between the bands, then the distribution is consistent with
the data. We will use simultaneous confidence bands on all probability
plots in this chapter.
%----------------------------------------------------------------------
%----------------------------------------------------------------------
\section{Probability Plotting Positions}
\label{section:plotting.pos}
To construct a probability plot, one must decide how to plot the
nonparametric estimate of $F(t)$ on the probability scales described
in Section~\ref{section:linearize.loc.scale}.  With exact times, it
has been traditional to plot each failure time against an estimate of
the probability of failing at that time.  To follow this tradition, we
plot an estimate of $F(t)$ at some specified points in
time---typically the failure times when they are reported and the
upper endpoints of inspection intervals for inspection data.  Then we
need to define ``plotting positions,'' consisting of a corresponding
estimate of $F(t)$, at these points in time.

%----------------------------------------------------------------------
\subsection{Criteria for choosing plotting positions}
\label{section:plotpos.criteria}
Criteria for choosing plotting positions should depend on the
application or purpose for constructing the probability plot. The
following are some possible applications that will suggest criteria.

\mbox{ }\\
\noindent
%-------------------------------
{\bf Checking distributional assumptions.}
Probability plotting is used to
check if the observed data are well approximated
by the postulated parametric distribution $F(\realrv; \thetavec)$.
For this purpose, some bias in the slope
and location of the fitted line is not a serious problem.
For this reason, it is generally suggested that, for assessing
distributional assumptions, the choice of plotting positions,
in moderate-to-large samples, is not so
important. 

\mbox{ }\\
\noindent
%-------------------------------
{\bf Estimation of parameters.}
If the purpose of the probability plot is to use a fitted line to
estimate parameters of a particular distribution
(by using the slope and intercept of a line drawn through
the data points), the ``best'' plotting positions will depend
on the assumed underlying model and the
functions to be estimated (e.g., which quantile or moment is of interest). 
For complete data, letting $i$ index the ordered observations, there is some
general agreement that the plotting positions
\begin{displaymath}
p_{i}=\frac{i-.5}{n}
\end{displaymath}
provide a good choice for general-purpose use in probability plotting.

\mbox{ }\\
\noindent
%-------------------------------
{\bf Display of maximum likelihood fits with data.}
As shown in 
Chapter~\ref{chapter:parametric.ml.one.par},
maximum likelihood (ML) fitting of parametric models
is a convenient and general method for obtaining
estimates and predictions from censored data.  One important
application of a probability plot is to display the ML
fit graphically and to compare with the corresponding nonparametric
estimate.  In this case an important criterion is that the line
``fit'' the points well when the assumed model being fit
with ML agrees with the data. 
With a poor choice of plotting positions, the ML line may not 
fit the plotted points. Then the probability plot can give
the false impression that the parametric model and the data disagree,
even though any difference
between the points and the line will generally be small relative to
sampling error 
variability that would be observed by repeating the sampling process.
Plotting simultaneous confidence bands on the
probability plot will indicate the amount of sampling variability
one might expect to see.

%----------------------------------------------------------------------
\subsection{Choice of plotting positions}
There are three cases to consider: 1) continuous inspection 
(or small inspection intervals resulting in exact failures), 
2) interval-censored data
with relatively large intervals, and 3) arbitrarily censored data which
could include combinations of left censoring, right censoring,
and overlapping failure intervals.


\mbox{ }\\
\noindent
%-------------------------------
{\bf Continuous inspection data and single censoring.}  With
continuous inspection and single right censoring (or complete data),
the nonparametric estimate $\Fhat(t_{(i)})=i/n$ is a step function
increasing by an amount $1/n$ at each reported failure time.  Let
$\realrv_{(1)},
\realrv_{(2)}, \ldots $ be the ordered failure times.  From a plot of
the step function (e.g.,  Figure~\ref{figure:alloy.t7987.cdf.sim.ps}),
we see that $\Fhat(t)$ steps up at each reported failure time.
Plotting at the bottom (top) of the step would lead to bias in the
plotted points and the ML line would tend to be above (below) the
plotted points.  Also, in situations where the last reported time is a
failure, it is not possible to plot a point at $\Fhat(t)=1$ (the
value of the nonparametric estimate at the last failure).  A
reasonable compromise plotting position is the midpoint of the jump
\begin{equation}
\label{equation:i.minus.p5}
\frac{i-.5}{n} = \frac{1}{2} [\Fhat (\realrv_{(i)})
 		+ \Fhat (\realrv_{(i-1)})].
\end{equation}
Another justification for this definition of plotting positions
[estimator of $F(t)$ at $t$] is that the median of the $i$th order
statistic (i.e., the $i$th largest observation) in a sample of size
$n$ is approximately $ F^{-1} [(i-.5)/n]$.  For complete or
singly-censored data, this plotting position has been useful for a
variety of different distributions and purposes [see, for example,
pages 293-294 of Hahn and Shapiro (1967) for discussion of
alternative plotting positions].
 
%-------------------------------------------------------------------
\begin{example}
\label{example:alloy.t7987.data}
{\bf Probability plots of fatigue life data for Alloy T7987.} 
Table~\ref{table:alloy.t7987} gives the fatigue life (rounded to
the nearest thousand cycles) for 67 specimens
of Alloy T7987 that failed before having accumulated 300 thousand cycles of
testing. There were, in addition, 5 ``runout'' specimens that survived
until 300 thousand cycles without failure.
%-------------------------------------------------------------------
\begin{table}
\caption{Number of cycles (in thousands) of fatigue life for 67 of 72
Alloy T7987 specimens that failed before 300 thousand cycles.}
\centering\small
\begin{tabular}{*{10}{r}}
\\[-.5ex]
\hline
% alloy  T7987 fatigue data
% thousands of cycles
 94 &  96  &  99 &   99 &  104  &  108  & 112  & 114  & 117  & 117      \\
118  & 121  & 121  & 123  & 129   &  131  & 133  & 135  & 136  & 139     \\
139  & 140  & 141  &141  & 143  & 144 &  149 &  149  & 152  & 153     \\ 
159  & 159  & 159  & 159 & 162  & 168 &  168 &  169 &  170  & 170     \\ 
171  & 172  & 173  & 176  & 177  & 180  & 180 &  184 & 	187 &  188      \\
189  & 190 &  196 &  197  & 203  &  205 &  211  & 213  & 224  & 226      \\
227  & 256  & 257  & 269  & 271  & 274    & 291    \\
\hline      
\end{tabular}
\label{table:alloy.t7987}
\end{table}
%-------------------------------------------------------------------
Figure~\ref{figure:alloy.t7987.cdf.sim.ps}
shows a plot of $\Fhat(t)$, the nonparametric estimate of
the fatigue life cdf.
Some of the step increases are integer multiples of $1/n$ because of the
ties resulting from rounding.
%-------------------------------------------------------------------
\begin{figure}
\splusbookfigure{\figurehome/alloy.t7987.cdf.sim.ps}
\caption{Linear-scales plot of nonparametric estimate 
of $F(t)$ for the Alloy T7987
fatigue life and
simultaneous approximate 95\% confidence bands for $F(t)$.}
\label{figure:alloy.t7987.cdf.sim.ps}
\end{figure}
%-------------------------------------------------------------------
The points in the Weibull probability plot in
Figure~\ref{figure:alloy.t7987.npp.weib.ps} are, for each reported
failure point, plotted at a probability corresponding to half the
jump-height of each step in
Figure~\ref{figure:alloy.t7987.cdf.sim.ps}. 
Figure~\ref{figure:alloy.t7987.npp.weib.ps} indicates
that the Weibull distribution does not provide a good fit to the
data.
%-------------------------------------------------------------------
\begin{figure}
\splusbookfigure{\figurehome/alloy.t7987.npp.weib.ps}
\caption{Weibull probability plot
of the Alloy T7987 fatigue life data and
simultaneous approximate 95\% confidence bands for $F(t)$.}
\label{figure:alloy.t7987.npp.weib.ps}
\end{figure}
%-------------------------------------------------------------------
Figure~\ref{figure:alloy.t7987.npp.lnor.ps}
indicates that the lognormal distribution provides a much better fit
than the Weibull distribution. Both distributions, however, show
concave behavior in the lower tail, an indication that a threshold parameter
for either the Weibull or lognormal distribution
would improve the fit to the data. This is investigated further
in Section \ref{section:probplot.extensions}.
%-------------------------------------------------------------------
\begin{figure}
\splusbookfigure{\figurehome/alloy.t7987.npp.lnor.ps}
\caption{Lognormal probability plot
for the Alloy T7987
fatigue life and
simultaneous approximate 95\% confidence bands for $F(t)$.}
\label{figure:alloy.t7987.npp.lnor.ps}
\end{figure}
%-------------------------------------------------------------------
\end{example}


\mbox{ }\\
\noindent
%-------------------------------
{\bf Continuous inspection data and multiple censoring.}
With continuous inspection and multiple right censoring 
the usual nonparametric estimate $\Fhat(t)$
is again a step function with steps at each reported failure time 
(but, due to censoring
between failures, the step increases may be different from $1/n$).
Corresponding to the definition for
single censoring in (\ref{equation:i.minus.p5}), we modify $\Fhat(t_{(i)})$
to get plotting positions for multiple censoring as \{$\realrv_{(i)}$ versus
$p_{i}$\} with
\begin{equation}
\label{equation:averagekm}
 p_{i}=\frac{1}{2} [\Fhat (\realrv_{(i)})\, 
+ \,\Fhat (\realrv_{(i-1)})].
\end{equation}


\begin{example}
{\bf Comparison of Weibull and lognormal probability plots for the
shock absorber data.} For the shock absorber data
in Example~\ref{example:shock.absorber.data} and Appendix
Table~\ref{atable:shockabs.data},
the nonparametric estimate of $F(t)$ is given in
Figure~\ref{figure:shockabsB.cdf.pw.ps}.
Figures~\ref{figure:shockabsB.npp.weib.ps} and
\ref{figure:shockabsB.npp.lnor.ps} show, respectively, Weibull and
lognormal probability plots of these data, along with approximate 95\%
nonparametric simultaneous confidence bands.
The Weibull distribution appears to provide a better
description of these data.
With the large amount of uncertainty
expressed by the simultaneous confidence bands, however, we certainly
could not rule out the lognormal distribution as an adequate
distribution.
\end{example}
%-------------------------------------------------------------------
\begin{figure}
\splusbookfigure{\figurehome/shockabsB.npp.weib.ps}
\caption{Weibull probability plot of the shock absorber data with
simultaneous approximate 95\% confidence bands for $F(t)$.}
\label{figure:shockabsB.npp.weib.ps}
\end{figure}
%-------------------------------------------------------------------
%-------------------------------------------------------------------
\begin{figure}
\splusbookfigure{\figurehome/shockabsB.npp.lnor.ps}
\caption{Lognormal probability plot of the shock absorber data with
simultaneous approximate 95\% confidence bands for
$F(t)$.}
\label{figure:shockabsB.npp.lnor.ps}
\end{figure}
%------------------------------------------------------------------- 

\mbox{ }\\
\noindent
%-------------------------------
{\bf Interval-censored inspection data.}
With interval-censored data, if there are failures in each interval,
$\Fhat(t)$ is defined at the upper endpoint
of each interval (see
Section~\ref{section:def.of.nonparametric.estimate}).  Let
$(\realrv_{0}, \realrv_{1}], \ldots, (\realrv_{m-1}, \realrv_{m}]$ be the
intervals preceding the $m$ inspection times.  The upper endpoints of
the inspection intervals $\realrv_{i}, i=1,2,\ldots$, 
are convenient plotting times.  For corresponding
plotting positions here use $p_{i}=\Fhat(\realrv_{i})$.  
The justification for this choice is that, with no censoring, 
from standard binomial theory,
\begin{equation}
\label{equation:interval.points.justification}
\E [\Fhat(\realrv_{i})]=F(\realrv_{i}).
\end{equation}
With losses (multiple censoring), 
(\ref{equation:interval.points.justification}) will be
approximately true. When there are no
censored observations beyond $\realrv_{m}$,
$F(\realrv_{m})=1$ and this point cannot be plotted on
probability paper. 

\begin{example}
{\bf Exponential probability plot for the heat exchanger tube data.}
Figure~\ref{figure:heatexch.npp.exp.ps} is an exponential
probability plot of the heat exchanger data showing the
nonparametric estimate of $F(t)$ computed in
Example~\ref{example:npe.heat.exchanger}.  Also shown are 95\%
nonparametric simultaneous confidence bands for $F(\realrv)$. These
bands are very wide due to the small number of observed cracks in
the combined heat exchanger data.  The bands are not symmetric
because we used the logit transformation (described in
Section~\ref{section:np.pointwise.ci}) to improve the large-sample
approximation.  The points on this plot fall roughly along a
straight line, indicating that there is no evidence here to
contradict an exponential distribution assumption. Of course, the
width of the simultaneous confidence bands for $F(\realrv)$ also
indicates that it is certainly possible that the heat exchanger tube
life has a distribution far from the exponential distribution.
%-------------------------------------------------------------------
\begin{figure}
\splusbookfigure{\figurehome/heatexch.npp.exp.ps}
\caption{Exponential probability plot of the heat-exchanger
tube cracking data with 
simultaneous approximate 95\% confidence bands for $F(t)$.}
\label{figure:heatexch.npp.exp.ps}
\end{figure}
\end{example}

\mbox{ }\\
\noindent
%------------------------
{\bf Arbitrarily censored data.}  With mixtures of left censoring,
right censoring, and observations reported as exact failures,
$\Fhat(t)$ can consist of a mixture of sets of points and horizontal
lines of increasing height.  Such estimates require a compromise
between the other two cases.

\begin{example}
%-------------------------------------------------------------------
{\bf Turbine wheel data.} The turbine wheel data from
Example~\ref{example:turbine.wheel.data} and given in
Table~\ref{table:turbine.data} consist of a set of overlapping left-
and right-censored observations.
Figure~\ref{figure:turbine.npp.exp.ps} is an exponential probability
plot of the turbine wheel data.
Figure~\ref{figure:turbine.npp.lnor.ps} is a lognormal probability
plot of the same data. Both plots contain 95\% simultaneous
nonparametric confidence bands.

It is clear that the lognormal distribution fits these data better
than the exponential distribution.  For these data, the Weibull
probability plot (not shown here) was very similar to the lognormal
probability plot.  The great width of the 95\% simultaneous
confidence bands indicates, however, that none of these distributions
could be ruled out. As a practical matter, however, it is generally
more conservative to use the more general Weibull or lognormal
distributions. Unless the assumption could be based on physical
experience with related data or other information apart from the
data, use of the exponential distribution with such sparse data
would generally give an unrealistically small indication of
sampling uncertainty.
%-------------------------------------------------------------------
\begin{figure}
\splusbookfigure{\figurehome/turbine.npp.exp.ps}
\caption{Exponential probability plot of the turbine wheel
inspection data with 
simultaneous approximate 95\% confidence bands for $F(\realrv)$.}
\label{figure:turbine.npp.exp.ps}
\end{figure}
%-------------------------------------------------------------------
%-------------------------------------------------------------------
\begin{figure}
\splusbookfigure{\figurehome/turbine.npp.lnor.ps}
\caption{Lognormal probability plot of the turbine wheel
inspection data with 
simultaneous approximate 95\% confidence bands for $F(t)$.}
\label{figure:turbine.npp.lnor.ps}
\end{figure}
%-------------------------------------------------------------------
\end{example}


%----------------------------------------------------------------------
%----------------------------------------------------------------------
\section{Probability Plots with Specified Shape Parameters}
\label{section:probplot.extensions} 
The methods in Section~\ref{section:linearize.loc.scale} can
extend to constructing probability plots for distributions that are not
members of the log-location-scale family. In addition to the other
applications described in Section~\ref{section:plotpos.criteria}, such
plots help graphically identify the possibility of improving fit by
using a nonzero threshold
parameter (Sections~\ref{section:intro.to.threshold.distributions} and
\ref{section:threshold.dist.ml}).

Some distributions are not in the location-scale or
log-location-scale families and cannot be transformed into such a
distribution (e.g, the gamma and generalized gamma distributions and
other distributions covered in
Chapter~\ref{chapter:other.parametric.models}).  Such distributions
have one or more unknown shape parameters (if a distribution has a
single shape parameter whose value is assumed known, the
distribution can be considered to be a location-scale
distribution). It is still possible to construct a probability plot
for distributions with an unknown shape parameter, but the plotting
scales depend on the given value or estimate for the shape
parameter.  There are two approaches to specifying an unknown shape
parameter for a probability plot:
\begin{itemize}
\item
Plot the data with different given values of the shape parameter in an
attempt to find a value that will give a probability plot that is
nearly linear.
\item
Use parametric maximum likelihood methods 
to estimate the shape parameter and use the estimated value to
construct probability plotting scales. This is discussed, starting in
Chapter~\ref{chapter:parametric.ml.one.par}, and continuing in
Chapters~\ref{chapter:parametric.ml.ls}, and \ref{chapter:ml.other.parametric}
\end{itemize}
These two approaches generally lead to approximately the same plot,
and with modern computing software, either is reasonably easy to
implement as shown below.

%-------------------------------------------------------------------
\subsection{Linearizing the gamma cdf}
\label{section:probplot.gamma}
As shown in Section~\ref{section:gamma.moment.quant},
the quantile function for the gamma distribution is
$\rvquan_{p}=\incgamma^{-1}(p;\kappa) \theta$.
The quantile function for the three-parameter
gamma distribution, allowing the distribution to start at $\gamma$
instead of 0
(Section~\ref{section:intro.to.threshold.distributions}), is
\begin{displaymath}
\rvquan_{p}= \gamma + \incgamma^{-1}(p;\kappa) \theta.
\end{displaymath}
This implies that \{$\rvquan_{p}$ versus $\incgamma^{-1}(p;\kappa)$\}
plots as a straight line.  In contrast to the exponential
probability plot, the probability scale for the gamma probability plot
depends on specification of the shape parameter $\kappa$.  As with the
exponential probability plots (described in
Section~\ref{section:linearizing.exponential.probplot}), $\gamma$ is
the intercept on the time scale [because $\incgamma^{-1}(p;\kappa)=0$
when $p=0$].  When plotting time is on the horizontal axis, the slope of
the cdf line equals $1/\expmean.$ Thus, changing $\expmean$
changes the slope of the line, and changing $\gamma$ changes the
horizontal position of the line.
 
%-------------------------------------------------------------------
\begin{example}
\label{example:gamma.probplot.alloy.t7987}
{\bf Gamma probability plots for fatigue data for alloy T7987.}
Here we return to the Alloy T7987 fatigue life data introduced in
Example~\ref{example:alloy.t7987.data}.
Figure~\ref{figure:alloy.t7987.npp.gamma.all.ps}
shows gamma probability plots with shape parameter
$\kappa=.8$, 1.2, 2, and 5. These plots show the effect
that choosing different gamma shape parameters will have on the
curvature in the probability plot. Among these shape parameters,
$\kappa=2$ seems to give the best fit to the data.  Each of these
plots also indicates the need for a threshold parameter $\gamma$ that is
approximately 90 (Table \ref{table:alloy.t7987} shows that the
smallest observation in the data set was 94). There is, for these
data, some physical justification for a threshold parameter. For some
alloys, the amount of time that it takes for a fatigue crack to
initiate and grow to failure may be on the order of hundreds of
thousands of cycles, particularly if deformation caused by loading is
primarily elastic.
%-------------------------------------------------------------------
\begin{figure}
\splusbookfigure{\figurehome/alloy.t7987.npp.gamma.all.ps}
\caption{Gamma probability plots with $\kappa=.8$, 1.2, 2, and 5
for the Alloy T7987 fatigue life data with
simultaneous approximate 95\% confidence bands for $F(t)$.}
\label{figure:alloy.t7987.npp.gamma.all.ps}
\end{figure}
%-------------------------------------------------------------------
\end{example}


%-------------------------------------------------------------------
\subsection{Linearizing the Weibull cdf using a linear time scale and
specified shape parameter}
\label{section:probplot.linear.weibull}
The quantile
function for the three-parameter Weibull distribution
\begin{displaymath}
\rvquan_{p}=	 \gamma + \weibscale[-\log(1-p)]^{1/\beta}.
\end{displaymath} 
can be obtained by inverting the cdf given in
Section~\ref{section:threshold.cdf}.  This expression shows that
\{$\rvquan_{p}$ versus $[-\log(1-p)]^{1/\beta}$\} plots as a
straight line.  Unlike the standard log-time-scale Weibull
probability plot described in
Section~\ref{section:probplot.weibull}, the probability scale for
the linear-time-scale Weibull probability plot requires a given
value of the shape parameter $\beta$. However, the plots provide
instead, a graphical estimate of the threshold parameter $\gamma$
(which was previously constrained to be 0). As with the gamma and
exponential probability plots, $\gamma$ is the intercept on the time
scale (because $[-\log(1-p)]^{1/\beta}=0$ when $p=0$).  When time is
on the horizontal axis, the slope of the cdf line is equal to
$1/\weibscale.$ As with the gamma and exponential probability plots,
changing $\weibscale$ changes the slope of the line, and changing
$\gamma$ changes the horizontal position of the line.

\begin{example} 
\label{example:alloy.t7987.weibull.plot.compare}
{\bf Comparison of log and linear time-scale Weibull
probability plots for fatigue life data for alloy T7987.} 
Although the standard log-data-scale Weibull probability plot in
Figure~\ref{figure:alloy.t7987.npp.weib.ps} (with threshold parameter
$\gamma=0$, implicitly) indicated a poor fit, the linear-data-scale
Weibull probability plot with specified $\beta=1.4$ (this value was
determined by trial to provide the best fit, visually) in
Figure~\ref{figure:alloy.t7987.npp.weib.1p4.ps} indicates that a
Weibull distribution with a shape parameter $\beta=1.4$ and threshold
parameter of approximately $\gamma=90$ will provide a good fit to the
data.
%-------------------------------------------------------------------
\begin{figure}
\splusbookfigure{\figurehome/alloy.t7987.npp.weib.1p4.ps}
\caption{Linear-scale Weibull plot with $\beta=1.4$
for the Alloy T7987 fatigue life with
simultaneous approximate 95\% confidence bands for $F(t)$.}
\label{figure:alloy.t7987.npp.weib.1p4.ps}
\end{figure}
%-------------------------------------------------------------------
\end{example}


%-------------------------------------------------------------------
\subsection{Linearizing the generalized gamma cdf}
\label{section:probplot.gng}
As shown in Section~\ref{section:gng.moment.quant}, the quantile
function for the generalized gamma distribution (GENG) is
$\rvquan_{p}=\theta \left[
\incgamma^{-1}(p;\kappa)\right]^{\frac{1}{\beta}}$.  This leads to
\begin{displaymath}
\log(\rvquan_{p})= \log(\theta)  + \log[\incgamma^{-1}(p;\kappa)]\frac{1}{\beta},
\end{displaymath}
implying that \{$\log(\rvquan_{p})$ versus
$\log[\incgamma^{-1}(p;\kappa)]$\} plots as a straight line.
Unlike the standard Weibull probability plot in
Section~\ref{section:probplot.weibull}, the probability scale for the
GENG probability plot requires a given value of the shape
parameter $\kappa$.  The scale parameter $\theta$ is the intercept on the
time scale, corresponding to the time where the cdf crosses the
horizontal line at $\log[\incgamma^{-1}(p;\kappa)]=0$. 
The slope of the line on the
graph with time on the horizontal axis is $\beta$. 



\begin{example} 
\label{example:bearing.gng.plot.compare}
{\bf GENG
probability plots for ball bearing fatigue data.} 
Example~\ref{example:ball.bearing.data} introduced data on the number of
revolutions to failure for 23 ball bearings.
%-------------------------------------------------------------------
\begin{sidewaysfigure}
\splusfigurelandscapesize{\figurehome/bearing.mul.gng.ps}{8in}
\caption{GENG probability plots of the ball bearing
fatigue data with specified $\kappa$=
.1, 1, 4, and 20.}
\label{figure:bearing.mul.gng.ps}
\end{sidewaysfigure}
%-------------------------------------------------------------------
Figure~\ref{figure:bearing.mul.gng.ps} shows GENG
probability plots with specified values of $\kappa$= .1, 1, 4, and 20.
The linear right-hand axis shows the gamma standard quantiles corresponding
to $\log[\incgamma^{-1}(p;\kappa)]$.
As explained in Section~\ref{section:gng.special.cases}, the value of
$\kappa$=1 corresponds to a Weibull distribution and $\kappa \rightarrow
\infty$ corresponds to the lognormal distribution. 
The value $\kappa$=20 was chosen to
correspond, roughly, to the lognormal distribution.
None of the values of $\kappa$ could be ruled out, but values greater
than 1 seem to fit better.   The value of
$\kappa$=4 was chosen by trial and error as a compromise
between the Weibull and lognormal distributions.  These examples show
that the ranges of the standard quantile scales depend strongly on the
specified value of $\kappa$. Relatedly, the value of $p$ corresponding
to $\log[\incgamma^{-1}(p;\kappa)]=0$ depends strongly on $\kappa$
(and for $\kappa=20$ it is off of the scale). This is an indication of
the potential problems, alluded to in 
Section~\ref{section:gng.reparameterization},
associated with statistically estimating the 
three traditional GENG parameters.
\end{example}


\subsection{Summary of probability plotting methods}
Table~\ref{table:linearization.of.cdfs}
summarizes the linearizing transformations
given in Sections~\ref{section:linearize.loc.scale}
and \ref{section:probplot.extensions}.
This table also indicates which parameters need to be specified
and which can be estimated from the
slope and time-scale intercept of a fitted line.
\begin{sidewaystable}
\caption{Summary of Probability Plot Scales to
Linearize cdfs}
\centering\small
%\begin{tabular}{lllllll}
\begin{tabular}{lrcrccc}
\\[-.5ex]
\hline
& & 
\multicolumn{2}{c}{Linearizing}&& \multicolumn{2}{c}{Identified}
\\
& & 
\multicolumn{2}{c}{Transformation}& & \multicolumn{2}{c}{Parameters}
\\
\cline{3-4}  \cline{6-7}
       &                               & \multicolumn{1}{c}{Time (Data)}          &
\multicolumn{1}{c}{Probability}&Specified&\multicolumn{1}{c}{Time-Scale}
\\
Family &\multicolumn{1}{c}{cdf}  &  \multicolumn{1}{c}{Scale}
& \multicolumn{1}{c}{Scale}&Parameter& Intercept&Slope \\
\hline
\\
Exponential & $1-\exp \left (-\,  \frac{t-\threshold}{\expmean}\right )$ & $\rvquan_{p}$
& $-\log(1-p)$ &   & $\gamma$  & $\frac{1}{\theta} 
\approx \frac{1}{\rvquan_{.63}}$  \\ \\
%

Smallest Extreme Value & $\Phi_{{\rm sev}} \left ( \frac{\grealrv-\mu}{\sigma} \right )$ & $\grvquan_{p}$
& $\Phi^{-1}_{{\rm sev}}(p)$  &    & $\mu \approx y_{.63}$   &  $\frac{1}{\sigma}$   \\ \\
%
Weibull (two-parameter) & $\Phi_{{\rm sev}} \left [ \frac{\log(\realrv)-\mu}{\sigma} \right ]$ & $\log(\rvquan_{p})$
& $\Phi^{-1}_{{\rm sev}}(p)$ &    & $\weibscale=e^{\mu}\approx t_{.63}$  &  $\beta=\frac{1}{\sigma}$    \\ \\
%
Weibull (three-parameter) & $\Phi_{{\rm sev}} \left [ \frac{\log(\realrv-\gamma)-\mu}{\sigma} \right ]$ & $\rvquan_{p}$
& $\exp[\Phi^{-1}_{{\rm sev}}(p)\sigma]$ &  $\beta=\frac{1}{\sigma}$  &  $\gamma$  &  $\frac{1}{\weibscale}\approx \frac{1}{\rvquan_{.63}}$    \\ \\
%
Normal & $\Phi_{{\rm nor}} \left ( \frac{\grealrv-\mu}{\sigma} \right
)$ 
& $\grvquan_{p}$
& $\Phi^{-1}_{{\rm nor}}(p)$    & & $\mu=y_{.5}$  & $\frac{1}{\sigma}$   \\ \\
%
Lognormal (two-parameter) & $\Phi_{{\rm nor}} \left [ \frac{\log(\realrv)-\mu}{\sigma} \right ]$ & $\log(\rvquan_{p})$
& $\Phi^{-1}_{{\rm nor}}(p)$  &   &  $e^{\mu}=\rvquan_{.5}$  & $\frac{1}{\sigma}$   \\ \\
%
Lognormal (three-parameter) & $\Phi_{{\rm nor}} \left [ \frac{\log(\realrv-\gamma)-\mu}{\sigma} \right ]$ & $\rvquan_{p}$
& $\exp[\Phi^{-1}_{{\rm nor}}(p)\sigma]$  & $\sigma$    &  $\gamma$  & 
	$e^{-\mu}=\frac{1}{\rvquan_{.5}}$   \\ \\
Gamma (3-parameter) & $\incgamma\left(\frac{\realrv-\gamma}{\expmean};\gammashape\right)$ & $\rvquan_{p}$
& $\incgamma^{-1}(p;\gammashape)$    & $\gammashape$ &  $\gamma$   &  $\frac{1}{\theta}$    \\ \\
%
Generalized Gamma
& $\incgamma\left[\left(\frac{\realrv}{\expmean}\right)^{\beta};\gammashape \right]$ & $\log(\rvquan_{p})$
& $\log[\incgamma^{-1}(p;\gammashape)]$  &  $\gammashape$   &  $\theta$  &
$\beta$   \\ [.5ex]
\hline
\end{tabular}\\
\begin{minipage}[t]{5.5in}
The functions defined under ``cdf'' and ``Probability Scale'' are
defined in Chapters~\ref{chapter:ls.parametric.models} and
\ref{chapter:other.parametric.models}.
\end{minipage}
\label{table:linearization.of.cdfs}
\end{sidewaystable}

%----------------------------------------------------------------------
%----------------------------------------------------------------------
\section{Notes on the Application of Probability Plotting}
\label{section.probplot.notes}
\subsection{Using simulation to help interpret probability plots}
\begin{itemize}
\item 
When the points on the probability plot follow a curved pattern (as in
Figure~\ref{figure:alloy.t7987.npp.weib.ps}), a smooth curve drawn
through the points will still provide a useful graphical nonparametric
estimate of the cdf. The curve provides quantile or failure
probability estimates. This kind of plot is, for some, easier to
interpret than the cdf plot with linear probability axes (compare
Figure~\ref{figure:alloy.t7987.cdf.sim.ps} with
Figure~\ref{figure:alloy.t7987.npp.weib.ps}).
\item 
As we have illustrated in our examples, analysts should try
probability plotting with different assumed distributions and compare
the results.  Of course, finding a probability plot that indicates a
good fit to the data does not guarantee that the model will be
adequate for the desired purpose. This is a judgment that must be
made in the context of the particular application.
\item 
When assessing linearity, one must generally allow for the fact
that, for most distributions, there will be more variability in the
extreme observations.  Judgment about the departure from
linearity to expect comes with experience.  Even experienced data
analysts find, however, that it helpful to either
\begin{itemize}
\item 
Plot simultaneous nonparametric confidence bands (e.g., the methods
described in Section~\ref{section:np.simultaneous.cb}
and Section~\ref{section:graphical.gof}) to help assess the
sampling uncertainty in the nonparametric estimate of $F(t)$ or
\item 
Use simulation methods to assess sampling variability directly. For
example, one could generate simulated censored data from a particular
distribution and plot the nonparametric estimates from a series of such
data sets to get a sense of the deviations from linearity that one
would expect under specific assumed distributions.
\end{itemize}
\end{itemize}

%-------------------------------------------------------------------
\begin{sidewaysfigure}
\splusfigurelandscapesize{\figurehome/multiple.probplot.nor.sim.ps}{8in}
\caption{Simulated normal data on normal probability plots.
Five replications for each sample size.}
\label{figure:multiple.probplot.nor.sim.ps}
\end{sidewaysfigure}
%-------------------------------------------------------------------
%-------------------------------------------------------------------
\begin{sidewaysfigure}
\splusfigurelandscapesize{\figurehome/multiple.probplot.exp.sim.ps}{8in}
\caption{Simulated exponential data plotted on normal probability plots.
Five replications for each sample size.}
\label{figure:multiple.probplot.exp.sim.ps}
\end{sidewaysfigure}
%-------------------------------------------------------------------
To illustrate the use of simulation,
Figure~\ref{figure:multiple.probplot.nor.sim.ps} shows 
probability plots of simulated normal distribution samples 
of size $n$=10, 20, and 40.
There are five probability plots
from each sample size to allow an assessment of the repeatability or
consistency of such plots. What we see is that
there is very little consistency in the $n=10$ plots. Even though the
data were normally distributed, the pattern in the plots can deviate
importantly from a straight line. With the larger sample sizes,
however, there is more consistency across the repeated plots, except for
the variability in the tails of the distribution. 
Figure~\ref{figure:multiple.probplot.exp.sim.ps} has similar
normal probability plots, but in that case the
simulated data were from an exponential distribution.
In this case we see that some of the plots with $n=10$ do not
deviate too much from a straight line and, to some extent,
are similar to the probability plots of the normal data in
Figure~\ref{figure:multiple.probplot.nor.sim.ps}. For the larger sample
sizes, however, there is enough consistency to indicate that samples
of size 20 to 40 are sufficiently large to distinguish between
data  from exponential and normal distributions.

Inexperienced analysts tend to expect plots to be straighter than
they are.  Simulations of this kind can and should be used to help
analysts ``calibrate'' their interpretation of probability plots,
particularly in unfamiliar situations.

%-------------------------------------------------------------------
\subsection{Possible reason for a bend in a probability plot}
Probability plots with a sharp bend or change in slope generally
indicate an abrupt change in a failure process.  Causes for such
behavior could include two or more failure modes or a mixture of
different subpopulations. Such causes should be investigated and
will often suggest how to improve product reliability.
\begin{example}
%-------------------------------------------------------------------
{\bf Bleed system failure.}
%-------------------------------------------------------------------
\begin{sidewaysfigure}
\splusfigurelandscapesize{\figurehome/bleed.mixture.ps}{8in}
\caption{Bleed system data: (NW) Linear plot of the cdf, (NE) Weibull probability plot 
for all bases, (SW) Weibull probability plots for Base D alone,
(SE) Weibull plot for all bases except Base D.}
\label{figure:bleed.mixture.ps}
\end{sidewaysfigure}
%-------------------------------------------------------------------
Appendix Table~\ref{atable:bleed.data} gives failure and running time for 2256
bleed systems. The data were abstracted from Abernethy, Breneman,
Medlin, and Reinman~(1983) who present an analysis similar to
the one done here.

The top row of Figure~\ref{figure:bleed.mixture.ps} shows a plot of
the nonparametric estimate and a corresponding Weibull probability
plot of the data. The
different slopes on this probability plot before and after 600 hours suggests some kind of
change. Closer examination of the data showed that 9 of the 19
failures had occurred at Base D. In the bottom row of 
Figure~\ref{figure:bleed.mixture.ps}, separate analyses of the Base D data
and the data from the other bases indicated different life
distributions. The large slope ($\beta \approx 5$) for Base D indicated
strong wearout behavior.  
The relatively small slope for the other bases ($\beta
\approx .85$) suggested infant mortality or accidental failures. After
investigation it was determined that the early-failure
problem at base D was caused
by salt air (Base D was near the ocean). A change in maintenance
procedures there solved the dominant bleed system reliability problem.
%-------------------------------------------------------------------
%-------------------------------------------------------------------
\end{example}

%-------------------------------------------------------------------
\subsection{Use of grid lines and special scales on probability plots}
\label{section:grid.lines}
In the past, most data were plotted by hand on pre-prepared
probability paper that contained grid lines. The grid lines make it
easier to plot points by hand and allowed one to more precisely read
numbers from the plot. While grid lines are useful, some analysts feel
that grid lines can get in the way of interpreting a plot. Computer
programs should provide an option to include grid lines or not.
Because our interest is primarily in graphical perception of the
information on a plot, we will generally not use grid lines on our
computer-generated data analysis plots.  If we are interested in
particular numbers that would be read from the graph, the numbers are
available from tabular computer output. For purposes of illustration,
however, the following example uses probability plots with grid lines.
It is also possible to put special scales on probability paper to
facilitate graphical estimation of the parameter related to the slope
of a line on the plot. Such scales are on some commercial probability
papers but can also be put on computer-generated plots.

\begin{example}
%-------------------------------------------------------------------
{\bf V7 transmitter tube failure data.}
%-------------------------------------------------------------------
Figure~\ref{figure:v7tube.npp.exp.ps} is an exponential probability plot
of the V7 transmitter tube data from
Example~\ref{example:heat.transmitter.tube.data}. We see some departure from
a straight line in the plot, but the width of the
confidence bands makes it clear that this could be the result of
random variability.
%-------------------------------------------------------------------
\begin{figure}
\splusbookfigure{\figurehome/v7tube.npp.exp.ps}
\caption{Exponential probability plot of the V7 transmitter
tube failure data with 
simultaneous approximate 95\% confidence bands for $F(\realrv)$.}
\label{figure:v7tube.npp.exp.ps}
\end{figure}
%-------------------------------------------------------------------

Figures~\ref{figure:v7tube.npp.weib.ps} and
\ref{figure:v7tube.npp.lnor.ps} are, respectively, Weibull 
and lognormal probability plots
of the V7 transmitter tube failure data. Comparing 
Figures~\ref{figure:v7tube.npp.exp.ps},
\ref{figure:v7tube.npp.weib.ps},
and \ref{figure:v7tube.npp.lnor.ps} suggests that none of these
distributions can be ruled out but that the lognormal
distribution provides the best fit among these distributions.  Vacuum
tubes have parts (filaments and cathode coatings) that will
deteriorate with use, suggesting that the exponential distribution
would not be an appropriate model.

%-------------------------------------------------------------------
\begin{figure}
\splusbookfigure{\figurehome/v7tube.npp.weib.ps}
\caption{Weibull probability plot of the V7 transmitter
tube failure data with
simultaneous approximate 95\% confidence bands for $F(\realrv)$.}
\label{figure:v7tube.npp.weib.ps}
\end{figure}
%-------------------------------------------------------------------
%-------------------------------------------------------------------
\begin{figure}
\splusbookfigure{\figurehome/v7tube.npp.lnor.ps}
\caption{Lognormal probability plot of the V7 transmitter
tube failure data with
simultaneous approximate 95\% confidence bands for $F(\realrv)$.}
\label{figure:v7tube.npp.lnor.ps}
\end{figure}
Figures~\ref{figure:v7tube.npp.exp.ps}, \ref{figure:v7tube.npp.weib.ps} and
\ref{figure:v7tube.npp.lnor.ps} also contain special scales
that allow one to graphically estimate $\theta$, $\beta$ and $\sigma$,
respectively, without doing any computations. To do this,
draw a line, as shown, parallel to a line through
the data points, going through the
mark ``$\oplus$'' and read, respectively,
the estimates $\theta=32$, $\beta=.75$ and $\sigma=1.06$ scales
on the left-hand side of the graphs.
%-------------------------------------------------------------------
\end{example}


%----------------------------------------------------------------------
%----------------------------------------------------------------------
\section*{Bibliographic Notes}
Most of the literature on methods for probability plotting
is concerned with complete (uncensored) data. Chapter 8 of Hahn 
and Shapiro~(1967) provides
a nice summary of basic theory and methods and illustrates, with
simulated data, the variability
that one expects to see in the points on a probability
plot.  Harter~(1984) reviews some history concerning the choice of
plotting positions. Chernoff and Lieberman~(1954), Blom~(1958), and
Barnett~(1976) used good estimation of model parameters as a criterion
for choosing plotting positions. David~(1981, page 208) gives 
an excellent review of the results on these last three
papers. For multiply censored data Lawless
(1982, page 88) suggests the use of the half-step correction to the
nonparametric estimate of $F(t)$ defined in
(\ref{equation:averagekm}).
Nelson and Thompson~(1971) provide Weibull probability papers
that can be copied and used for making
probability plots ``by hand.'' Such papers are also available
commercially.


For multiply censored data Nelson~(1972, 1982, Chapter~4), proposed
the use of a hazard plot. A hazard plot can be viewed as a type of
probability plot with special plotting positions corresponding to
the Nelson-Aalen nonparametric estimate of the cdf (See
Exercise~\ref{exercise:nelson.aalen.estimator}).  Nelson~(1982, page
135) suggests modified hazard plotting positions obtained by averaging
the hazard step function at the jumps.  These are similar to the
modified plotting positions in (\ref{equation:averagekm}).  Nelson
comments ``The modified positions agree better with a distribution
fitted by maximum likelihood.'' An alternative that would serve the
same purpose would average the estimates in the probability scale
instead, similar to (\ref{equation:averagekm}).  Wilk, Gnanadesikan,
and Huyett~(1962a) show how to construct probability plots for the
gamma distribution.  Nair~(1981) describes, in more detail, the theory
and applications of simultaneous confidence bands as a tool for
assessing distributional goodness of fit.


%----------------------------------------------------------------------
%----------------------------------------------------------------------
\section*{Exercises}

\begin{exercise}
For the $\LOGLOGIS(\mu, \sigma)$ distribution
with cdf 	 
\begin{displaymath}
F(\realrv)=\Phi_{\logis} \left [
\frac{\log(\realrv)-\mu}{\sigma}
\right ], \quad \realrv > 0; \quad -\infty< \mu <\infty, \sigma
> 0,  	 
\end{displaymath} 	 
\begin{enumerate} 	 
\item 	 
Find the probability scales that will linearize all of the cdfs
in the logistic family.
\item
Use the scales to generate a proper labeled graph, and display the
$\LOGLOGIS(1, 1)$ and the $\LOGLOGIS(1, 2)$ cdfs.
\item 
What quantile of this distribution corresponds to the scale
parameter $\exp(\mu)$?
\end{enumerate}
\end{exercise}

\begin{exercise}
Starting with an ordinary piece of graph paper with linear divisions,
perform the following steps to create Weibull probability paper with
time ranging between $10$ and $1000$ and probability ranging between
$.001$ and $.999.$ Refer to Figure~\ref{figure:loglinear.weibull.ps} 
for an example. Alternatively, program a spread-sheet or statistical 
package to do the same thing with computer graphics.
\begin{enumerate}
\item
Find values of $\log[-\log(1-p)]$ for $p=.001$ and $p=.999.$ Use these
to develop a linear axis on the right-hand side of the graph.
\item
For selected values of $p$ between .001 and .999 (e.g., .001,
.01, .1, .3, .5, .7, .9, .99, .999) compute $\log[-\log(1-p)].$ Find
this value on the right-hand side axis to determine the location of
the $p$ label on the left-hand side axis.
\item
Find values of $\log(t)$ for $t=10$ and $t=1000.$ Use these to develop
a linear axis for $\log(t)$ on the top of the page.
\item
For selected values of $t$ between 10 and 1000 (e.g., 10, 20, 50,
100, 200, 500, 1000), compute $\log(t)$ and use the location on the
top axis to determine the corresponding locations for the time labels
on the bottom axis.
\end{enumerate}
\end{exercise}

\begin{exercise}
Consider the scale  parameter $\eta$ for the Weibull distribution.
\begin{enumerate}
\item
Show that $\eta=\exp(\mu)$ for the Weibull distribution is
approximately equal to the $.63$ quantile.
\item
Discuss the practical importance of estimating $\eta$ for a
population of integrated circuits to be installed in new personal
computers.
\item
Is it possible to get a good graphical estimate of $\eta$ from a
probability plot based on a life test for which only 3.5\% of the
integrated circuits failed by the end of the test?
\item
For what Weibull ``parameters'' (i.e., functions of $\eta$ and $\beta$)
can one get a good graphical estimates
from such data?
\end{enumerate}
\end{exercise}

\begin{exercise}
Use the following 10 simulated observations from a Weibull
distribution with $\weibscale=1$ and $\beta=2$ (so that $\mu=0$ and
$\sigma=.5$) to make a Weibull probability plot and use it to
obtain graphical estimates of the parameters
$\eta$ and $\beta$. 
$\realrv_{i}=$.74, 1.21, .22, .37, 1.28,
.73, .99, .67, .71, .33.
How do the estimates compare with the ``true parameter values''?
\end{exercise}


\begin{exercise} 
\label{exercise:bbear.nonpar.probplot}
Consider the ball bearing fatigue data
given in Example~\ref{example:ball.bearing.data}
and Table~\ref{table:lz.bbearing.data}.

%-------------------------------------------------------------------
\begin{enumerate}
\item
Compute a nonparametric estimate of $F(\realrv)$, the proportion of
units failing as function of time. Plot your estimate on paper with
linear scales.
\item
\label{exer.part:lnormal.prob.plot}
Make a lognormal probability plot of the data. This is accomplished
by ordering the failure times
 in increasing order, $\realrv_{(1)} \le \ldots \le \realrv_{(23)}$. 
Then plot
$\realrv_{(i)}$ versus $(i-.5)/n$ on lognormal 
probability paper.
\item
\label{exer.part:weibull.prob.plot}
Do the same as in part~\ref{exer.part:lnormal.prob.plot} but on
Weibull probability paper. 
\item
Comment on the adequacy
of the lognormal and Weibull models to describe these data.
\end{enumerate}
\end{exercise}


\begin{exercise}
Use the answers to Exercise~\ref{exercise:earth.orbit.cdf}
to do the following:
\begin{enumerate}
\item
Make a Weibull probability plot displaying the
device failure data.
\item
Use the plotted points to estimate the proportion of devices that
will fail before 10,000 hours of operation.
\item
Comment on whether the Weibull distribution fits the data well.
\item
Use the slope and location of this line to estimate the Weibull
distribution parameters.
\item
Use the plotted points to estimate the proportion of devices that
will fail before 100,000 hours. Comment on the usefulness of this
estimate.
\end{enumerate}
\end{exercise}

\begin{exercise} 
\label{exercise:fatigue.experiment}
A sample of 100 specimens of a titanium alloy were subjected to a
fatigue test to determine time to crack initiation. The test was run
up to a limit of 100,000 cycles.  The observed times of crack
initiation (in units of 1,000 of cycles) were: $18, 32, 39, 53, 59,
68, 77, 78, 93$. No crack had initiated in any of the other 91
specimens.
\begin{enumerate}
\item
Compute a nonparametric estimate, $\Fhat(\realrv)$, of the cdf
$F(\realrv)$ using both the simple binomial method and the Kaplan
Meier method (in this case these two methods provide the same
answer).
\item
Plot $\Fhat(t)$ on linear axes.
\item
Use $\Fhat(t)$ to compute plotting positions and plot the data
on Weibull paper. Use the plot to obtain an
estimate of the Weibull shape parameter~$\beta$.
\item
Comment on the adequacy of the Weibull distribution.
\item
Comment on the adequacy of the available data if the purpose of the 
experiment was to estimate $t_{.1}$.
\end{enumerate}
\end{exercise}


\begin{exercise}
For the high-cycle fatigue life data in Exercise~\ref{exercise:parida.1991},
construct probability plots for the exponential, lognormal, Weibull,
and gamma distributions (trying several values of $\gammashape$
for the gamma distribution). Which distributions appear suitable for
describing the shape of the distribution in the lower tail?
\end{exercise}



\begin{exercise}
Using the life test data on silicon photodiode detectors 
from Exercise~\ref{exercise:weis.et.al.1986},
construct probability plots for the
exponential, Weibull, and lognormal distributions.
Which distributions look like they might provide
an adequate model for photodiode detector life?
\end{exercise}


\begin{exercise}
Figures~\ref{figure:loglinear.lognormal.ps} and 
\ref{figure:loglinear.weibull.ps} have horizontal lines at the 
standardized quantile value of 0.
\begin{enumerate}
\item
For Figure~\ref{figure:loglinear.lognormal.ps}, 
explain why the dotted line crosses the $F(t)$
scale at $F(t)=.5$.
\item
For Figure~\ref{figure:loglinear.weibull.ps}, compute the value of $F(t)$ 
where the dotted line crossed the 0 on the standardized quantile scale.
\end{enumerate}
\end{exercise}

\begin{exercise}
\label{example:graphical.T7987}
Using the linear scales on the top and right of 
Figure~\ref{figure:alloy.t7987.npp.lnor.ps},
we can use a straight line drawn through the data points to
obtain graphical estimates of the lognormal distribution fit to
the Alloy T7987 data. Use these estimates to compute a parametric estimate
of F(200) by substituting them into (\ref{equation:lognormal.cdf}).
\end{exercise}

\begin{exercise}
\label{example:graphical.T7987.a}
Use Figures~\ref{figure:alloy.t7987.cdf.sim.ps}, 
\ref{figure:alloy.t7987.npp.weib.ps}, and 
\ref{figure:alloy.t7987.npp.lnor.ps} to obtain nonparametric graphical
estimates of $F(200)$ for the Alloy T7987 data. Are the answers
similar? Explain why or why not. Compare your answers with those
obtained in Exercise~\ref{example:graphical.T7987}. Explain the reason
for observed differences.
\end{exercise}

\begin{exercise}
Use the linear scales on the top and right of 
Figure~\ref{figure:shockabsB.npp.weib.ps}
to compute graphical estimates of the Weibull distribution parameters
for the shock absorber data.
\end{exercise}


\begin{exercise1}
Consider the family of gamma distributions with scale parameter
$\expmean$ and shape parameter $\gammashape$, as in
equation~(\ref{equation:gamma.cdf}). Show that for a fixed value
of $\gammashape$, the probability plotting scales $\{\rvquan_{p},
\incgamma^{-1}(p;\gammashape)\}$ provide linearizing scales of
the distribution for all values of the parameter $\expmean$.
\end{exercise1}


\begin{exercise1}
Consider an uncensored sample (i.e., all observations reported as
exact failures) $\realrv_{(1)} \le \cdots \le \realrv_{(n)}$ of
failure times used to make a Weibull probability plot. Let
$(\realrv_{(i)}, p_{i}), i=1, \ldots, n$ be the points on the
probability plot, where $p_{i}$ is defined in
(\ref{equation:averagekm}). A simple method for estimating the
Weibull~$(\eta, \beta)$ parameters is the following: Use least
squares to fit a straight line through the points using $\log
(t_{(i)})$ as the response $(y)$ and $\Phi^{-1}_{\sev}(p_i)$ as the explanatory variable
$(x)$.  Then use the intercept and the slope of the line,
respectively, to estimate the parameters $\mu$ and $\sigma$.  Denote
these estimates by $\sigmahat_{{\rm ols}}$ and $\muhat_{{\rm ols}}$.
Then estimates for $(\eta, \beta)$ are $(\exp(\muhat_{{\rm
ols}}),1/\sigmahat_{{\rm ols}})$.

\begin{enumerate} 	 
\item 
Derive the equations for the estimates $\sigmahat_{{\rm ols}}$, and
$\muhat_{{\rm ols}}$.
\item 
Do the assumptions that assure optimality of the ordinary least
squares estimators hold in this case?  Give details for your answer.
\item 
Is the standard R-squared statistic used in regression a useful measure
of goodness of fit for this problem? Why or why not?
\end{enumerate}
\end{exercise1}
