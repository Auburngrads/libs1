%chapter 17
%original by wqmeeker  12 Jan 94
%edited by wqmeeker 9 Apr 94
%edited by wqmeeker 24 Apr 94
%edited by wqmeeker 27 Apr 94
%edited by wqmheeker  1 June 94
%edited by driker 31 August 95
%edited by driker 11 nov 95
%edited by wqmeeker  10 dec 95  restructuring
%edited by driker 20 dec 95
%edited by wqmeeker  22-31 dec 95 restructuring and adding new sections
%edited by wqmeeker  1 jan 96
%edited by driker 12 jan 96 
%edited by driker 1 july 96 
%edited by wqmeeker 28 aug 96 
%edited by driker 20 nov 96
%edited by driker 23 july 97

\setcounter{chapter}{16}

\chapter{Failure-Time Regression Analysis}

\label{chapter:regression.analysis}


%{\Large {\bf William Q. Meeker and Luis A. Escobar}}\\
%Iowa State University and Louisiana State University\\[5ex]
{\large {\bf \today}}\\[2ex]
Part of 
{\em Statistical Methods for Reliability Data}\\
Copyright 1997 W. Q. Meeker and L. A. Escobar. \\[1ex]
To be published by John Wiley \& Sons Inc. in 1998.



%----------------------------------------------------------------------
%----------------------------------------------------------------------
\section*{Objectives}
This chapter explains
\begin{itemize} 
\item 
Modeling life as a function of explanatory variables.
\item 
Graphical methods for displaying censored
regression data.
\item
Time-scaling transformation functions and other forms
of relationships between life and explanatory variables.
\item 
Simple regression models to relate life to explanatory
variables.
\item
Likelihood methods to draw conclusions from censored
regression data.
\item 
How to detect departures from an assumed regression model.
\item 
Extensions to more elaborate nonstandard regression models that can
be used to relate life to explanatory variables, including quadratic
regression, models in which $\sigma$ depends on explanatory
variables, and proportional hazards models.
\end{itemize}

%----------------------------------------------------------------------
%----------------------------------------------------------------------

%---------------------------------------------------------------------
%----------------------------------------------------------------------
\section*{Overview}
This chapter builds on the material in
Chapter~\ref{chapter:parametric.ml.ls} and other earlier chapters.
It shows how to model time as a function of explanatory variables.
Section~\ref{section:fail.time.regr.models}
introduces the idea of time acceleration and related failure-time
regression models and contrasts them with other regression models.
Section~\ref{section:simple.linear.reg} uses a simple regression
example to explain and illustrate the most important concepts in 
this chapter. Section~\ref{section:regr.ci}
shows how to extend methods from
Chapter~\ref{chapter:parametric.ml.ls} to compute confidence intervals
for model
parameters and functions of parameters (e.g., quantiles and failure
probabilities).  Sections~\ref{section:quad.regr},
\ref{section:multiple.reg.models}, and
\ref{section:product.comparison} describe applications requiring 
other, more complicated, regression models.
Section~\ref{section:censored.data.regression.diagnostics} shows how
to adapt traditional regression diagnostics to nonnormal
distributions
and problems involving
censored data.  Section~\ref{section:prohaz.fail} describes the Cox
proportional hazard model, which has been used occasionally for
analyzing field failure data.

%----------------------------------------------------------------------
\section{Introduction}
Chapters~\ref{chapter:nonparametric.estimation} through
\ref{chapter:ml.other.parametric} presented and illustrated the
use of models to describe a single failure-time distribution. In
this chapter we present and illustrate methods for including
explanatory variables in failure-time models.  Some of the material
in this chapter will be familiar to those who have previously
studied traditional statistical regression analysis. There the mean
of a normal distribution is modeled as a linear function of one or
more (possibly transformed) explanatory variables. The more general
regression methods presented here were needed to solve practical
problems in reliability testing and data analysis.
\begin{example}
\label{example:comptime.data}
{\bf Computer program execution time data.} Appendix
Table~\ref{atable:comptime.data} and
Figure~\ref{figure:comptime.data.ps} give the amount of time it took
to execute a particular computer program, on a multi-user computer
system, as a function of the system load (obtained with the Unix {\tt
uptime} command) at the point in time when execution was beginning.
The figure shows that, as one might expect, it takes longer to execute
a program when the system is more heavily loaded. The figure also
indicates that there is more variability at higher levels of system
load.  Execution-time predictions are useful for scheduling subsequent
steps in a multi-step series-parallel computing problem.
\end{example}

%----------------------------------------------------------------------
\begin{figure}
\splusbookfigure{\figurehome/comptime.data.ps}
\caption{Scatter plot of computer program execution time versus
system load.}
\label{figure:comptime.data.ps}
\end{figure}

\begin{example}
\label{example:super.alloy.data}
{\bf Low-cycle fatigue data on a nickel-base superalloy.}  Appendix
Table~\ref{atable:super.alloy.data} and
Figure~\ref{figure:super.alloy.data} give low-cycle fatigue life
data for a strain-controlled test on 26 cylindrical specimens of a
nickel-base superalloy.  The data were originally described and
analyzed in Nelson (1984 and 1990a, page 272).  Four of the
specimens were removed from test before failure.  These censored
observations are indicated by a $\Delta$ in
Figure~\ref{figure:super.alloy.data}.  In addition to number of
cycles, each specimen has a level of pseudostress (Young's modulus
times strain).  The purpose of Nelson's analysis was to estimate the
curve giving the number of cycles at which .1\% of the population of
such specimens would fail, as a function of the pseudostress.
Figure~\ref{figure:super.alloy.data} shows that (as expected)
failures occur sooner at high pseudostress.  Also, at lower stress
there is more spread in the failure times.  The following sections
will explore several models for the relationship between cycles to
failure and pseudostress.
\end{example}
%----------------------------------------------------------------------
\begin{figure}
\splusbookfigure{\figurehome/super.alloy.data.ps}
\caption{Scatter plot of low-cycle super 
alloy fatigue life versus pseudostress for specimens of a
nickel-base superalloy. Points marked with $\Delta$ are right
censored observations; others are failures.}
\label{figure:super.alloy.data}
\end{figure}
%----------------------------------------------------------------------
\section{Failure-time Regression Models}
\label{section:fail.time.regr.models}
A model with explanatory variables sometimes explains or predicts
why some units fail quickly and other units survive a long time. 
Also, if
important explanatory variables are ignored in an analysis, it is
possible that resulting estimates of quantities of interest (e.g.,
distribution quantiles or failure probabilities) could be seriously
biased.  In reliability studies possible explanatory variables include
\begin{itemize} 
\item 
Continuous variables like stress, temperature, voltage, and pressure.
\item
Discrete variables like number of hardening treatments
or number of simultaneous users of a system.
\item 
Categorical variables like manufacturer, design, and location.
\end{itemize}
The general idea of a regression model is to express the failure time
distribution as a function of $k$ explanatory variables
denoted by $\xvec=(x_{1},\ldots,x_{k})$. For example
\begin{displaymath}
\Pr(T \leq t ;\xvec) = F(\realrv;\xvec) = F(\realrv).
\end{displaymath}
In some cases, to simplify notation, we will
suppress the dependency of $F(\realrv)$ on $\xvec$. Regression 
models can come from physical/chemical theory, curve fitting to
empirical observations, or some combination of theory and empiricism.
In science and engineering, new knowledge and theory are 
often developed and refined through iterative experimentation.

%----------------------------------------------------------------------
\subsection{Parameters as functions of explanatory variables}
An important class of regression models allows one or more of the
elements of the model parameter vector
$\thetavec=(\theta_1,\ldots,\theta_{r})$ to be a function of the
explanatory variables.  Generally one employs a function with a
specified form with one or more unknown parameters that need to be
estimated from data.  This is a generalization of statistical
regression analysis in which the most commonly used models have the
mean of the normal distribution depending linearly on a vector $\xvec$
of explanatory variables. For example, if $x_{i}$ is a scalar
explanatory variable for observation $i$, then the normal distribution
mean is
\begin{displaymath}
\mu_{i} = \beta_{0} + \beta_{1}x_{i}.
\end{displaymath}
In this case we think of $x_{i}$ as being a fixed part of
$\data_{i}$. When the $x_{i}$ values are themselves random, standard
statistical methods and models provide inferences that are conditional
on the fixed, observed values of these explanatory variables. Then the
unknown regression model coefficients ($\beta_{0}$ and $\beta_{1})$
replace the model parameter $\mu_{i}$ in a new definition of
$\thetavec$.  In some situations, there may be more than one model
parameter that will depend on one or more explanatory variables.

%----------------------------------------------------------------------
\subsection{The scale-accelerated failure time model}
\label{section:saft.model}
%-------------------------------------------------------------------
\begin{figure}
\splusbookfigure{\figurehome/sft.transform.ps}
\caption{SAFT models illustrating acceleration and deceleration.}
\label{figure:sft.transform.ps}
\end{figure}
%-------------------------------------------------------------------
The Scale-Accelerated Failure-Time (SAFT) model is commonly used
to describe the effect that
explanatory variables $\xvec$ have on time. This model
has a simple time-scaling
Acceleration Factor that is a function of $\xvec$ and is defined by
\begin{equation}
\label{equation:regr.saft}
\rv(\xvec)=\rv(\xvec_{0})/\AF(\xvec), \quad
\AF(\xvec) > 0, \quad \AF(\xvec_{0})=1
\end{equation}
where $\rv(\xvec)$ is the time at conditions $\xvec$ and
$\rv(\xvec_{0})$ is the corresponding time at some ``baseline'' conditions
$\xvec_{0}$.
Some commonly used forms for the time-scale factor
$\AF(\xvec)$ include the following loglinear
relationships (assuring that $\AF(\xvec)>0$):
\begin{itemize}
\item 
For a scalar $x$, $\AF(x)=1/\exp(\beta_{1}x)$ with $x_{0} =  0$.
\item
For a vector $\xvec=( x_{1}, \dots, x_{k} )$,
$\AF(\xvec)=1/\exp(\beta_{1}x_{1}+ \ldots +\beta_{k}x_{k})$ with
$\xvec_{0}  = \zerovec$. 
\end{itemize} 
When $\AF(\xvec) > 1$, the model accelerates time in the sense that
time moves more quickly at $\xvec$ than at $\xvec_{0}$ so that $\rv(\xvec)
< \rv(\xvec_{0})$.  When $0< \AF(\xvec) < 1$, $\rv(\xvec) >
\rv(\xvec_{0})$, and time at $\xvec$ is decelerated relative to time
at $\xvec_{0}$, but we still refer to the model as SAFT.  
Figure~\ref{figure:sft.transform.ps} illustrates that scale failure
time 
transformation functions
are straight lines starting at the origin:
\begin{itemize}
\item
Lines {\em below} the diagonal accelerate time relative to
time at $\xvec_{0}$.
\item
Lines {\em above} the diagonal decelerate time relative to
time at $\xvec_{0}$.
\end{itemize}

%----------------------------------------------------------------------
\begin{figure}
\splusbookfigure{\figurehome/lawless.lognor.aft.ps}
\caption{Weibull probability plot showing that two lognormal 
cdfs with a scale accelerated failure time (SAFT) regression
relationship are equidistant along the log-time axis.}
\label{figure:lawless.lognor.aft.ps}
\end{figure}
%----------------------------------------------------------------------
Figure~\ref{figure:sft.transform.ps} and 
equation~(\ref{equation:regr.saft})
describe the effect that $\AF(\xvec)$ has on time. In
terms of cdfs, with baseline cdf $F(\realrv;\xvec_{0})$, $F(\realrv;
\xvec)=F \left [
\AF(\xvec) \times \realrv ; \xvec_{0} \right ].$ This shows that 
if $\AF(\xvec) \ne 1$, the cdfs $F(\realrv; \xvec)$ and
$F(\realrv;\xvec_{0})$ do not cross each other. In terms of
distribution quantiles,
$\rvquan_{p}(\xvec)=\rvquan_{p}(\xvec_{0})/\AF(\xvec)$.  Then taking
logs, and isolating the term $\log[\AF(\xvec)],$ gives
\begin{equation}
\label{equation:log.saft}
\log[\rvquan_{p}(\xvec_{0})]
 - \log[\rvquan_{p}(\xvec)]=\log[\AF(\xvec)].
\end{equation}
Thus, as shown in Figure~\ref{figure:lawless.lognor.aft.ps}, in a cdf
plot or a probability plot with a log data scale, $F(\realrv, \xvec)$
is a translation of $F(\realrv, \xvec_{0})$ along the $\log(\realrv)$
axis.

SAFT models are relatively simple to interpret.  Also they are often
suggested by physical theory of failure some simple mechanisms (as
shown in Chapter~\ref{chapter:accelerated.test.models}). They do
{\em not}, however, hold universally (as shown in
Sections~\ref{section:prohaz.fail}, \ref{section:gen.time.trans},
and \ref{section:parallel.reaction}).

%----------------------------------------------------------------------
%----------------------------------------------------------------------
\section{Simple Linear Regression Models}
\label{section:simple.linear.reg}
This section describes and illustrates the use of simple linear
regression models (i.e., models with a single explanatory variable)
based on location-scale distributions, including the normal, logistic,
and smallest extreme value.  These models are closely related with the
traditional linear regression model. Maximum likelihood (ML) 
estimation is used instead of least squares, however, to handle 
censored data and nonnormal distributions. 
Because of the relationships described in
Sections~\ref{section:lognormal.distribution.def},
\ref{section:weibull.distribution}, and
\ref{section:loglogistic.distribution.definition}, location-scale models
can be used also to fit log-location-scale distributions (e.g.,
Weibull, lognormal, and loglogistic) regression models.  The methods
developed in this and some of the following sections in this chapter
can be viewed as a generalization of the material in
Chapter~\ref{chapter:parametric.ml.ls}, allowing $\mu$ (and later
$\sigma$) to depend on explanatory variables.  Although it would be
possible to use notation like $\mu(x)$ to make this dependency
explicit, we will keep the notation simple, as the dependency will
be clear from the context of the problem.

We will first illustrate the methods for the lognormal distribution.
The methods apply, however, directly to other log-location-scale
distributions or other distributions that can be transformed into
the location-scale form. Sections~\ref{section:quad.regr} through
\ref{section:gen.time.trans} illustrate various extensions of
the simple regression model. These extensions include quadratic
relationships, models with more than one explanatory variable, a model
for product comparison, and relationships based on more general
time-transformation functions.

%----------------------------------------------------------------------
\subsection{Location-scale regression model and likelihood}
\label{section:ls.regr.model}

With only one explanatory variable, the location-scale simple regression
model (including the normal, logistic, and smallest
extreme value distributions as special cases) is
\begin{equation}
\label{equation:ls.regr.cdf.model}
	\Pr(\grv \leq \grealrv) = F(\grealrv;\mu,\sigma) = F(\grealrv;\beta_{0}, \beta_{1},\sigma) =
\Phi\left ( \frac{ \grealrv -\mu }{\sigma}\right )
\end{equation}
where $\mu=\beta_{0}+\beta_{1} x$ and $\sigma$ does not depend on 
the explanatory variable $x$.
The quantile function for this model
\begin{equation}
\label{equation:ls.reg.quant}
	\grvquan_{p}(x) = \mu+ \Phi^{-1}(p)\sigma=
\beta_{0}+\beta_{1} x + \Phi^{-1}(p)\sigma
\end{equation}
is linear in $x$. As with the 
Chapter~\ref{chapter:parametric.ml.ls} models, choosing
$\Phi$ determines the shape of the distribution for a particular value
of $x$. One uses $\Phi_{\nor}$ for normal, $\Phi_{\logis}$ for logistic, and
$\Phi_{\sev}$ for the smallest extreme value distribution. In this
model $\beta_{0}$ can be interpreted as the value of $\mu$ when $x=0$
(when this has meaning) and $\beta_{1}$ is the change in $\mu$ [or
$\grvquan_{p}(x)$] for a one-unit increase in $x$.

%------------------------------
The likelihood for a sample of $n$ independent observational units with
right censored and exact-failure observations has the form
\begin{eqnarray}
\label{equation:ls.regr.likelihood}
	\like(\beta_{0},\beta_{1}, \sigma)&=& 	\prod_{i=1}^{n}
\like_{i}(\beta_{0},\beta_{1}, \sigma;\data_{i}) \nonumber \\ 	&=&
\prod_{i=1}^{n} 	\left [ \frac{1}{\sigma} \,
\phi	\left(\frac{\grealrv_{i} -
\mu_{i}}{\sigma}\right) 	\right ]^{\delta_{i}} 	\left [1-
\Phi\left( \frac{ \grealrv_{i} 	-\mu_{i} }
{\sigma} \right ) \right ]^{1-\delta_{i}}
\end{eqnarray}
where $\mu_{i}=\beta_{0}+\beta_{1} x_{i} $, $\delta_{i}=1$ for an
exact failure time, and $\delta_{i}=0$ for a right censored
observation. Similar terms could be added for left-censored or
interval-censored data, as described in
Section~\ref{section:likelihood.contributions}.  As in
Chapter~\ref{chapter:parametric.ml.ls}, ML estimates are obtained by
finding the values of $\beta_{0}$, $\beta_{1}$, and $\sigma$ that
maximize (\ref{equation:ls.regr.likelihood}) or, equivalently, the
corresponding log likelihood (preferred for numerical reasons).  With
an underlying normal distribution and complete data, there are simple
closed-form equations to compute the ML estimates. Otherwise, it is
necessary to use numerical optimization methods to maximize the
log likelihood.

%----------------------------------------------------------------------
\subsection{Log-location-scale regression model and likelihood}
\label{section:exp.regr.model}

Following the development in Section~\ref{section:ls.regr.model},
the log-location-scale distribution
simple regression model (including the lognormal, Weibull, and
loglogistic distributions as special cases) is
\begin{equation}
\label{equation:regr.cdf.model}
	\Pr(\rv \leq \realrv) = F(t;\mu,\sigma) = F(t;\beta_{0}, \beta_{1},\sigma) =
\Phi\left [ \frac{ \log(\realrv) -\mu }{\sigma}\right ]
\end{equation}
where $\mu=\beta_{0}+\beta_{1} x$ and $\sigma$ does not depend on $x$.
The quantile function for this model
\begin{equation}
\label{equation:weib.reg.quant}
	\log[\rvquan_{p}(x)] = \grvquan_{p}(x) = \mu+ \Phi^{-1}(p)\sigma=
\beta_{0}+\beta_{1} x + \Phi^{-1}(p)\sigma
\end{equation}
is linear in $x$. Such a relationship between $\rvquan_{p}(x)$ and
$x$ is sometimes known as a ``loglinear relationship.''  Choosing
$\Phi$ determines the shape of the distribution for a particular
value of $x$. One uses $\Phi_{\nor}$ for lognormal, $\Phi_{\logis}$
for loglogistic, and $\Phi_{\sev}$ for the Weibull
distribution. Again, $\beta_{1}$ can be interpreted as the change in
$\mu$ (or in $\log[\rvquan_{p}(x)]$) for a one-unit increase in
$x$. Relatedly, because the response is on the log scale,
$100\beta_{1}$ can be interpreted as the approximate percent
increase in $\rvquan_{p}(x)$ for a one-unit increase in
$x$. Re-expressing the quantile function as
\begin{displaymath}
	\rvquan_{p}(x)=\exp \left [\grvquan_{p}(x) \right ]= 	\exp
\left (\beta_{1} x	\right ) \, \rvquan_{p}(0)
\end{displaymath}
shows that this regression model is a SAFT model with $\AF(x)=1/\exp
(\beta_{1}x)$.

%------------------------------
The likelihood for a combination of $n$ independent exact-failure
and right-censored observations is
\begin{eqnarray}
\label{equation:regr.likelihood}
	\like(\beta_{0},\beta_{1}, \sigma)&=& 	\prod_{i=1}^{n}
\like_{i}(\beta_{0},\beta_{1}, \sigma;\data_{i}) \nonumber \\ 	&=&
\prod_{i=1}^{n} 	\left\{ \frac{1}{\sigma \realrv_{i}} \,
\phi	\left[\frac{ \log(\realrv_{i}) -
\mu_{i}}{\sigma}\right] 	\right\}^{\delta_{i}} 	\left\{1-
\Phi\left[\frac{ \log(\realrv_{i}) 	-\mu_{i} }
{\sigma}\right] \right\}^{1-\delta_{i}}
\end{eqnarray}
where $\mu_{i}=\beta_{0}+\beta_{1} x_{i} $,
$\delta_{i}=1$ for an exact failure time,
and $\delta_{i}=0$ for a right censored observation.
Similar terms could be added
for left-censored or interval-censored data, as described in
Section~\ref{section:likelihood.contributions}. 
As in Chapter~\ref{chapter:parametric.ml.ls}, ML estimates are 
obtained by finding the values of $\beta_{0}$, $\beta_{1}$, 
and $\sigma$ that maximize (\ref{equation:regr.likelihood})
or the corresponding log likelihood. Also, as described in Section~\ref{section:exp.loc.scale.like}, some computer programs omit the $1/t_{i}$ term in the
density part of the likelihood and
therefore caution should be used when comparing values of
log likelihoods computed with different software.

\begin{example}
{\bf Loglinear lognormal regression model for the computer program
execution times.}
\label{example:ct.regr.model}
%----------------------------------------------------------------------
\begin{figure}
\splusbookfigure{\figurehome/comptime.altplot.lnor.ps}
\caption{Computer program execution time versus system load
with fitted lognormal regression model. The lines show ML estimates of
the .1, .5, and .9 lognormal distribution quantiles as a function of
system load.}
\label{figure:comptime.altplot.lnor.ps}
\end{figure}
%----------------------------------------------------------------------
Continuing with Example~\ref{example:comptime.data},
Figure~\ref{figure:comptime.altplot.lnor.ps} shows the lognormal
simple regression model fit to the computer program execution time
data. Table~\ref{table:comptime.mles} summarizes the numerical
results. Fitting this model is equivalent to fitting a simple linear
relationship to the logarithms of the times.  The estimated densities
in Figure~\ref{figure:comptime.altplot.lnor.ps} are normal
distribution densities because they are plotted on a log-time scale.
This loglinear model provides a useful description of the data.
\end{example}

%----------------------------------------------------------------------
\begin{table}
\caption{ML estimates for the computer execution time example.}
\centering\small
\begin{tabular}{crrrr} \hline
\\[-.5ex]
 & & & \multicolumn{2}{c}{Approximate 95\%}\\
&\multicolumn{1}{c}{ML} &Standard  & \multicolumn{2}{c}{Confidence
Intervals}\\ \cline{4-5}
Parameter &  Estimate & \multicolumn{1}{c}{Error} & Lower & Upper \\
\hline 
 $\beta_{0} $ &4.49 &.11 &4.28 &4.71 \\[.7ex] 
$\beta_{1}$ &.290 &.05 &.20 &.38 \\[.7ex] 
$\sigma$ &.312 &.05 &.22 &.44\\
\hline
\\[-1.8ex]
\end{tabular}
\begin{minipage}[t]{4in}
The log likelihood is $\loglike= -89.50$.  The confidence intervals for
$\beta_0$, $\beta_1$, and $\log(\sigma)$ are based on the normal
approximation method.
\end{minipage}
\label{table:comptime.mles}
\end{table} 
%----------------------------------------------------------------------

%----------------------------------------------------------------------
%----------------------------------------------------------------------
\section{Standard Errors and Confidence Intervals for Regression Models}
\label{section:regr.ci}
This section shows how to compute estimates of standard errors and
confidence intervals for parameters and functions of parameters. We
use the computer execution time data and the simple regression model
in Section~\ref{section:simple.linear.reg} as an example. The general
theory is in Appendix Section~\ref{asection:ci.wald}.  Other examples in this
chapter, in the exercises, and in subsequent chapters illustrate how the
general ideas apply directly to more complicated models.  We will
focus primarily on normal-approximation confidence intervals.
Likelihood and bootstrap confidence intervals described in
Chapters~\ref{chapter:parametric.ml.ls} and \ref{chapter:bootstrap}
can also be used for these models.

Standard errors are provided by most computer software packages.
Normal-approximation confidence intervals may or may not be provided,
but are easy to compute given the ML estimates and standard errors.
Although they are superior, likelihood and simulation (bootstrap)
intervals typically require specialized procedures not commonly
available in today's commercial software.

%----------------------------------------------------------------------
\subsection{Standard errors and confidence intervals for parameters}
\label{section:se.and.ci.for.regr.param}
Appendix Section~\ref{asection:ci.wald} describes the general theory for
computing normal-approximation confidence intervals that can be
applied to regression models. These confidence intervals are based on
the large-sample approximate normal distribution of the ML estimators.
These intervals employ an estimate of the variance-covariance matrix
for the ML estimates of the model parameters
$\thetavechat=(\betahat_{0},\betahat_{1},\sigmahat)$. Extending the
ideas presented in Section~\ref{section:param.vcv.mat}, the ``local''
estimate $\vcvmathat_{\thetavechat}$ of $\vcvmat_{\thetavechat}$ is
the inverse of the ``observed'' information matrix, namely
\begin{eqnarray}
\label{equation:regr.local.est.vcv}
\vcvmathat_{\thetavechat} &=&
\left[ 
\begin{array}{lll}
\varhat(\betahat_{0})& \covhat(\betahat_{0},\betahat_{1})& 
	\covhat(\betahat_{0},\sigmahat)\\[.2ex]
\covhat(\betahat_{1},\betahat_{0}) & \varhat(\betahat_{1})&
	\covhat(\betahat_{1}, \sigmahat) \\[.2ex]
\covhat(\sigmahat,\betahat_{0})& 
	\covhat(\sigmahat,\betahat_{1})& \varhat(\sigmahat)
\end{array}
\right]\\[1ex]
&=&
\left[ 
\begin{array}{lll}
-\frac{\partial ^{2} \loglike(\beta_{0},\beta_{1},\sigma)}
	{\partial\beta_{0}^{2} }&
-\frac{\partial ^{2} \loglike(\beta_{0},\beta_{1},\sigma)}
	{\partial\beta_{0} \partial \beta_{1}}&
-\frac{\partial ^{2} \loglike(\beta_{0},\beta_{1},\sigma)}
	{\partial\beta_{0} \partial \sigma }\\[.2ex]
-\frac{\partial^{2} \loglike(\beta_{0},\beta_{1},\sigma)}
	{\partial \beta_{1} \partial \beta_{0}  }&
-\frac{\partial ^{2} \loglike(\beta_{0},\beta_{1},\sigma)}
	{\partial\beta^{2}_{1}  }&
-\frac{\partial ^{2} \loglike(\beta_{0},\beta_{1},\sigma)}
	{\partial\beta_{1} \partial \sigma }\\[.2ex]
-\frac{\partial ^{2} \loglike(\beta_{0},\beta_{1},\sigma)}
	{\partial \sigma \partial \beta_{0} }&
-\frac{\partial ^{2} \loglike(\beta_{0},\beta_{1},\sigma)}
	{\partial \sigma \partial \beta_{1} }&
-\frac{\partial ^{2} \loglike(\beta_{0},\beta_{1},\sigma)}
	{\partial \sigma^{2} }
\end{array}
\right]^{-1} \nonumber
\end{eqnarray}
where the partial derivatives are evaluated at
$\betahat_{0},\betahat_{1},\sigmahat$.

\begin{example}
\label{example:parvarcovarmatest}
{\bf Parameter variance-covariance matrix estimate for the computer
execution time example.} For the fitted 
computer execution time model, the
estimate of the variance-covariance matrix for the ML estimates
$\thetavechat=(\betahat_{0},\betahat_{1},\sigmahat)$ is
\begin{eqnarray}
\label{equation:ct.local.est.vcv}
\vcvmathat_{\thetavechat} &=&
\left[ 
\begin{array}{rrr}
.012     &-.0037 &0\\
-.0037 &.0021    &0\\
0&0&.0029
\end{array}
\right].
\end{eqnarray}
These quantities will be used in subsequent numerical examples.
\end{example}

Normal-approximation confidence intervals for regression model
parameters can be computed using the methods described in
Section~\ref{section:ci.ls.param}.

\begin{example}
\label{example:normal.theory.ci.for.ct.beta1}
{\bf Normal-approximation confidence interval for the regression
slope in the computer execution time example.} For the computer time
data, $\sehat_{\betahat_{1}}=\sqrt{.0021}=.046$ [see
(\ref{equation:ct.local.est.vcv}) and
Table~\ref{table:comptime.mles}] and
\begin{displaymath}
 [\undertilde{\beta_{1}},\quad \tilde{\beta_{1}}] =
 .2907 \pm  1.960 \times .0459  = [.20, \quad .38]
%splus   .2907 + 1.96* (.0459) = .380664
%splus   .2907 -  1.96* (.0459) = .200736
\end{displaymath}
provides an approximate 95\% confidence interval for $\beta_{1}$.
This interval is based on the approximation
$Z_{\betahat_{1}}=(\betahat_{1}-\beta_{1})/\sehat_{\betahat_{1}}
\approxdist
\NOR(0,1)$.
This interval implies that we are 95\% confident that a one-unit
increase in load will increase quantiles of the running time
distribution by an amount between 20\% and 38\%.
\end{example}

As described in Section~\ref{section:ci.ls.param}, the log
transformation generally improves normal-approximation confidence
intervals for positive parameters like $\sigma$.  In particular,
using (\ref{equation:normal.theory.ci.for.sigma}) provides an
approximate $100(1-\alpha)$\% confidence interval for $\sigma$ based
on the assumption that
$Z_{\log(\sigmahat)}=[\log(\sigmahat)-\log(\sigma)]/\sehat_{\log(\sigmahat)}$
can be approximated by a $\NOR(0,1)$ distribution.

\begin{example}
{\bf Normal-approximation confidence intervals
for  $\sigma$ in the computer execution time example.}
\label{example:normal.theory.ci.for.ct.sigma}
For the computer execution time example 
$\sehat_{\sigmahat}=\sqrt{.0029}=.054$. An approximate 95\% confidence
interval for $\sigma$ is
\begin{displaymath}
[ \undertilde{\sigma}, \quad \tilde{\sigma}] =
[.3125/1.400, \quad
.3125 \times 1.400]  = [ .22    , \quad  .44  ]
% splus  .3125/exp(1.96*(.05359)/.3125)  =  .2232937
% splus  .3125*exp(1.96*(.05359)/.3125)  =  .4373444
\end{displaymath}
where $w=\exp[1.96(.05359)/.3125]=1.400$.
\end{example}

%----------------------------------------------------------------------
\subsection{Standard errors and confidence intervals for 
distribution quantities at specific explanatory variable conditions}
\label{section:se.and.ci.for.regr.funct}

Appendix Section~\ref{asection:ci.wald} describes the general theory for
computing standard errors and normal-approximation confidence
intervals for functions of model parameters. This general theory can
be used to estimate distribution quantities (such as quantiles,
failure probabilities, or hazard function values) at specified levels
of the explanatory variables.  This section illustrates an equivalent
two-stage approach that will simplify the presentation of the method
(and computations). The presentation and examples are for simple
regression models with log-location-scale distributions, but the ideas
apply more generally.

As expressed in (\ref{equation:regr.cdf.model}) and
(\ref{equation:weib.reg.quant}), there are unknown values of $\mu$ and
$\sigma$ at each level of $x$ (or combination of levels of $\xvec$ in
the case of multiple explanatory variables). For specified values of
the explanatory variables, one can compute the ML estimates $(\muhat,
\sigmahat)$ of these parameters and an estimate of the corresponding
variance-covariance matrix. Then the methods given in
Sections~\ref{section:ci.ls.param} and
\ref{section:norm.conf.int.for.ls.fun} can
be applied directly to compute standard errors and
confidence intervals for quantities of interest at specified values of
the explanatory variable(s). 

The general formulas needed to compute an estimate of the
variance-covariance matrix of $(\muhat, \sigmahat)$ are
(\ref{equation:variance.approximation}) and
(\ref{equation:covariance.approximation}) in
Appendix Section~\ref{asection:delta.method}.  For the simple linear
regression model, at a particular value of $x$, $\muhat =
\betahat_{0}+
\betahat_{1}x$, $\sigma$ does not depend on $x$, and using a special
case of (\ref{equation:gcovariance.est}),
\begin{equation}
\label{equation:ls.regr.local.est.vcv}
\vcvmathat_{\muhat,\sigmahat}=
\left[ 
\begin{array}{ll}
\varhat(\muhat)& \covhat(\muhat,\sigmahat)\\
\covhat(\muhat,\sigmahat)& \varhat(\sigmahat)
\end{array}
\right]
\end{equation}
is obtained from $\varhat(\muhat)=\varhat(\betahat_{0})+2x
\covhat(\betahat_{1},\betahat_{0})+ x^{2} \varhat(\betahat_{1}) $ and
$\covhat(\muhat,\sigmahat)=
\covhat(\betahat_{0},\sigmahat) + x \covhat(\betahat_{1}, \sigmahat)$.
Then standard errors for functions of $(\muhat, \sigmahat)$ can be
computed by using the delta method, as in
(\ref{equation:se.for.function}).  The following examples show how the
the general approach, given in
Section~\ref{section:norm.conf.int.for.ls.fun}, can be used to find
confidence intervals for a function of regression model parameters.

\begin{example}
{\bf Confidence interval for $\rvquan_{p}(x)$ in the computer
execution time example.}
\label{example:regr.ci.for.quan}
Refer to Figure~\ref{figure:comptime.altplot.lnor.ps} and the results
from Example~\ref{example:ct.regr.model}.  For scheduling purposes, it
was necessary to estimate and obtain a 95\% confidence interval for
$\rvquan_{.9}(5)$, the time at which 90\% of jobs running at a system
load of 5 will have finished. From Table~\ref{table:comptime.mles},
\begin{eqnarray*}
\muhat & = & \betahat_{0} + \betahat_{1}x
	=  4.494 + .2907 \times 5  = 5.947 \\[1ex]
\rvquanhat_{.9}(5)  &=  & \exp(\muhat + \Phi_{\nor}^{-1}(p) \sigmahat)
		  = \exp( 5.947 + 1.2816 \times .3125) = 571.2.
\end{eqnarray*}
%splus  exp( 5.94732 + 1.2816 * .3124667 ) = 571.2217
%splus
From
(\ref{equation:ct.local.est.vcv}) 
computing the elements in  (\ref{equation:ls.regr.local.est.vcv}) gives
\begin{displaymath}
\vcvmathat_{\muhat,\sigmahat}=
\left[ 
\begin{array}{ll}
\varhat(\muhat)& \covhat(\muhat,\sigmahat)\\
\covhat(\muhat,\sigmahat)& \varhat(\sigmahat)
\end{array}
\right]=
\left[ 
\begin{array}{lr}
 .0277 & 0 \\
 0 & .00287
\end{array}
\right].
\end{displaymath}
Substituting into
(\ref{equation:ls.quant.sehat}) gives
\begin{displaymath}
\sehat_{\rvquanhat_{.9}(5)}=  571.22 [ .0277 +
	2 \times (1.2816) \times 0 + (1.2816)^{2} \times .00287]^{\frac{1}{2}}=102.9.
\end{displaymath}
Then an approximate  95\% confidence
interval for $\rvquan_{.9}(5)$ based on $Z_{\log[\rvquanhat_{.9}(5)]}
\approxdist \NOR(0,1)$ is obtained by 
substituting into (\ref{equation:norm.approx.for.quant}) giving
%splus
%splus
%splus   571.22 * sqrt(.027736 + 2*(1.2816)*0+(1.2816)^2 *.00287)= 102.8989
%splus
%splus
\begin{displaymath}
[ \undertilde{\rvquan_{.9}}, \quad \tilde{\rvquan}_{.9} ]=
[571.2/1.423, \quad
571.2 \times 1.423] = [401, \quad  813]
\end{displaymath}
where $w=\exp[1.96 \times 102.9 / 571.2] = 1.423$.
%splus
%splus  exp(1.96 *102.9 / 571.2) = 1.423457
%splus  571.2 * 1.4235
%splus  571.2 / 1.4235
%splus
Thus we are 95\% confident that when the system load is 5,
$\rvquan_{.9}$ lies between 401 and  813 seconds.
\end{example}
Normal-approximation confidence intervals for other quantities (like
failure probabilities or hazard values) at this or other levels of
system load can be found similarly using the methods of
Section~\ref{section:norm.conf.int.for.ls.fun}.

More accurate confidence intervals for regression models can be based
on the likelihood ratio approach used in
Chapter~\ref{chapter:parametric.ml.ls} or on the bootstrap methods
described in Chapter~\ref{chapter:bootstrap}. Implementation here is a
straightforward extension of the methods presented there and is not 
described in detail here.

%----------------------------------------------------------------------
\section{Regression Model with Quadratic-$\mu$ and Nonconstant-$\sigma$}
\label{section:quad.regr}
This section describes and illustrates the use of 
regression models that extend the simple linear model in
Section~\ref{section:simple.linear.reg}.

%----------------------------------------------------------------------
\subsection{Quadratic regression relationship
for $\mu$ and a constant spread parameter}

Consider
the log-quadratic relationship that uses
(\ref{equation:regr.cdf.model}) with $\mu=\beta_{0}+\beta_{1}
x+\beta_{2} x^2$ and $\sigma$ does not depend on $x$. Then
\begin{displaymath}
\log[\rvquan_{p}(x)] =\grvquan_{p}(x)=\beta_{0}+\beta_{1} x + \beta_{2} x^2 + \Phi^{-1}(p)\sigma
\end{displaymath}
is quadratic in $x$. The quantile function can be written as
\begin{displaymath}
\rvquan_{p}(x)=\exp \left [\grvquan_{p}(x) \right ]= \exp(\beta_{1} x+\beta_{2} x^2)
\, \rvquan_{p}(0).
\end{displaymath}
Thus this is a SAFT model with $\AF(x)=1/\exp(\beta_{1}
x+\beta_{2} x^2)$.  Substituting $\mu_{i} = \beta_{0} + \beta_{1}
x_{i} + \beta_{2} x_{i}^{2}$ into (\ref{equation:regr.likelihood})
gives a likelihood having the form $\like(\beta_{0},\beta_{1},
\beta_{2},\sigma)$.  In the Weibull model $\beta=1/\sigma$ is the
Weibull shape parameter. On the log-time scale (corresponding to the
smallest extreme value distribution), $\sigma$ is a scale parameter.
Generally we will refer to $\sigma$ as a ``spread'' parameter.

%------------------------------
\begin{example}
{\bf Model 1: log-quadratic Weibull
regression model for superalloy fatigue data.} 
\label{example:super.alloy.nf20}
%----------------------------------------------------------------------
\begin{figure}
\splusbookfigure{\figurehome/nf.altplot.nf20.weib.ps}
\caption{Superalloy fatigue data with fitted 
log-quadratic Weibull regression model 
with constant $\sigma$, plotted on log-log axes.}
\label{figure:nf.altplot.nf20.weib.ps}
\end{figure}
%----------------------------------------------------------------------
Continuing from Example~\ref{example:super.alloy.data}, we will fit a
curvilinear regression model to the superalloy fatigue data.
Figure~\ref{figure:nf.altplot.nf20.weib.ps} shows the fitted
log-quadratic Weibull regression model with constant $\sigma$, fit to
the superalloy fatigue data with $x=\log(\mbox{pseudostress})$. This
model provides a reasonable fit to the data, but there is, even on the
log scale, some evidence that the spread in the data is greater at
lower levels of stress.  The estimated densities in
Figure~\ref{figure:nf.altplot.nf20.weib.ps} are smallest extreme value
densities because they are plotted on a log-cycles scale.
Table~\ref{table:super.alloy.mles} summarizes the numerical results of
this and another model.  The quadratic model has to be used with
caution.  As can be visualized in
Figure~\ref{figure:nf.altplot.nf20.weib.ps}, extrapolation to levels
of pseudostress beyond 160 ksi would lead to nonsensical estimates of
longer life.  Section~\ref{section:fatigue.limit.model}
suggests an alternative model for these data.
%----------------------------------------------------------------------
\end{example}

%----------------------------------------------------------------------
\begin{table}
\caption{ML estimates for life versus
stress Weibull regression relationships for nickel-base superalloy
specimens.}
\centering\small
\begin{tabular}{lcrrrr} 
\\[-.5ex]
\hline
 & & & & \multicolumn{2}{c}{Approximate 95\%} \\
 & &\multicolumn{1}{c}{ML} &Standard  & \multicolumn{2}{c}{Confidence Intervals}\\  \cline{5-6}
& Parameter & Estimate& \multicolumn{1}{c}{Error} & Lower & Upper \\
\hline 
Model 1 &
 $\beta_{0}$ &217.61 &62.1 &95.9 & 339.3\\[.7ex] 
 &$\beta_{1}$ &$-85.52$ &26.53 &$-137.5$ &$-33.53$ \\[.7ex] 
 &$\beta_{2}$ &8.48 &2.83 &2.93 & 14.03\\[.7ex] 
 &$\sigma$ &.375 &.067 &.26  &.53 \\[1.2ex]
\hline 
\\[-1.8ex]
Model 2 &
$\beta_{0}^{[\mu]}$ &243.2 &58.12 &129.3 &357.1 \\[.7ex] 
 &$\beta_{1}^{[\mu]}$ &$ -96.54$ &24.73 &-145.0 &-48.07 \\[.7ex] 
 &$\beta_{2}^{[\mu]}$ &  9.67 &2.63 &4.52 &14.8 \\[.7ex] 
 &$\beta_{0}^{[\sigma]}$ &4.47 &4.17 &-3.71 & 12.6 \\[.7ex]
 &$\beta_{1}^{[\sigma]}$ &$-1.18$ &.89 &$-2.93$ &.58 \\[.7ex]
\hline
\\[-1.8ex]
\end{tabular}
\begin{minipage}[t]{4in}
The log likelihoods for Model 1 and Model 2 are, respectively,
$\loglike_{1}=-93.38$ and $\loglike_{2}=-92.58$.
\end{minipage}
\label{table:super.alloy.mles}
\end{table}
%----------------------------------------------------------------------

%----------------------------------------------------------------------
\subsection{Quadratic regression model with nonconstant spread
parameter ($\sigma$)}

This section illustrates the fitting of a regression model in which both
$\mu$ and $\sigma$ depend on an explanatory variable. The model is given
by (\ref{equation:regr.cdf.model}) with
$\mu=\beta_{0}^{[\mu]}+\beta_{1}^{[\mu]} x+\beta_{2}^{[\mu]}
x^2$ and $\log(\sigma)=\beta_{0}^{[\sigma]}+\beta_{1}^{[\sigma]} x$.
The log-quantile function for this model is
\begin{equation}
\label{equation:quad.mu.lin.sigma.model}
	\log[\rvquan_{p}(x)] = \grvquan_{p}(x)=\beta_{0}^{[\mu]}+\beta_{1}^{[\mu]} x +
\beta_{2}^{[\mu]} x^2 +
\Phi^{-1}(p)\exp \left( \beta_{0}^{[\sigma]}+ \beta_{1}^{[\sigma]} x
\right )
\end{equation}
which is {\em not} quadratic in $x$. Also because
$\rvquan_{p}(x)/\rvquan_{p}(0)=\exp[\grvquan_{p}(x)-\grvquan_{p}(0)]$
depends on $p$, this model is {\em not} a SAFT model.
Substituting $\mu_{i}=\beta_{0}^{[\mu]}+
\beta_{1}^{[\mu]} x_{i}+\beta_{2}^{[\mu]} x_{i}^{2}$
and $\sigma_{i}=\exp \left (\beta_{0}^{[\sigma]}+\beta_{1}^{[\sigma]}
x_{i}
\right )$ into (\ref{equation:regr.likelihood}) 
gives a likelihood having the form
$\like(\beta_{0}^{[\mu]},\beta_{1}^{[\mu]},\beta_{2}^{[\mu]},
\beta_{0}^{[\sigma]},\beta_{1}^{[\sigma]}).$

%------------------------------
\begin{example}
{\bf Model 2: Weibull log-quadratic
regression model with nonconstant $\sigma$ for the superalloy fatigue data.} 
\label{example:super.alloy.nf21}
%----------------------------------------------------------------------
\begin{figure}
\splusbookfigure{\figurehome/nf.altplot.nf21.weib.ps}
\caption{Superalloy fatigue 
data with fitted log-quadratic Weibull regression model
with nonconstant $\sigma$.}
\label{figure:nf.altplot.nf21.weib.ps}
\end{figure}
%----------------------------------------------------------------------
Continuing from Examples~\ref{example:super.alloy.data} and
\ref{example:super.alloy.nf20},
Figure~\ref{figure:nf.altplot.nf21.weib.ps} shows the log-quadratic
Weibull regression model with nonconstant $\sigma$ fit to the super
alloy fatigue data.  The numerical results are summarized in
Table~\ref{table:super.alloy.mles}.  This model seems to provide a
reasonably good fit to these data. At first sight, the failure at 
13,949 cycles with
pseudostress equal to 85.2 ksi appears to be an outlier.  Relative to the
long lower tail of the smallest extreme distribution, however, the
observation is not surprisingly extreme (see
Exercise~\ref{exercise:outlier.test}).
%splus > 2*(-92.58-(-93.38))
Comparing Model 1 and Model 2 [e.g.,
$-2(\loglike_{1}-\loglike_{2})=1.6 < \chisquare_{(.90;1)} = 2.71$]
indicates that the evidence for nonconstant $\sigma$ in the data is
not strong. On the other hand, having $\sigma$ decrease with stress
or strain is typical in fatigue data and this is what we see in data
points plotted in Figure~\ref{figure:nf.altplot.nf20.weib.ps}.
Thus, it would be reasonable to use a model with decreasing $\sigma$
in this case, even
in the absence of ``statistical significance.''
\end{example}

%----------------------------------------------------------------------
\subsection{Further comments on the use of empirical regression models}
As described in Example~\ref{example:super.alloy.nf20}, the quadratic
relationship should not be used for extrapolation.  The 
fatigue-data models used in
Examples~\ref{example:super.alloy.nf20} and
\ref{example:super.alloy.nf21} are purely empirical, without any
physical basis. In general this is true of other quadratic and higher
order polynomial relationships. Such relationships 
can be useful, providing a smooth
curve to describe a population or a process {\em within the range of
the data}, but should not be used to extrapolate outside the
range of one's data.

The nonconstant-$\sigma$ model also has a potential extrapolation
pitfall (even if the relationship for $\mu$ is linear).  Refer to
Figure~\ref{figure:nf.altplot.nf21.weib.ps}.  Because of the
potentially longer lower tail of the distribution at low levels of
stress, depending on the values of the parameters of the model, it is
possible to have the lower-tail quantile of the cycles to failure
distribution decrease as one moves to lower levels of pseudostress,
predicting shorter life at lower stress, leading physically
nonsensical extrapolations (see
Exercise~\ref{exercise:reverse.quant}).
Section~\ref{section:fatigue.limit.model} suggests an alternative
model for these data.

%----------------------------------------------------------------------
\subsection{Comment on numerical methods}
\label{section:numerical.methods}
For certain model/data combinations, the shape of the likelihood can
be such that the commonly used optimization software will find it
difficult or impossible to find the maximum of the function.  This
is a problem that is analogous to the numerical problem of finding
regression coefficients when there is a strong degree of linear
dependence (multicollinearity) among the explanatory variables.
High quality software for least squares regression does not use the
standard textbook formulas based on matrix inversion.  Similarly,
software developers for ML iterations need to consider carefully the
numerical analysis aspects of the calculations to be
programmed. Good software will deal with these numerical issues in a
manner that is transparent to the user.  That is, when a
reparameterization or reformulation is used, this need not be
brought to the user's attention (but such information
should, perhaps, be available as
an option).

The superalloy regression models are a case in point. Nelson~(1984)
centers the $x$ variables, which alleviates the collinearity problem
in polynomial regression. If, however, the different explanatory
variables are close to being linearly dependent (multicollinear) or
if a model without an intercept term is required, then a different
approach is needed.  Escobar and Meeker~(1998c) describe an
appropriate reparameterization for multiple linear regression with
censored data. In general having robust ML iterations (i.e., having
a high probability of finding the maximum, given that one exists)
requires
\begin{itemize}
\item
A parameterization that makes the likelihood have a shape that
is not too different from a quadratic with major axes corresponding to
the transformed model parameters (resulting in transformed parameter estimates
that are approximately uncorrelated), and
\item
Starting values that are not too far from the maximum.
\end{itemize}
For further discussion of numerical methods for ML, see Section~8.6 of
Nelson~(1982), Ross~(1990), and Escobar and Meeker~(1998c).

%----------------------------------------------------------------------
%----------------------------------------------------------------------
\section{Checking Model Assumptions}
\label{section:censored.data.regression.diagnostics}

An important part of any statistical analysis is diagnostic checking
for departures from model assumptions. In conducting a failure time
regression analysis we recommend the use of graphical methods, using
generalizations of usual regression diagnostics (including residual
analysis). These diagnostic
methods can be used in a manner that is similar to
their use in ordinary regression analysis, except that interpretation
is often complicated by the censoring. The analysis can also be
complicated when fitting underlying nonnormal distributions.
%----------------------------------------------------------------------
%----------------------------------------------------------------------
\subsection{Definition of residuals}
\label{section:resid.def}
Consider a set of independent (possibly censored) observations
$\grealrv_{i}$, $i=1,\dots,n$ from location-scale distributions with
cdfs $\Phi[(y-\mu_{i})/\sigma_{i}]$, where $\mu_{i}$ and
$\sigma_{i}$ may be functions of regression parameters and
explanatory variables $\xvec_{i}$. This would include data from
normal, logistic, or smallest extreme value regression models. A
natural and commonly used definition of standardized residuals for
this model is
\begin{equation}
\label{equation:estimated.residual.definition}
\epsilonhat_{i}=
\frac{\grealrv_{i} - \muhat_{i}}{\sigmahat_{i}} 
\end{equation}
where $\muhat_{i}$ and $\sigmahat_{i}$ are the ML estimates of
$\mu_{i}$ and $\sigma_{i}$.  Then, under the assumed regression
model, these residuals should look like a (possibly censored) random
sample from a standardized (i.e., $\mu=0$ and $\sigma=1$)
location-scale distribution (e.g., standard normal, smallest extreme
value, or logistic).  When $y_{i}$ is a censored observation, the
corresponding residual is also censored. For example, if $y_{i}$ is
a right-censored observation, the corresponding $\epsilonhat_{i}$ is
also right censored (we only know that the actual residual would
have been larger than the censored residual).

For regression data from log-location-scale distributions with cdfs
$\Phi\{[\log(\realrv)-\mu_{i}]/\sigma_{i}\}$ (e.g., regression with
the lognormal, loglogistic, or Weibull data) a natural extension of
the definition of standardized residuals is
\begin{equation}
\label{equation:exp.estimated.residual.definition}
\epsilonhat_{i} = \exp \left [\frac{\log(\realrv_{i}) -
\muhat_{i}}{\sigmahat_{i}}\right]  = \left[ \frac{\realrv_{i}}
{\exp(\muhat_{i})}\right]^{1/\sigmahat_{i}}.
\end{equation}
Again, when $t_{i}$ is a censored observation, the corresponding
residual is also censored.  Under the assumed regression model,
these residuals should look like they came from a standardized
(i.e., $\mu=0$ and $\sigma=1$) log-location-scale distribution.
Adequacy of the fitted distribution can be assessed by making a
probability plot of the (possibly censored) residuals using the
methods in Chapter~\ref{chapter:probability.plotting}. We will refer
to (\ref{equation:estimated.residual.definition}) and
(\ref{equation:exp.estimated.residual.definition}) as ``censored
Cox-Snell'' residuals because they are special cases of general
Cox-Snell~(1968) residuals as applied to censored data. Nelson
(1973) shows how to use such residuals to detect departures from
distributional assumptions and check for other kinds of model
inadequacies.  These ideas are illustrated in
Section~\ref{section:regression.diag}.

Cox-Snell residuals can also be used for checking model assumptions
for non-location-scale distributions. The general definition of
Cox-Snell residuals is as follows.  Consider observed failure times
$\realrv_{i}$ corresponding to random variables $\rv_{i}$
($i=1,\dots,n$). Suppose that the $T_{i}$ are
functions of explanatory variables $\xvec_{i}$, model parameters
$\thetavec$, and a set of iid random deviations $\epsilon_{i}$ ($i=1,\dots, n$)
having a distribution with no unknown parameters. This function
defines the assumed model for $T$. Then if there is a function
$w_{i}(\rv;
\xvec,\thetavec)$ such that
\begin{displaymath}
\epsilon_{i} = w_{i}(\rv_{i}; \xvec_{i},\thetavec)
\end{displaymath}
the Cox-Snell residuals are defined as and can be computed from
\begin{displaymath}
\epsilonhat_{i}= w(\realrv_{i}; \xvec_{i},\thetavechat).
\end{displaymath}
Here $\thetavechat$ is the ML estimate of $\thetavec$
obtained using the assumed model, 
the data $(\realrv_{i}, \xvec_{i})$ ($i=1,\dots,n$), 
and the censoring information.
For example, for the lognormal loglinear regression
a choice is 
$w_{i}(\realrv;\xvec_{i}, \thetavec)=
\exp \left \{[\log(\realrv)-\mu_{i}]/\sigma_{i}\right \}$,
which yields the residuals in
(\ref{equation:exp.estimated.residual.definition}).

For a given model and data set, Cox-Snell residuals are not uniquely
defined because any parameter-free one-to-one transformations of the
$w_{i}(\realrv; \xvec_{i}; \thetavechat)$ values would also satisfy
the Cox-Snell definition [with a corresponding change in the
definition of the deviations $\epsilon_{i}$ ($i=1,\dots, n$) in the
assumed model].  For example, for the lognormal regression above,
one can choose $w_{i}(\realrv;\xvec_{i},
\thetavec)= [\log(\realrv)-\mu_{i}]/\sigma_{i}$ which provides
Cox-Snell residuals that are unrestricted in sign in contrast with
the residuals in (\ref{equation:exp.estimated.residual.definition})
which are all positive.

In general, when the observations are independent (possibly
censored) and the $\rv_{i}$ have strictly increasing cdfs
$F(\realrv; \xvec_{i}, \thetavec)$, a natural choice is
$w_{i}(\realrv;\xvec_{i},\thetavec)= F(\realrv; \xvec_{i},
\thetavec)$ which provides Cox-Snell residuals on $(0,1)$ given by
\begin{equation}
\label{equation:general.cox.snell}
\uhat_{i}=F(\realrv_{i};\xvec_{i}, \thetavechat).
\end{equation}
Because
$F(\rv_{i};\xvec_{i}, \thetavec)$ has a $\UNIF(0,1)$ distribution (see
Exercise~\ref{exercise:F.of.T.dist}), these residuals should look
like a (possibly censored) random sample from a $\UNIF(0,1)$
distribution.  For
example, with the log-location-scale regression model above with 
$F(\realrv;\xvec_{i},\thetavec)=
\Phi\{[\log(\realrv)-\mu_{i}]/\sigma_{i}\}$, a set of Cox-Snell residuals
in $(0,1)$  are 
\begin{displaymath}
\uhat_{i}=F(\realrv_{i};\xvec_{i}, \thetavechat)=
\Phi\left[
\frac{
\log(\realrv_{i})-\muhat_{i}
     }
     {
\sigmahat_{i}
     }
\right ].
\end{displaymath}
These residuals are related to the residuals in
(\ref{equation:exp.estimated.residual.definition}) through the
transformation $\epsilonhat_{i}=\exp[\Phi^{-1}(\uhat_{i})]$.  Thus
the information in the $\uhat_{i}$ residuals is equivalent to that
in the $\epsilonhat_{i}$ residuals.

To assess the adequacy of the distributional assumption, one can use
a P-P plot (e.g., page 66 of Crowder et al.~1991) of the residuals
defined in~(\ref{equation:general.cox.snell}).  To do this, obtain
the nonparametric estimate of the cdf of the residuals using the
methods in Chapter~\ref{chapter:nonparametric.estimation}.  Then
obtain probability plotting positions $p_{\uhat}$ at each point
$\uhat$ at which the nonparametric estimate jumps (see
Section~\ref{section:plotting.pos}).  The P-P plot is obtained by
plotting $\{\uhat \quad \mbox{versus} \quad p_{\uhat} \}$ on linear
axes.  Strong departures from linearity in this plot indicate a
departure from the assumed model.  Note that the P-P plot can be
viewed as a probability plot corresponding to a $\UNIF(0,1)$
distribution.

Alternatively, one can transform the (possibly censored) $\uhat$
values to $-\log(1-\uhat)$ which, under the assumed model, should
look like a (possibly censored) sample from an $\EXP(1)$
distribution. In this case, the adequacy of the distributional
assumption can be checked by making an exponential or a Weibull
probability plot of the $-\log(1-\uhat)$ values, using the methods in
Chapter~\ref{chapter:probability.plotting}.  The Weibull probability
plot provides much better resolution in the lower tail of the
distribution.

Lawless~(1982, pages 280-281) describes the use of probability plots
for Cox-Snell residual analysis and Collett~(1994, page 158)
describes closely related Weibull Q-Q plots for Cox-Snell residuals.

 
%----------------------------------------------------------------------
\subsection{Regression diagnostics}
\label{section:regression.diag}
Some suggested regression model diagnostics include
\begin{itemize}
\item
{\bf Plot of standardized residuals versus fitted values.} As
mentioned in Section~\ref{section:resid.def}, fitted values can be
defined in several different ways.  For any of the suggested
definitions, plotting the residuals versus fitted values can help
detect  nonlinearity not modeled in the underlying relationship or
nonconstant variability in life. Heavy censoring can, however,
make such plots difficult to interpret.
\item
{\bf Probability plot of standardized residuals.} When the data at
different levels of the explanatory variables are censored at a common
censoring time, it is possible that the computed residuals will be
multiply censored. In such cases, one can use the methods
in Chapters~\ref{chapter:nonparametric.estimation} and
\ref{chapter:probability.plotting} to produce the probability plot.

\item
{\bf Other residual plots.} Residuals can be plotted in a variety of
other ways. For example, one might plot residuals against other
potential explanatory variables not in the model
to see if they provide any explanatory
power. If data were collected sequentially over time, then plotting
residuals versus observation order can help to detect process trends
and cycles.
\item
{\bf Influence (or sensitivity) analysis.} It is important to
assess the degree to which estimates depend on model assumptions and
other uncertain inputs to the data analysis process. In some cases, simple
reanalysis under alternative model assumptions will suffice.  It is,
however, possible to systematize the process by, for example, dropping
out one observation at a time and refitting the model to detect highly
influential observations.
\end{itemize}
Additionally, most analytical tests commonly used to detect departures
from an assumed model can be suitably generalized, at least
approximately, for censored data (especially using likelihood ratio
tests).

%----------------------------------------------------------------------
\begin{figure}
\splusbookfigure{\figurehome/nf.fitvsres.nf11.weib.ps}
\caption{Plot of standardized residuals
versus fitted values for the Weibull regression model with
log-quadratic $\mu$ and constant $\sigma$,
fit to the superalloy data on log-log axes.}
\label{figure:nf.fitvsres.nf11.weib.ps}
\end{figure}
%----------------------------------------------------------------------
%----------------------------------------------------------------------
\begin{figure}
\splusbookfigure{\figurehome/nf.residual.nf11.weib.ps}
\caption{Weibull probability plot of the standardized residuals
from the Weibull regression model with log-quadratic $\mu$ and
constant $\sigma$, fit to the superalloy data.}
\label{figure:nf.residual.nf11.weib.ps}
\end{figure}
%----------------------------------------------------------------------

\begin{example}
{\bf Residual plots for Model 1 and the superalloy fatigue data.}
Figure~\ref{figure:nf.fitvsres.nf11.weib.ps} shows the standardized
residuals from (\ref{equation:exp.estimated.residual.definition}) versus
the fitted values for Model 1. This figure gives a strong indication
that there is more variability among the observations with larger
fitted values.  Figure~\ref{figure:nf.residual.nf11.weib.ps} is a
Weibull probability plot of the standardized residuals, revealing
the early outlying observation. Similar plots for Model 2 (not shown
here) are better behaved, but still show the early outlying
observation.
\end{example}

%----------------------------------------------------------------------
%----------------------------------------------------------------------
\section{Models with Two or More Explanatory Variables}
\label{section:multiple.reg.models}
This section illustrates the use of regression models having two
different explanatory variables. The approach generalizes 
to problems involving more than two explanatory variables.

%----------------------------------------------------------------------
\subsection{Model-free graphical analysis of two-variable regression data.}

As in simple regression, it is useful to view two-variable
regression data graphically before making assumptions about the
relationship between life and the explanatory variables.
Probability plots and scatter-plots are useful tools for doing this.

%----------------------------------------------------------------------
\begin{figure}
\splusbookfigure{\figurehome/zelencap.scatter.ps}
\caption{Scatter-plot of glass capacitor life test data.}
\label{figure:zelencap.scatter.ps}
\end{figure}
%----------------------------------------------------------------------
%----------------------------------------------------------------------
\begin{figure}
\splusbookfigure{\figurehome/zelencap.groupi.ps}
\caption{Weibull probability plots of glass capacitor life data
at each temperature and voltage condition, each with
its Weibull ML estimate of $F(t)$ plotted as a straight line.}
\label{figure:zelencap.groupi.ps}
\end{figure}
%----------------------------------------------------------------------
%----------------------------------------------------------------------
\begin{table}
\caption{Glass capacitor life test failure times
and Weibull ML estimates for each temperature/voltage combination.}
\centering\small
\begin{tabular}{l  l  rrrr  } 
\\[-.5ex]
\hline
&& \multicolumn{4}{c}{Applied Voltage}\\
\cline{3-6}
Temperature && 200 & 250 & 300 & 350 \\
\hline 
               && 439& 572& 315& 258 \\
        &\multicolumn{1}{c}{Hours} & 904& 690& 315& 258 \\
               &\multicolumn{1}{c}{to Failure} &1092& 904& 439& 347 \\
170$\degreesc$ &&1105&1090& 628& 588 \\
\cline{2-6}
\\[-1.8ex]
      &$\muhat$                   &7.13    &7.10    &6.57    &6.54  \\
      &$\sehat_{\muhat}$          &.15     &.16     &.21     & .26 \\
      &$\sigmahat$                &.263    &.279    &.372    & .46 \\
      &$\sehat_{\sigmahat}$       &.13     &.13     &.17     & .21 \\
      &$\max\loglike(\mu,\sigma)$ &$-31.8$ &$-31.7$ &$-30.2$ &$-30.3$  \\
\hline
\hline
\\[-1.8ex]
               && 959& 216& 241& 241 \\
                &\multicolumn{1}{c}{Hours} & 1065&315 & 315 & 241 \\
               &\multicolumn{1}{c}{to Failure} & 1065& 455& 332& 435 \\
180$\degreesc$ && 1087 & 473& 380& 455 \\
\cline{2-6}
\\[-1.8ex]
      &$\muhat$                   &7.01    &6.28     &6.00    &6.24  \\
      &$\sehat_{\muhat}$          &.21     &.16      &.09     & .17 \\
      &$\sigmahat$                &.04     &.28      &.17     & .30 \\
      &$\sehat_{\sigmahat}$       &.02     &.13      &.08     & .14 \\
      &$\max\loglike(\mu,\sigma)$ &$-24.8$ &$-28.4$  &$-26.0$ &$-28.4$  \\
\hline
\hline
\\[-1.8ex]
\end{tabular}\\
\begin{minipage}[t]{4in}
There were eight capacitors tested at each combination of temperature
and voltage. Testing at each combination was terminated after the
fourth failure, yielding failure (Type~II) censored data. The data
are from Zelen~(1959).
\end{minipage}
\label{table:zelen.cap.data}
\end{table}
%splus
%splusParameter Estimation Results 
%splus             MLE         se    z-ratio  95% lower   95% upper 
%splus   b0  1.9219862 5.74022806  0.3348275 -9.3288608  13.1728332
%splus   b1  0.5357155 0.21815297  2.4556872  0.1081356   0.9632953
%splus   b2 -1.6233388 0.27930177 -5.8121322 -2.1707703  -1.0759073
%splussigma  0.3553966 0.05416938  6.5608396  0.2492246   0.4615686

\begin{example}
\label{example:zelen.data}
{\bf The effect of voltage and temperature on capacitor life.}
Table~\ref{table:zelen.cap.data} gives data from a factorial
experiment on the life of glass capacitors as a function of
voltage and operating temperature.  The data were originally analyzed
in Zelen~(1959), using a two-parameter exponential distribution.
Figure~\ref{figure:zelencap.scatter.ps} is a scatter-plot of life in
hours versus voltage, with different symbols for different
temperatures. A small amount of ``jitter'' was introduced into the
voltage variable before plotting so that the graph would separate 
the tied values.  Figure~\ref{figure:zelencap.groupi.ps}, using 
methods from Chapters~\ref{chapter:probability.plotting} and
\ref{chapter:parametric.ml.ls}, contains 
Weibull probability plots for each of
the 8 individual test conditions. The straight lines on these plots
are the ML estimates of Weibull cdfs.
Table~\ref{table:zelen.cap.data} summarizes numerical results.
\end{example}
%----------------------------------------------------------------------
\subsection{Two-variable regression model without interaction}

The log-location-scale two-variable regression model uses
(\ref{equation:regr.cdf.model}) with $\mu=\beta_{0}+\beta_{1} x_{1} +
\beta_{2} x_{2}$. When $\sigma$ does not depend on
$\xvec=(x_{1},x_{2})$,
the $p$ quantile of the life distribution at a specified $\xvec$ is 
\begin{equation}
\label{equation:two.var.regr.model}
\log[\rvquan_{p}(\xvec)] =\grvquan_{p}(\xvec)=\beta_{0}+
	\beta_{1} x_{1} + \beta_{2} x_{2} + \Phi^{-1}(p) \sigma.
\end{equation}
Re-expressing the quantile function gives
$\rvquan_{p}(\xvec)=\exp \left [\grvquan_{p}(\xvec) \right ]=
\exp(\beta_{1} x_{1} + \beta_{2} x_{2}) \, \rvquan_{p}(\zerovec)$
showing that this is a SAFT model with $\AF(\xvec)=1/\exp(\beta_{1} x_{1} +
\beta_{2} x_{2})$.  Substituting $\mu_{i} = \beta_{0} + \beta_{1}
x_{1i} +
\beta_{2} x_{2i}$ into (\ref{equation:regr.likelihood}) gives a
likelihood having the form $\like(\beta_{0},\beta_{1},
\beta_{2},\sigma)$.  The model
generalizes easily when there are more than two explanatory variables.
In the model, $\beta_{0}$ is the value of $\mu$ when $x_{1}=x_{2}=0$
(if this makes sense physically), $\beta_{1}$ is the change in $\mu$
for a one-unit change in $x_{1}$ (holding $x_{2}$ constant), 
and $\beta_{2}$ is the change in $\mu$
for a one-unit change in $x_{2}$ (holding $x_{1}$ constant).
In particular, the effect of changing $x_{1}$ does not depend on the
level of $x_{2}$ and vice versa.
This model is known as the ``additive''
or ``no-interaction'' two-variable regression model. 

%----------------------------------------------------------------------
\subsection{Two-variable regression model with interaction}
\label{section:two.var.with.int}
The log-location-scale two-variable regression model {\em
with} interaction uses (\ref{equation:regr.cdf.model}) with
$\mu=\beta_{0}+\beta_{1} x_{1} +
\beta_{2} x_{2}+
\beta_{3} x_{1}x_{2}$. If $\sigma$ does not
depend on $\xvec$, the log-quantile function is
\begin{displaymath}
\log[\rvquan_{p}(\xvec)] =\grvquan_{p}(\xvec)=\beta_{0}+
	\beta_{1} x_{1} + \beta_{2} x_{2}+ \beta_{3}x_{1}x_{2} 
+ \Phi^{-1}(p) \sigma.
\end{displaymath}
The distribution quantile can be re-expressed as
$\rvquan_{p}(\xvec)=\exp \left [\grvquan_{p}(\xvec) \right ]= 
		\exp(\beta_{1} x_{1} + \beta_{2} x_{2}+
                \beta_{3}x_{1}x_{2}) \, \rvquan_{p}(0)$.
Thus this is a SAFT model with  $\AF(\xvec)=1/\exp(\beta_{1}
x+\beta_{2}x_{2}+
\beta_{3} x_{1}x_{2})$.
Substituting $\mu_{i}  = \beta_{0} + \beta_{1} x_{1i} +
\beta_{2} x_{2i}+
\beta_{3} x_{1i}x_{2i}$ into (\ref{equation:regr.likelihood}) gives a
likelihood having the form $\like(\beta_{0},\beta_{1},
\beta_{2},\beta_{3},\sigma)$. 
In this model $\beta_{1} + \beta_{3}x_{2}$ is the change in $\mu$ for
a one-unit change in $x_{1}$. Similarly, $\beta_{2} + \beta_{3}x_{1}$
is the change in $\mu$ for a one-unit change in $x_{2}$.

%----------------------------------------------------------------------
\begin{table}
\caption{Glass capacitor life test ML estimates for 
the Weibull regression model.}
\centering\small
\begin{tabular}{lcrrrr} 
\\[-.5ex]
\hline
& & & & \multicolumn{2}{c}{Approximate 95\%}\\
&&\multicolumn{1}{c}{ML} &Standard & \multicolumn{2}{c}{Confidence
Interval}\\ \cline{5-6}
& Parameter & Estimate&
\multicolumn{1}{c}{Error} & Lower & Upper \\
\hline 
Model 1 
&$\beta_{0}$ &13.41 &2.30 &$8.90$ &17.91 \\[.7ex] 
&$\beta_{1}$ &$-.02$  &.01  & $-.05$  & $-.004$\\[.7ex] 
&$\beta_{2}$ &$-.006$  &.001 &$-.008$ &$-.004$ \\[.7ex] 
&$\sigma$ &0.36 &.055 &.27 & .49 \\[1.2ex]
\hline 
\\[-1.8ex]
Model 2 
&$\beta_{0}$ &9.41 &10.5 &$-11.2$ &30.1 \\[.7ex] 
&$\beta_{1}$ & $-.0062$ &.060 &$-.12$ &.11 \\[.7ex] 
&$\beta_{2}$ &$.0086$ &.037 & $-.065$ &.082 \\[.7ex] 
&$\beta_{3}$ &$.000082$ &$.00021$ & $-.00050$ &$.00034$ \\[.7ex] 
&$\sigma$ &.362 &.055 &.27 &.49 \\[1.2ex]
\hline 
\\[-1.8ex]
\end{tabular}
\begin{minipage}[t]{4in}
The log likelihoods for Models 1 and 2 are, respectively, $\loglike_{1}=-244.24 $
and $\loglike_{2}=-244.17 $.
\end{minipage}
\label{table:zelen.mles}
\end{table}

\begin{example}
\label{example:zelen.data.int.model}
{\bf Regression models for glass capacitor life.} The top half of
Table~\ref{table:zelen.mles} provides information on the ML fit of
Model 1, the two-variable no-interaction regression model with
$x_{1}=\mbox{temperature in $\degreesc$}$ and $x_{2}=\mbox{Voltage}$.
The bottom half of Table~\ref{table:zelen.mles} provides information
on the ML estimates for Model 2.  This model includes the interaction
term $x_{3}=x_{1} x_{2}$.  Comparing the log-likelihood values for the
two models indicates that the interaction term has not improved the
model's ability to explain variability in the failure times. In
particular, the log-likelihood ratio statistic $-2 \times (\loglike_{1}
- \loglike_{2}) = .14$ is small relative to $\chisquare_{(.75;1)}=1.32$
(from standard chi-square quantile tables).
Figure~\ref{figure:zelencap.groupm.int.ps}, is similar to
Figure~\ref{figure:zelencap.groupi.ps}, but in this case, the parallel
lines are the Model 2 ML estimates for each of the 8 test conditions.
The lines are parallel because $\sigma$ in the model does not depend
on $\xvec=(x_{1},x_{2})$.  

The plotted points in Figure~\ref{figure:zelencap.cellquan.int.ps}
are the individual estimates of $\rvquan_{.5}$ for each of the 8
combinations of the glass capacitor test conditions (computed from
the ML estimates in Table~\ref{table:zelen.cap.data}). The lines in
Figure~\ref{figure:zelencap.cellquan.int.ps} give the Model 2 fitted
relationship between life and voltage for temperatures of
170$\degreesc$ and 180$\degreesc$. The Model 1 plot (not shown here)
was similar, but the 170$\degreesc$ and 180$\degreesc$ lines were
exactly parallel.  Figure~\ref{figure:zelencap.groupm.int.resid.ps}
is a plot of the Model 2 regression residuals. There does not seem
to be any serious departure from the fitted model or the Weibull
distribution assumptions in the model.
%----------------------------------------------------------------------
\begin{figure}
\splusbookfigure{\figurehome/zelencap.groupm.int.ps}
\caption{Weibull probability plots of data
at individual temperature and voltage test conditions, with the
Weibull regression Model 2 ML estimates of $F(t)$ at each set of
conditions for the glass capacitor
data.}
\label{figure:zelencap.groupm.int.ps}
\end{figure}
%----------------------------------------------------------------------
%----------------------------------------------------------------------
\begin{figure}
\splusbookfigure{\figurehome/zelencap.cellquan.int.ps}
\caption{Individual estimates of Weibull $t_{.5}$ plotted for 
each glass capacitor test condition. The
straight lines depict the with-interaction (Model 2) 
Weibull regression model estimate of $t_{.5}$
versus voltage for $170\degreesc$ and 
$180\degreesc$.}
\label{figure:zelencap.cellquan.int.ps}
\end{figure}
%----------------------------------------------------------------------
%----------------------------------------------------------------------
\begin{figure}
\splusbookfigure{\figurehome/zelencap.groupm.int.resid.ps}
\caption{Weibull probability plots of the
residuals from Model 2 fit to the glass capacitor
data.}
\label{figure:zelencap.groupm.int.resid.ps}
\end{figure}
%----------------------------------------------------------------------
\end{example}

%----------------------------------------------------------------------
%----------------------------------------------------------------------
\section{Product Comparison: An Indicator-Variable Regression Model}
\label{section:product.comparison}
This section illustrates methods for comparing samples from two
different populations or processes, which we generically call
``groups.'' In the first analysis 
(Examples~\ref{example:snubber.separate} and 
\ref{example:snubber.comparing.quantiles}) the samples are simply analyzed
separately to make a comparison.  The second analysis 
(Example~\ref{example:snubber.common.sigma}) uses an
indicator-variable regression model to compare the samples under the
assumption that the spread parameter $\sigma$ is the same for both
groups.

\subsection{Comparison of groups using separate analyses.}
The simplest method for comparing two groups is to analyze the
groups separately.

\begin{example}
\label{example:snubber.separate}
{\bf A comparison of snubber designs---separate analyses.} Appendix
Table~\ref{atable:snubber.data} gives data from Nelson
(1982, page 529), 
from a life test to compare two different snubber designs.  A
snubber is a component in an electric toaster.  Following the analysis
used by Nelson, Figure~\ref{figure:snubber.indiv.norm.ps} shows the
results of fitting separate normal distributions to the data from
each snubber design.  The different slopes reflect the different
standard deviation estimates. The numerical results are summarized
in Table~\ref{table:snubber.mles}, under Model 1.
\end{example}
%----------------------------------------------------------------------
\begin{figure}
\splusbookfigure{\figurehome/snubber.indiv.norm.ps}
\caption{Normal probability plot showing separate analyses 
comparing the old and new snubber designs (old design ``$\circ$'' and
the new design ``$+$'').  The solid (dotted) line is the ML estimate
of the normal distribution cdf for the old (new) design.}
\label{figure:snubber.indiv.norm.ps}
\end{figure}
%----------------------------------------------------------------------
%----------------------------------------------------------------------
%splus old design: Log Likelihood at maximum: -146.7548
%splus           MLE       se   z-ratio 95% lower 95% upper 
%splus   mu 908.1341 76.19281 11.918894  758.7962 1057.4720
%splussigma 362.4326 63.40395  5.716247  238.1609  486.7044
%splus
%splus new design Log Likelihood at maximum: -138.6036
%splus            MLE        se  z-ratio 95% lower 95% upper 
%splus   mu 1126.5972 123.20691 9.143945  885.1116 1368.0827
%splussigma  545.9695  99.50092 5.487080  350.9477  740.9913
%splus
%splus pooled sigma Log Likelihood at maximum: -286.6587
%splus            MLE        se    z-ratio 95% lower 95% upper 
%splus   b0 974.62787  89.11084 10.9372538  799.9706 1149.2851
%splus   b1  86.67246 114.20675  0.7589084 -137.1728  310.5177
%splussigma 458.56430  57.66753  7.9518637  345.5360  571.5926
%splus 
%splus
\begin{table}
\caption{ML estimates for the snubber life test data.}
\centering\small
\begin{tabular}{lcrrrr} 
\\[-.5ex]
\hline
& & & & \multicolumn{2}{c}{Approximate 95\%}\\
& &\multicolumn{1}{c}{ML} &Standard & 
\multicolumn{2}{c}{Confidence Interval}\\  \cline{5-6}
& Parameter & Estimate& \multicolumn{1}{c}{Error} & Lower & Upper \\
\hline 
Model 1  &$\mu_{\old} $ & 908 &76.2 &759 & 1057 \\[.7ex] 
&$\sigma_{\old}$ &362 &63.4 &238 &487\\[1.2ex]
&$\mu_{\new} $ & 1126 &123 &885 &1368 \\[.7ex] 
&$\sigma_{\new}$ &546 &100 &351 &741 \\[.7ex] 
\hline 
\\[-1.8ex]
Model 2 
&$\beta_{0}$ &975  &89.1  &800    & 1149  \\[.7ex]
&$\beta_{1}$ &86.7 &114   &$-137$ &  311  \\[.7ex] 
&$\sigma$    & 459 & 57.7 & 346   & 572 \\[1.2ex]
\hline 
\\[-1.8ex]
\end{tabular}
\begin{minipage}[t]{4in}
For Model 1, $\loglike_{\old}=-138.6$ for the old design and
$\loglike_{\new}=-146.8$ for the new design (totaling
$\loglike_{1}=\loglike_{\old}+\loglike_{\new} =-285.4$). For Model 2,
$\loglike_{2}=-286.7$.
\label{table:snubber.mles}
\end{minipage}
\end{table}

As illustrated in Figure~\ref{figure:snubber.indiv.norm.ps}, comparing
distributions may not be straightforward in applications where the
cdfs from the two groups cross. If we took the lines in
Figure~\ref{figure:snubber.indiv.norm.ps} to be true cdfs, the
implication would be that the old design is better up until about 550
cycles, after which the new design is better. The ambiguity arises
because the estimate for $\sigma$ from the new design is larger.
It is, however, possible to compare particular points on the cdfs.

\begin{example}
\label{example:snubber.comparing.quantiles}
{\bf A comparison of snubber designs---comparing quantiles.} To make a
quantitative comparison between the two designs, we will estimate the
difference between the values of $y_{.5}(\new)$ and
$y_{.5}(\old)$ (we use $y$ instead of $t$ here because we are
fitting a normal distribution which has a theoretical range extending
to negative numbers). For the normal distribution, this is the same as
comparing the means of the two groups.
Using the results for Model 1 in Table~\ref{table:snubber.mles},
\begin{eqnarray*}
\muhat_{\new} - \muhat_{\old} & = & 1126 - 908 = 218\\[1ex]
\sehat_{\muhat_{\new} - \muhat_{\old}} &=&
\sqrt{\sehat^{2}_{\muhat_{\new}} + \sehat^{2}_{\muhat_{\old}}}
\\
  & = & \sqrt{(76.2)^{2} +  (123)^{2}   } = 144.7
\end{eqnarray*}
%splus> 1126 - 908 = 218
%splus> sqrt((76.2)^2 +  (123)^2  )= 144.6908
and an approximate 95\% confidence interval for
$\delta = \mu_{\new} -
\mu_{\old}$ is
\begin{eqnarray*}
[\undertilde{\delta}, \quad \tilde{\delta}] &=& 
	\muhat_{\new} - \muhat_{\old} \pm 
     \norquan_{(1-\alpha/2)} \sehat_{\muhat_{\new} -
\muhat_{\old}}\\
	&=& 218 \pm 1.96 \times 144.7 = [-66, \quad 501].
\end{eqnarray*}
Because this interval contains 0, we conclude that there is
not a convincing difference
between the mean cycles to failure for the two designs.
%splus> 218 + 1.96 * 144.7 = 501.612
%splus> 218 - 1.96 * 144.7 = -65.612
\end{example}

%----------------------------------------------------------------------
\subsection{Comparison of groups using combined analyses}
In some situations it might be reasonable to use a model in which 
$\sigma$ is the same across the groups, with differences only in the
values of $\mu$. 
The analysis can be made with a simple regression relationship using
$\mu=\beta_{0}+\beta_{1} x$ where
$x=0$ for one group and $x=1$ for the other. Then substituting $x$ into
the model gives $\mu(0)=\beta_{0}$ and
$\mu(1)=\beta_{0}+\beta_{1}$.
Furthermore $\delta = t_{p}(1) -t_{p}(0)=\mu(1) -
\mu(0) = \beta_{1}$, making comparisons with a common $\sigma$
less ambiguous because, in this model, $\delta$ does not depend on 
which quantile is compared.

%----------------------------------------------------------------------
\begin{figure}
\splusbookfigure{\figurehome/snubber.gmodel.norm.ps}
\caption{Normal probability plot summarizing 
the fitted common-$\sigma$ models for 
the old and new snubber designs. 
Observations from the old design are indicated by a ``$\circ$'' and
observations from the new design are indicated by a ``$+$.''
The solid (dotted) line is the ML estimate of the normal distribution
cdf for the old (new) design. The dashed lines are
approximate 95\% pointwise
confidence intervals for the old design.}
\label{figure:snubber.gmodel.norm.ps}
\end{figure}
%----------------------------------------------------------------------

\begin{example}
{\bf A comparison of snubber designs---common $\sigma$ analysis.}
\label{example:snubber.common.sigma}
%----------------------------------------------------------------------
Figure~\ref{figure:snubber.gmodel.norm.ps} displays estimates of the
cdfs for the new and old designs. Because in this model $\sigma$ is
the same for both designs, the fitted lines are parallel.
Table~\ref{table:snubber.mles} gives corresponding numerical results.
The dotted curves on Figure~\ref{figure:snubber.gmodel.norm.ps} are
pointwise approximate 95\% confidence intervals for $F(t)$ for the old
design. Although these intervals do not answer precisely the question
of interest, they do indicate that there is considerable variability
in the estimates, relative to the difference between the point
estimates of the distributions for the old and the new designs.  An
approximate 95\% confidence interval for $\delta = t_{p}(\new)
-t_{p}(\old)=\mu_{\new} -
\mu_{\old}=\beta_{1}$ is
\begin{eqnarray*}
[\undertilde{\delta}, \quad \tilde{\delta}]= 
[\undertilde{\beta_{1}}, \quad \tilde{\beta_{1}}] &=& 
	\betahat_{1} \pm 
     \norquan_{(1-\alpha/2)} \sehat_{\betahat_{1}}\\
	&=& 86.7 \pm 1.96 \times 114 = [-137, \quad  311].
\end{eqnarray*}
Because this interval contains zero, we conclude, as before,  
that there is not a convincing difference
between the new and the old designs.
\end{example}

%----------------------------------------------------------------------
%----------------------------------------------------------------------
\section{The Proportional Hazards Failure Time Model}
\label{section:prohaz.fail}

%----------------------------------------------------------------------
\subsection{Proportional hazards relationships}
The proportional hazards (PH) model relates the hazard functions at
conditions $\xvec$ and baseline conditions $\xvec_{0}$ by
\begin{equation}
\label{equation:ph.def.h}
h(\realrv; \xvec)=\Psi(\xvec) h(\realrv; \xvec_{0})
\end{equation}
for all $\realrv>0$, where $\Psi(\xvec)$, like $\AF(\xvec)$ in
Section~\ref{section:saft.model}, is a positive function with
$\Psi(\xvec_{0})=1$.  The proportional hazards model 
can also be
written as
\begin{equation}
\label{equation:ph.def.s}
S(t;\xvec) = [S(t;\xvec_{0})]^{\Psi(\xvec)}
\end{equation}
or
\begin{equation}
\label{equation:ph.def.f}
F(t;\xvec) = 1- [1-F(t;\xvec_{0})]^{\Psi(\xvec)}.
\end{equation}
Again, if
$\Psi(\xvec) \ne 1$, then $F(t;\xvec)$ and $F(t;\xvec_{0})$ do not
cross. When $\Psi(\xvec) >1$, the model accelerates time in the sense
that $F(t;\xvec) > F(t;\xvec_{0})$ for all $t$.  When $\Psi(\xvec)<1$,
the model decelerates time in the sense that $F(t;\xvec) <
F(t;\xvec_{0})$ for all $t$. 

Re-expressing (\ref{equation:ph.def.f}) leads to
$1-F(t;\xvec)=[1-F(t;\xvec_{0})]^{\Psi(\xvec)}$ and taking logs
(twice) gives
\begin{equation}
\label{equation:log.log.phft}
\log  \left \{-\log   \left [1- F(t;\xvec)   \right ]\right \}
- \log \left \{-\log \left [1-F(t;\xvec_{0}) \right ]\right \} = \log
\left [ \Psi(\xvec) \right ].
\end{equation}
Thus when $F(\realrv; \xvec)$ and $F(\realrv; \xvec_{0})$ are related by
a PH model, they are vertically equidistant at any given $t$ on
Weibull probability paper, as shown in
Figure~\ref{figure:lawless.lognor.ph.ps}. This graphical
relationship is useful for assessing the reasonableness of a PH
regression model.

%----------------------------------------------------------------------
\begin{figure}
\splusbookfigure{\figurehome/lawless.lognor.ph.ps}
\caption{Weibull probability plot showing a
proportional hazards accelerated failure time regression relationship
with a lognormal baseline distribution (lower line). The upper line
corresponds to a cdf from the power-lognormal distribution.}
\label{figure:lawless.lognor.ph.ps}
\end{figure}
%----------------------------------------------------------------------

A proportional hazards model can also be expressed as a failure time
transformation model. In particular, if $\rv(\xvec_{0}) \sim
F(\realrv;\xvec_{0})$ and if $\rv(\xvec)$ and $\rv(\xvec_{0})$ are
related by the time transformation function
\begin{equation}
\label{equation:ph.time.trans}
\rv(\xvec)=F^{-1}\left (
		1- \left \{ 			1 -F \left
[\rv(\xvec_{0});
\xvec_{0}
			         \right ] 		 \right
\}^{1/\Psi(\xvec)};
\xvec_{0}
			      \right )
\end{equation}
then it can be shown that $\rv(\xvec)$ and $\rv(\xvec_{0})$ have the
PH relationship in (\ref{equation:ph.def.h}).  This time
transformation function is illustrated in
Figure~\ref{figure:ph.transform.ps}.  In this example, the amount of
acceleration (or deceleration), $\rv(\xvec_{0})/\rv(\xvec)$, depends
on the position in time.

%-------------------------------------------------------------------
\begin{figure}
\splusbookfigure{\figurehome/ph.transform.ps}
\caption{Proportional hazards model with a lognormal baseline
hazard function expressed as a time transformation function.}
\label{figure:ph.transform.ps}
\end{figure}

%----------------------------------------------------------------------
\begin{figure}
\splusbookfigure{\figurehome/lawless.weib.ph.ps}
\caption{Weibull probability plot of
two Weibull distributions. The parallel straight lines
here come from a regression relationship that is both
accelerated failure time and proportional hazards.}
\label{figure:lawless.weib.ph.ps}
\end{figure}
%----------------------------------------------------------------------

%----------------------------------------------------------------------
\subsection{The Weibull proportional hazards model}
For the Weibull distribution (and only the Weibull distribution), a
PH regression model is also a SAFT regression model.  This can be
seen by noting that the Weibull distribution is the only
distribution in which both (\ref{equation:log.saft}) and
(\ref{equation:log.log.phft}) hold. Relatedly, Weibull probability
plots of Weibull cdfs with the same $\sigma$ are translations of each
other in {\em both} the probability and the $\log(\realrv)$
scale. Thus the Weibull cdfs at $\xvec$ and $\xvec_{0}$ are parallel
straight lines on Weibull probability paper, as shown in
Figure~\ref{figure:lawless.weib.ph.ps}.

%----------------------------------------------------------------------
\subsection{Other proportional hazards models}

Except in the case of the Weibull distribution, the PH regression
relationship changes the shape of the underlying distribution as a
function of the explanatory variables $\xvec$.  That is, in general,
the PH model does not preserve the form of baseline distribution.
For example, if $\rv(\xvec_{0})$ has a lognormal distribution then
$\rv(\xvec)$ has a power lognormal distribution
(Section~\ref{section:power.dist}).

%----------------------------------------------------------------------
\subsection{The semiparametric (Cox) proportional hazards model}
In its semiparametric form, the PH model in
(\ref{equation:ph.def.h}) is known as the Cox
proportional hazards model. In this model, the form of the hazard
function $h(t ; \xvec_{0})$ is unspecified. One can estimate the regression
coefficients [the parametric part of the model $\Psi(\xvec)$] and obtain
nonparametric estimates of the hazard (or cdf or survival) functions
at any specified values of the explanatory variable(s).  The Cox PH
model is widely used in biomedical applications and especially in
the analysis of clinical-trial data.


%----------------------------------------------------------------------
\subsection{PH model applications in reliability}
Because models of failure based on physics and chemistry (see
Chapter~\ref{chapter:accelerated.test.models}) generally suggests
a SAFT model or other non-PH models, the main area for potential
application of PH models would appear to be in the analysis of field
reliability data for which it is necessary to adjust for covariates
like operating environment, use rate, and so on.  Bendell~(1985) and
Dale~(1985) describe applications of semiparametric proportional
hazards modeling to the analysis of reliability data.

Landers and Kolarik~(1987) describe an application where a
parametric PH model was used in the analysis of field reliability
data. They, however, used a Weibull baseline distribution and so
their model was really the same as the SAFT Weibull regression model
used in Sections~\ref{section:fail.time.regr.models} through
\ref{section:quad.regr}.

%----------------------------------------------------------------------
%----------------------------------------------------------------------
\section{General Time Transformation Functions}
\label{section:gen.time.trans}
In Sections~\ref{section:fail.time.regr.models} and
\ref{section:prohaz.fail} we expressed both SAFT and PH models as
special time transformation functions. Time transformation functions
provide a general model for relating time at one level of $\xvec$
with time at another level of $\xvec$. The time transformation model
maps time at one level of $\xvec_{0}$ to time at another level of
$\xvec$. This can be expressed as
\begin{displaymath}
\rv(\xvec)=\Upsilon \left [\rv(\xvec_{0}),\xvec \right ]
\end{displaymath}
where $\xvec_{0}$ are again {\em baseline} conditions. To be a time
transformation, the function $\Upsilon \left (\realrv,\xvec \right )$
must have the following properties:
\begin{itemize}
\item
For any $\xvec$, $\Upsilon \left (0,\xvec \right )=0$, as in 
Figure~\ref{figure:general.transform.ps}.
\item
$\Upsilon \left (\realrv,\xvec \right ) $ is nonnegative, i.e.,
$\Upsilon \left (\realrv,\xvec \right ) \ge 0$ for all $\realrv$ and
$\xvec$.
\item
For fixed $\xvec$, $\Upsilon \left (\realrv,\xvec \right )$ is
monotone increasing in $\realrv$.
\item
When evaluated at $\xvec_{0}$, the transformation is the identity 
transformation (i.e., $\Upsilon \left (\realrv,\xvec_{0} \right )=\realrv
\,\, \mbox{for all} \,\, \realrv $). 
\end{itemize}
A quantile of the distribution of $\rv(\xvec)$ can be determined as a
function of the corresponding quantile of the distribution of
$\rv(\xvec_{0})$ and $\xvec$.  In particular, $\rvquan_{p}(\xvec)=
\Upsilon \left [\rvquan_{p}(\xvec_{0}),\xvec \right ]
$ for $0 \le p \le 1.$ As shown in
Figure~\ref{figure:general.transform.ps}, a plot of
$\rv(\xvec_{0})$ versus $\rv(\xvec)$ can imply a particular class of
transformation functions. In particular,
%-------------------------------------------------------------------
\begin{figure}
\splusbookfigure{\figurehome/general.transform.ps}
\caption{General failure time transformation  model.}
\label{figure:general.transform.ps}
\end{figure}
\begin{itemize} 
\item
$T(\xvec)$ entirely below the diagonal line implies acceleration.
\item
$T(\xvec)$ entirely above the diagonal line implies deceleration.
\item
$T(\xvec)$ can cross the diagonal, in which case the transformation is
accelerating over some times and decelerating over other times. In
this case the cdfs of $\rv(\xvec)$ and $\rv(\xvec_{0})$ cross.
\end{itemize}

%----------------------------------------------------------------------
%----------------------------------------------------------------------

\section*{Bibliographic Notes}
Seber~(1977) and Neter, Kutner, Nachtsheim, and Wasserman~(1996) are
useful references on linear regression analysis for complete
(uncensored) data. Seber and Wild~(1989) describe methods for
nonlinear regression.  Lawless~(1982), Cox and Oakes~(1984), and
Nelson~(1990a) describe applications of failure-time regression
models and illustrate fitting methods for censored
data. Nelson~(1990a) describes failure time regression modeling,
analysis, and test plans with applications to accelerated life
testing, a topic covered in
Chapters~\ref{chapter:accelerated.test.models} and
\ref{chapter:analyzing.alt.data} of this book. The ideas
behind Figures~\ref{figure:lawless.lognor.aft.ps},
\ref{figure:lawless.weib.ph.ps}, and 
\ref{figure:lawless.lognor.ph.ps} came from Lawless~(1986).
Nelson~(1984) analyzed the superalloy fatigue data used in
Examples~\ref{example:super.alloy.data},
\ref{example:super.alloy.nf20},
and \ref{example:super.alloy.nf21}.  Nelson~(1984) also used a model
in which both $\mu$ and $\sigma$ (parameters of the lognormal
distribution) depend on a stress variable and outlined the pitfalls
of using this model.  Our development has followed this
work. Escobar and Meeker~(1992) give expressions for computing the
elements in (\ref{equation:regr.local.est.vcv}) and present methods
for doing influence analysis with censored data. Meeker and
LuValle~(1995) describe a regression model in which two different
chemical reaction rate constants depend on temperature, leading to a
non-SAFT model.  Martin~(1982) describes the use of general time
transformation functions used in life testing.  General time
transformation functions are discussed, in the context of
accelerated testing, in Chapter 17 of Ushakov~(1994).

Nelson~(1973) describes methods for residual analysis with censored
data.  Schmee and Hahn~(1979) present an iterative least squares
method of finding estimates for regression parameters with censored
data. The method is useful for finding starting values for ML
estimation. 

Kalbfleisch and Prentice~(1980), Lawless~(1982) and Cox and
Oakes~(1984) provide a detailed treatment of the important facets of
the Cox proportional hazards model.  Bagdonavi\v{c}ius and
Nikulin~(1995, 1997) describe general classes of semiparametric
failure-time regression methods, methods of estimation, and
asymptotic theory for the estimators.

\section*{Exercises}

%-------------------------------------------------------
\begin{exercise}
The confidence intervals for Example~\ref{example:ct.regr.model} given
in Table~\ref{table:comptime.mles} were computed by using the normal
approximation method that is commonly used for censored data problems.
For this data set, with the lognormal model, it is possible to use
standard ``exact'' methods to compute confidence intervals for these
parameters, based on a simple ordinary least squares regression
relating the logarithms of execution time to system load.  Formulas
for these exact methods are available in almost any elementary
statistics text covering regression analysis.  Use a computer program
to do the necessary least squares computations and compute, by hand,
the ``exact'' confidence intervals.  Compare the estimates and
confidence intervals. What is your conclusion?
\end{exercise}

%-------------------------------------------------------
\begin{exercise}
Refer to the loglinear model in (\ref{equation:weib.reg.quant}).
Show why $100\beta_{1}$ can be interpreted as the approximate percent increase
in $\rvquan_{p}(x)$ for a one-unit increase in $x$.
\end{exercise}

%-------------------------------------------------------
\begin{exercise}
\label{exercise:rcfatigue.graphical}
McCool~(1980) gives the results of a life test on rolling contact
fatigue of ceramic ball bearings. Ten specimens were tested at each
of 4 levels of stress. The ordered failure times are given in the
following table.  McCool indicates that it is customary to model such
data with the two-parameter Weibull distribution with a shape parameter
that does not depend on stress.\\[3ex]
\begin{tabular}{ll}
Stress ($10^{6}$ psi) &  Ordered Life Times ($10^{6}$ revolutions) \\
\hline
.87 & 1.67, 2.20, 2.51, 3.00, 3.90, 4.70, 7.53, 14.70, 27.80, 37.40 \\
.99& .80, 1.00, 1.37, 2.25, 2.95, 3.70, 6.07, 6.65, 7.05, 7.37 \\ 
1.09& .012, .18, .20, .24, .26, .32, .32, .42, .44, .88 \\ 
1.18 & .073, .098, .117, .135, .175, .262, .270, .350, .386, .456 \\
\hline
\\[2ex]
\end{tabular}
\begin{enumerate}
\item
\label{exer.part:refatigue.logplot}
Plot the failure times versus stress on log-log axes (or alternatively,
take logs and plot on linear axes).
\item
It is often suggested that median failure time is proportional to a
power transformation of stress. That is,  $t_{.5}=
e^{\beta_{0}} \times {(\Stress)}^{\beta_{1}}$ or $\log(t_{.5})=
\beta_{0}+ \beta_{1}\log(\Stress)$. Is the suggestion
reasonable in this case?  Plot the sample medians on the graph in
part~\ref{exer.part:refatigue.logplot} to help answer this question.
\item
\label{exer.part:refatigue.graphical}
Use a hand-drawn line
through sample medians versus stress points to obtain a graphical estimate
of the exponent (or slope) $\beta_{1}$.
\item
Make separate Weibull probability plots for the data at each level of
stress, plotting them all on the same graph. What does this plot
suggest about the customary assumption that the Weibull shape
parameter is the same for all levels of stress?  Provide possible
explanations for the observed differences in the estimates of the
Weibull shape parameter at each level of stress.
\end{enumerate}
\end{exercise}

%-------------------------------------------------------
\begin{exercise}
Figure~\ref{figure:super.alloy.data} shows that there is more spread 
(variability) in the
observed log failure times at low stress, as compared with high stress.
Provide a simple, nontechnical, intuitive explanation for this
common behavior of fatigue data.
\end{exercise} 

%-------------------------------------------------------
\begin{exercise1}
\label{exercise:rcfatigue.ls.ml}
Refer to Exercise~\ref{exercise:rcfatigue.graphical}.
Suppose that log life can be adequately described by a normal distribution.
Consider the regression model $\mu=\beta_{0}+\beta_{1}\log(\Stress)$
with constant $\sigma$, where $\sigma$ is the standard deviation of
log life, the same for any fixed levels of stress.
\begin{enumerate}
\item
Use ordinary least squares to fit this model to the rolling contact
fatigue data. Compare your answer to the graphical estimate from
part~\ref{exer.part:refatigue.graphical} of
Exercise~\ref{exercise:rcfatigue.graphical}.
\item
Use a computer program that does maximum likelihood estimation and fit
the lognormal regression model (\ref{equation:weib.reg.quant}) to the
rolling contact fatigue data.
\item
Compute and compare estimates from these two different methods of
estimation for, the median time to failure at $1.05 \times 10^{6}$
psi, the .01 quantile at $1.05 \times 10^{6}$ psi, and the .01
quantile at $.85 \times 10^{6}$ psi.  Comment on the results of this
comparison.
\end{enumerate}
\end{exercise1}

%-------------------------------------------------------
\begin{exercise}
Use the results in Table~\ref{table:super.alloy.mles} for
Example~\ref{example:super.alloy.nf20} to compute an 80\%
normal-approximation confidence interval for $\sigma$.  Explain the
interpretation of this interval.
\end{exercise}

%-------------------------------------------------------
\begin{exercise}
Use the results in Table~\ref{table:super.alloy.mles}
for Example~\ref{example:super.alloy.nf21} to do the following: 
\begin{enumerate}
\item
\label{exer.part:nc.sig.beta.ci}
Compute an 90\% normal-approximation confidence interval for
$\beta_{1}^{[\sigma]}$.
\item
Explain how this confidence interval can be used to judge whether
$\sigma$ depends on the level of pseudo stress or not.
\item
Do a likelihood ratio test to determine whether or not there exists strong
evidence that $\sigma$ depends on the level of pseudo stress.
\item
Explain the steps you would follow to compute a likelihood-ratio-based
confidence interval for $\beta_{1}^{[\sigma]}$.
\end{enumerate}
\end{exercise}

%-------------------------------------------------------
\begin{exercise1}
\label{exercise:outlier.test}
Refer to Example~\ref{example:super.alloy.nf21} and
Figure~\ref{figure:nf.altplot.nf21.weib.ps}. 
It has been suggested that the failure at 13,949 cycles with
pseudostress equal to 85.2 ksi appears to be an outlier.
\begin{enumerate}
\item
\label{exer.part:prob.ext}
Use the ML estimates for Model 2 in
Table~\ref{table:super.alloy.mles} as if they were 
the actual parameter values and  compute the probability
that one would have a failure before 13,949 cycles when running at 
a pseudostress equal to 85.2 ksi.
\item
\label{exer.part:nf21.outlier}
Following the approach in part~\ref{exer.part:prob.ext}, if 26 units
were to be tested at a pseudostress equal to 85.2 ksi, what is the
probability that the {\em earliest} failure would occur before 13,949
cycles?
\item
There were 26 observations in the superalloy fatigue example. Explain
why the probability in part~\ref{exer.part:nf21.outlier} is the
relevant probability to consider when judging whether the single
observation in the example departs importantly from the assumed model
or not.
\end{enumerate}
\end{exercise1}

%-------------------------------------------------------
\begin{exercise}
\label{exercise:reverse.quant}
Consider the model used in Example~\ref{example:super.alloy.nf21} and
the corresponding ML estimates for Model 2 in
Table~\ref{table:super.alloy.mles}. 
\begin{enumerate}
\item
\label{exer.part:reverse.quant}
Show that it is possible
to have $\rvquan_{p}(x_{1}) < \rvquan_{p}(x_{2})$ when $x_{1} <
x_{2}$.
\item
Explain why the relationship in part~\ref{exer.part:reverse.quant}
is physically unreasonable.
\end{enumerate}
\end{exercise}

%-------------------------------------------------------
\begin{exercise1}
Follow the general approach outlined in
Section~\ref{section:se.and.ci.for.regr.funct} and
Example~\ref{example:regr.ci.for.quan} to do the following for Model 2
used in Example~\ref{example:super.alloy.nf21}:
\begin{enumerate}
\item
Using the results in Table~\ref{table:super.alloy.mles}, compute
estimates $(\muhat, \sigmahat)$ for pseudostress 100 ksi.
\item
Derive the expressions needed to compute
$\vcvmathat_{\muhat,\sigmahat}$ as a function of the elements of the
variance-covariance matrix of the regression parameters in Model 2.
\end{enumerate}
\end{exercise1}

%-------------------------------------------------------
\begin{exercise1}
\label{exercise:rcfatigue.sigma.test}
Refer to Exercise~\ref{exercise:rcfatigue.graphical}. There is some
evidence that the Weibull shape parameter depends on
stress, but it might be argued that
the observed differences are due to random variation in the data.
\begin{enumerate}
\item
Fit separate two-parameter Weibull distributions to the data
at each level of stress.
\item
Fit a regression model with an indicator variable allowing for a
different Weibull scale parameter $\weibscale=\exp(\mu)$ at each
level of stress. Hold $\sigma=1/\beta$ (the reciprocal of the
Weibull shape parameter $\beta$) constant over all levels of stress.
\item
Do a likelihood ratio test to see if there is evidence that
the values of $\sigma$ differ with stress.
\end{enumerate}
\end{exercise1}

%-------------------------------------------------------
\begin{exercise} 
Return to the capacitor life test data in
Example~\ref{example:zelen.data}.  Physical theory suggests that the
Weibull shape parameter $\beta=1/\sigma$ will depend on temperature if
there is unit-to-unit variability in both the initial level of
dielectric degradation and in dielectric degradation rate. Insulation
engineers were interested in using these data to see if there was
evidence for a temperature or voltage effect on $\beta=1/\sigma$.
\begin{enumerate}
\item
\label{exer.part:zelen.indiv}
Use the fitted model 
in Table~\ref{table:zelen.cap.data} giving the Weibull
distribution estimates at each combination of the individual
experimental conditions. Compute the sum of the log likelihoods for all
of the conditions.
\item
Plot, on Weibull probability paper, the estimates of the Weibull cdfs
for each test condition, as done in
Figure~\ref{figure:zelencap.groupi.ps}.
\item
\label{exer.part:zelen.csig}
Fit an indicator-variable regression model to the capacitor life test data
such that there is a constant value of $\sigma$ but that
a separate value of $\mu$ is estimated at each test condition.
\item
Compare the sum of the likelihoods from
part~\ref{exer.part:zelen.indiv} with the likelihood obtained from the
model in part~\ref{exer.part:zelen.csig}. Use these results to test
if there is evidence for nonconstant $\sigma$ in these data.
What do you conclude?
\end{enumerate}
\end{exercise}

%-------------------------------------------------------
\begin{exercise1}
Write a time transformation function corresponding to
model~(\ref{equation:quad.mu.lin.sigma.model}).
\end{exercise1}

%-------------------------------------------------------
\begin{exercise1} 
Show that (\ref{equation:ph.def.h}) implies (\ref{equation:ph.def.s}).
\end{exercise1}

%-------------------------------------------------------
\begin{exercise}
Refer to Example~\ref{example:zelen.data.int.model} and
Figure~\ref{figure:zelencap.cellquan.int.ps}.  The two points
corresponding to the subexperiments at 250 and 300 volts and
180$\degreesc$ seem to be out of line from the other points. Use the
individual subexperiment results given in
Table~\ref{table:zelen.cap.data} to compute a confidence interval for
$t_{.5}$ at 300 volts and 180$\degreesc$.  What does this suggest
about the two outlying points?
\end{exercise}

%-------------------------------------------------------
\begin{exercise}
Refer to Example~\ref{example:zelen.data.int.model} and the numerical
results in Table~\ref{table:zelen.mles}. For both models,
compute the ML estimates of
$t_{.5}$ for the 8 factor-level combinations and plot these on a graph
like that in Figure~\ref{figure:zelencap.cellquan.int.ps}.  What do
you conclude?
\end{exercise}

%-------------------------------------------------------
\begin{exercise}
Consider the two-variable regression model
with interaction given in Section~\ref{section:two.var.with.int}.
Show that $\beta_{1} + \beta_{3}x_{2}$ is the change in $\mu$ for
a one-unit change in $x_{1}$.
\end{exercise}

%-------------------------------------------------------
\begin{exercise}
In some reliability applications it is common to use transformations
of explanatory variables in regression modeling (models involving such
transformations are described in
Chapter~\ref{chapter:accelerated.test.models}). Also, it might be
possible to find a distribution other than the Weibull that will
provide a better fit to the data.  Use the glass capacitor data
in Examples~\ref{example:zelen.data} and
\ref{example:zelen.data.int.model}, and fit the following
alternative models.
\begin{enumerate}
\item
\label{exer.part:zelen.tran.expvar}
In the two variable regression model, use the transformations
$x_{1}=1/(\mbox{temp}\degreesc+ 273.15)$ (known as the Arrhenius
relationship) and $x_{2}=\log(\mbox{Voltage})$ (known as the inverse
power relationship). 
Compare estimates of $t_{.5}$ at the different levels of
temperature and voltage used in the life test. Is there evidence
that these factors affect life? Explain.
\item
Repeat the analysis in part~\ref{exer.part:zelen.tran.expvar},
using the normal distribution instead of the Weibull distribution.
Again, compare estimates of $t_{.5}$ at the different levels of
temperature and voltage used in the life test.
\end{enumerate}
\end{exercise}

%-------------------------------------------------------
\begin{exercise}
The Weibull SAFT regression model is also a PH model. Thus fitting a
parametric PH model with a Weibull baseline distribution is equivalent
to fitting a Weibull regression model in which $\mu=\log(\weibscale)$
is a function of explanatory variables and $\sigma=1/\beta$ is
constant.
\begin{enumerate}
\item
Why is the Cox PH model called ``semiparametric?''
\item
What are some advantages of using the semiparametric Cox PH model,
when compared to using the parametric Weibull model.
\item
What are some advantages of using the parametric Weibull model, when
compared to using the semiparametric Cox PH model?  
\end{enumerate}
\end{exercise}

%-------------------------------------------------------
\begin{exercise1}
Show that a parametric SAFT model is also a parametric PH model if and
only if the underlying distribution is a Weibull.
Hint: divide the proof in the following steps:
\begin{enumerate}
\item
(Sufficient Condition). Assume that $\rv(\xvec_{0})$ has a Weibull
distribution. Then show that
$\rv(\xvec)=\rv(\xvec_{0})/\AF(\xvec)$ has
Weibull distribution and that
$h(\realrv; \xvec)=\Psi(\xvec) h(\realrv; \xvec_{0})$.
\item
(Necessary Condition). Assume that
$\rv(\xvec)=\rv(\xvec_{0})/\AF(\xvec)$ and
that
$h(\realrv; \xvec)=\Psi(\xvec) h(\realrv; \xvec_{0})$.
First show that
$S[\AF(\xvec) t;\xvec_{0}] = [S(t;\xvec_{0})]^{\Psi(\xvec)}$. 
Then argue that this can only be true if
a Weibull probability plot of $F(\realrv;\xvec_{0})$
is a straight line.
\end{enumerate}
\end{exercise1}
