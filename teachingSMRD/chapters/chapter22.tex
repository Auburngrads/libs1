%chapter 22
%original by wqmeeker  3 aug 1994
%edited by driker 11 dec 96
%edited by wqmeeker 14-15 dec 96
%edited by driker 19 dec 96
%edited by driker 7 apr 97
%edited by driker 2 july 97
%edited by driker 25 july 97
\setcounter{chapter}{21}

%\setcounter{page}{m}

\chapter{Case Studies and Further Applications}
\label{chapter:case.studies}


%{\Large {\bf William Q. Meeker and Luis A. Escobar}}\\
%Iowa State University and Louisiana State University\\[5ex]
{\large {\bf \today}}\\[2ex]
Part of 
{\em Statistical Methods for Reliability Data}\\
Copyright 1997 W. Q. Meeker and L. A. Escobar. \\[1ex]
To be published by John Wiley \& Sons Inc. in 1998.



%----------------------------------------------------------------------
%----------------------------------------------------------------------
\section*{Objectives}
This chapter 
\begin{itemize} 
\item 
Describes additional applications of the
reliability data analysis methods in this book.
\item 
Shows how to extend the general methods covered in the earlier
chapters to handle other special models and applications.
\item 
Provides additional discussion of some important practical aspects of
reliability data analysis applications.
\end{itemize}

%----------------------------------------------------------------------
%----------------------------------------------------------------------
\section*{Overview}
This chapter presents several case studies that illustrate some
additional important concepts and pitfalls of reliability data
analysis, shows how to integrate ideas taken from several different
places in the book, and presents some important additional examples.

Section~\ref{section:cen.mixed.pop} describes a serious problem that
can arise when different cohorts of units are censored unequally,
having the potential to lead to misleading conclusions about product
life.  Section~\ref{section:bayes.acceleration} applies Bayesian
methods from Chapter~\ref{chapter:singledist.bayes} to an
accelerated testing example that was first introduced in
Chapter~\ref{chapter:analyzing.alt.data}.  This example shows
how the introduction of prior information can importantly improve
the precision with which one can estimate a failure-time
distribution with an accelerated life test.
Section~\ref{section:crisk.mixture.model} illustrates the use of a
model that can be used to describe the failure-time distribution of
a product that has both infant mortality and wearout failure causes.
Section~\ref{section:fatigue.limit.model} suggests a physically
motivated model that nicely describes the features and complicated
relationship between fatigue life and applied stress or strain.
Finally,
Section~\ref{section:planning.accelerated.degradation.tests} shows
how to use simulation methods to plan an accelerated degradation test.

%----------------------------------------------------------------------
%----------------------------------------------------------------------
\section{Dangers of Censoring in a Mixed Population}
\label{section:cen.mixed.pop}

The life distribution of a product can change from one production
period to the next.  Changes in the design or method of manufacture
may improve product reliability.  This is especially likely for new products
undergoing reliability improvement efforts.  In other cases,
reliability may deteriorate due to the adverse consequences of a cost
reduction, change in raw materials, or a relaxation of process
monitoring standards.  Thus, field-tracking life data usually involve
a mixture of failure-time distributions.

\subsection{A conceptual example}
Suppose that a product had been manufactured in equal quantities in each of
two short production periods, one year apart.  Units from the two
periods had exponential life distributions (constant hazard rate) with
mean times to failure of $\theta_{1}=1$ year (constant hazard rate
$\lambda_{1}=1/\theta_{1}=1$) and $\theta_{2}=5$ years
($\lambda_{2}=.2$), respectively.  Due to their earlier availability,
the units made in period 1 were put into service approximately one
year earlier than those from period 2. An analysis of the
failure-time data, is performed two years after the first group (or
equivalently, one year after the second group) was put into service.

For the first year of operation, based upon combining the units from
both production periods, the average failure rate would be
approximately .6 [i.e., ($\lambda_{1}+\lambda_{2})/2 = (1 + .2)/2]$.
For the second year of operation, based upon the units only from
production period 1, the failure rate would be approximately 1.
Thus, because the production period 2 units with the lower failure
rate are mixed with production period 1 units with the higher
failure rate for the first year of operation (but not the second),
there is an incorrect indication of an increasing failure rate.  This is
so despite the fact that a population consisting of a mixture of two
different exponential distributions has a decreasing failure rate
(see Exercises~\ref{exercise:simple.mix.exp} and
\ref{exercise:mix.exp.haz.dec}).

\subsection{A numerical example}
To illustrate the dangers of censoring in this simple setting, we
simulated data from two populations with known characteristics.  We used
samples of 1000 units each from exponential distributions with
$\theta_{1}=1$ year (production period 1) and $\theta_{2}=5$ years
(production period 2), respectively.  The interval data are given in
Table~\ref{table:sim.mix.data}. The data available after two years
from the start of production (one year in service for the second
group) are plotted on Weibull paper in Figure~\ref{figure:sim.mix1.ps}
both separately and combined for the two groups.  Weibull probability
plots of the data after two years of operation for both groups are
given in Figure~\ref{figure:sim.mix2.ps}.  This is the plot that would
have been obtained after two years if both groups had been put into
service at the same time.  Curvature in the combined-sample plots in
Figures~\ref{figure:sim.mix1.ps} and
\ref{figure:sim.mix2.ps} suggests some
deviation from a Weibull distribution. Because the data were
generated from a mixture of exponentials, a single Weibull
distribution is not strictly correct in either case.

\begin{table}
\caption{Simulated data from two different production periods.}
\centering\small
\begin{tabular}{r@{\hspace{2ex}}rr@{\hspace{2ex}}r@{\hspace{3ex}}rr}
\\[-.5ex]
\hline
& \multicolumn{3}{c}{Production Period 1}
& \multicolumn{2}{c}{Production Period 2}\\
 & \multicolumn{3}{c}{$\theta=1$ year}
& \multicolumn{2}{c}{$\theta=5$ years}  \\[.5ex]
&  \multicolumn{3}{c}{Data After} 
&\multicolumn{2}{c}{Data After} \\
\cline{5-6}
Time to Failure&  \multicolumn{3}{c}{2 Years of} & \multicolumn{1}{c}{1 Year of}
& \multicolumn{1}{c}{2 Years of}  \\
\multicolumn{1}{c}{(years)}&& \multicolumn{1}{c}{Operation} && \multicolumn{1}{c}{Operation}
& \multicolumn{1}{c}{Operation}  \\
\hline
   0.0-0.2    &&  185  &   & 33    & 33  \\
   0.2-0.4    &&  163  &   & 33    & 33  \\
   0.4-0.6    &&  134  &    & 35    & 35  \\
   0.6-0.8    &&  90   &   & 39    & 39  \\
   0.8-1.0    &&  83   &   & 45    & 45  \\
   1.0-1.2    &&  58   &   &     & 34  \\
   1.2-1.4    &&  42   &   &     & 33  \\
   1.4-1.6    &&  44   &   &     & 31  \\
   1.6-1.8    &&  36   &   &     & 24  \\
   1.8-2.0    &&  35   &   &     & 31  \\
  $> 1.0$     &&       &   & 815   &   \\
  $> 2.0$    &&   130  &   &       & 662  \\
\hline      
\end{tabular}\\
\begin{minipage}[t]{4in}
The total sample size for each production period was 1000 units.
Data from Hahn and Meeker~(1982b).
\end{minipage}
\label{table:sim.mix.data}
\end{table}

%-------------------------------------------------------------------
\begin{figure}
\splusbookfigure{\figurehome/sim.mix1.ps}
\caption{Weibull probability plot of failure data for units from
different production periods after two years since production startup
(but only one year of operation for the second production period).}
\label{figure:sim.mix1.ps}
\end{figure}
%-------------------------------------------------------------------

%-------------------------------------------------------------------
\begin{figure}
\splusbookfigure{\figurehome/sim.mix2.ps}
\caption{Weibull probability plot of failure data for units from
different production periods after two years operating time
for both production periods.}
\label{figure:sim.mix2.ps}
\end{figure}
%-------------------------------------------------------------------

Analysis 1 is based upon the combined data after two years since
production startup (one year of operation for units from production
period 2), i.e., the crosses in Figure~\ref{figure:sim.mix1.ps}.
Analysis 2 is based upon the combined data after two years of
operation for the units from both production periods, i.e., the
crosses in Figure~\ref{figure:sim.mix2.ps}
\begin{itemize}
\item
{\bf Analysis 1.} A Weibull distribution is fit to the data with
unequal right-censoring for the two production periods.  A 95\%
confidence interval for the Weibull distribution shape parameter is
$[1.02,\quad 1.15]$, incorrectly indicating an increasing hazard
rate with time (as expected from the conceptual example).
\item
{\bf Analysis 2.}  A Weibull distribution is fit to the data with
equal right-censoring at 2-years for both production periods. A 95\%
confidence interval for the Weibull distribution shape parameter is
$[.82,\quad .92].$ This indicates a decreasing hazard rate with
time, as expected from theory (see Proschan 1963).
\end{itemize}

An appropriate analysis of the unequally censored data would fit
separate exponential distributions for the two production
periods. This analysis (left as an exercise) gives estimates of
failure probabilities and quantiles that agree well with the true
model from which the data were simulated.

\subsection{Analyzing data from different production periods}

Because reliability can change over time, it is advisable, when possible, to
conduct separate analyses for each production period, to compare the
results, and to combine them only if this seems appropriate.  Separate
analyses, however, are not possible in the following circumstances.

\begin{itemize}
\item
A unit's production period is not known.
\item
Production periods are not well defined. For example, when
production is continuous, there may not be well defined points in
time where the process has changed.
\item
The data are too scanty for reasonable dissection.
\end{itemize}

In most practical data analysis problems, available data could be
viewed as having come from two or more populations. Analyses are
most often done with the pooled data. This is appropriate when
interest centers on failure-time distribution of the mixture and
\begin{itemize}
\item
There are only small differences among the populations or
\item
The amount of censoring is approximately the same over the
different populations.
\end{itemize}

The effect of fitting a simple distribution to mixtures of two (or more)
populations with unequal censoring depends on the degree of
dissimilarity between or among the subpopulations and on the
relative number of units produced in the two periods.  The example
in this section illustrates the desirability of doing separate
analyses for units from different production periods, especially
when field exposure periods differ for the different groups.  If
separate analyses are not possible because of the sparsity of the
data or limitations in identifying the production period, one needs
to recognize that a seriously incorrect model can give misleading
results.

%----------------------------------------------------------------------
%----------------------------------------------------------------------
\section{Using Prior Information in Accelerated Testing}
\label{section:bayes.acceleration}
This section uses an extension of the Bayesian methods presented in
Chapter~\ref{chapter:singledist.bayes}, to reanalyze the data from
Example~\ref{example:new.tech.ic}.  Lerch and Meeker~(1998) present
similar examples. The computational methods used here, like
those used in Chapter~\ref{chapter:singledist.bayes}, follow
Smith and Gelfand~(1992).

Example~\ref{example:new.tech.ic} illustrated the analysis of
accelerated life test data on a new-technology IC device. As a
contrast, Example~\ref{example:new.tech.ic.beta.known} showed how
much smaller the confidence intervals on $F(t)$ would be if the
Arrhenius activation energy were known. Generally it is unreasonable
to assume that a parameter like activation energy is known
exactly. For some applications, however, it may be useful or even
important to bring outside knowledge into the analysis. Otherwise it
would be necessary to spend scarce resources to conduct experiments
to learn what is already known. In some applications, knowledgeable
reliability engineers can, for example, specify the approximate
activation energy for different expected failure modes.  Translating
the information about activation energy into a prior distribution
will allow the use of Bayesian methods like those introduced in
Chapter~\ref{chapter:singledist.bayes}. This section shows how to
incorporate prior information on the activation energy for a failure
mode into an analysis of the new-technology IC device data.

\subsection{Prior distributions}

Section~\ref{section:prior.dist} describes different kinds of prior
information.  This section reanalyzes the new-technology IC
device ALT data in order to compare 
\begin{itemize}
\item
A diffuse (wide uniform) prior distribution for $\Ea$.
\item
A given value (degenerate prior distribution) for $\Ea$.
\item
The engineers' prior information, converted into an informative
prior distribution for $\Ea$.
\end{itemize}

On the basis of previous experience with a similar failure mode, the
engineers responsible for this device felt that it would be safe to
presume that, with a ``high degree of certainty,'' the activation energy
$\Ea$ is somewhere in the interval .80 to .95. They also felt that a
normal distribution could be used to describe the uncertainty in
$\Ea$. We use normal distribution 3-$\sd$ limits (i.e., mean $\pm$
three standard deviations) to correspond to an interval with a
high degree of certainty, corresponding to about 99.7\%
probability. This is an informative prior distribution for
$\Ea$. The engineers did not have any firm information about the
other parameters of the model. To specify prior distributions for
the other parameters, it is then appropriate to choose a diffuse
prior. A convenient choice is a UNIF distribution that extends far
beyond the range of the data and physical possibility. As
described in Chapter~\ref{chapter:singledist.bayes}, the parameters
used to specify the joint posterior distributions should be given in
terms of parameters that can be specified somewhat independently and
conveniently. For this example, the prior distributions for $\sigma$
was specified as $\UNIF(.2,.9)$ and the prior distribution for
$t_{.1}$ at 250$\degreesc$ was specified to be $\UNIF(500,7000)$
hours. Comparison with the ML estimates from
Example~\ref{example:new.tech.ic} shows that the corresponding joint
uniform distribution is relatively diffuse.

Figure~\ref{figure:nt.device.Ea.dist.ps} compares a NOR prior
distribution with a 3-$\sd$ range of $(.80,.95)$ and a
$\UNIF(.4,1.4)$ (diffuse) prior for $\Ea$. The corresponding marginal
posterior distributions for $\Ea$ are also shown. 
%-------------------------------------------------------------------
\begin{figure}
\splusbookfigure{\figurehome/nt.device.Ea.dist.ps} 
\caption{Plot of diffuse and informative prior 
distributions for the new technology device activation energy $\Ea$
along with corresponding posterior distributions.}
\label{figure:nt.device.Ea.dist.ps}
\end{figure}
%-------------------------------------------------------------------
The center of the marginal posterior distribution for $\Ea$
corresponding to the informative NOR prior is very close to that of
the prior itself. This is mostly because the prior is strong
relative to the information in the data.
Figure~\ref{figure:nt.device.Ea.dist.ps} also shows a posterior
distribution corresponding to the uniform (diffuse) prior. The
corresponding joint posterior is approximately proportional to the
profile likelihood for $\Ea$. The uniform prior has had little
effect on the posterior and therefore is approximately
noninformative.

In addition to activation energy $\Ea$, the reliability engineers
also wanted to estimate life at 100$\degreesc$.
Figure~\ref{figure:nt.device.bayes.quant.ps} shows different
posterior distributions for $t_{.01}$ at 100$\degreesc$.
%-------------------------------------------------------------------
\begin{figure}
\splusbookfigure{\figurehome/nt.device.bayes.quant.ps} 
\caption{Plot of the marginal posterior distribution of $\rvquan_{.01}$
at 100$\degreesc$ for the new technology device, based on different
assumptions. NW corner: all data and an informative prior for $\Ea$. NE
corner: all data and a diffuse prior for $\Ea$. SW corner: drop
300$\degreesc$ data and an informative prior for $\Ea$. SE corner: drop
300$\degreesc$ data and given $\Ea = .8$. The vertical lines are two-sided
95\% Bayesian confidence intervals for $t_{.01}$ at 100$\degreesc$.}
\label{figure:nt.device.bayes.quant.ps}
\end{figure}
%-------------------------------------------------------------------
The plots on the top row of
Figure~\ref{figure:nt.device.bayes.quant.ps} compare posteriors
computed under the informative and diffuse prior distributions for
$\Ea$. This comparison shows the strong effect of using the prior
information in this application.

Recall from Example~\ref{example:new.tech.ic} that there was some
concern (because of the different slopes in
Figure~\ref{figure:icdevice02.groupi.lognor.ps}) about the
possibility of a new failure mode occurring at
300$\degreesc$. Sometimes physical failure mode analysis is useful
for assessing such uncertainties. In this application the
information was inconclusive.

When using ML estimation or when using Bayesian methods with a
diffuse prior for $\Ea$, it is necessary to have failures at two or
more levels of temperature in order to be able to extrapolate to
100$\degreesc$.  With a given value of $\Ea$ or an informative prior
distribution on $\Ea$, however, it is possible to use Bayesian methods
to estimate $t_{.01}$ at 100$\degreesc$ with failures at only one
level of temperature.  The posterior distributions in the bottom row
of Figure~\ref{figure:nt.device.bayes.quant.ps} assess the effect of
dropping the 300$\degreesc$ data, leaving failures only at
250$\degreesc$. Comparing the graphs in the NW and SW corners, the
effect of dropping the 300$\degreesc$ data results in a small
leftward shift in the posterior. Relative to the confidence
intervals, however, the shift is small. Comparing the two plots in
the bottom row suggests that using a given value of $\Ea=.8$ results
in an interval that is probably unreasonably narrow and potentially
misleading. If the engineering information and previous experience
used to specify the informative prior on $\Ea$ is credible for the
new device, then the SW analysis provides an appropriate compromise
between the commonly used extremes of assuming nothing about $\Ea$
and assuming that $\Ea$ is known.

If one tried to compute the posterior after dropping the
300$\degreesc$, using a uniform prior distribution on $\Ea$, the
posterior distribution would be strongly dependent on the range of
the uniform distribution.  This is because with failures only at
250$\degreesc$, there is no information on how large $\Ea$ might
be. In this case there would  be no approximately uninformative prior
distribution.

To put the meaning of the results in perspective, the analysis
based on the informative prior distribution for $\Ea$ after dropping
the suspect data at 300$\degreesc$ would be more credible than the
alternatives. The 95\% Bayesian confidence intervals for $t_{.01}$
at 100$\degreesc$ for this analysis are $[.6913, \quad 2.192]$
million hours or $[79, \quad 250]$ years. This does not imply that
the devices will last this long (we are quite sure that they will
not!). Instead, the results of the analysis suggest that, if the
Arrhenius model is correct, this particular failure mode is unlikely
to occur until far beyond the technological life of the system into
which the IC would be used. It is likely, however, that there are
other failure modes (perhaps with smaller $\Ea$) that will be
observed, particularly at lower levels of temperature (see also the
discussion of failure mode masking in
Section~\ref{:section.masked.failure.mode}).

%----------------------------------------------------------------------
%----------------------------------------------------------------------
\section{An LFP/Competing Risk Model}
\label{section:crisk.mixture.model}

Chan and Meeker~(1998) describe a model that combines components
from the LFP model (for infant mortality, as described in
Section~\ref{section:lfp.model}) with a competing risk model (for
longer-term wearout, as described in
Section~\ref{section:competing.risk}).  This model is called the
Generalized Limited Failure Population (GLFP) model.  This section
briefly describes the GLFP model and results of using maximum likelihood
to estimate the parameters of the model.

%----------------------------------------------------------------------
\subsection{Background}

Consider the Vendor 1
data in Examples~\ref{example:electronic.subsystem.data},
\ref{example:pretest.and.left.trun}, and \ref{example:pretest.mle}. 
Most of the early failures were known to have been caused by
defective integrated circuits (Mode 1). Only a small proportion of the
circuit packs would contain an Integrated Circuit (IC) with such a
defect (something like 1\% or 2\% was expected for this particular
technology).  After about 2000 to 4000 hours, however, the failure
rate began to increase and there was some evidence (both in the data
and some limited physical failure analysis) that the latter failures
were being caused by a combination of a corrosion and another
chemical degradation failure mode to which all units would
eventually succumb (Mode 2).  This can be seen in the Weibull probability
plot shown in Figure~\ref{figure:vendor1.cdf.ps}, where the plotted
points change direction after 2000 hours.
%----------------------------------------------------------------------
\begin{figure}
\splusbookfigure{\figurehome/vendor1.cdf.ps}
\caption{Weibull probability plot comparing the Weibull/Weibull and the
Weibull/lognormal GLFP competing risk $F(t;\thetavechat$) ML estimates for the
Vendor 1 circuit pack failure data.}
\label{figure:vendor1.cdf.ps}
\end{figure}
%----------------------------------------------------------------------
Although something like this secondary failure mode had been
expected, the managers responsible for the operation of the system
in which the circuit packs were to be deployed, were concerned at
how early such failures were beginning to appear. They were
interested in obtaining a prediction for the proportion of units
that would fail in the first 5 years (43,800 hours) of operation
(approximate technological life of the system).

%----------------------------------------------------------------------
\subsection{The GLFP model}
 Let $T_{1}$ denote the ``infant mortality'' failure time for a
unit. If a unit is not defective, then $T_{1}=\infty$.  As with the
LFP model described in Section~\ref{section:lfp.model}, the cdf for
$T_{1}$, conditional on the unit being defective, is
$F_{1}(t;\thetavec_{1})$, where $\thetavec_{1}$ is a vector of
unknown parameters.  The unconditional cdf of $T_{1}$ is
$pF_{1}(t;\thetavec_{1})$, where $p$ is the proportion of defective
units in the population.  Similarly, let $T_{2}$ denote the unit's
wearout failure time and let $F_{2}(t;\thetavec_{2})$ denote the cdf
of $T_{2}$. 
The unit's actual
failure time is $T =\min(T_{1},T_{2})$. 
As in (\ref{equation:two.comp.series}), if $T_{1}$ and
$T_{2}$ are independent, the cdf of failure time $T$ is
\begin{equation}
\label{equation:glfp.cdf}
F_{T}(t;\thetavec) =
\Pr(T \leq t) = 1 - [1 - pF_{1}(t;\thetavec_{1})][1 -
F_{2}(t;\thetavec_{2})]
\end{equation}
where $\thetavec = (\thetavec_{1},\thetavec_{2}).$ The pdf of $T$ is
then given by 
\begin{equation}
\label{equation:glfp.pdf}
f_{T}(t;\thetavec) = \frac{dF_{T}(t;\thetavec)}{dt}
= pf_{1}(t;\thetavec_{1})[1-F_{2}(t;\thetavec_{2})] +
f_{2}(t;\thetavec_{2})[1-pF_{1}(t;\thetavec_{1})]
\end{equation}
where
$f_{1}(t;\thetavec_{1})$ and $f_{2}(t;\thetavec_{2})$ are the pdfs
of $T_{1}$ and $T_{2}$, respectively.
Chan and Meeker~(1998) used Weibull and lognormal distributions for
$F_{1}(t;\thetavec_{1})$ and $F_{2}(t;\thetavec_{2})$. Other
distributions could, however, be substituted without difficulty.

%----------------------------------------------------------------------
\subsection{Likelihood contributions}
%----------------------------------------------------------------------
The likelihood for the circuit pack data can be written using the
general form for independent right-censored and interval-censored
observations given in
(\ref{equation:general.parametric.likelihood}). The contributions
for the individual observations for the GLFP model depend, however,
on whether the cause of failure is known or not and, if so, on which
type of failure occurred.

If the cause of failure is {\em known}, then either $T_{1}$ or
$T_{2}$ is also known.  Otherwise if failure cause is {\em unknown}
then only $T$ is known.  If $T_{1}<T_{2}$ the unit fails from a
defective IC and $T_{2}$ is not observed. Similarly, if
$T_{2}<T_{1}$, the unit fails from a wearout mode and $T_{1}$ is not
observed.

If unit $i$ is known to have failed between times $t_{i-1}$ and
$t_{i}$ from failure Mode 1, the probability of the
observation is 
\begin{eqnarray*} 
\like_{i}(\thetavec) &=&
\Pr[(t_{i-1} < T \leq t_{i})
\cap (T_{1}<T_{2})] \\
               &=& \Pr[(t_{i-1} < T_{1} \leq t_{i}) \cap 
                               (T_{1}<T_{2})] \\   
                       &=& \int_{t_{i-1}}^{t_{i}}
\int_{s}^{\infty} pf_{1}(s;\thetavec_{1})f_{2}
             (v;\thetavec_{2})\,dv\,ds \\ &=&
          \int_{t_{i-1}}^{t_{i}} p f_{1}(s;\thetavec_{1})
          [1-F_{2}(s;\thetavec_{2})]\,ds
\end{eqnarray*} 
where, as before, $T =$ min$(T_{1},T_{2}).$ Similarly, if the cause
of failure is Mode 2, 
\begin{displaymath}
\like_{i}(\thetavec) =
\int_{t_{i-1}}^{t_{i}}f_{2}(s;\thetavec_{2})
	[1-pF_{1}(s;\thetavec_{1})]\,ds.
\end{displaymath}

If the cause of failure is not known, then
\begin{eqnarray*}
      \like_{i}(\thetavec) &=& \Pr(t_{i-1} < T \leq t_{i}) \\
    &=& F_{T}(t_{i};\thetavec)- F_{T}(t_{i-1};\thetavec) \\
    &=& [1 - pF_{1}(t_{i-1};\thetavec_{1})][1 -F_{2}(t_{i-1};
\thetavec_{2})] 
- [1 - pF_{1}(t_{i};\thetavec_{1})][1 - F_{2}(t_{i};\thetavec_{2})].
\end{eqnarray*}
For a right-censored observation at time $t_{i}$ (i.e., the failure
time is after $t_{i}$), then
\begin{eqnarray*}   
     \like_{i}(\thetavec) &=& \Pr(T > t_{i}) \\ &=&
   [1-pF_{1}(t_{i};\thetavec_{1})][1-F_{2}(t_{i};\thetavec_{2})].
\end{eqnarray*} 
Chan and Meeker~(1998) also give expressions for left-censored and
exact failure-time observations.

%----------------------------------------------------------------------
\subsection{ML estimates}
%----------------------------------------------------------------------
To develop predictions for the proportion failing in 5 years, we fit
the Weibull/Weibull and Weibull/lognormal models to the available
data. We did this by supposing, on the basis of engineering
judgment, that the failures before 200 hours were due to defective
components and that failures after 5000 hours were due to a wearout
mechanism. No assumption was used for the failures between 200 and
5000 hours.  Figure~\ref{figure:vendor1.cdf.ps} shows a Weibull
probability plot comparing the Weibull/Weibull and the
Weibull/lognormal GLFP competing risk models for the Vendor 1
circuit pack failure data.  Although the two different distributions
provide excellent agreement within the range of the data, they
differ importantly in extrapolation.  It is interesting to note
that, as we have seen in other similar examples, the lognormal
extrapolation provides a much more optimistic (smaller) prediction
of the future proportion of units that will fail.
%-----------------------------------------------
\begin{figure}
\splusbookfigure{\figurehome/vendor1.lognor.cdf.ci.ps}
\caption{Weibull probability plot showing the ML estimate
of the Weibull/lognormal GLFP competing risk cdf with pointwise
approximate 95\% confidence intervals for the Vendor 1 circuit pack
failure data.}
\label{figure:vendor1.lognor.cdf.ci.ps}
\end{figure}
%-----------------------------------------------
Figure~\ref{figure:vendor1.lognor.cdf.ci.ps} shows the ML estimate
of the Weibull/lognormal GLFP competing risk cdf along with
pointwise approximate 95\% confidence intervals for the Vendor 1
circuit pack failure data.  At $10^{4}$ hours, the confidence
interval is narrow because of the large sample size ($n=4993$). At
$10^{5}$ hours, the confidence interval is wide, ranging from about
.3 to .85. This is due to the large amount of extrapolation. It is
important to note, however, that the width of this confidence
interval does {\em not} reflect the deviation (which almost
certainly exists) from the assumed Weibull/lognormal
GLFP distribution. Indeed, as seen in Figure~\ref{figure:vendor1.cdf.ps},
the Weibull/Weibull ML estimate at $10^{5}$ hours is almost 1.

Figure~\ref{figure:vendor1.hazard.ps} shows a plot of the
corresponding Weibull/Weibull and the Weibull/lognormal GLFP
competing risk hazard function ML estimates.
%-----------------------------------------------
\begin{figure}
\splusbookfigure{\figurehome/vendor1.hazard.ps}
\caption{Plot comparing the Weibull/Weibull and the
Weibull/lognormal GLFP competing risk mode hazard function estimates
for the Vendor 1 circuit pack failure data.}
\label{figure:vendor1.hazard.ps}
\end{figure}
%-----------------------------------------------
The hazard function
was computed as $h_{T}(t)=f_{T}(t;\thetavec)/[1-
F_{T}(t;\thetavec)]$ where $F_{T}(t;\thetavec)$ and
$f_{T}(t;\thetavec)$ are defined in (\ref{equation:glfp.cdf}) and
(\ref{equation:glfp.pdf}), respectively. The plot shows the hazard
decreasing until the corrosion/degradation failure mode becomes
active, at which time the hazard increases markedly. We also fit
lognormal/Weibull and the lognormal/lognormal GLFP models, but the
estimates of the cdf and hazard function estimates were, for all
practical purposes, the same as the Weibull/Weibull and
Weibull/lognormal models, respectively. The shape of $h_{T}(t)$
results from adding the decreasing hazard of $T_{1}$ to the increasing
hazard for $T_{2}$.

%----------------------------------------------------------------------
%----------------------------------------------------------------------
\section{Fatigue-limit Regression Model}
\label{section:fatigue.limit.model}

 
Section~\ref{section:quad.regr} illustrated the fitting of a
quadratic regression model to nickel-base superalloy fatigue data
from Nelson~(1984).  Fitting a quadratic function is relatively easy
to do and may be satisfactory for some purposes.  Alternative
functional forms, however, may provide a better description of the
data. Pascual and Meeker~(1998a) use ML methods to
fit an alternative regression model that contains a fatigue limit
parameter. The material in this section has been adapted from this
paper.

Under the fatigue-limit model, specimens operated at levels of
stress below the fatigue-limit will never fail.  The fatigue-limit
model also allows the standard deviation of fatigue life to be a
function of stress.  The purpose of the analysis is to obtain an
estimate of the small quantiles of the fatigue-life distribution.

\begin{figure}
\splusbookfigure{\figurehome/nelson.scatter.ps}
\caption{Log-log S-N (stress versus number of cycles)
plot for the superalloy data with ML
estimates of the .05, .5 and .95 quantiles from the fatigue-limit
model ($\bullet$ failure, $\triangleright$~censored).}
\label{figure:nelson.scatter.ps}
\end{figure}

%----------------------------------------------------------------------
\subsection{The fatigue-limit model}
Let $x_{1}$,\,\ldots, $x_{n}$ denote pseudostress levels of $n$
specimens and let $t_{1}$,\,\ldots, $t_{n}$ be actual failure times
or censoring times. Censoring times may vary from specimen to specimen.  Let
$\gamma$ be the fatigue limit.  At each pseudostress level with
$x_{i}>\gamma$, fatigue life $t_{i}$ is modeled with a lognormal
distribution, i.e., the cumulative proportion failing function and
its derivative are given by
\begin{eqnarray}
\label{equation:fatigue.limit.cdf}
\Pr(\rv \leq t)=
F[\realrv;\mu(x),\sigma(x)]&=&\Phi\hspace{-1.2 mm}\left[\frac{\log(\realrv)-\mu(x)}{\sigma(x)}\right]\\
f[\realrv;\mu(x),\sigma(x)]&=&\frac{1}{
	\sigma(x)\realrv}\phi\hspace{-1.2 mm}
	\left[\frac{\log(\realrv)-\mu(x)}{\sigma(x)}\right], 
	\;\quad\realrv>0. \nonumber
\end{eqnarray}
For example, using $\Phi_{\nor}$ and $\phi_{\nor}$ implies that
$\log(\rv)$ is modeled with a normal distribution with mean $\mu(x)$
and standard deviation $\sigma(x)$.  These parameters are related to
stress according to
\begin{eqnarray}
\label{equation:mu.model}
\mu(x)&=&\mbox{E}[\log(\rv)]=
	\beta^{[\mu]}_{0}+\beta^{[\mu]}_{1}\log(x-\gamma),
\;x>\gamma\\
\label{equation:sigma.model}
\sigma(x)&=&\sqrt{\mbox{Var}[\log(\rv)]}=
	\exp[\beta^{[\sigma]}_{0}+\beta^{[\sigma]}_{1}\log(x)],
	\;x>\gamma
\end{eqnarray}
where $\beta^{[\mu]}_{0}$, $\beta^{[\mu]}_{1}$,
$\beta^{[\sigma]}_{0}$, $\beta^{[\sigma]}_{1}$ and $\gamma$ are
unknown parameters to be estimated from data.  If $x_{\rm minf}$ is the
smallest observed stress level that yields a failure, then $\gamma$
must be in the interval $[0,x_{\rm minf})$.
 
Note that when $\beta^{[\sigma]}_{1}=0$, the model has a constant
standard deviation.  In most fatigue data, the standard deviation
decreases as stress increases, which corresponds to
$\beta^{[\sigma]}_{1}<0$.  The scatter plot of the superalloy data
(and the fitted model) in Figure~\ref{figure:nelson.scatter.ps} has
cycles to failure on the horizontal axis, as is commonly done in the
fatigue literature. This scatter-plot indicates more scatter at the
lower stress levels and less at the higher stress levels.

The value of $\gamma$ determines the amount of curvature present in
the plotted S-N curve for values of stress that are not far from
$x_{\rm minf}$. When $\gamma$ is close to zero, the S-N curve is close
to linear.  Larger values of $\gamma$ result in more curvature in
the plot.  When $\gamma=0$, the model is equivalent to the simple
linear regression model used in
Section~\ref{section:simple.linear.reg}.  Curvature in
Figure~\ref{figure:nelson.scatter.ps} suggests the inclusion of a
fatigue limit $\gamma$ in the model.  Although a fixed fatigue limit
may be unrealistic for describing a population of specimens, the
fatigue limit provides a physically appealing alternative to the
quadratic term in the $\mu(x)$ relationship used in
Section~\ref{section:quad.regr} for describing S-N curvature.
 
The maximum likelihood methods described in the next section use the
following assumptions: a) specimens are tested independently and b)
for $x > \gamma$ the times at which observations were censored are
independent of actual failure times that would be observed if the
experiment were to be run until failure.

\subsection{Maximum likelihood estimation}
The parameters of the fatigue limit model can be estimated by using
the method of maximum likelihood in a manner that is very similar to
that described in Chapter~\ref{chapter:regression.analysis}.  As
before, we use
$\thetavec=(\beta^{[\mu]}_{0},~\beta^{[\mu]}_{1},~\beta^{[\sigma]}_{0},~\beta^{[\sigma]}_{1},~\gamma)$
to denote the vector of model parameters. The log likelihood
function is
\begin{displaymath}
\loglike(\thetavec)=\log[\like(\thetavec)]=\sum_{i=1}^{n}\loglikei(\thetavec)
\end{displaymath}
where 
\begin{displaymath}
\loglikei(\thetavec)=\delta_{i}\{\log[\phi(z_{i})]-\log[\sigma(x_{i})t_{i})]\}+(1-\delta_{i})\log[1-\Phi(z_{i})]
\end{displaymath}
where $\delta_{i}=1$ ($\delta_{i}=0$) if observation $i$ is a
failure (right-censored observation)
and $z_{i}=[\log(t_{i})-\mu(x_{i})]/\sigma(x_{i})$.
 
The ML estimate $\thetavec$ is
the set of parameter values that maximize
$\loglike(\thetavec)$.  Table~\ref{table:mle.nelson} gives the
ML estimates of all model parameters resulting from fitting the
fatigue-limit model to the data. This table also shows normal-approximation
and likelihood confidence intervals for the parameters.
Figure~\ref{figure:nelson.scatter.ps} shows curves of the ML estimates
of the .05, .5 and .95 quantiles of fatigue life.

\begin{table}[htb]\centering
\begin{tabular}{crrr} \hline
Parameter        & Estimate & \multicolumn{2}{c}{Approximate}\\
                 &          & \multicolumn{2}{c}{95\% Confidence Interval}\\ \cline{3-4}
&&Normal-Theory & Likelihood-Ratio\\  \hline  
$\beta^{[\mu]}_{0}$    & 14.75  & (12.06, \quad 17.44)   & (12.90, \quad 21.45)     \\  
$\beta^{[\mu]}_{1}$    & $-1.39$  & $(-2.02, \quad -.76)$   &
$(-2.81, \quad -.92)$  \\  
$\beta^{[\sigma]}_{0}$ & 10.97  & (3.82, \quad 18.12)    & (3.22, \quad 17.90)      \\ 
$\beta^{[\sigma]}_{1}$ & $-2.50$  & $(-4.04, \quad -.96)$   &
$(-3.98, \quad -.81)$ \\  
$\gamma$               & 75.71  & (67.35, \quad 84.06)   & (49.98, \quad 79.79)  \\ \hline
\end{tabular}
\caption{Maximum likelihood results for the superalloy data}
\label{table:mle.nelson}
\end{table}

Nelson~(1984, pages 72-73) comments that the quadratic
fatigue life models produce quantiles larger at an
intermediate stress than at a lower stress. Such a relationship is physically
implausible. Although such behavior is also theoretically possible for the
fatigue-limit model, it seems to be less of a problem and
does not occur within the range of interest
for these data.
 
\subsection{Profile likelihoods and likelihood-ratio-based
confidence regions} 

The profile likelihood for $\gamma$ is defined by
\begin{displaymath}
R(\gamma)=\max_{\thetavec}
	\left[\frac{\like(\thetavec,\gamma)}
	{\like(\thetavechat)}\right]
\end{displaymath}
and is shown, for the superalloy data, in
Figure~\ref{figure:nelson.like.ps}.  The likelihood confidence
interval for $\gamma$ in Table~\ref{table:mle.nelson} is indicated by
the vertical lines where $R(\gamma)$ intersects the horizontal
critical level. Note that the upper bound of the normal-approximation
interval exceeds $x_{\rm minf}=80.3$. This will never happen with the
likelihood interval.

The confidence intervals in Table~\ref{table:mle.nelson} indicate that
the parameters $\beta^{[\mu]}_{1}$, $\beta^{[\sigma]}_{1}$ and
$\gamma$ are different from zero. The confidence intervals for
$\beta^{[\sigma]}_{1}$ indicate that the standard deviation of fatigue
life depends on the stress level and, moreover, that the standard
deviation decreases as stress increases, a commonly observed
phenomenon in metal fatigue data.  The confidence intervals for
$\gamma$ support the inclusion of a fatigue limit as suggested by the
curvature in Figure~\ref{figure:nelson.scatter.ps}. Similar confidence
intervals could also be computed for functions of the parameters, following
the methods described in Chapter~\ref{chapter:parametric.ml.ls}.

\begin{figure}
\splusbookfigure{\figurehome/nelson.like.ps}
\caption{Profile likelihood plot for the Fatigue Limit $\gamma$ for the
superalloy data}
\label{figure:nelson.like.ps}
\end{figure}

%----------------------------------------------------------------------
%----------------------------------------------------------------------
\section{Planning Accelerated Degradation Tests}
\label{section:planning.accelerated.degradation.tests}

Morse and Meeker~(1998) describe the use of simulation methods to
help plan accelerated degradation tests.  This section has been
adapted from their work.

The motivating problem arose from an extension of the accelerated
degradation analysis described in
Example~\ref{example:device-b.degradation.data}.  The Device-B
accelerated test had been designed as an accelerated life test. The
engineers were going to define failure time to be the time at which
power output first dropped .5 decibels (dB) below initial output.
After learning about the accelerated degradation analysis methods,
the engineers wanted to know how to design an accelerated test with
degradation analysis in mind and how to assess the potential
advantage of using degradation methods. In particular, they were
interested in seeing whether, with a different test plan, they could
expect to obtain better precision for estimating $\Fhat(130000)$,
the proportion failing at 130 thousand hours at $80\degreesc$ junction
temperature.

\subsection{Design parameters and test constraints}

The evaluation of alternative test plans will be based on the
information obtained in the initial study (see
Examples~\ref{example:device-b.degradation.data} through
\ref{example:device.b.basic.estimate}).  The test plans will be
evaluated under the following constraints:
%
\begin{itemize} 
\item The test will use 
3 levels of accelerated temperature.
\begin{itemize}
\item The highest test temperature will be 237$\degreesc$.
\item The middle temperature will be halfway between 237$\degreesc$ and the
low temperature (on the Arrhenius scale).
\end{itemize}
\item 20\% of the units on test will be allocated to the middle
level of temperature.
\item 
There is a constraint on the overall number of test positions and
the length of the test. To get useful information, it is necessary
to test units at lower temperatures for longer periods of time. The
time to leave a unit on test at a particular level of temperature (censoring
time) will be found through a censoring function:
\begin{itemize}
\item Censoring time = $ - 15621 + 730 \times 11605/(273.15+\Tempc{})$ hours. 
This was the approximate censoring function used in the original
study.
\item 
A total of 
67,000 test position-hours will be available for testing, as in
the original study (an average of about 22 test positions over 3000 hours).
\end{itemize}
%
The parameters of the study that will be varied are
\begin{itemize}
\item The low level of test-plan temperature, denoted by $\Temp_{L}$.
\item The proportion of units allocated to $\Temp_{L}$,
denoted by $\pi_{L}$.
\end{itemize}
\end{itemize}
Because the highest accelerated temperature is $\Temp_{H}=
237\degreesc$ and the
middle temperature is halfway between the low and high, specifying
$\Temp_{L}$ fixes all three accelerated temperatures.  Likewise,
because 20\% of the units will always be allocated to the middle
temperature $(\pi_{M}=.2)$, 
specifying the proportion allocated to $\Temp_{L}$
fixes the allocations to all temperatures.
\subsection{Evaluation of test plan properties}

Chapters~\ref{chapter:test-planning} and \ref{chapter:alt.test.planning} 
showed how to evaluate the properties of proposed test plans by 
\begin{itemize}
\item Using large-sample approximations (easy to compute for simple
problems, but the approximation may not be adequate with small samples).
\item Using Monte Carlo simulation (requires much more computing time, but
does not rely on large-sample approximations).
\end{itemize}
For the nonlinear accelerated degradation models, easy-to-compute large-sample
approximations have not been derived. To answer the engineers questions
quickly, Monte Carlo simulation was used.

\subsection{Simulation procedure}

This section shows how to simulate the accelerated degradation
experiment. Model parameter estimates from the original study were
used as planning values (see
Example~\ref{example:device.b.basic.estimate}). Test plan properties
can be computed for test plans specified by different combinations
of $\Temp_{L}$ and $\pi_{L}$.  These evaluations can be used to
determine if a different test plan would yield more precise
estimates of the failure-time distribution.  The procedure for doing
this is as follows:
%
\begin{enumerate}
\item Fix the test plan by specifying $\Temp_{L}$ and $\pi_{L}$
from a candidate list of combinations.
\item Randomly generate degradation sample paths for each of the $n$
units according to the specified test plan, using parameter estimates from
the original study as planning values. Add simulated measurement error.
\item Fit the degradation model to the $n$ simulated paths and compute
approximate ML estimates of the model parameters. 
\item Compute $\Fhat(130000)$ at 80$\degreesc$ as a function of the
parameter estimates. 
\item Repeat steps 2 to 4 $N$ times obtaining $\Fhat_1, \ldots, \Fhat_N$.
\item Compute some statistic quantifying the variation in these
point estimates (e.g., a sample standard deviation).
\item Return to step 1 until the list of $\Temp_{L}$ and $\pi_{L}$
combinations
has been exhausted.
\end{enumerate}
%
For example, Figure~\ref{figure:adtplan.sim.cdfs.ps} shows 
$\Fhat(t)$ curves computed from 
20 simulated degradation analysis experiments,
using a low accelerated temperature of $\Temp_{L}=130\degreesc$
and allocation $\pi_{L}=.1$. Overall the $\Fhat$ estimates in 
Figure ~\ref{figure:adtplan.sim.cdfs.ps} show a considerable amount of
variability. At 130 thousand hours, however, most of the simulated $\Fhat$
estimates are close to 0. Of course, if the specified planning values
are optimistic [in terms of having $\Fhat(130000)$ be much less than 
the true $F(130000)$], these calculations could be misleading.
Thus, one should use sensitivity analysis to investigate the effect that
deviations from the assumed (uncertain) inputs will have on conclusions.

%-------------------------------------------------------------------
\begin{figure}
\splusbookfigure{\figurehome/adtplan.sim.cdfs.ps} 
\caption{An example of variation in failure-time distributions for
$(\Temp_{L}, \, \pi_{L}) = (130 \degreesc, \, .1)$ test parameter
combination.}
\label{figure:adtplan.sim.cdfs.ps}
\end{figure}
%-------------------------------------------------------------------
%
%-------------------------------------------------------------------
\begin{figure}
\splusbookfigure{\figurehome/adtplan.contour.ps}
\caption{Contour plot of sample standard deviations of
the simulated degradation $\Fhat(130000)$ values for different
proposed compromise accelerated degradation plans.}
\label{figure:adtplan.contour.ps}
\end{figure}
%-------------------------------------------------------------------
\subsection{Evaluation of test plans over a 
grid of $\boldsymbol{(\Temp_{L}, \, \pi_{L})}$ values}

Test plan simulations were run at each combination of $(\Temp_{L}, \,
\pi_{L})$ values for $\Temp_{L}= 130(5)160$ and $\pi_{L}=
.05(.05).30$. At each point in the grid, 2000 tests were
simulated. For 32 out of the 42 combinations of $(\Temp_{L}, \,
\pi_{L})$ values, the ML estimation algorithm  converged for all 2000 samples. For 
the combination $(\Temp_{L}, \, \pi_{L})=(135,.25)$, about 6\% of
the simulated tests resulted in samples that did not converge and
about 1\% failed to converge at $(\Temp_{L}, \, \pi_{L})=(130, \,
.20)$ and (130, \,.30). In the other 7 cells with some convergence
difficulties, the percentage was on the order of .5\% or less. The
nonconverging samples were omitted from summarizing computations.

At each point the sample standard deviation of the
$\Fhat(130000)$ values was computed.
Figure~\ref{figure:adtplan.contour.ps} is a contour plot of the
results.  The original test plan used $\Temp_{L}=150\degreesc$ and
$\pi_{L}=.2$. The orientation of the contour lines suggests that
higher precision can be obtained for estimating $F(130000)$ by using
tests having lower levels of $\Temp_{L}$. At lower ranges of
temperature, precision seems not to be highly dependent on
$\pi_{L}$. The final recommendation was to choose a plan with
$(\Temp_{L}, \, \pi_{L})=(130, \, .15)$. In comparison with the original
test plan, this would reduce the amount of extrapolation in
temperature, provide a test plan with somewhat more precision, and be
safely away from the points that had convergence difficulties.


%----------------------------------------------------------------------
%----------------------------------------------------------------------
\section*{Bibliographic Notes}

The example in Section~\ref{section:cen.mixed.pop} was adapted from
Hahn and Meeker~(1982b).  In addition to the unequal censoring
example presented in Section~\ref{section:cen.mixed.pop}, Hahn and
Meeker~(1982a, 1982b) describe several other potential pitfalls that
arise in the analysis of life data.  The material in
Section~\ref{section:bayes.acceleration} builds on work done in
Lerch and Meeker~(1998).  The example in
Section~\ref{section:crisk.mixture.model} was extracted from Chan and
Meeker~(1998) who also describe in detail a Monte Carlo-based method
for finding likelihood-based confidence intervals for functions of
model parameters when one is faced with a model with many
parameters.  The material in
Section~\ref{section:fatigue.limit.model} was adapted from Pascual
and Meeker~(1997).  In addition to analyzing the superalloy data,
Pascual and Meeker~(1997) also investigate the effect that censoring
has on the ability to estimate the fatigue-limit parameter. Pascual
and Meeker~(1998a) describe a related regression model for fatigue
data, suggested in Nelson~(1990a, page 93). In this model, the
fatigue-limit parameter is allowed to vary from unit to unit.  Morse
and Meeker~(1998) provide more information on the use of simulation
methods to help plan accelerated degradation tests. Boulanger and
Escobar~(1994) also describe methods for planning accelerated
degradation tests.
